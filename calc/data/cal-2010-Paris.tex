\documentclass[a4paper,11pt,landscape]{article}
\usepackage[sort&compress,square,numbers]{natbib}

\usepackage[xetex]{graphicx}
%\usepackage{fullpage}
\usepackage{multirow}
\usepackage[normalsections]{savetrees}
\usepackage{euler}
\usepackage{fontspec}
\usepackage{xltxtra}
\usepackage{url}
\usepackage{multicol}
\usepackage{bbding}
% PDF SETUP
% ---- FILL IN HERE THE DOC TITLE AND AUTHOR
%\usepackage[bookmarks, colorlinks, breaklinks, pdftitle={Karthik Raman - vita},pdfauthor={Karthik Raman}]{hyperref} 
\usepackage[dvipsnames]{xcolor} 
\usepackage{wasysym} 
%\hypersetup{linkcolor=Sepia,citecolor=blue,filecolor=black,urlcolor=Blue} 


\defaultfontfeatures{Scale=MatchLowercase,Mapping=tex-text}
\setmainfont{Scala Sans LF}
\setsansfont{Sanskrit 2003:script=deva}

\newcommand{\caldata}[8]{%
\begin{minipage}{2.6cm}
\begin{minipage}[t]{1.6cm}
\vspace{.2ex}
%\mbox{}\\
%\SunshineOpenCircled
\mbox{{\sun\tiny\UParrow} \small #2}\\
\mbox{{\sun\tiny\DOWNarrow} \small  #6}\\
\scriptsize
\mbox{(\textsf{स} #3)}\\
\mbox{#7}\\
\mbox{#8}\\
\mbox{\textsf{राहु~} #4}\\
\mbox{\textsf{यम~} #5}\\
\end{minipage}\begin{minipage}[c]{1.cm}
\vspace{.4ex}
\begin{flushright} \textcolor{blue}{\font\x="Plantin Std" at 24 pt\x #1}
\end{flushright}
\end{minipage}
\end{minipage}
}

\addtolength{\headsep}{-3ex}
\pagestyle{empty}
\newcommand{\ahoratram}{\textsf{अहोरात्रम्}}
\newcommand{\ashwini}{अश्विनी}
\newcommand{\apabharani}{अपभरणी}
\newcommand{\krittika}{कृत्तिका}
\newcommand{\rohini}{रोहिणी}
\newcommand{\mrigashirsha}{मृगशीर्ष}
\newcommand{\ardra}{आर्द्रा}
\newcommand{\punarvasu}{पुनर्वसू}
\newcommand{\pushya}{पुष्य}
\newcommand{\ashresha}{आश्रेषा}
\newcommand{\magha}{मघा}
\newcommand{\purvaphalguni}{पूर्वफल्गुनी}
\newcommand{\uttaraphalguni}{उत्तरफल्गुनी}
\newcommand{\hasta}{हस्त}
\newcommand{\chitra}{चित्रा}
\newcommand{\svati}{स्वाति}
\newcommand{\vishakha}{विशाखा}
\newcommand{\anuradha}{अनूराधा}
\newcommand{\jyeshtha}{ज्येष्ठा}
\newcommand{\mula}{मूला}
\newcommand{\purvashadha}{पूर्वाषाढा}
\newcommand{\uttarashadha}{उत्तराषाढा}
\newcommand{\shravana}{श्रवण}
\newcommand{\shravishtha}{श्रविष्ठा}
\newcommand{\shatabhishak}{शतभिषक्}
\newcommand{\proshthapada}{प्रोष्ठपदा}
\newcommand{\uttaraproshthapada}{उत्तरप्रोष्ठपदा}
\newcommand{\revati}{रेवती}

\newcommand{\spra}{शुक्ल प्रथमा}
\newcommand{\sdvi}{शुक्ल द्वितीया}
\newcommand{\stri}{शुक्ल तृतीया}
\newcommand{\scha}{शुक्ल चतुर्थी}
\newcommand{\spanc}{शुक्ल पञ्चमी}
\newcommand{\ssha}{शुक्ल षष्ठी}
\newcommand{\ssap}{शुक्ल सप्तमी}
\newcommand{\sasht}{शुक्ल अष्टमी}
\newcommand{\snav}{शुक्ल नवमी}
\newcommand{\sdas}{शुक्ल दशमी}
\newcommand{\seka}{शुक्ल एकादशी}
\newcommand{\sdva}{शुक्ल द्वादशी}
\newcommand{\stra}{शुक्ल त्रयोदशी}
\newcommand{\schaturdashi}{शुक्ल चतुर्दशी}
\newcommand{\purnima}{\fullmoon~पूर्णिमा}
\newcommand{\kpra}{कृष्ण प्रथमा}
\newcommand{\kdvi}{कृष्ण द्वितीया}
\newcommand{\ktri}{कृष्ण तृतीया}
\newcommand{\kcha}{कृष्ण चतुर्थी}
\newcommand{\kpanc}{कृष्ण पञ्चमी}
\newcommand{\ksha}{कृष्ण षष्ठी}
\newcommand{\ksap}{कृष्ण सप्तमी}
\newcommand{\kasht}{कृष्ण अष्टमी}
\newcommand{\knav}{कृष्ण नवमी}
\newcommand{\kdas}{कृष्ण दशमी}
\newcommand{\keka}{कृष्ण एकादशी}
\newcommand{\kdva}{कृष्ण द्वादशी}
\newcommand{\ktra}{कृष्ण त्रयोदशी}
\newcommand{\kchaturdashi}{कृष्ण चतुर्दशी}
\newcommand{\ama}{\newmoon~अमावस्या}
\begin{document}
\pagestyle{empty}
\begin{center}
\mbox{}\\[2.5in]
\hrule\mbox{}
\mbox{}\\[1ex]
\mbox{}
{\font\x="Warnock Pro" at 60 pt\x 2010\\[0.5cm]}
\mbox{}
{\font\x="Warnock Pro" at 48 pt\x \uppercase{Paris}\\[0.3cm]}
\hrule
\begin{tabular}{|c|c|c|c|c|c|c|}
\multicolumn{7}{c}{\Large \bfseries JANUARY 2010}\\[3mm]
\hline
\textbf{SUN} & \textbf{MON} & \textbf{TUE} & \textbf{WED} & \textbf{THU} & \textbf{FRI} & \textbf{SAT} \\ \hline
{}  &
{}  &
{}  &
{}  &
{}  &
\caldata{1}{\sunmonth{धनुः}{17}{}}{\sundata{08:45}{17:02}{10:24}}{\textsf{\krishna~प्रथमा} {\tiny \RIGHTarrow} 16:43\hspace{2ex}}{\textsf{पुनर्वसू} {\tiny \RIGHTarrow} 23:26\hspace{2ex}}{11:51-12:53}{14:57-15:59} 
&
%मकर
\caldata{2}{\sunmonth{धनुः}{18}{}}{\sundata{08:45}{17:03}{10:24}}{\textsf{\krishna~द्वितीया} {\tiny \RIGHTarrow} 13:14\hspace{2ex}}{\textsf{पुष्य} {\tiny \RIGHTarrow} 20:43\hspace{2ex}}{10:49-11:51}{13:56-14:58} 
\\ \hline
%मकर
\caldata{3}{\sunmonth{धनुः}{19}{}}{\sundata{08:45}{17:04}{10:24}}{\textsf{\krishna~तृतीया} {\tiny \RIGHTarrow} 09:51\hspace{2ex}}{\textsf{आश्रेषा} {\tiny \RIGHTarrow} 18:10\hspace{2ex}}{16:01-17:04}{12:54-13:56} 
&
%मकर
\caldata{4}{\sunmonth{धनुः}{20}{}}{\sundata{08:45}{17:05}{10:25}}{\textsf{\krishna~पञ्चमी} {\tiny \RIGHTarrow} 04:04(+1)}{\textsf{मघा} {\tiny \RIGHTarrow} 15:56\hspace{2ex}}{09:47-10:50}{11:52-12:55} 
&
%मकर
\caldata{5}{\sunmonth{धनुः}{21}{}}{\sundata{08:45}{17:06}{10:25}}{\textsf{\krishna~षष्ठी} {\tiny \RIGHTarrow} 01:53(+1)}{\textsf{पूर्वफल्गुनी} {\tiny \RIGHTarrow} 14:09\hspace{2ex}}{15:00-16:03}{10:50-11:52} 
&
%मकर
\caldata{6}{\sunmonth{धनुः}{22}{}}{\sundata{08:45}{17:08}{10:25}}{\textsf{\krishna~सप्तमी} {\tiny \RIGHTarrow} 00:18(+1)}{\textsf{उत्तरफल्गुनी} {\tiny \RIGHTarrow} 12:53\hspace{2ex}}{12:56-13:59}{09:47-10:50} 
&
%मकर
\caldata{7}{\sunmonth{धनुः}{23}{}}{\sundata{08:44}{17:09}{10:25}}{\textsf{\krishna~अष्टमी} {\tiny \RIGHTarrow} 23:20\hspace{2ex}}{\textsf{हस्त} {\tiny \RIGHTarrow} 12:15\hspace{2ex}}{13:59-15:02}{08:44-09:47} 
&
%मकर
\caldata{8}{\sunmonth{धनुः}{24}{}}{\sundata{08:44}{17:10}{10:25}}{\textsf{\krishna~नवमी} {\tiny \RIGHTarrow} 23:01\hspace{2ex}}{\textsf{चित्रा} {\tiny \RIGHTarrow} 12:14\hspace{2ex}}{11:53-12:57}{15:03-16:06} 
&
%मकर
\caldata{9}{\sunmonth{धनुः}{25}{}}{\sundata{08:43}{17:11}{10:24}}{\textsf{\krishna~दशमी} {\tiny \RIGHTarrow} 23:20\hspace{2ex}}{\textsf{स्वाति} {\tiny \RIGHTarrow} 12:52\hspace{2ex}}{10:50-11:53}{14:00-15:04} 
\\ \hline
%मकर
\caldata{10}{\sunmonth{धनुः}{26}{}}{\sundata{08:43}{17:13}{10:25}}{\textsf{\krishna~एकादशी} {\tiny \RIGHTarrow} 00:14(+1)}{\textsf{विशाखा} {\tiny \RIGHTarrow} 14:05\hspace{2ex}}{16:09-17:13}{12:58-14:01} 
&
%मकर
\caldata{11}{\sunmonth{धनुः}{27}{}}{\sundata{08:43}{17:14}{10:25}}{\textsf{\krishna~द्वादशी} {\tiny \RIGHTarrow} 01:39(+1)}{\textsf{अनूराधा} {\tiny \RIGHTarrow} 15:49\hspace{2ex}}{09:46-10:50}{11:54-12:58} 
&
%मकर
\caldata{12}{\sunmonth{धनुः}{28}{}}{\sundata{08:42}{17:15}{10:24}}{\textsf{\krishna~त्रयोदशी} {\tiny \RIGHTarrow} 03:30(+1)}{\textsf{ज्येष्ठा} {\tiny \RIGHTarrow} 18:01\hspace{2ex}}{15:06-16:10}{10:50-11:54} 
&
%मकर
\caldata{13}{\sunmonth{धनुः}{29}{{\textsf{धनुः} {\tiny \RIGHTarrow} 08:01(+1)}}}{\sundata{08:41}{17:17}{10:24}}{\textsf{\krishna~चतुर्दशी} {\tiny \RIGHTarrow} 05:42(+1)}{\textsf{मूला} {\tiny \RIGHTarrow} 20:33\hspace{2ex}}{12:59-14:03}{09:45-10:50} 
&
%मकर
\caldata{14}{\sunmonth{मकर}{1}{}}{\sundata{08:41}{17:18}{10:24}}{\textsf{\newmoon~अमावस्या} {\tiny \RIGHTarrow} 08:11(+1)}{\textsf{पूर्वाषाढा} {\tiny \RIGHTarrow} 23:23\hspace{2ex}}{14:04-15:08}{08:41-09:45} 
&
%मकर
\caldata{15}{\sunmonth{मकर}{2}{}}{\sundata{08:40}{17:19}{10:23}}{\textsf{\shukla~प्रथमा} {\tiny \RIGHTarrow} \textsf{अहोरात्रम्}}{\textsf{उत्तराषाढा} {\tiny \RIGHTarrow} 02:24(+1)}{11:54-12:59}{15:09-16:14} 
&
%मकर
\caldata{16}{\sunmonth{मकर}{3}{}}{\sundata{08:39}{17:21}{10:23}}{\textsf{\shukla~प्रथमा} {\tiny \RIGHTarrow} 10:50\hspace{2ex}}{\textsf{श्रवण} {\tiny \RIGHTarrow} 05:31(+1)}{10:49-11:54}{14:05-15:10} 
\\ \hline
%मकर
\caldata{17}{\sunmonth{मकर}{4}{}}{\sundata{08:39}{17:22}{10:23}}{\textsf{\shukla~द्वितीया} {\tiny \RIGHTarrow} 13:34\hspace{2ex}}{\textsf{श्रविष्ठा} {\tiny \RIGHTarrow} 08:37(+1)}{16:16-17:22}{13:00-14:05} 
&
%मकर
\caldata{18}{\sunmonth{मकर}{5}{}}{\sundata{08:38}{17:24}{10:23}}{\textsf{\shukla~तृतीया} {\tiny \RIGHTarrow} 16:14\hspace{2ex}}{\textsf{शतभिषक्} {\tiny \RIGHTarrow} \textsf{अहोरात्रम्}}{09:43-10:49}{11:55-13:01} 
&
%मकर
\caldata{19}{\sunmonth{मकर}{6}{}}{\sundata{08:37}{17:25}{10:22}}{\textsf{\shukla~चतुर्थी} {\tiny \RIGHTarrow} 18:42\hspace{2ex}}{\textsf{शतभिषक्} {\tiny \RIGHTarrow} 11:35\hspace{2ex}}{15:13-16:19}{10:49-11:55} 
&
%मकर
\caldata{20}{\sunmonth{मकर}{7}{}}{\sundata{08:36}{17:27}{10:22}}{\textsf{\shukla~पञ्चमी} {\tiny \RIGHTarrow} 20:51\hspace{2ex}}{\textsf{प्रोष्ठपदा} {\tiny \RIGHTarrow} 14:16\hspace{2ex}}{13:01-14:07}{09:42-10:48} 
&
%मकर
\caldata{21}{\sunmonth{मकर}{8}{}}{\sundata{08:35}{17:28}{10:21}}{\textsf{\shukla~षष्ठी} {\tiny \RIGHTarrow} 22:30\hspace{2ex}}{\textsf{उत्तरप्रोष्ठपदा} {\tiny \RIGHTarrow} 16:31\hspace{2ex}}{14:08-15:14}{08:35-09:41} 
&
%मकर
\caldata{22}{\sunmonth{मकर}{9}{}}{\sundata{08:34}{17:30}{10:21}}{\textsf{\shukla~सप्तमी} {\tiny \RIGHTarrow} 23:32\hspace{2ex}}{\textsf{रेवती} {\tiny \RIGHTarrow} 18:13\hspace{2ex}}{11:55-13:02}{15:16-16:23} 
&
%मकर
\caldata{23}{\sunmonth{मकर}{10}{}}{\sundata{08:33}{17:31}{10:20}}{\textsf{\shukla~अष्टमी} {\tiny \RIGHTarrow} 23:52\hspace{2ex}}{\textsf{अश्विनी} {\tiny \RIGHTarrow} 19:14\hspace{2ex}}{10:47-11:54}{14:09-15:16} 
\\ \hline
%मकर
\caldata{24}{\sunmonth{मकर}{11}{}}{\sundata{08:32}{17:33}{10:20}}{\textsf{\shukla~नवमी} {\tiny \RIGHTarrow} 23:25\hspace{2ex}}{\textsf{अपभरणी} {\tiny \RIGHTarrow} 19:32\hspace{2ex}}{16:25-17:33}{13:02-14:10} 
&
%मकर
\caldata{25}{\sunmonth{मकर}{12}{}}{\sundata{08:31}{17:35}{10:19}}{\textsf{\shukla~दशमी} {\tiny \RIGHTarrow} 22:11\hspace{2ex}}{\textsf{कृत्तिका} {\tiny \RIGHTarrow} 19:04\hspace{2ex}}{09:39-10:47}{11:55-13:03} 
&
%मकर
\caldata{26}{\sunmonth{मकर}{13}{}}{\sundata{08:30}{17:36}{10:19}}{\textsf{\shukla~एकादशी} {\tiny \RIGHTarrow} 20:13\hspace{2ex}}{\textsf{रोहिणी} {\tiny \RIGHTarrow} 17:52\hspace{2ex}}{15:19-16:27}{10:46-11:54} 
&
%मकर
\caldata{27}{\sunmonth{मकर}{14}{}}{\sundata{08:28}{17:38}{10:18}}{\textsf{\shukla~द्वादशी} {\tiny \RIGHTarrow} 17:37\hspace{2ex}}{\textsf{मृगशीर्ष} {\tiny \RIGHTarrow} 16:02\hspace{2ex}}{13:03-14:11}{09:36-10:45} 
&
%मकर
\caldata{28}{\sunmonth{मकर}{15}{}}{\sundata{08:27}{17:39}{10:17}}{\textsf{\shukla~त्रयोदशी} {\tiny \RIGHTarrow} 14:31\hspace{2ex}}{\textsf{आर्द्रा} {\tiny \RIGHTarrow} 13:41\hspace{2ex}}{14:12-15:21}{08:27-09:36} 
&
%मकर
\caldata{29}{\sunmonth{मकर}{16}{}}{\sundata{08:26}{17:41}{10:17}}{\textsf{\shukla~चतुर्दशी} {\tiny \RIGHTarrow} 11:01\hspace{2ex}}{\textsf{पुनर्वसू} {\tiny \RIGHTarrow} 10:56\hspace{2ex}}{11:54-13:03}{15:22-16:31} 
&
%मकर
\caldata{30}{\sunmonth{मकर}{17}{}}{\sundata{08:25}{17:43}{10:16}}{\textsf{\krishna~प्रथमा} {\tiny \RIGHTarrow} 03:30(+1)}{\textsf{आश्रेषा} {\tiny \RIGHTarrow} 04:55(+1)}{10:44-11:54}{14:13-15:23} 
\\ \hline
%मकर
\caldata{31}{\sunmonth{मकर}{18}{}}{\sundata{08:23}{17:44}{10:15}}{\textsf{\krishna~द्वितीया} {\tiny \RIGHTarrow} 23:52\hspace{2ex}}{\textsf{मघा} {\tiny \RIGHTarrow} 02:02(+1)}{16:33-17:44}{13:03-14:13} 
&
%मकर
{}  &
{}  &
{}  &
{}  &
{}  &
\\ \hline
\end{tabular}


%\clearpage
\begin{tabular}{|c|c|c|c|c|c|c|}
\multicolumn{7}{c}{\Large \bfseries FEBRUARY 2010}\\[3mm]
\hline
\textbf{SUN} & \textbf{MON} & \textbf{TUE} & \textbf{WED} & \textbf{THU} & \textbf{FRI} & \textbf{SAT} \\ \hline
{}  &
\caldata{1}{\sunmonth{मकर}{19}{}}{\sundata{08:22}{17:46}{10:14}}{\textsf{\krishna~तृतीया} {\tiny \RIGHTarrow} 20:31\hspace{2ex}}{\textsf{पूर्वफल्गुनी} {\tiny \RIGHTarrow} 23:29\hspace{2ex}}{09:32-10:43}{11:53-13:04} 
&
%मकर
\caldata{2}{\sunmonth{मकर}{20}{}}{\sundata{08:21}{17:48}{10:14}}{\textsf{\krishna~चतुर्थी} {\tiny \RIGHTarrow} 17:38\hspace{2ex}}{\textsf{उत्तरफल्गुनी} {\tiny \RIGHTarrow} 21:24\hspace{2ex}}{15:26-16:37}{10:42-11:53} 
&
%मकर
\caldata{3}{\sunmonth{मकर}{21}{}}{\sundata{08:19}{17:49}{10:13}}{\textsf{\krishna~पञ्चमी} {\tiny \RIGHTarrow} 15:19\hspace{2ex}}{\textsf{हस्त} {\tiny \RIGHTarrow} 19:56\hspace{2ex}}{13:04-14:15}{09:30-10:41} 
&
%मकर
\caldata{4}{\sunmonth{मकर}{22}{}}{\sundata{08:18}{17:51}{10:12}}{\textsf{\krishna~षष्ठी} {\tiny \RIGHTarrow} 13:44\hspace{2ex}}{\textsf{चित्रा} {\tiny \RIGHTarrow} 19:12\hspace{2ex}}{14:16-15:27}{08:18-09:29} 
&
%मकर
\caldata{5}{\sunmonth{मकर}{23}{}}{\sundata{08:16}{17:53}{10:11}}{\textsf{\krishna~सप्तमी} {\tiny \RIGHTarrow} 12:57\hspace{2ex}}{\textsf{स्वाति} {\tiny \RIGHTarrow} 19:14\hspace{2ex}}{11:52-13:04}{15:28-16:40} 
&
%मकर
\caldata{6}{\sunmonth{मकर}{24}{}}{\sundata{08:15}{17:54}{10:10}}{\textsf{\krishna~अष्टमी} {\tiny \RIGHTarrow} 12:59\hspace{2ex}}{\textsf{विशाखा} {\tiny \RIGHTarrow} 20:04\hspace{2ex}}{10:39-11:52}{14:16-15:29} 
\\ \hline
%मकर
\caldata{7}{\sunmonth{मकर}{25}{}}{\sundata{08:13}{17:56}{10:09}}{\textsf{\krishna~नवमी} {\tiny \RIGHTarrow} 13:48\hspace{2ex}}{\textsf{अनूराधा} {\tiny \RIGHTarrow} 21:38\hspace{2ex}}{16:43-17:56}{13:04-14:17} 
&
%मकर
\caldata{8}{\sunmonth{मकर}{26}{}}{\sundata{08:12}{17:58}{10:09}}{\textsf{\krishna~दशमी} {\tiny \RIGHTarrow} 15:19\hspace{2ex}}{\textsf{ज्येष्ठा} {\tiny \RIGHTarrow} 23:48\hspace{2ex}}{09:25-10:38}{11:51-13:05} 
&
%मकर
\caldata{9}{\sunmonth{मकर}{27}{}}{\sundata{08:10}{17:59}{10:07}}{\textsf{\krishna~एकादशी} {\tiny \RIGHTarrow} 17:22\hspace{2ex}}{\textsf{मूला} {\tiny \RIGHTarrow} 02:27(+1)}{15:31-16:45}{10:37-11:50} 
&
%मकर
\caldata{10}{\sunmonth{मकर}{28}{}}{\sundata{08:09}{18:01}{10:07}}{\textsf{\krishna~द्वादशी} {\tiny \RIGHTarrow} 19:48\hspace{2ex}}{\textsf{पूर्वाषाढा} {\tiny \RIGHTarrow} 05:25(+1)}{13:05-14:19}{09:23-10:37} 
&
%मकर
\caldata{11}{\sunmonth{मकर}{29}{}}{\sundata{08:07}{18:03}{10:06}}{\textsf{\krishna~त्रयोदशी} {\tiny \RIGHTarrow} 22:27\hspace{2ex}}{\textsf{उत्तराषाढा} {\tiny \RIGHTarrow} \textsf{अहोरात्रम्}}{14:19-15:34}{08:07-09:21} 
&
%मकर
\caldata{12}{\sunmonth{मकर}{30}{{\textsf{मकर} {\tiny \RIGHTarrow} 21:00\hspace{2ex}}}}{\sundata{08:05}{18:04}{10:04}}{\textsf{\krishna~चतुर्दशी} {\tiny \RIGHTarrow} 01:10(+1)}{\textsf{उत्तराषाढा} {\tiny \RIGHTarrow} 08:32\hspace{2ex}}{11:49-13:04}{15:34-16:49} 
&
%मकर
\caldata{13}{\sunmonth{कुम्भ}{1}{}}{\sundata{08:04}{18:06}{10:04}}{\textsf{\newmoon~अमावस्या} {\tiny \RIGHTarrow} 03:50(+1)}{\textsf{श्रवण} {\tiny \RIGHTarrow} 11:40\hspace{2ex}}{10:34-11:49}{14:20-15:35} 
\\ \hline
%मकर
\caldata{14}{\sunmonth{कुम्भ}{2}{}}{\sundata{08:02}{18:07}{10:03}}{\textsf{\shukla~प्रथमा} {\tiny \RIGHTarrow} 06:21(+1)}{\textsf{श्रविष्ठा} {\tiny \RIGHTarrow} 14:42\hspace{2ex}}{16:51-18:07}{13:04-14:20} 
&
%मकर
\caldata{15}{\sunmonth{कुम्भ}{3}{}}{\sundata{08:00}{18:09}{10:01}}{\textsf{\shukla~द्वितीया} {\tiny \RIGHTarrow} \textsf{अहोरात्रम्}}{\textsf{शतभिषक्} {\tiny \RIGHTarrow} 17:34\hspace{2ex}}{09:16-10:32}{11:48-13:04} 
&
%मकर
\caldata{16}{\sunmonth{कुम्भ}{4}{}}{\sundata{07:58}{18:11}{10:00}}{\textsf{\shukla~द्वितीया} {\tiny \RIGHTarrow} 08:38\hspace{2ex}}{\textsf{प्रोष्ठपदा} {\tiny \RIGHTarrow} 20:10\hspace{2ex}}{15:37-16:54}{10:31-11:47} 
&
%मकर
\caldata{17}{\sunmonth{कुम्भ}{5}{}}{\sundata{07:57}{18:12}{10:00}}{\textsf{\shukla~तृतीया} {\tiny \RIGHTarrow} 10:36\hspace{2ex}}{\textsf{उत्तरप्रोष्ठपदा} {\tiny \RIGHTarrow} 22:28\hspace{2ex}}{13:04-14:21}{09:13-10:30} 
&
%मकर
\caldata{18}{\sunmonth{कुम्भ}{6}{}}{\sundata{07:55}{18:14}{09:58}}{\textsf{\shukla~चतुर्थी} {\tiny \RIGHTarrow} 12:10\hspace{2ex}}{\textsf{रेवती} {\tiny \RIGHTarrow} 00:21(+1)}{14:21-15:39}{07:55-09:12} 
&
%मकर
\caldata{19}{\sunmonth{कुम्भ}{7}{}}{\sundata{07:53}{18:16}{09:57}}{\textsf{\shukla~पञ्चमी} {\tiny \RIGHTarrow} 13:17\hspace{2ex}}{\textsf{अश्विनी} {\tiny \RIGHTarrow} 01:47(+1)}{11:46-13:04}{15:40-16:58} 
&
%मकर
\caldata{20}{\sunmonth{कुम्भ}{8}{}}{\sundata{07:51}{18:17}{09:56}}{\textsf{\shukla~षष्ठी} {\tiny \RIGHTarrow} 13:53\hspace{2ex}}{\textsf{अपभरणी} {\tiny \RIGHTarrow} 02:40(+1)}{10:27-11:45}{14:22-15:40} 
\\ \hline
%मकर
\caldata{21}{\sunmonth{कुम्भ}{9}{}}{\sundata{07:49}{18:19}{09:55}}{\textsf{\shukla~सप्तमी} {\tiny \RIGHTarrow} 13:52\hspace{2ex}}{\textsf{कृत्तिका} {\tiny \RIGHTarrow} 02:56(+1)}{17:00-18:19}{13:04-14:22} 
&
%मकर
\caldata{22}{\sunmonth{कुम्भ}{10}{}}{\sundata{07:47}{18:21}{09:53}}{\textsf{\shukla~अष्टमी} {\tiny \RIGHTarrow} 13:14\hspace{2ex}}{\textsf{रोहिणी} {\tiny \RIGHTarrow} 02:34(+1)}{09:06-10:25}{11:44-13:04} 
&
%मकर
\caldata{23}{\sunmonth{कुम्भ}{11}{}}{\sundata{07:46}{18:22}{09:53}}{\textsf{\shukla~नवमी} {\tiny \RIGHTarrow} 11:56\hspace{2ex}}{\textsf{मृगशीर्ष} {\tiny \RIGHTarrow} 01:32(+1)}{15:43-17:02}{10:25-11:44} 
&
%मकर
\caldata{24}{\sunmonth{कुम्भ}{12}{}}{\sundata{07:44}{18:24}{09:52}}{\textsf{\shukla~दशमी} {\tiny \RIGHTarrow} 10:00\hspace{2ex}}{\textsf{आर्द्रा} {\tiny \RIGHTarrow} 23:54\hspace{2ex}}{13:04-14:24}{09:04-10:24} 
&
%मकर
\caldata{25}{\sunmonth{कुम्भ}{13}{}}{\sundata{07:42}{18:25}{09:50}}{\textsf{\shukla~द्वादशी} {\tiny \RIGHTarrow} 04:23(+1)}{\textsf{पुनर्वसू} {\tiny \RIGHTarrow} 21:45\hspace{2ex}}{14:23-15:44}{07:42-09:02} 
&
%मकर
\caldata{26}{\sunmonth{कुम्भ}{14}{}}{\sundata{07:40}{18:27}{09:49}}{\textsf{\shukla~त्रयोदशी} {\tiny \RIGHTarrow} 00:57(+1)}{\textsf{पुष्य} {\tiny \RIGHTarrow} 19:12\hspace{2ex}}{11:42-13:03}{15:45-17:06} 
&
%मकर
\caldata{27}{\sunmonth{कुम्भ}{15}{}}{\sundata{07:38}{18:29}{09:48}}{\textsf{\shukla~चतुर्दशी} {\tiny \RIGHTarrow} 21:20\hspace{2ex}}{\textsf{आश्रेषा} {\tiny \RIGHTarrow} 16:23\hspace{2ex}}{10:20-11:42}{14:24-15:46} 
\\ \hline
%मकर
\caldata{28}{\sunmonth{कुम्भ}{16}{}}{\sundata{07:36}{18:30}{09:46}}{\textsf{\fullmoon~पूर्णिमा} {\tiny \RIGHTarrow} 17:39\hspace{2ex}}{\textsf{मघा} {\tiny \RIGHTarrow} 13:27\hspace{2ex}}{17:08-18:30}{13:03-14:24} 
&
%मकर
{}  &
{}  &
{}  &
{}  &
{}  &
\\ \hline
\end{tabular}


%\clearpage
\begin{tabular}{|c|c|c|c|c|c|c|}
\multicolumn{7}{c}{\Large \bfseries MARCH 2010}\\[3mm]
\hline
\textbf{SUN} & \textbf{MON} & \textbf{TUE} & \textbf{WED} & \textbf{THU} & \textbf{FRI} & \textbf{SAT} \\ \hline
{}  &
\caldata{1}{\sunmonth{कुम्भ}{17}{}}{\sundata{07:34}{18:32}{09:45}}{\textsf{\krishna~प्रथमा} {\tiny \RIGHTarrow} 14:04\hspace{2ex}}{\textsf{पूर्वफल्गुनी} {\tiny \RIGHTarrow} 10:35\hspace{2ex}}{08:56-10:18}{11:40-13:03} 
&
%मकर
\caldata{2}{\sunmonth{कुम्भ}{18}{}}{\sundata{07:32}{18:33}{09:44}}{\textsf{\krishna~द्वितीया} {\tiny \RIGHTarrow} 10:46\hspace{2ex}}{\textsf{उत्तरफल्गुनी} {\tiny \RIGHTarrow} 07:58\hspace{2ex}}{15:47-17:10}{10:17-11:39} 
&
%मकर
\caldata{3}{\sunmonth{कुम्भ}{19}{}}{\sundata{07:30}{18:35}{09:43}}{\textsf{\krishna~तृतीया} {\tiny \RIGHTarrow} 07:55\hspace{2ex}}{\textsf{चित्रा} {\tiny \RIGHTarrow} 04:22(+1)}{13:02-14:25}{08:53-10:16} 
&
%मकर
\caldata{4}{\sunmonth{कुम्भ}{20}{}}{\sundata{07:28}{18:37}{09:41}}{\textsf{\krishna~पञ्चमी} {\tiny \RIGHTarrow} 04:24(+1)}{\textsf{स्वाति} {\tiny \RIGHTarrow} 03:38(+1)}{14:26-15:49}{07:28-08:51} 
&
%मकर
\caldata{5}{\sunmonth{कुम्भ}{21}{}}{\sundata{07:26}{18:38}{09:40}}{\textsf{\krishna~षष्ठी} {\tiny \RIGHTarrow} 03:53(+1)}{\textsf{विशाखा} {\tiny \RIGHTarrow} 03:43(+1)}{11:38-13:02}{15:50-17:14} 
&
%मकर
\caldata{6}{\sunmonth{कुम्भ}{22}{}}{\sundata{07:24}{18:40}{09:39}}{\textsf{\krishna~सप्तमी} {\tiny \RIGHTarrow} 04:14(+1)}{\textsf{अनूराधा} {\tiny \RIGHTarrow} 04:40(+1)}{10:13-11:37}{14:26-15:51} 
\\ \hline
%मकर
\caldata{7}{\sunmonth{कुम्भ}{23}{}}{\sundata{07:22}{18:41}{09:37}}{\textsf{\krishna~अष्टमी} {\tiny \RIGHTarrow} 05:25(+1)}{\textsf{ज्येष्ठा} {\tiny \RIGHTarrow} 06:24(+1)}{17:16-18:41}{13:01-14:26} 
&
%मकर
\caldata{8}{\sunmonth{कुम्भ}{24}{}}{\sundata{07:20}{18:43}{09:36}}{\textsf{\krishna~नवमी} {\tiny \RIGHTarrow} 07:17(+1)}{\textsf{मूला} {\tiny \RIGHTarrow} \textsf{अहोरात्रम्}}{08:45-10:10}{11:36-13:01} 
&
%मकर
\caldata{9}{\sunmonth{कुम्भ}{25}{}}{\sundata{07:18}{18:44}{09:35}}{\textsf{\krishna~दशमी} {\tiny \RIGHTarrow} \textsf{अहोरात्रम्}}{\textsf{मूला} {\tiny \RIGHTarrow} 08:49\hspace{2ex}}{15:52-17:18}{10:09-11:35} 
&
%मकर
\caldata{10}{\sunmonth{कुम्भ}{26}{}}{\sundata{07:16}{18:46}{09:34}}{\textsf{\krishna~दशमी} {\tiny \RIGHTarrow} 09:41\hspace{2ex}}{\textsf{पूर्वाषाढा} {\tiny \RIGHTarrow} 11:43\hspace{2ex}}{13:01-14:27}{08:42-10:08} 
&
%मकर
\caldata{11}{\sunmonth{कुम्भ}{27}{}}{\sundata{07:14}{18:48}{09:32}}{\textsf{\krishna~एकादशी} {\tiny \RIGHTarrow} 12:21\hspace{2ex}}{\textsf{उत्तराषाढा} {\tiny \RIGHTarrow} 14:51\hspace{2ex}}{14:27-15:54}{07:14-08:40} 
&
%मकर
\caldata{12}{\sunmonth{कुम्भ}{28}{}}{\sundata{07:12}{18:49}{09:31}}{\textsf{\krishna~द्वादशी} {\tiny \RIGHTarrow} 15:04\hspace{2ex}}{\textsf{श्रवण} {\tiny \RIGHTarrow} 18:01\hspace{2ex}}{11:33-13:00}{15:54-17:21} 
&
%मकर
\caldata{13}{\sunmonth{कुम्भ}{29}{}}{\sundata{07:10}{18:51}{09:30}}{\textsf{\krishna~त्रयोदशी} {\tiny \RIGHTarrow} 17:39\hspace{2ex}}{\textsf{श्रविष्ठा} {\tiny \RIGHTarrow} 21:02\hspace{2ex}}{10:05-11:32}{14:28-15:55} 
\\ \hline
%मकर
\caldata{14}{\sunmonth{मीन}{1}{{\textsf{कुम्भ} {\tiny \RIGHTarrow} 17:52\hspace{2ex}}}}{\sundata{07:07}{18:52}{09:28}}{\textsf{\krishna~चतुर्दशी} {\tiny \RIGHTarrow} 19:59\hspace{2ex}}{\textsf{शतभिषक्} {\tiny \RIGHTarrow} 23:49\hspace{2ex}}{17:23-18:52}{12:59-14:27} 
&
%मकर
\caldata{15}{\sunmonth{मीन}{2}{}}{\sundata{07:05}{18:54}{09:26}}{\textsf{\newmoon~अमावस्या} {\tiny \RIGHTarrow} 21:58\hspace{2ex}}{\textsf{प्रोष्ठपदा} {\tiny \RIGHTarrow} 02:15(+1)}{08:33-10:02}{11:30-12:59} 
&
%मकर
\caldata{16}{\sunmonth{मीन}{3}{}}{\sundata{07:03}{18:55}{09:25}}{\textsf{\shukla~प्रथमा} {\tiny \RIGHTarrow} 23:35\hspace{2ex}}{\textsf{उत्तरप्रोष्ठपदा} {\tiny \RIGHTarrow} 04:20(+1)}{15:57-17:26}{10:01-11:30} 
&
%मकर
\caldata{17}{\sunmonth{मीन}{4}{}}{\sundata{07:01}{18:57}{09:24}}{\textsf{\shukla~द्वितीया} {\tiny \RIGHTarrow} 00:48(+1)}{\textsf{रेवती} {\tiny \RIGHTarrow} 06:02(+1)}{12:59-14:28}{08:30-10:00} 
&
%मकर
\caldata{18}{\sunmonth{मीन}{5}{}}{\sundata{06:59}{18:58}{09:22}}{\textsf{\shukla~तृतीया} {\tiny \RIGHTarrow} 01:36(+1)}{\textsf{अश्विनी} {\tiny \RIGHTarrow} \textsf{अहोरात्रम्}}{14:28-15:58}{06:59-08:28} 
&
%मकर
\caldata{19}{\sunmonth{मीन}{6}{}}{\sundata{06:57}{19:00}{09:21}}{\textsf{\shukla~चतुर्थी} {\tiny \RIGHTarrow} 02:01(+1)}{\textsf{अश्विनी} {\tiny \RIGHTarrow} 07:21\hspace{2ex}}{11:28-12:58}{15:59-17:29} 
&
%मकर
\caldata{20}{\sunmonth{मीन}{7}{}}{\sundata{06:55}{19:01}{09:20}}{\textsf{\shukla~पञ्चमी} {\tiny \RIGHTarrow} 02:00(+1)}{\textsf{अपभरणी} {\tiny \RIGHTarrow} 08:15\hspace{2ex}}{09:56-11:27}{14:28-15:59} 
\\ \hline
%मकर
\caldata{21}{\sunmonth{मीन}{8}{}}{\sundata{06:53}{19:03}{09:19}}{\textsf{\shukla~षष्ठी} {\tiny \RIGHTarrow} 01:32(+1)}{\textsf{कृत्तिका} {\tiny \RIGHTarrow} 08:45\hspace{2ex}}{17:31-19:03}{12:58-14:29} 
&
%मकर
\caldata{22}{\sunmonth{मीन}{9}{}}{\sundata{06:51}{19:04}{09:17}}{\textsf{\shukla~सप्तमी} {\tiny \RIGHTarrow} 00:35(+1)}{\textsf{रोहिणी} {\tiny \RIGHTarrow} 08:49\hspace{2ex}}{08:22-09:54}{11:25-12:57} 
&
%मकर
\caldata{23}{\sunmonth{मीन}{10}{}}{\sundata{06:49}{19:06}{09:16}}{\textsf{\shukla~अष्टमी} {\tiny \RIGHTarrow} 23:10\hspace{2ex}}{\textsf{मृगशीर्ष} {\tiny \RIGHTarrow} 08:24\hspace{2ex}}{16:01-17:33}{09:53-11:25} 
&
%मकर
\caldata{24}{\sunmonth{मीन}{11}{}}{\sundata{06:46}{19:07}{09:14}}{\textsf{\shukla~नवमी} {\tiny \RIGHTarrow} 21:15\hspace{2ex}}{\textsf{आर्द्रा} {\tiny \RIGHTarrow} 07:31\hspace{2ex}}{12:56-14:29}{08:18-09:51} 
&
%मकर
\caldata{25}{\sunmonth{मीन}{12}{}}{\sundata{06:44}{19:09}{09:13}}{\textsf{\shukla~दशमी} {\tiny \RIGHTarrow} 18:53\hspace{2ex}}{\textsf{पुष्य} {\tiny \RIGHTarrow} 04:21(+1)}{14:29-16:02}{06:44-08:17} 
&
%मकर
\caldata{26}{\sunmonth{मीन}{13}{}}{\sundata{06:42}{19:10}{09:11}}{\textsf{\shukla~एकादशी} {\tiny \RIGHTarrow} 16:08\hspace{2ex}}{\textsf{आश्रेषा} {\tiny \RIGHTarrow} 02:11(+1)}{11:22-12:56}{16:03-17:36} 
&
%मकर
\caldata{27}{\sunmonth{मीन}{14}{}}{\sundata{06:40}{19:12}{09:10}}{\textsf{\shukla~द्वादशी} {\tiny \RIGHTarrow} 13:05\hspace{2ex}}{\textsf{मघा} {\tiny \RIGHTarrow} 23:47\hspace{2ex}}{09:48-11:22}{14:30-16:04} 
\\ \hline
%मकर
\caldata{28}{\sunmonth{मीन}{15}{}}{\sundata{07:38}{20:13}{10:09}}{\textsf{\shukla~त्रयोदशी} {\tiny \RIGHTarrow} 10:51\hspace{2ex}}{\textsf{पूर्वफल्गुनी} {\tiny \RIGHTarrow} 22:17\hspace{2ex}}{18:38-20:13}{13:55-15:29} 
&
%मकर
\caldata{29}{\sunmonth{मीन}{16}{}}{\sundata{07:36}{20:15}{10:07}}{\textsf{\fullmoon~पूर्णिमा} {\tiny \RIGHTarrow} 04:26(+1)}{\textsf{उत्तरफल्गुनी} {\tiny \RIGHTarrow} 19:52\hspace{2ex}}{09:10-10:45}{12:20-13:55} 
&
%मकर
\caldata{30}{\sunmonth{मीन}{17}{}}{\sundata{07:34}{20:16}{10:06}}{\textsf{\krishna~प्रथमा} {\tiny \RIGHTarrow} 01:37(+1)}{\textsf{हस्त} {\tiny \RIGHTarrow} 17:40\hspace{2ex}}{17:05-18:40}{10:44-12:19} 
&
%मकर
\caldata{31}{\sunmonth{मीन}{18}{}}{\sundata{07:32}{20:18}{10:05}}{\textsf{\krishna~द्वितीया} {\tiny \RIGHTarrow} 23:17\hspace{2ex}}{\textsf{चित्रा} {\tiny \RIGHTarrow} 15:52\hspace{2ex}}{13:55-15:30}{09:07-10:43} 
&
%मकर
{}  &
{}  &
\\ \hline
\end{tabular}


%\clearpage
\begin{tabular}{|c|c|c|c|c|c|c|}
\multicolumn{7}{c}{\Large \bfseries APRIL 2010}\\[3mm]
\hline
\textbf{SUN} & \textbf{MON} & \textbf{TUE} & \textbf{WED} & \textbf{THU} & \textbf{FRI} & \textbf{SAT} \\ \hline
{}  &
{}  &
{}  &
{}  &
\caldata{1}{\sunmonth{मीन}{19}{}}{\sundata{07:30}{20:19}{10:03}}{\textsf{\krishna~तृतीया} {\tiny \RIGHTarrow} 21:34\hspace{2ex}}{\textsf{स्वाति} {\tiny \RIGHTarrow} 14:39\hspace{2ex}}{15:30-17:06}{07:30-09:06} 
&
%मकर
\caldata{2}{\sunmonth{मीन}{20}{}}{\sundata{07:28}{20:21}{10:02}}{\textsf{\krishna~चतुर्थी} {\tiny \RIGHTarrow} 20:37\hspace{2ex}}{\textsf{विशाखा} {\tiny \RIGHTarrow} 14:09\hspace{2ex}}{12:17-13:54}{17:07-18:44} 
&
%मकर
\caldata{3}{\sunmonth{मीन}{21}{}}{\sundata{07:26}{20:22}{10:01}}{\textsf{\krishna~पञ्चमी} {\tiny \RIGHTarrow} 20:29\hspace{2ex}}{\textsf{अनूराधा} {\tiny \RIGHTarrow} 14:28\hspace{2ex}}{10:40-12:17}{15:31-17:08} 
\\ \hline
%मकर
\caldata{4}{\sunmonth{मीन}{22}{}}{\sundata{07:23}{20:24}{09:59}}{\textsf{\krishna~षष्ठी} {\tiny \RIGHTarrow} 21:12\hspace{2ex}}{\textsf{ज्येष्ठा} {\tiny \RIGHTarrow} 15:36\hspace{2ex}}{18:46-20:24}{13:53-15:31} 
&
%मकर
\caldata{5}{\sunmonth{मीन}{23}{}}{\sundata{07:21}{20:25}{09:57}}{\textsf{\krishna~सप्तमी} {\tiny \RIGHTarrow} 22:40\hspace{2ex}}{\textsf{मूला} {\tiny \RIGHTarrow} 17:31\hspace{2ex}}{08:59-10:37}{12:15-13:53} 
&
%मकर
\caldata{6}{\sunmonth{मीन}{24}{}}{\sundata{07:19}{20:27}{09:56}}{\textsf{\krishna~अष्टमी} {\tiny \RIGHTarrow} 00:46(+1)}{\textsf{पूर्वाषाढा} {\tiny \RIGHTarrow} 20:02\hspace{2ex}}{17:10-18:48}{10:36-12:14} 
&
%मकर
\caldata{7}{\sunmonth{मीन}{25}{}}{\sundata{07:17}{20:28}{09:55}}{\textsf{\krishna~नवमी} {\tiny \RIGHTarrow} 03:15(+1)}{\textsf{उत्तराषाढा} {\tiny \RIGHTarrow} 22:58\hspace{2ex}}{13:52-15:31}{08:55-10:34} 
&
%मकर
\caldata{8}{\sunmonth{मीन}{26}{}}{\sundata{07:15}{20:30}{09:54}}{\textsf{\krishna~दशमी} {\tiny \RIGHTarrow} 05:52(+1)}{\textsf{श्रवण} {\tiny \RIGHTarrow} 02:05(+1)}{15:31-17:11}{07:15-08:54} 
&
%मकर
\caldata{9}{\sunmonth{मीन}{27}{}}{\sundata{07:13}{20:31}{09:52}}{\textsf{\krishna~एकादशी} {\tiny \RIGHTarrow} \textsf{अहोरात्रम्}}{\textsf{श्रविष्ठा} {\tiny \RIGHTarrow} 05:08(+1)}{12:12-13:52}{17:11-18:51} 
&
%मकर
\caldata{10}{\sunmonth{मीन}{28}{}}{\sundata{07:11}{20:33}{09:51}}{\textsf{\krishna~एकादशी} {\tiny \RIGHTarrow} 08:23\hspace{2ex}}{\textsf{शतभिषक्} {\tiny \RIGHTarrow} \textsf{अहोरात्रम्}}{10:31-12:11}{15:32-17:12} 
\\ \hline
%मकर
\caldata{11}{\sunmonth{मीन}{29}{}}{\sundata{07:09}{20:34}{09:50}}{\textsf{\krishna~द्वादशी} {\tiny \RIGHTarrow} 10:35\hspace{2ex}}{\textsf{शतभिषक्} {\tiny \RIGHTarrow} 07:55\hspace{2ex}}{18:53-20:34}{13:51-15:32} 
&
%मकर
\caldata{12}{\sunmonth{मीन}{30}{}}{\sundata{07:07}{20:36}{09:48}}{\textsf{\krishna~त्रयोदशी} {\tiny \RIGHTarrow} 12:21\hspace{2ex}}{\textsf{प्रोष्ठपदा} {\tiny \RIGHTarrow} 10:16\hspace{2ex}}{08:48-10:29}{12:10-13:51} 
&
%मकर
\caldata{13}{\sunmonth{मीन}{31}{{\textsf{मीन} {\tiny \RIGHTarrow} 03:21(+1)}}}{\sundata{07:05}{20:37}{09:47}}{\textsf{\krishna~चतुर्दशी} {\tiny \RIGHTarrow} 13:38\hspace{2ex}}{\textsf{उत्तरप्रोष्ठपदा} {\tiny \RIGHTarrow} 12:09\hspace{2ex}}{17:14-18:55}{10:28-12:09} 
&
%मकर
\caldata{14}{\sunmonth{मेष}{1}{}}{\sundata{07:03}{20:39}{09:46}}{\textsf{\newmoon~अमावस्या} {\tiny \RIGHTarrow} 14:26\hspace{2ex}}{\textsf{रेवती} {\tiny \RIGHTarrow} 13:34\hspace{2ex}}{13:51-15:33}{08:45-10:27} 
&
%मकर
\caldata{15}{\sunmonth{मेष}{2}{}}{\sundata{07:01}{20:40}{09:44}}{\textsf{\shukla~प्रथमा} {\tiny \RIGHTarrow} 14:45\hspace{2ex}}{\textsf{अश्विनी} {\tiny \RIGHTarrow} 14:33\hspace{2ex}}{15:32-17:15}{07:01-08:43} 
&
%मकर
\caldata{16}{\sunmonth{मेष}{3}{}}{\sundata{06:59}{20:42}{09:43}}{\textsf{\shukla~द्वितीया} {\tiny \RIGHTarrow} 14:40\hspace{2ex}}{\textsf{अपभरणी} {\tiny \RIGHTarrow} 15:06\hspace{2ex}}{12:07-13:50}{17:16-18:59} 
&
%मकर
\caldata{17}{\sunmonth{मेष}{4}{}}{\sundata{06:57}{20:43}{09:42}}{\textsf{\shukla~तृतीया} {\tiny \RIGHTarrow} 14:12\hspace{2ex}}{\textsf{कृत्तिका} {\tiny \RIGHTarrow} 15:19\hspace{2ex}}{10:23-12:06}{15:33-17:16} 
\\ \hline
%मकर
\caldata{18}{\sunmonth{मेष}{5}{}}{\sundata{06:55}{20:45}{09:41}}{\textsf{\shukla~चतुर्थी} {\tiny \RIGHTarrow} 13:24\hspace{2ex}}{\textsf{रोहिणी} {\tiny \RIGHTarrow} 15:12\hspace{2ex}}{19:01-20:45}{13:50-15:33} 
&
%मकर
\caldata{19}{\sunmonth{मेष}{6}{}}{\sundata{06:53}{20:46}{09:39}}{\textsf{\shukla~पञ्चमी} {\tiny \RIGHTarrow} 12:19\hspace{2ex}}{\textsf{मृगशीर्ष} {\tiny \RIGHTarrow} 14:47\hspace{2ex}}{08:37-10:21}{12:05-13:49} 
&
%मकर
\caldata{20}{\sunmonth{मेष}{7}{}}{\sundata{06:51}{20:48}{09:38}}{\textsf{\shukla~षष्ठी} {\tiny \RIGHTarrow} 10:55\hspace{2ex}}{\textsf{आर्द्रा} {\tiny \RIGHTarrow} 14:05\hspace{2ex}}{17:18-19:03}{10:20-12:04} 
&
%मकर
\caldata{21}{\sunmonth{मेष}{8}{}}{\sundata{06:50}{20:49}{09:37}}{\textsf{\shukla~सप्तमी} {\tiny \RIGHTarrow} 09:15\hspace{2ex}}{\textsf{पुनर्वसू} {\tiny \RIGHTarrow} 13:06\hspace{2ex}}{13:49-15:34}{08:34-10:19} 
&
%मकर
\caldata{22}{\sunmonth{मेष}{9}{}}{\sundata{06:48}{20:51}{09:36}}{\textsf{\shukla~अष्टमी} {\tiny \RIGHTarrow} 07:18\hspace{2ex}}{\textsf{पुष्य} {\tiny \RIGHTarrow} 11:50\hspace{2ex}}{15:34-17:20}{06:48-08:33} 
&
%मकर
\caldata{23}{\sunmonth{मेष}{10}{}}{\sundata{06:46}{20:52}{09:35}}{\textsf{\shukla~दशमी} {\tiny \RIGHTarrow} 02:38(+1)}{\textsf{आश्रेषा} {\tiny \RIGHTarrow} 10:20\hspace{2ex}}{12:03-13:49}{17:20-19:06} 
&
%मकर
\caldata{24}{\sunmonth{मेष}{11}{}}{\sundata{06:44}{20:54}{09:34}}{\textsf{\shukla~एकादशी} {\tiny \RIGHTarrow} 00:03(+1)}{\textsf{मघा} {\tiny \RIGHTarrow} 08:38\hspace{2ex}}{10:16-12:02}{15:35-17:21} 
\\ \hline
%मकर
\caldata{25}{\sunmonth{मेष}{12}{}}{\sundata{06:42}{20:55}{09:32}}{\textsf{\shukla~द्वादशी} {\tiny \RIGHTarrow} 21:25\hspace{2ex}}{\textsf{पूर्वफल्गुनी} {\tiny \RIGHTarrow} 06:47\hspace{2ex}}{19:08-20:55}{13:48-15:35} 
&
%मकर
\caldata{26}{\sunmonth{मेष}{13}{}}{\sundata{06:40}{20:57}{09:31}}{\textsf{\shukla~त्रयोदशी} {\tiny \RIGHTarrow} 18:50\hspace{2ex}}{\textsf{हस्त} {\tiny \RIGHTarrow} 03:09(+1)}{08:27-10:14}{12:01-13:48} 
&
%मकर
\caldata{27}{\sunmonth{मेष}{14}{}}{\sundata{06:39}{20:58}{09:30}}{\textsf{\shukla~चतुर्दशी} {\tiny \RIGHTarrow} 16:26\hspace{2ex}}{\textsf{चित्रा} {\tiny \RIGHTarrow} 01:38(+1)}{17:23-19:10}{10:13-12:01} 
&
%मकर
\caldata{28}{\sunmonth{मेष}{15}{}}{\sundata{06:37}{20:59}{09:29}}{\textsf{\fullmoon~पूर्णिमा} {\tiny \RIGHTarrow} 14:22\hspace{2ex}}{\textsf{स्वाति} {\tiny \RIGHTarrow} 00:30(+1)}{13:48-15:35}{08:24-10:12} 
&
%मकर
\caldata{29}{\sunmonth{मेष}{16}{}}{\sundata{06:35}{21:01}{09:28}}{\textsf{\krishna~प्रथमा} {\tiny \RIGHTarrow} 12:45\hspace{2ex}}{\textsf{विशाखा} {\tiny \RIGHTarrow} 23:54\hspace{2ex}}{15:36-17:24}{06:35-08:23} 
&
%मकर
\caldata{30}{\sunmonth{मेष}{17}{}}{\sundata{06:33}{21:02}{09:26}}{\textsf{\krishna~द्वितीया} {\tiny \RIGHTarrow} 11:44\hspace{2ex}}{\textsf{अनूराधा} {\tiny \RIGHTarrow} 23:56\hspace{2ex}}{11:58-13:47}{17:24-19:13} 
&
%मकर
\\ \hline
\end{tabular}


%\clearpage
\begin{tabular}{|c|c|c|c|c|c|c|}
\multicolumn{7}{c}{\Large \bfseries MAY 2010}\\[3mm]
\hline
\textbf{SUN} & \textbf{MON} & \textbf{TUE} & \textbf{WED} & \textbf{THU} & \textbf{FRI} & \textbf{SAT} \\ \hline
{}  &
{}  &
{}  &
{}  &
{}  &
{}  &
\caldata{1}{\sunmonth{मेष}{18}{}}{\sundata{06:32}{21:04}{09:26}}{\textsf{\krishna~तृतीया} {\tiny \RIGHTarrow} 11:25\hspace{2ex}}{\textsf{ज्येष्ठा} {\tiny \RIGHTarrow} 00:41(+1)}{10:10-11:59}{15:37-17:26} 
\\ \hline
%मकर
\caldata{2}{\sunmonth{मेष}{19}{}}{\sundata{06:30}{21:05}{09:25}}{\textsf{\krishna~चतुर्थी} {\tiny \RIGHTarrow} 11:50\hspace{2ex}}{\textsf{मूला} {\tiny \RIGHTarrow} 02:08(+1)}{19:15-21:05}{13:47-15:36} 
&
%मकर
\caldata{3}{\sunmonth{मेष}{20}{}}{\sundata{06:28}{21:07}{09:23}}{\textsf{\krishna~पञ्चमी} {\tiny \RIGHTarrow} 13:00\hspace{2ex}}{\textsf{पूर्वाषाढा} {\tiny \RIGHTarrow} 04:15(+1)}{08:17-10:07}{11:57-13:47} 
&
%मकर
\caldata{4}{\sunmonth{मेष}{21}{}}{\sundata{06:27}{21:08}{09:23}}{\textsf{\krishna~षष्ठी} {\tiny \RIGHTarrow} 14:47\hspace{2ex}}{\textsf{उत्तराषाढा} {\tiny \RIGHTarrow} \textsf{अहोरात्रम्}}{17:27-19:17}{10:07-11:57} 
&
%मकर
\caldata{5}{\sunmonth{मेष}{22}{}}{\sundata{06:25}{21:10}{09:22}}{\textsf{\krishna~सप्तमी} {\tiny \RIGHTarrow} 17:02\hspace{2ex}}{\textsf{उत्तराषाढा} {\tiny \RIGHTarrow} 06:54\hspace{2ex}}{13:47-15:38}{08:15-10:06} 
&
%मकर
\caldata{6}{\sunmonth{मेष}{23}{}}{\sundata{06:23}{21:11}{09:20}}{\textsf{\krishna~अष्टमी} {\tiny \RIGHTarrow} 19:30\hspace{2ex}}{\textsf{श्रवण} {\tiny \RIGHTarrow} 09:52\hspace{2ex}}{15:38-17:29}{06:23-08:14} 
&
%मकर
\caldata{7}{\sunmonth{मेष}{24}{}}{\sundata{06:22}{21:12}{09:20}}{\textsf{\krishna~नवमी} {\tiny \RIGHTarrow} 21:56\hspace{2ex}}{\textsf{श्रविष्ठा} {\tiny \RIGHTarrow} 12:53\hspace{2ex}}{11:55-13:47}{17:29-19:20} 
&
%मकर
\caldata{8}{\sunmonth{मेष}{25}{}}{\sundata{06:20}{21:14}{09:18}}{\textsf{\krishna~दशमी} {\tiny \RIGHTarrow} 00:08(+1)}{\textsf{शतभिषक्} {\tiny \RIGHTarrow} 15:44\hspace{2ex}}{10:03-11:55}{15:38-17:30} 
\\ \hline
%मकर
\caldata{9}{\sunmonth{मेष}{26}{}}{\sundata{06:19}{21:15}{09:18}}{\textsf{\krishna~एकादशी} {\tiny \RIGHTarrow} 01:52(+1)}{\textsf{प्रोष्ठपदा} {\tiny \RIGHTarrow} 18:11\hspace{2ex}}{19:23-21:15}{13:47-15:39} 
&
%मकर
\caldata{10}{\sunmonth{मेष}{27}{}}{\sundata{06:17}{21:17}{09:17}}{\textsf{\krishna~द्वादशी} {\tiny \RIGHTarrow} 03:03(+1)}{\textsf{उत्तरप्रोष्ठपदा} {\tiny \RIGHTarrow} 20:08\hspace{2ex}}{08:09-10:02}{11:54-13:47} 
&
%मकर
\caldata{11}{\sunmonth{मेष}{28}{}}{\sundata{06:16}{21:18}{09:16}}{\textsf{\krishna~त्रयोदशी} {\tiny \RIGHTarrow} 03:37(+1)}{\textsf{रेवती} {\tiny \RIGHTarrow} 21:31\hspace{2ex}}{17:32-19:25}{10:01-11:54} 
&
%मकर
\caldata{12}{\sunmonth{मेष}{29}{}}{\sundata{06:14}{21:19}{09:15}}{\textsf{\krishna~चतुर्दशी} {\tiny \RIGHTarrow} 03:36(+1)}{\textsf{अश्विनी} {\tiny \RIGHTarrow} 22:18\hspace{2ex}}{13:46-15:39}{08:07-10:00} 
&
%मकर
\caldata{13}{\sunmonth{मेष}{30}{}}{\sundata{06:13}{21:21}{09:14}}{\textsf{\newmoon~अमावस्या} {\tiny \RIGHTarrow} 03:02(+1)}{\textsf{अपभरणी} {\tiny \RIGHTarrow} 22:33\hspace{2ex}}{15:40-17:34}{06:13-08:06} 
&
%मकर
\caldata{14}{\sunmonth{मेष}{31}{{\textsf{मेष} {\tiny \RIGHTarrow} 00:10(+1)}}}{\sundata{06:12}{21:22}{09:14}}{\textsf{\shukla~प्रथमा} {\tiny \RIGHTarrow} 02:01(+1)}{\textsf{कृत्तिका} {\tiny \RIGHTarrow} 22:20\hspace{2ex}}{11:53-13:47}{17:34-19:28} 
&
%मकर
\caldata{15}{\sunmonth{वृषभ}{1}{}}{\sundata{06:10}{21:24}{09:12}}{\textsf{\shukla~द्वितीया} {\tiny \RIGHTarrow} 00:38(+1)}{\textsf{रोहिणी} {\tiny \RIGHTarrow} 21:45\hspace{2ex}}{09:58-11:52}{15:41-17:35} 
\\ \hline
%मकर
\caldata{16}{\sunmonth{वृषभ}{2}{}}{\sundata{06:09}{21:25}{09:12}}{\textsf{\shukla~तृतीया} {\tiny \RIGHTarrow} 22:58\hspace{2ex}}{\textsf{मृगशीर्ष} {\tiny \RIGHTarrow} 20:53\hspace{2ex}}{19:30-21:25}{13:47-15:41} 
&
%मकर
\caldata{17}{\sunmonth{वृषभ}{3}{}}{\sundata{06:08}{21:26}{09:11}}{\textsf{\shukla~चतुर्थी} {\tiny \RIGHTarrow} 21:06\hspace{2ex}}{\textsf{आर्द्रा} {\tiny \RIGHTarrow} 19:47\hspace{2ex}}{08:02-09:57}{11:52-13:47} 
&
%मकर
\caldata{18}{\sunmonth{वृषभ}{4}{}}{\sundata{06:06}{21:27}{09:10}}{\textsf{\shukla~पञ्चमी} {\tiny \RIGHTarrow} 19:05\hspace{2ex}}{\textsf{पुनर्वसू} {\tiny \RIGHTarrow} 18:33\hspace{2ex}}{17:36-19:31}{09:56-11:51} 
&
%मकर
\caldata{19}{\sunmonth{वृषभ}{5}{}}{\sundata{06:05}{21:29}{09:09}}{\textsf{\shukla~षष्ठी} {\tiny \RIGHTarrow} 16:59\hspace{2ex}}{\textsf{पुष्य} {\tiny \RIGHTarrow} 17:12\hspace{2ex}}{13:47-15:42}{08:00-09:56} 
&
%मकर
\caldata{20}{\sunmonth{वृषभ}{6}{}}{\sundata{06:04}{21:30}{09:09}}{\textsf{\shukla~सप्तमी} {\tiny \RIGHTarrow} 14:48\hspace{2ex}}{\textsf{आश्रेषा} {\tiny \RIGHTarrow} 15:48\hspace{2ex}}{15:42-17:38}{06:04-07:59} 
&
%मकर
\caldata{21}{\sunmonth{वृषभ}{7}{}}{\sundata{06:03}{21:31}{09:08}}{\textsf{\shukla~अष्टमी} {\tiny \RIGHTarrow} 12:36\hspace{2ex}}{\textsf{मघा} {\tiny \RIGHTarrow} 14:23\hspace{2ex}}{11:51-13:47}{17:39-19:35} 
&
%मकर
\caldata{22}{\sunmonth{वृषभ}{8}{}}{\sundata{06:02}{21:32}{09:08}}{\textsf{\shukla~नवमी} {\tiny \RIGHTarrow} 10:25\hspace{2ex}}{\textsf{पूर्वफल्गुनी} {\tiny \RIGHTarrow} 12:59\hspace{2ex}}{09:54-11:50}{15:43-17:39} 
\\ \hline
%मकर
\caldata{23}{\sunmonth{वृषभ}{9}{}}{\sundata{06:01}{21:34}{09:07}}{\textsf{\shukla~दशमी} {\tiny \RIGHTarrow} 08:17\hspace{2ex}}{\textsf{उत्तरफल्गुनी} {\tiny \RIGHTarrow} 11:39\hspace{2ex}}{19:37-21:34}{13:47-15:44} 
&
%मकर
\caldata{24}{\sunmonth{वृषभ}{10}{}}{\sundata{06:00}{21:35}{09:07}}{\textsf{\shukla~एकादशी} {\tiny \RIGHTarrow} 06:16\hspace{2ex}}{\textsf{हस्त} {\tiny \RIGHTarrow} 10:27\hspace{2ex}}{07:56-09:53}{11:50-13:47} 
&
%मकर
\caldata{25}{\sunmonth{वृषभ}{11}{}}{\sundata{05:59}{21:36}{09:06}}{\textsf{\shukla~त्रयोदशी} {\tiny \RIGHTarrow} 02:58(+1)}{\textsf{चित्रा} {\tiny \RIGHTarrow} 09:27\hspace{2ex}}{17:41-19:38}{09:53-11:50} 
&
%मकर
\caldata{26}{\sunmonth{वृषभ}{12}{}}{\sundata{05:58}{21:37}{09:05}}{\textsf{\shukla~चतुर्दशी} {\tiny \RIGHTarrow} 01:50(+1)}{\textsf{स्वाति} {\tiny \RIGHTarrow} 08:46\hspace{2ex}}{13:47-15:44}{07:55-09:52} 
&
%मकर
\caldata{27}{\sunmonth{वृषभ}{13}{}}{\sundata{05:57}{21:38}{09:05}}{\textsf{\fullmoon~पूर्णिमा} {\tiny \RIGHTarrow} 01:09(+1)}{\textsf{विशाखा} {\tiny \RIGHTarrow} 08:29\hspace{2ex}}{15:45-17:42}{05:57-07:54} 
&
%मकर
\caldata{28}{\sunmonth{वृषभ}{14}{}}{\sundata{05:56}{21:39}{09:04}}{\textsf{\krishna~प्रथमा} {\tiny \RIGHTarrow} 01:02(+1)}{\textsf{अनूराधा} {\tiny \RIGHTarrow} 08:41\hspace{2ex}}{11:49-13:47}{17:43-19:41} 
&
%मकर
\caldata{29}{\sunmonth{वृषभ}{15}{}}{\sundata{05:55}{21:40}{09:04}}{\textsf{\krishna~द्वितीया} {\tiny \RIGHTarrow} 01:29(+1)}{\textsf{ज्येष्ठा} {\tiny \RIGHTarrow} 09:26\hspace{2ex}}{09:51-11:49}{15:45-17:43} 
\\ \hline
%मकर
\caldata{30}{\sunmonth{वृषभ}{16}{}}{\sundata{05:54}{21:41}{09:03}}{\textsf{\krishna~तृतीया} {\tiny \RIGHTarrow} 02:33(+1)}{\textsf{मूला} {\tiny \RIGHTarrow} 10:47\hspace{2ex}}{19:42-21:41}{13:47-15:45} 
&
%मकर
\caldata{31}{\sunmonth{वृषभ}{17}{}}{\sundata{05:54}{21:42}{09:03}}{\textsf{\krishna~चतुर्थी} {\tiny \RIGHTarrow} 04:10(+1)}{\textsf{पूर्वाषाढा} {\tiny \RIGHTarrow} 12:43\hspace{2ex}}{07:52-09:51}{11:49-13:48} 
&
%मकर
{}  &
{}  &
{}  &
{}  &
\\ \hline
\end{tabular}


%\clearpage
\begin{tabular}{|c|c|c|c|c|c|c|}
\multicolumn{7}{c}{\Large \bfseries JUNE 2010}\\[3mm]
\hline
\textbf{SUN} & \textbf{MON} & \textbf{TUE} & \textbf{WED} & \textbf{THU} & \textbf{FRI} & \textbf{SAT} \\ \hline
{}  &
{}  &
\caldata{1}{\sunmonth{वृषभ}{18}{}}{\sundata{05:53}{21:43}{09:03}}{\textsf{\krishna~पञ्चमी} {\tiny \RIGHTarrow} \textsf{अहोरात्रम्}}{\textsf{उत्तराषाढा} {\tiny \RIGHTarrow} 15:08\hspace{2ex}}{17:45-19:44}{09:50-11:49} 
&
%मकर
\caldata{2}{\sunmonth{वृषभ}{19}{}}{\sundata{05:52}{21:44}{09:02}}{\textsf{\krishna~पञ्चमी} {\tiny \RIGHTarrow} 06:14\hspace{2ex}}{\textsf{श्रवण} {\tiny \RIGHTarrow} 17:56\hspace{2ex}}{13:48-15:47}{07:51-09:50} 
&
%मकर
\caldata{3}{\sunmonth{वृषभ}{20}{}}{\sundata{05:52}{21:45}{09:02}}{\textsf{\krishna~षष्ठी} {\tiny \RIGHTarrow} 08:36\hspace{2ex}}{\textsf{श्रविष्ठा} {\tiny \RIGHTarrow} 20:54\hspace{2ex}}{15:47-17:46}{05:52-07:51} 
&
%मकर
\caldata{4}{\sunmonth{वृषभ}{21}{}}{\sundata{05:51}{21:46}{09:02}}{\textsf{\krishna~सप्तमी} {\tiny \RIGHTarrow} 11:02\hspace{2ex}}{\textsf{शतभिषक्} {\tiny \RIGHTarrow} 23:50\hspace{2ex}}{11:49-13:48}{17:47-19:46} 
&
%मकर
\caldata{5}{\sunmonth{वृषभ}{22}{}}{\sundata{05:51}{21:47}{09:02}}{\textsf{\krishna~अष्टमी} {\tiny \RIGHTarrow} 13:16\hspace{2ex}}{\textsf{प्रोष्ठपदा} {\tiny \RIGHTarrow} 02:31(+1)}{09:50-11:49}{15:48-17:48} 
\\ \hline
%मकर
\caldata{6}{\sunmonth{वृषभ}{23}{}}{\sundata{05:50}{21:48}{09:01}}{\textsf{\krishna~नवमी} {\tiny \RIGHTarrow} 15:07\hspace{2ex}}{\textsf{उत्तरप्रोष्ठपदा} {\tiny \RIGHTarrow} 04:45(+1)}{19:48-21:48}{13:49-15:48} 
&
%मकर
\caldata{7}{\sunmonth{वृषभ}{24}{}}{\sundata{05:50}{21:49}{09:01}}{\textsf{\krishna~दशमी} {\tiny \RIGHTarrow} 16:26\hspace{2ex}}{\textsf{रेवती} {\tiny \RIGHTarrow} \textsf{अहोरात्रम्}}{07:49-09:49}{11:49-13:49} 
&
%मकर
\caldata{8}{\sunmonth{वृषभ}{25}{}}{\sundata{05:49}{21:49}{09:01}}{\textsf{\krishna~एकादशी} {\tiny \RIGHTarrow} 17:04\hspace{2ex}}{\textsf{रेवती} {\tiny \RIGHTarrow} 06:22\hspace{2ex}}{17:49-19:49}{09:49-11:49} 
&
%मकर
\caldata{9}{\sunmonth{वृषभ}{26}{}}{\sundata{05:49}{21:50}{09:01}}{\textsf{\krishna~द्वादशी} {\tiny \RIGHTarrow} 17:02\hspace{2ex}}{\textsf{अश्विनी} {\tiny \RIGHTarrow} 07:17\hspace{2ex}}{13:49-15:49}{07:49-09:49} 
&
%मकर
\caldata{10}{\sunmonth{वृषभ}{27}{}}{\sundata{05:49}{21:51}{09:01}}{\textsf{\krishna~त्रयोदशी} {\tiny \RIGHTarrow} 16:19\hspace{2ex}}{\textsf{अपभरणी} {\tiny \RIGHTarrow} 07:33\hspace{2ex}}{15:50-17:50}{05:49-07:49} 
&
%मकर
\caldata{11}{\sunmonth{वृषभ}{28}{}}{\sundata{05:48}{21:51}{09:00}}{\textsf{\krishna~चतुर्दशी} {\tiny \RIGHTarrow} 15:00\hspace{2ex}}{\textsf{कृत्तिका} {\tiny \RIGHTarrow} 07:11\hspace{2ex}}{11:49-13:49}{17:50-19:50} 
&
%मकर
\caldata{12}{\sunmonth{वृषभ}{29}{}}{\sundata{05:48}{21:52}{09:00}}{\textsf{\newmoon~अमावस्या} {\tiny \RIGHTarrow} 13:11\hspace{2ex}}{\textsf{रोहिणी} {\tiny \RIGHTarrow} 06:18\hspace{2ex}}{09:49-11:49}{15:50-17:51} 
\\ \hline
%मकर
\caldata{13}{\sunmonth{वृषभ}{30}{}}{\sundata{05:48}{21:53}{09:01}}{\textsf{\shukla~प्रथमा} {\tiny \RIGHTarrow} 10:59\hspace{2ex}}{\textsf{आर्द्रा} {\tiny \RIGHTarrow} 03:18(+1)}{19:52-21:53}{13:50-15:51} 
&
%मकर
\caldata{14}{\sunmonth{वृषभ}{31}{}}{\sundata{05:48}{21:53}{09:01}}{\textsf{\shukla~द्वितीया} {\tiny \RIGHTarrow} 08:30\hspace{2ex}}{\textsf{पुनर्वसू} {\tiny \RIGHTarrow} 01:28(+1)}{07:48-09:49}{11:49-13:50} 
&
%मकर
\caldata{15}{\sunmonth{मिथुन}{1}{{\textsf{वृषभ} {\tiny \RIGHTarrow} 06:43\hspace{2ex}}}}{\sundata{05:48}{21:54}{09:01}}{\textsf{\shukla~तृतीया} {\tiny \RIGHTarrow} 05:50\hspace{2ex}}{\textsf{पुष्य} {\tiny \RIGHTarrow} 23:34\hspace{2ex}}{17:52-19:53}{09:49-11:50} 
&
%मकर
\caldata{16}{\sunmonth{मिथुन}{2}{}}{\sundata{05:48}{21:54}{09:01}}{\textsf{\shukla~पञ्चमी} {\tiny \RIGHTarrow} 00:28(+1)}{\textsf{आश्रेषा} {\tiny \RIGHTarrow} 21:42\hspace{2ex}}{13:51-15:51}{07:48-09:49} 
&
%मकर
\caldata{17}{\sunmonth{मिथुन}{3}{}}{\sundata{05:48}{21:55}{09:01}}{\textsf{\shukla~षष्ठी} {\tiny \RIGHTarrow} 21:56\hspace{2ex}}{\textsf{मघा} {\tiny \RIGHTarrow} 19:57\hspace{2ex}}{15:52-17:53}{05:48-07:48} 
&
%मकर
\caldata{18}{\sunmonth{मिथुन}{4}{}}{\sundata{05:48}{21:55}{09:01}}{\textsf{\shukla~सप्तमी} {\tiny \RIGHTarrow} 19:36\hspace{2ex}}{\textsf{पूर्वफल्गुनी} {\tiny \RIGHTarrow} 18:24\hspace{2ex}}{11:50-13:51}{17:53-19:54} 
&
%मकर
\caldata{19}{\sunmonth{मिथुन}{5}{}}{\sundata{05:48}{21:55}{09:01}}{\textsf{\shukla~अष्टमी} {\tiny \RIGHTarrow} 17:31\hspace{2ex}}{\textsf{उत्तरफल्गुनी} {\tiny \RIGHTarrow} 17:06\hspace{2ex}}{09:49-11:50}{15:52-17:53} 
\\ \hline
%मकर
\caldata{20}{\sunmonth{मिथुन}{6}{}}{\sundata{05:48}{21:55}{09:01}}{\textsf{\shukla~नवमी} {\tiny \RIGHTarrow} 15:45\hspace{2ex}}{\textsf{हस्त} {\tiny \RIGHTarrow} 16:06\hspace{2ex}}{19:54-21:55}{13:51-15:52} 
&
%मकर
\caldata{21}{\sunmonth{मिथुन}{7}{}}{\sundata{05:48}{21:56}{09:01}}{\textsf{\shukla~दशमी} {\tiny \RIGHTarrow} 14:19\hspace{2ex}}{\textsf{चित्रा} {\tiny \RIGHTarrow} 15:27\hspace{2ex}}{07:49-09:50}{11:51-13:52} 
&
%मकर
\caldata{22}{\sunmonth{मिथुन}{8}{}}{\sundata{05:48}{21:56}{09:01}}{\textsf{\shukla~एकादशी} {\tiny \RIGHTarrow} 13:17\hspace{2ex}}{\textsf{स्वाति} {\tiny \RIGHTarrow} 15:11\hspace{2ex}}{17:54-19:55}{09:50-11:51} 
&
%मकर
\caldata{23}{\sunmonth{मिथुन}{9}{}}{\sundata{05:49}{21:56}{09:02}}{\textsf{\shukla~द्वादशी} {\tiny \RIGHTarrow} 12:39\hspace{2ex}}{\textsf{विशाखा} {\tiny \RIGHTarrow} 15:20\hspace{2ex}}{13:52-15:53}{07:49-09:50} 
&
%मकर
\caldata{24}{\sunmonth{मिथुन}{10}{}}{\sundata{05:49}{21:56}{09:02}}{\textsf{\shukla~त्रयोदशी} {\tiny \RIGHTarrow} 12:29\hspace{2ex}}{\textsf{अनूराधा} {\tiny \RIGHTarrow} 15:56\hspace{2ex}}{15:53-17:54}{05:49-07:49} 
&
%मकर
\caldata{25}{\sunmonth{मिथुन}{11}{}}{\sundata{05:49}{21:56}{09:02}}{\textsf{\shukla~चतुर्दशी} {\tiny \RIGHTarrow} 12:46\hspace{2ex}}{\textsf{ज्येष्ठा} {\tiny \RIGHTarrow} 16:59\hspace{2ex}}{11:51-13:52}{17:54-19:55} 
&
%मकर
\caldata{26}{\sunmonth{मिथुन}{12}{}}{\sundata{05:50}{21:56}{09:03}}{\textsf{\fullmoon~पूर्णिमा} {\tiny \RIGHTarrow} 13:33\hspace{2ex}}{\textsf{मूला} {\tiny \RIGHTarrow} 18:30\hspace{2ex}}{09:51-11:52}{15:53-17:54} 
\\ \hline
%मकर
\caldata{27}{\sunmonth{मिथुन}{13}{}}{\sundata{05:50}{21:56}{09:03}}{\textsf{\krishna~प्रथमा} {\tiny \RIGHTarrow} 14:47\hspace{2ex}}{\textsf{पूर्वाषाढा} {\tiny \RIGHTarrow} 20:28\hspace{2ex}}{19:55-21:56}{13:53-15:53} 
&
%मकर
\caldata{28}{\sunmonth{मिथुन}{14}{}}{\sundata{05:51}{21:56}{09:04}}{\textsf{\krishna~द्वितीया} {\tiny \RIGHTarrow} 16:28\hspace{2ex}}{\textsf{उत्तराषाढा} {\tiny \RIGHTarrow} 22:51\hspace{2ex}}{07:51-09:52}{11:52-13:53} 
&
%मकर
\caldata{29}{\sunmonth{मिथुन}{15}{}}{\sundata{05:51}{21:56}{09:04}}{\textsf{\krishna~तृतीया} {\tiny \RIGHTarrow} 18:31\hspace{2ex}}{\textsf{श्रवण} {\tiny \RIGHTarrow} 01:34(+1)}{17:54-19:55}{09:52-11:52} 
&
%मकर
\caldata{30}{\sunmonth{मिथुन}{16}{}}{\sundata{05:52}{21:56}{09:04}}{\textsf{\krishna~चतुर्थी} {\tiny \RIGHTarrow} 20:50\hspace{2ex}}{\textsf{श्रविष्ठा} {\tiny \RIGHTarrow} 04:31(+1)}{13:54-15:54}{07:52-09:53} 
&
%मकर
{}  &
{}  &
\\ \hline
\end{tabular}


%\clearpage
\begin{tabular}{|c|c|c|c|c|c|c|}
\multicolumn{7}{c}{\Large \bfseries JULY 2010}\\[3mm]
\hline
\textbf{SUN} & \textbf{MON} & \textbf{TUE} & \textbf{WED} & \textbf{THU} & \textbf{FRI} & \textbf{SAT} \\ \hline
{}  &
{}  &
{}  &
{}  &
\caldata{1}{\sunmonth{मिथुन}{17}{}}{\sundata{05:52}{21:55}{09:04}}{\textsf{\krishna~पञ्चमी} {\tiny \RIGHTarrow} 23:15\hspace{2ex}}{\textsf{शतभिषक्} {\tiny \RIGHTarrow} \textsf{अहोरात्रम्}}{15:53-17:54}{05:52-07:52} 
&
%मकर
\caldata{2}{\sunmonth{मिथुन}{18}{}}{\sundata{05:53}{21:55}{09:05}}{\textsf{\krishna~षष्ठी} {\tiny \RIGHTarrow} 01:36(+1)}{\textsf{शतभिषक्} {\tiny \RIGHTarrow} 07:31\hspace{2ex}}{11:53-13:54}{17:54-19:54} 
&
%मकर
\caldata{3}{\sunmonth{मिथुन}{19}{}}{\sundata{05:54}{21:55}{09:06}}{\textsf{\krishna~सप्तमी} {\tiny \RIGHTarrow} 03:41(+1)}{\textsf{प्रोष्ठपदा} {\tiny \RIGHTarrow} 10:22\hspace{2ex}}{09:54-11:54}{15:54-17:54} 
\\ \hline
%मकर
\caldata{4}{\sunmonth{मिथुन}{20}{}}{\sundata{05:54}{21:54}{09:06}}{\textsf{\krishna~अष्टमी} {\tiny \RIGHTarrow} 05:19(+1)}{\textsf{उत्तरप्रोष्ठपदा} {\tiny \RIGHTarrow} 12:53\hspace{2ex}}{19:54-21:54}{13:54-15:54} 
&
%मकर
\caldata{5}{\sunmonth{मिथुन}{21}{}}{\sundata{05:55}{21:54}{09:06}}{\textsf{\krishna~नवमी} {\tiny \RIGHTarrow} \textsf{अहोरात्रम्}}{\textsf{रेवती} {\tiny \RIGHTarrow} 14:54\hspace{2ex}}{07:54-09:54}{11:54-13:54} 
&
%मकर
\caldata{6}{\sunmonth{मिथुन}{22}{}}{\sundata{05:56}{21:54}{09:07}}{\textsf{\krishna~नवमी} {\tiny \RIGHTarrow} 06:19\hspace{2ex}}{\textsf{अश्विनी} {\tiny \RIGHTarrow} 16:17\hspace{2ex}}{17:54-19:54}{09:55-11:55} 
&
%मकर
\caldata{7}{\sunmonth{मिथुन}{23}{}}{\sundata{05:57}{21:53}{09:08}}{\textsf{\krishna~दशमी} {\tiny \RIGHTarrow} 06:35\hspace{2ex}}{\textsf{अपभरणी} {\tiny \RIGHTarrow} 16:56\hspace{2ex}}{13:55-15:54}{07:56-09:56} 
&
%मकर
\caldata{8}{\sunmonth{मिथुन}{24}{}}{\sundata{05:57}{21:53}{09:08}}{\textsf{\krishna~एकादशी} {\tiny \RIGHTarrow} 06:07\hspace{2ex}}{\textsf{कृत्तिका} {\tiny \RIGHTarrow} 16:51\hspace{2ex}}{15:54-17:54}{05:57-07:56} 
&
%मकर
\caldata{9}{\sunmonth{मिथुन}{25}{}}{\sundata{05:58}{21:52}{09:08}}{\textsf{\krishna~त्रयोदशी} {\tiny \RIGHTarrow} 02:58(+1)}{\textsf{रोहिणी} {\tiny \RIGHTarrow} 16:04\hspace{2ex}}{11:55-13:55}{17:53-19:52} 
&
%मकर
\caldata{10}{\sunmonth{मिथुन}{26}{}}{\sundata{05:59}{21:51}{09:09}}{\textsf{\krishna~चतुर्दशी} {\tiny \RIGHTarrow} 00:30(+1)}{\textsf{मृगशीर्ष} {\tiny \RIGHTarrow} 14:41\hspace{2ex}}{09:57-11:56}{15:54-17:53} 
\\ \hline
%मकर
\caldata{11}{\sunmonth{मिथुन}{27}{}}{\sundata{06:00}{21:51}{09:10}}{\textsf{\newmoon~अमावस्या} {\tiny \RIGHTarrow} 21:37\hspace{2ex}}{\textsf{आर्द्रा} {\tiny \RIGHTarrow} 12:48\hspace{2ex}}{19:52-21:51}{13:55-15:54} 
&
%मकर
\caldata{12}{\sunmonth{मिथुन}{28}{}}{\sundata{06:01}{21:50}{09:10}}{\textsf{\shukla~प्रथमा} {\tiny \RIGHTarrow} 18:27\hspace{2ex}}{\textsf{पुनर्वसू} {\tiny \RIGHTarrow} 10:33\hspace{2ex}}{07:59-09:58}{11:56-13:55} 
&
%मकर
\caldata{13}{\sunmonth{मिथुन}{29}{}}{\sundata{06:02}{21:49}{09:11}}{\textsf{\shukla~द्वितीया} {\tiny \RIGHTarrow} 15:08\hspace{2ex}}{\textsf{पुष्य} {\tiny \RIGHTarrow} 08:06\hspace{2ex}}{17:52-19:50}{09:58-11:57} 
&
%मकर
\caldata{14}{\sunmonth{मिथुन}{30}{}}{\sundata{06:03}{21:48}{09:12}}{\textsf{\shukla~तृतीया} {\tiny \RIGHTarrow} 11:49\hspace{2ex}}{\textsf{मघा} {\tiny \RIGHTarrow} 03:08(+1)}{13:55-15:53}{08:01-09:59} 
&
%मकर
\caldata{15}{\sunmonth{मिथुन}{31}{}}{\sundata{06:04}{21:47}{09:12}}{\textsf{\shukla~चतुर्थी} {\tiny \RIGHTarrow} 08:36\hspace{2ex}}{\textsf{पूर्वफल्गुनी} {\tiny \RIGHTarrow} 00:58(+1)}{15:53-17:51}{06:04-08:01} 
&
%मकर
\caldata{16}{\sunmonth{कर्कटक}{1}{{\textsf{मिथुन} {\tiny \RIGHTarrow} 17:33\hspace{2ex}}}}{\sundata{06:05}{21:47}{09:13}}{\textsf{\shukla~षष्ठी} {\tiny \RIGHTarrow} 03:07(+1)}{\textsf{उत्तरफल्गुनी} {\tiny \RIGHTarrow} 23:09\hspace{2ex}}{11:58-13:56}{17:51-19:49} 
&
%मकर
\caldata{17}{\sunmonth{कर्कटक}{2}{}}{\sundata{06:06}{21:46}{09:14}}{\textsf{\shukla~सप्तमी} {\tiny \RIGHTarrow} 01:04(+1)}{\textsf{हस्त} {\tiny \RIGHTarrow} 21:47\hspace{2ex}}{10:01-11:58}{15:53-17:51} 
\\ \hline
%मकर
\caldata{18}{\sunmonth{कर्कटक}{3}{}}{\sundata{06:07}{21:45}{09:14}}{\textsf{\shukla~अष्टमी} {\tiny \RIGHTarrow} 23:32\hspace{2ex}}{\textsf{चित्रा} {\tiny \RIGHTarrow} 20:56\hspace{2ex}}{19:47-21:45}{13:56-15:53} 
&
%मकर
\caldata{19}{\sunmonth{कर्कटक}{4}{}}{\sundata{06:08}{21:44}{09:15}}{\textsf{\shukla~नवमी} {\tiny \RIGHTarrow} 22:34\hspace{2ex}}{\textsf{स्वाति} {\tiny \RIGHTarrow} 20:39\hspace{2ex}}{08:05-10:02}{11:59-13:56} 
&
%मकर
\caldata{20}{\sunmonth{कर्कटक}{5}{}}{\sundata{06:10}{21:43}{09:16}}{\textsf{\shukla~दशमी} {\tiny \RIGHTarrow} 22:10\hspace{2ex}}{\textsf{विशाखा} {\tiny \RIGHTarrow} 20:57\hspace{2ex}}{17:49-19:46}{10:03-11:59} 
&
%मकर
\caldata{21}{\sunmonth{कर्कटक}{6}{}}{\sundata{06:11}{21:42}{09:17}}{\textsf{\shukla~एकादशी} {\tiny \RIGHTarrow} 22:20\hspace{2ex}}{\textsf{अनूराधा} {\tiny \RIGHTarrow} 21:47\hspace{2ex}}{13:56-15:52}{08:07-10:03} 
&
%मकर
\caldata{22}{\sunmonth{कर्कटक}{7}{}}{\sundata{06:12}{21:40}{09:17}}{\textsf{\shukla~द्वादशी} {\tiny \RIGHTarrow} 23:01\hspace{2ex}}{\textsf{ज्येष्ठा} {\tiny \RIGHTarrow} 23:08\hspace{2ex}}{15:52-17:48}{06:12-08:08} 
&
%मकर
\caldata{23}{\sunmonth{कर्कटक}{8}{}}{\sundata{06:13}{21:39}{09:18}}{\textsf{\shukla~त्रयोदशी} {\tiny \RIGHTarrow} 00:10(+1)}{\textsf{मूला} {\tiny \RIGHTarrow} 00:55(+1)}{12:00-13:56}{17:47-19:43} 
&
%मकर
\caldata{24}{\sunmonth{कर्कटक}{9}{}}{\sundata{06:14}{21:38}{09:18}}{\textsf{\shukla~चतुर्दशी} {\tiny \RIGHTarrow} 01:43(+1)}{\textsf{पूर्वाषाढा} {\tiny \RIGHTarrow} 03:07(+1)}{10:05-12:00}{15:51-17:47} 
\\ \hline
%मकर
\caldata{25}{\sunmonth{कर्कटक}{10}{}}{\sundata{06:16}{21:37}{09:20}}{\textsf{\fullmoon~पूर्णिमा} {\tiny \RIGHTarrow} 03:37(+1)}{\textsf{उत्तराषाढा} {\tiny \RIGHTarrow} 05:37(+1)}{19:41-21:37}{13:56-15:51} 
&
%मकर
\caldata{26}{\sunmonth{कर्कटक}{11}{}}{\sundata{06:17}{21:36}{09:20}}{\textsf{\krishna~प्रथमा} {\tiny \RIGHTarrow} 05:47(+1)}{\textsf{श्रवण} {\tiny \RIGHTarrow} \textsf{अहोरात्रम्}}{08:11-10:06}{12:01-13:56} 
&
%मकर
\caldata{27}{\sunmonth{कर्कटक}{12}{}}{\sundata{06:18}{21:34}{09:21}}{\textsf{\krishna~द्वितीया} {\tiny \RIGHTarrow} \textsf{अहोरात्रम्}}{\textsf{श्रवण} {\tiny \RIGHTarrow} 08:24\hspace{2ex}}{17:45-19:39}{10:07-12:01} 
&
%मकर
\caldata{28}{\sunmonth{कर्कटक}{13}{}}{\sundata{06:19}{21:33}{09:21}}{\textsf{\krishna~द्वितीया} {\tiny \RIGHTarrow} 08:08\hspace{2ex}}{\textsf{श्रविष्ठा} {\tiny \RIGHTarrow} 11:21\hspace{2ex}}{13:56-15:50}{08:13-10:07} 
&
%मकर
\caldata{29}{\sunmonth{कर्कटक}{14}{}}{\sundata{06:21}{21:32}{09:23}}{\textsf{\krishna~तृतीया} {\tiny \RIGHTarrow} 10:34\hspace{2ex}}{\textsf{शतभिषक्} {\tiny \RIGHTarrow} 14:21\hspace{2ex}}{15:50-17:44}{06:21-08:14} 
&
%मकर
\caldata{30}{\sunmonth{कर्कटक}{15}{}}{\sundata{06:22}{21:30}{09:23}}{\textsf{\krishna~चतुर्थी} {\tiny \RIGHTarrow} 12:58\hspace{2ex}}{\textsf{प्रोष्ठपदा} {\tiny \RIGHTarrow} 17:18\hspace{2ex}}{12:02-13:56}{17:43-19:36} 
&
%मकर
\caldata{31}{\sunmonth{कर्कटक}{16}{}}{\sundata{06:23}{21:29}{09:24}}{\textsf{\krishna~पञ्चमी} {\tiny \RIGHTarrow} 15:11\hspace{2ex}}{\textsf{उत्तरप्रोष्ठपदा} {\tiny \RIGHTarrow} 20:03\hspace{2ex}}{10:09-12:02}{15:49-17:42} 
\\ \hline
%मकर
\end{tabular}


%\clearpage
\begin{tabular}{|c|c|c|c|c|c|c|}
\multicolumn{7}{c}{\Large \bfseries AUGUST 2010}\\[3mm]
\hline
\textbf{SUN} & \textbf{MON} & \textbf{TUE} & \textbf{WED} & \textbf{THU} & \textbf{FRI} & \textbf{SAT} \\ \hline
\caldata{1}{\sunmonth{कर्कटक}{17}{}}{\sundata{06:25}{21:27}{09:25}}{\textsf{\krishna~षष्ठी} {\tiny \RIGHTarrow} 17:03\hspace{2ex}}{\textsf{रेवती} {\tiny \RIGHTarrow} 22:27\hspace{2ex}}{19:34-21:27}{13:56-15:48} 
&
%मकर
\caldata{2}{\sunmonth{कर्कटक}{18}{}}{\sundata{06:26}{21:26}{09:26}}{\textsf{\krishna~सप्तमी} {\tiny \RIGHTarrow} 18:26\hspace{2ex}}{\textsf{अश्विनी} {\tiny \RIGHTarrow} 00:20(+1)}{08:18-10:11}{12:03-13:56} 
&
%मकर
\caldata{3}{\sunmonth{कर्कटक}{19}{}}{\sundata{06:27}{21:24}{09:26}}{\textsf{\krishna~अष्टमी} {\tiny \RIGHTarrow} 19:11\hspace{2ex}}{\textsf{अपभरणी} {\tiny \RIGHTarrow} 01:36(+1)}{17:39-19:31}{10:11-12:03} 
&
%मकर
\caldata{4}{\sunmonth{कर्कटक}{20}{}}{\sundata{06:29}{21:23}{09:27}}{\textsf{\krishna~नवमी} {\tiny \RIGHTarrow} 19:12\hspace{2ex}}{\textsf{कृत्तिका} {\tiny \RIGHTarrow} 02:08(+1)}{13:56-15:47}{08:20-10:12} 
&
%मकर
\caldata{5}{\sunmonth{कर्कटक}{21}{}}{\sundata{06:30}{21:21}{09:28}}{\textsf{\krishna~दशमी} {\tiny \RIGHTarrow} 18:28\hspace{2ex}}{\textsf{रोहिणी} {\tiny \RIGHTarrow} 01:53(+1)}{15:46-17:38}{06:30-08:21} 
&
%मकर
\caldata{6}{\sunmonth{कर्कटक}{22}{}}{\sundata{06:31}{21:20}{09:28}}{\textsf{\krishna~एकादशी} {\tiny \RIGHTarrow} 16:58\hspace{2ex}}{\textsf{मृगशीर्ष} {\tiny \RIGHTarrow} 00:54(+1)}{12:04-13:55}{17:37-19:28} 
&
%मकर
\caldata{7}{\sunmonth{कर्कटक}{23}{}}{\sundata{06:33}{21:18}{09:30}}{\textsf{\krishna~द्वादशी} {\tiny \RIGHTarrow} 14:47\hspace{2ex}}{\textsf{आर्द्रा} {\tiny \RIGHTarrow} 23:14\hspace{2ex}}{10:14-12:04}{15:46-17:36} 
\\ \hline
%मकर
\caldata{8}{\sunmonth{कर्कटक}{24}{}}{\sundata{06:34}{21:16}{09:30}}{\textsf{\krishna~त्रयोदशी} {\tiny \RIGHTarrow} 12:00\hspace{2ex}}{\textsf{पुनर्वसू} {\tiny \RIGHTarrow} 21:01\hspace{2ex}}{19:25-21:16}{13:55-15:45} 
&
%मकर
\caldata{9}{\sunmonth{कर्कटक}{25}{}}{\sundata{06:36}{21:15}{09:31}}{\textsf{\krishna~चतुर्दशी} {\tiny \RIGHTarrow} 08:44\hspace{2ex}}{\textsf{पुष्य} {\tiny \RIGHTarrow} 18:24\hspace{2ex}}{08:25-10:15}{12:05-13:55} 
&
%मकर
\caldata{10}{\sunmonth{कर्कटक}{26}{}}{\sundata{06:37}{21:13}{09:32}}{\textsf{\shukla~प्रथमा} {\tiny \RIGHTarrow} 01:21(+1)}{\textsf{आश्रेषा} {\tiny \RIGHTarrow} 15:32\hspace{2ex}}{17:34-19:23}{10:16-12:05} 
&
%मकर
\caldata{11}{\sunmonth{कर्कटक}{27}{}}{\sundata{06:38}{21:11}{09:32}}{\textsf{\shukla~द्वितीया} {\tiny \RIGHTarrow} 21:37\hspace{2ex}}{\textsf{मघा} {\tiny \RIGHTarrow} 12:36\hspace{2ex}}{13:54-15:43}{08:27-10:16} 
&
%मकर
\caldata{12}{\sunmonth{कर्कटक}{28}{}}{\sundata{06:40}{21:10}{09:34}}{\textsf{\shukla~तृतीया} {\tiny \RIGHTarrow} 18:05\hspace{2ex}}{\textsf{पूर्वफल्गुनी} {\tiny \RIGHTarrow} 09:46\hspace{2ex}}{15:43-17:32}{06:40-08:28} 
&
%मकर
\caldata{13}{\sunmonth{कर्कटक}{29}{}}{\sundata{06:41}{21:08}{09:34}}{\textsf{\shukla~चतुर्थी} {\tiny \RIGHTarrow} 14:52\hspace{2ex}}{\textsf{उत्तरफल्गुनी} {\tiny \RIGHTarrow} 07:11\hspace{2ex}}{12:06-13:54}{17:31-19:19} 
&
%मकर
\caldata{14}{\sunmonth{कर्कटक}{30}{}}{\sundata{06:43}{21:06}{09:35}}{\textsf{\shukla~पञ्चमी} {\tiny \RIGHTarrow} 12:09\hspace{2ex}}{\textsf{चित्रा} {\tiny \RIGHTarrow} 03:37(+1)}{10:18-12:06}{15:42-17:30} 
\\ \hline
%मकर
\caldata{15}{\sunmonth{कर्कटक}{31}{}}{\sundata{06:44}{21:04}{09:36}}{\textsf{\shukla~षष्ठी} {\tiny \RIGHTarrow} 10:03\hspace{2ex}}{\textsf{स्वाति} {\tiny \RIGHTarrow} 02:49(+1)}{19:16-21:04}{13:54-15:41} 
&
%मकर
\caldata{16}{\sunmonth{कर्कटक}{32}{{\textsf{कर्कटक} {\tiny \RIGHTarrow} 01:56(+1)}}}{\sundata{06:45}{21:03}{09:36}}{\textsf{\shukla~सप्तमी} {\tiny \RIGHTarrow} 08:40\hspace{2ex}}{\textsf{विशाखा} {\tiny \RIGHTarrow} 02:45(+1)}{08:32-10:19}{12:06-13:54} 
&
%मकर
\caldata{17}{\sunmonth{सिंह}{1}{}}{\sundata{06:47}{21:01}{09:37}}{\textsf{\shukla~अष्टमी} {\tiny \RIGHTarrow} 08:02\hspace{2ex}}{\textsf{अनूराधा} {\tiny \RIGHTarrow} 03:23(+1)}{17:27-19:14}{10:20-12:07} 
&
%मकर
\caldata{18}{\sunmonth{सिंह}{2}{}}{\sundata{06:48}{20:59}{09:38}}{\textsf{\shukla~नवमी} {\tiny \RIGHTarrow} 08:09\hspace{2ex}}{\textsf{ज्येष्ठा} {\tiny \RIGHTarrow} 04:43(+1)}{13:53-15:39}{08:34-10:20} 
&
%मकर
\caldata{19}{\sunmonth{सिंह}{3}{}}{\sundata{06:50}{20:57}{09:39}}{\textsf{\shukla~दशमी} {\tiny \RIGHTarrow} 08:57\hspace{2ex}}{\textsf{मूला} {\tiny \RIGHTarrow} 06:36(+1)}{15:39-17:25}{06:50-08:35} 
&
%मकर
\caldata{20}{\sunmonth{सिंह}{4}{}}{\sundata{06:51}{20:55}{09:39}}{\textsf{\shukla~एकादशी} {\tiny \RIGHTarrow} 10:20\hspace{2ex}}{\textsf{पूर्वाषाढा} {\tiny \RIGHTarrow} \textsf{अहोरात्रम्}}{12:07-13:53}{17:24-19:09} 
&
%मकर
\caldata{21}{\sunmonth{सिंह}{5}{}}{\sundata{06:52}{20:53}{09:40}}{\textsf{\shukla~द्वादशी} {\tiny \RIGHTarrow} 12:10\hspace{2ex}}{\textsf{पूर्वाषाढा} {\tiny \RIGHTarrow} 08:59\hspace{2ex}}{10:22-12:07}{15:37-17:22} 
\\ \hline
%मकर
\caldata{22}{\sunmonth{सिंह}{6}{}}{\sundata{06:54}{20:51}{09:41}}{\textsf{\shukla~त्रयोदशी} {\tiny \RIGHTarrow} 14:18\hspace{2ex}}{\textsf{उत्तराषाढा} {\tiny \RIGHTarrow} 11:40\hspace{2ex}}{19:06-20:51}{13:52-15:37} 
&
%मकर
\caldata{23}{\sunmonth{सिंह}{7}{}}{\sundata{06:55}{20:49}{09:41}}{\textsf{\shukla~चतुर्दशी} {\tiny \RIGHTarrow} 16:38\hspace{2ex}}{\textsf{श्रवण} {\tiny \RIGHTarrow} 14:33\hspace{2ex}}{08:39-10:23}{12:07-13:52} 
&
%मकर
\caldata{24}{\sunmonth{सिंह}{8}{}}{\sundata{06:57}{20:47}{09:43}}{\textsf{\fullmoon~पूर्णिमा} {\tiny \RIGHTarrow} 19:04\hspace{2ex}}{\textsf{श्रविष्ठा} {\tiny \RIGHTarrow} 17:32\hspace{2ex}}{17:19-19:03}{10:24-12:08} 
&
%मकर
\caldata{25}{\sunmonth{सिंह}{9}{}}{\sundata{06:58}{20:46}{09:43}}{\textsf{\krishna~प्रथमा} {\tiny \RIGHTarrow} 21:30\hspace{2ex}}{\textsf{शतभिषक्} {\tiny \RIGHTarrow} 20:31\hspace{2ex}}{13:52-15:35}{08:41-10:25} 
&
%मकर
\caldata{26}{\sunmonth{सिंह}{10}{}}{\sundata{06:59}{20:44}{09:44}}{\textsf{\krishna~द्वितीया} {\tiny \RIGHTarrow} 23:52\hspace{2ex}}{\textsf{प्रोष्ठपदा} {\tiny \RIGHTarrow} 23:26\hspace{2ex}}{15:34-17:17}{06:59-08:42} 
&
%मकर
\caldata{27}{\sunmonth{सिंह}{11}{}}{\sundata{07:01}{20:42}{09:45}}{\textsf{\krishna~तृतीया} {\tiny \RIGHTarrow} 02:05(+1)}{\textsf{उत्तरप्रोष्ठपदा} {\tiny \RIGHTarrow} 02:13(+1)}{12:08-13:51}{17:16-18:59} 
&
%मकर
\caldata{28}{\sunmonth{सिंह}{12}{}}{\sundata{07:02}{20:40}{09:45}}{\textsf{\krishna~चतुर्थी} {\tiny \RIGHTarrow} 04:03(+1)}{\textsf{रेवती} {\tiny \RIGHTarrow} 04:45(+1)}{10:26-12:08}{15:33-17:15} 
\\ \hline
%मकर
\caldata{29}{\sunmonth{सिंह}{13}{}}{\sundata{07:04}{20:38}{09:46}}{\textsf{\krishna~पञ्चमी} {\tiny \RIGHTarrow} 05:39(+1)}{\textsf{अश्विनी} {\tiny \RIGHTarrow} 06:56(+1)}{18:56-20:38}{13:51-15:32} 
&
%मकर
\caldata{30}{\sunmonth{सिंह}{14}{}}{\sundata{07:05}{20:36}{09:47}}{\textsf{\krishna~षष्ठी} {\tiny \RIGHTarrow} 06:47(+1)}{\textsf{अपभरणी} {\tiny \RIGHTarrow} \textsf{अहोरात्रम्}}{08:46-10:27}{12:09-13:50} 
&
%मकर
\caldata{31}{\sunmonth{सिंह}{15}{}}{\sundata{07:07}{20:33}{09:48}}{\textsf{\krishna~सप्तमी} {\tiny \RIGHTarrow} \textsf{अहोरात्रम्}}{\textsf{अपभरणी} {\tiny \RIGHTarrow} 08:38\hspace{2ex}}{17:11-18:52}{10:28-12:09} 
&
%मकर
{}  &
{}  &
{}  &
\\ \hline
\end{tabular}


%\clearpage
\begin{tabular}{|c|c|c|c|c|c|c|}
\multicolumn{7}{c}{\Large \bfseries SEPTEMBER 2010}\\[3mm]
\hline
\textbf{SUN} & \textbf{MON} & \textbf{TUE} & \textbf{WED} & \textbf{THU} & \textbf{FRI} & \textbf{SAT} \\ \hline
{}  &
{}  &
{}  &
\caldata{1}{\sunmonth{सिंह}{16}{}}{\sundata{07:08}{20:31}{09:48}}{\textsf{\krishna~सप्तमी} {\tiny \RIGHTarrow} 07:20\hspace{2ex}}{\textsf{कृत्तिका} {\tiny \RIGHTarrow} 09:46\hspace{2ex}}{13:49-15:29}{08:48-10:28} 
&
%मकर
\caldata{2}{\sunmonth{सिंह}{17}{}}{\sundata{07:09}{20:29}{09:49}}{\textsf{\krishna~अष्टमी} {\tiny \RIGHTarrow} 07:12\hspace{2ex}}{\textsf{रोहिणी} {\tiny \RIGHTarrow} 10:14\hspace{2ex}}{15:29-17:09}{07:09-08:49} 
&
%मकर
\caldata{3}{\sunmonth{सिंह}{18}{}}{\sundata{07:11}{20:27}{09:50}}{\textsf{\krishna~दशमी} {\tiny \RIGHTarrow} 04:41(+1)}{\textsf{मृगशीर्ष} {\tiny \RIGHTarrow} 09:58\hspace{2ex}}{12:09-13:49}{17:08-18:47} 
&
%मकर
\caldata{4}{\sunmonth{सिंह}{19}{}}{\sundata{07:12}{20:25}{09:50}}{\textsf{\krishna~एकादशी} {\tiny \RIGHTarrow} 02:22(+1)}{\textsf{आर्द्रा} {\tiny \RIGHTarrow} 09:00\hspace{2ex}}{10:30-12:09}{15:27-17:06} 
\\ \hline
%मकर
\caldata{5}{\sunmonth{सिंह}{20}{}}{\sundata{07:14}{20:23}{09:51}}{\textsf{\krishna~द्वादशी} {\tiny \RIGHTarrow} 23:26\hspace{2ex}}{\textsf{पुनर्वसू} {\tiny \RIGHTarrow} 07:22\hspace{2ex}}{18:44-20:23}{13:48-15:27} 
&
%मकर
\caldata{6}{\sunmonth{सिंह}{21}{}}{\sundata{07:15}{20:21}{09:52}}{\textsf{\krishna~त्रयोदशी} {\tiny \RIGHTarrow} 20:03\hspace{2ex}}{\textsf{आश्रेषा} {\tiny \RIGHTarrow} 02:22(+1)}{08:53-10:31}{12:09-13:48} 
&
%मकर
\caldata{7}{\sunmonth{सिंह}{22}{}}{\sundata{07:16}{20:19}{09:52}}{\textsf{\krishna~चतुर्दशी} {\tiny \RIGHTarrow} 16:21\hspace{2ex}}{\textsf{मघा} {\tiny \RIGHTarrow} 23:24\hspace{2ex}}{17:03-18:41}{10:31-12:09} 
&
%मकर
\caldata{8}{\sunmonth{सिंह}{23}{}}{\sundata{07:18}{20:17}{09:53}}{\textsf{\newmoon~अमावस्या} {\tiny \RIGHTarrow} 12:30\hspace{2ex}}{\textsf{पूर्वफल्गुनी} {\tiny \RIGHTarrow} 20:20\hspace{2ex}}{13:47-15:24}{08:55-10:32} 
&
%मकर
\caldata{9}{\sunmonth{सिंह}{24}{}}{\sundata{07:19}{20:15}{09:54}}{\textsf{\shukla~प्रथमा} {\tiny \RIGHTarrow} 08:38\hspace{2ex}}{\textsf{उत्तरफल्गुनी} {\tiny \RIGHTarrow} 17:23\hspace{2ex}}{15:24-17:01}{07:19-08:56} 
&
%मकर
\caldata{10}{\sunmonth{सिंह}{25}{}}{\sundata{07:21}{20:13}{09:55}}{\textsf{\shukla~तृतीया} {\tiny \RIGHTarrow} 01:48(+1)}{\textsf{हस्त} {\tiny \RIGHTarrow} 14:43\hspace{2ex}}{12:10-13:47}{17:00-18:36} 
&
%मकर
\caldata{11}{\sunmonth{सिंह}{26}{}}{\sundata{07:22}{20:11}{09:55}}{\textsf{\shukla~चतुर्थी} {\tiny \RIGHTarrow} 23:09\hspace{2ex}}{\textsf{चित्रा} {\tiny \RIGHTarrow} 12:31\hspace{2ex}}{10:34-12:10}{15:22-16:58} 
\\ \hline
%मकर
\caldata{12}{\sunmonth{सिंह}{27}{}}{\sundata{07:23}{20:08}{09:56}}{\textsf{\shukla~पञ्चमी} {\tiny \RIGHTarrow} 21:13\hspace{2ex}}{\textsf{स्वाति} {\tiny \RIGHTarrow} 10:58\hspace{2ex}}{18:32-20:08}{13:45-15:21} 
&
%मकर
\caldata{13}{\sunmonth{सिंह}{28}{}}{\sundata{07:25}{20:06}{09:57}}{\textsf{\shukla~षष्ठी} {\tiny \RIGHTarrow} 20:04\hspace{2ex}}{\textsf{विशाखा} {\tiny \RIGHTarrow} 10:10\hspace{2ex}}{09:00-10:35}{12:10-13:45} 
&
%मकर
\caldata{14}{\sunmonth{सिंह}{29}{}}{\sundata{07:26}{20:04}{09:57}}{\textsf{\shukla~सप्तमी} {\tiny \RIGHTarrow} 19:45\hspace{2ex}}{\textsf{अनूराधा} {\tiny \RIGHTarrow} 10:13\hspace{2ex}}{16:54-18:29}{10:35-12:10} 
&
%मकर
\caldata{15}{\sunmonth{सिंह}{30}{}}{\sundata{07:28}{20:02}{09:58}}{\textsf{\shukla~अष्टमी} {\tiny \RIGHTarrow} 20:16\hspace{2ex}}{\textsf{ज्येष्ठा} {\tiny \RIGHTarrow} 11:06\hspace{2ex}}{13:45-15:19}{09:02-10:36} 
&
%मकर
\caldata{16}{\sunmonth{सिंह}{31}{{\textsf{सिंह} {\tiny \RIGHTarrow} 01:52(+1)}}}{\sundata{07:29}{20:00}{09:59}}{\textsf{\shukla~नवमी} {\tiny \RIGHTarrow} 21:31\hspace{2ex}}{\textsf{मूला} {\tiny \RIGHTarrow} 12:44\hspace{2ex}}{15:18-16:52}{07:29-09:02} 
&
%मकर
\caldata{17}{\sunmonth{कन्या}{1}{}}{\sundata{07:31}{19:58}{10:00}}{\textsf{\shukla~दशमी} {\tiny \RIGHTarrow} 23:21\hspace{2ex}}{\textsf{पूर्वाषाढा} {\tiny \RIGHTarrow} 14:58\hspace{2ex}}{12:11-13:44}{16:51-18:24} 
&
%मकर
\caldata{18}{\sunmonth{कन्या}{2}{}}{\sundata{07:32}{19:56}{10:00}}{\textsf{\shukla~एकादशी} {\tiny \RIGHTarrow} 01:35(+1)}{\textsf{उत्तराषाढा} {\tiny \RIGHTarrow} 17:39\hspace{2ex}}{10:38-12:11}{15:17-16:50} 
\\ \hline
%मकर
\caldata{19}{\sunmonth{कन्या}{3}{}}{\sundata{07:33}{19:54}{10:01}}{\textsf{\shukla~द्वादशी} {\tiny \RIGHTarrow} 04:02(+1)}{\textsf{श्रवण} {\tiny \RIGHTarrow} 20:35\hspace{2ex}}{18:21-19:54}{13:43-15:16} 
&
%मकर
\caldata{20}{\sunmonth{कन्या}{4}{}}{\sundata{07:35}{19:51}{10:02}}{\textsf{\shukla~त्रयोदशी} {\tiny \RIGHTarrow} 06:33(+1)}{\textsf{श्रविष्ठा} {\tiny \RIGHTarrow} 23:36\hspace{2ex}}{09:07-10:39}{12:11-13:43} 
&
%मकर
\caldata{21}{\sunmonth{कन्या}{5}{}}{\sundata{07:36}{19:49}{10:02}}{\textsf{\shukla~चतुर्दशी} {\tiny \RIGHTarrow} \textsf{अहोरात्रम्}}{\textsf{शतभिषक्} {\tiny \RIGHTarrow} 02:35(+1)}{16:45-18:17}{10:39-12:10} 
&
%मकर
\caldata{22}{\sunmonth{कन्या}{6}{}}{\sundata{07:38}{19:47}{10:03}}{\textsf{\shukla~चतुर्दशी} {\tiny \RIGHTarrow} 09:00\hspace{2ex}}{\textsf{प्रोष्ठपदा} {\tiny \RIGHTarrow} 05:26(+1)}{13:42-15:13}{09:09-10:40} 
&
%मकर
\caldata{23}{\sunmonth{कन्या}{7}{}}{\sundata{07:39}{19:45}{10:04}}{\textsf{\fullmoon~पूर्णिमा} {\tiny \RIGHTarrow} 11:16\hspace{2ex}}{\textsf{उत्तरप्रोष्ठपदा} {\tiny \RIGHTarrow} \textsf{अहोरात्रम्}}{15:12-16:43}{07:39-09:09} 
&
%मकर
\caldata{24}{\sunmonth{कन्या}{8}{}}{\sundata{07:41}{19:43}{10:05}}{\textsf{\krishna~प्रथमा} {\tiny \RIGHTarrow} 13:19\hspace{2ex}}{\textsf{उत्तरप्रोष्ठपदा} {\tiny \RIGHTarrow} 08:05\hspace{2ex}}{12:11-13:42}{16:42-18:12} 
&
%मकर
\caldata{25}{\sunmonth{कन्या}{9}{}}{\sundata{07:42}{19:41}{10:05}}{\textsf{\krishna~द्वितीया} {\tiny \RIGHTarrow} 15:06\hspace{2ex}}{\textsf{रेवती} {\tiny \RIGHTarrow} 10:29\hspace{2ex}}{10:41-12:11}{15:11-16:41} 
\\ \hline
%मकर
\caldata{26}{\sunmonth{कन्या}{10}{}}{\sundata{07:43}{19:39}{10:06}}{\textsf{\krishna~तृतीया} {\tiny \RIGHTarrow} 16:35\hspace{2ex}}{\textsf{अश्विनी} {\tiny \RIGHTarrow} 12:35\hspace{2ex}}{18:09-19:39}{13:41-15:10} 
&
%मकर
\caldata{27}{\sunmonth{कन्या}{11}{}}{\sundata{07:45}{19:37}{10:07}}{\textsf{\krishna~चतुर्थी} {\tiny \RIGHTarrow} 17:42\hspace{2ex}}{\textsf{अपभरणी} {\tiny \RIGHTarrow} 14:20\hspace{2ex}}{09:14-10:43}{12:12-13:41} 
&
%मकर
\caldata{28}{\sunmonth{कन्या}{12}{}}{\sundata{07:46}{19:34}{10:07}}{\textsf{\krishna~पञ्चमी} {\tiny \RIGHTarrow} 18:23\hspace{2ex}}{\textsf{कृत्तिका} {\tiny \RIGHTarrow} 15:42\hspace{2ex}}{16:37-18:05}{10:43-12:11} 
&
%मकर
\caldata{29}{\sunmonth{कन्या}{13}{}}{\sundata{07:48}{19:32}{10:08}}{\textsf{\krishna~षष्ठी} {\tiny \RIGHTarrow} 18:34\hspace{2ex}}{\textsf{रोहिणी} {\tiny \RIGHTarrow} 16:34\hspace{2ex}}{13:40-15:08}{09:16-10:44} 
&
%मकर
\caldata{30}{\sunmonth{कन्या}{14}{}}{\sundata{07:49}{19:30}{10:09}}{\textsf{\krishna~सप्तमी} {\tiny \RIGHTarrow} 18:12\hspace{2ex}}{\textsf{मृगशीर्ष} {\tiny \RIGHTarrow} 16:54\hspace{2ex}}{15:07-16:34}{07:49-09:16} 
&
%मकर
{}  &
\\ \hline
\end{tabular}


%\clearpage
\begin{tabular}{|c|c|c|c|c|c|c|}
\multicolumn{7}{c}{\Large \bfseries OCTOBER 2010}\\[3mm]
\hline
\textbf{SUN} & \textbf{MON} & \textbf{TUE} & \textbf{WED} & \textbf{THU} & \textbf{FRI} & \textbf{SAT} \\ \hline
{}  &
{}  &
{}  &
{}  &
{}  &
\caldata{1}{\sunmonth{कन्या}{15}{}}{\sundata{07:51}{19:28}{10:10}}{\textsf{\krishna~अष्टमी} {\tiny \RIGHTarrow} 17:12\hspace{2ex}}{\textsf{आर्द्रा} {\tiny \RIGHTarrow} 16:37\hspace{2ex}}{12:12-13:39}{16:33-18:00} 
&
%मकर
\caldata{2}{\sunmonth{कन्या}{16}{}}{\sundata{07:52}{19:26}{10:10}}{\textsf{\krishna~नवमी} {\tiny \RIGHTarrow} 15:35\hspace{2ex}}{\textsf{पुनर्वसू} {\tiny \RIGHTarrow} 15:44\hspace{2ex}}{10:45-12:12}{15:05-16:32} 
\\ \hline
%मकर
\caldata{3}{\sunmonth{कन्या}{17}{}}{\sundata{07:54}{19:24}{10:12}}{\textsf{\krishna~दशमी} {\tiny \RIGHTarrow} 13:22\hspace{2ex}}{\textsf{पुष्य} {\tiny \RIGHTarrow} 14:15\hspace{2ex}}{17:57-19:24}{13:39-15:05} 
&
%मकर
\caldata{4}{\sunmonth{कन्या}{18}{}}{\sundata{07:55}{19:22}{10:12}}{\textsf{\krishna~एकादशी} {\tiny \RIGHTarrow} 10:38\hspace{2ex}}{\textsf{आश्रेषा} {\tiny \RIGHTarrow} 12:14\hspace{2ex}}{09:20-10:46}{12:12-13:38} 
&
%मकर
\caldata{5}{\sunmonth{कन्या}{19}{}}{\sundata{07:57}{19:20}{10:13}}{\textsf{\krishna~त्रयोदशी} {\tiny \RIGHTarrow} 03:56(+1)}{\textsf{मघा} {\tiny \RIGHTarrow} 09:47\hspace{2ex}}{16:29-17:54}{10:47-12:13} 
&
%मकर
\caldata{6}{\sunmonth{कन्या}{20}{}}{\sundata{07:58}{19:18}{10:14}}{\textsf{\krishna~चतुर्दशी} {\tiny \RIGHTarrow} 00:19(+1)}{\textsf{उत्तरफल्गुनी} {\tiny \RIGHTarrow} 04:13(+1)}{13:38-15:03}{09:23-10:48} 
&
%मकर
\caldata{7}{\sunmonth{कन्या}{21}{}}{\sundata{08:00}{19:16}{10:15}}{\textsf{\newmoon~अमावस्या} {\tiny \RIGHTarrow} 20:46\hspace{2ex}}{\textsf{हस्त} {\tiny \RIGHTarrow} 01:28(+1)}{15:02-16:27}{08:00-09:24} 
&
%मकर
\caldata{8}{\sunmonth{कन्या}{22}{}}{\sundata{08:01}{19:14}{10:15}}{\textsf{\shukla~प्रथमा} {\tiny \RIGHTarrow} 17:27\hspace{2ex}}{\textsf{चित्रा} {\tiny \RIGHTarrow} 23:01\hspace{2ex}}{12:13-13:37}{16:25-17:49} 
&
%मकर
\caldata{9}{\sunmonth{कन्या}{23}{}}{\sundata{08:02}{19:12}{10:16}}{\textsf{\shukla~द्वितीया} {\tiny \RIGHTarrow} 14:32\hspace{2ex}}{\textsf{स्वाति} {\tiny \RIGHTarrow} 21:01\hspace{2ex}}{10:49-12:13}{15:00-16:24} 
\\ \hline
%मकर
\caldata{10}{\sunmonth{कन्या}{24}{}}{\sundata{08:04}{19:10}{10:17}}{\textsf{\shukla~तृतीया} {\tiny \RIGHTarrow} 12:10\hspace{2ex}}{\textsf{विशाखा} {\tiny \RIGHTarrow} 19:39\hspace{2ex}}{17:46-19:10}{13:37-15:00} 
&
%मकर
\caldata{11}{\sunmonth{कन्या}{25}{}}{\sundata{08:05}{19:08}{10:17}}{\textsf{\shukla~चतुर्थी} {\tiny \RIGHTarrow} 10:32\hspace{2ex}}{\textsf{अनूराधा} {\tiny \RIGHTarrow} 19:01\hspace{2ex}}{09:27-10:50}{12:13-13:36} 
&
%मकर
\caldata{12}{\sunmonth{कन्या}{26}{}}{\sundata{08:07}{19:06}{10:18}}{\textsf{\shukla~पञ्चमी} {\tiny \RIGHTarrow} 09:44\hspace{2ex}}{\textsf{ज्येष्ठा} {\tiny \RIGHTarrow} 19:13\hspace{2ex}}{16:21-17:43}{10:51-12:14} 
&
%मकर
\caldata{13}{\sunmonth{कन्या}{27}{}}{\sundata{08:09}{19:04}{10:20}}{\textsf{\shukla~षष्ठी} {\tiny \RIGHTarrow} 09:49\hspace{2ex}}{\textsf{मूला} {\tiny \RIGHTarrow} 20:15\hspace{2ex}}{13:36-14:58}{09:30-10:52} 
&
%मकर
\caldata{14}{\sunmonth{कन्या}{28}{}}{\sundata{08:10}{19:02}{10:20}}{\textsf{\shukla~सप्तमी} {\tiny \RIGHTarrow} 10:45\hspace{2ex}}{\textsf{पूर्वाषाढा} {\tiny \RIGHTarrow} 22:02\hspace{2ex}}{14:57-16:19}{08:10-09:31} 
&
%मकर
\caldata{15}{\sunmonth{कन्या}{29}{}}{\sundata{08:12}{19:00}{10:21}}{\textsf{\shukla~अष्टमी} {\tiny \RIGHTarrow} 12:22\hspace{2ex}}{\textsf{उत्तराषाढा} {\tiny \RIGHTarrow} 00:25(+1)}{12:15-13:36}{16:18-17:39} 
&
%मकर
\caldata{16}{\sunmonth{कन्या}{30}{}}{\sundata{08:13}{18:58}{10:22}}{\textsf{\shukla~नवमी} {\tiny \RIGHTarrow} 14:31\hspace{2ex}}{\textsf{श्रवण} {\tiny \RIGHTarrow} 03:13(+1)}{10:54-12:14}{14:56-16:16} 
\\ \hline
%मकर
\caldata{17}{\sunmonth{तुला}{1}{{\textsf{कन्या} {\tiny \RIGHTarrow} 13:50\hspace{2ex}}}}{\sundata{08:15}{18:56}{10:23}}{\textsf{\shukla~दशमी} {\tiny \RIGHTarrow} 16:58\hspace{2ex}}{\textsf{श्रविष्ठा} {\tiny \RIGHTarrow} 06:13(+1)}{17:35-18:56}{13:35-14:55} 
&
%मकर
\caldata{18}{\sunmonth{तुला}{2}{}}{\sundata{08:16}{18:54}{10:23}}{\textsf{\shukla~एकादशी} {\tiny \RIGHTarrow} 19:31\hspace{2ex}}{\textsf{शतभिषक्} {\tiny \RIGHTarrow} \textsf{अहोरात्रम्}}{09:35-10:55}{12:15-13:35} 
&
%मकर
\caldata{19}{\sunmonth{तुला}{3}{}}{\sundata{08:18}{18:52}{10:24}}{\textsf{\shukla~द्वादशी} {\tiny \RIGHTarrow} 21:57\hspace{2ex}}{\textsf{शतभिषक्} {\tiny \RIGHTarrow} 09:13\hspace{2ex}}{16:13-17:32}{10:56-12:15} 
&
%मकर
\caldata{20}{\sunmonth{तुला}{4}{}}{\sundata{08:19}{18:50}{10:25}}{\textsf{\shukla~त्रयोदशी} {\tiny \RIGHTarrow} 00:09(+1)}{\textsf{प्रोष्ठपदा} {\tiny \RIGHTarrow} 12:02\hspace{2ex}}{13:34-14:53}{09:37-10:56} 
&
%मकर
\caldata{21}{\sunmonth{तुला}{5}{}}{\sundata{08:21}{18:48}{10:26}}{\textsf{\shukla~चतुर्दशी} {\tiny \RIGHTarrow} 02:03(+1)}{\textsf{उत्तरप्रोष्ठपदा} {\tiny \RIGHTarrow} 14:35\hspace{2ex}}{14:52-16:11}{08:21-09:39} 
&
%मकर
\caldata{22}{\sunmonth{तुला}{6}{}}{\sundata{08:22}{18:46}{10:26}}{\textsf{\fullmoon~पूर्णिमा} {\tiny \RIGHTarrow} 03:34(+1)}{\textsf{रेवती} {\tiny \RIGHTarrow} 16:47\hspace{2ex}}{12:16-13:34}{16:10-17:28} 
&
%मकर
\caldata{23}{\sunmonth{तुला}{7}{}}{\sundata{08:24}{18:44}{10:28}}{\textsf{\krishna~प्रथमा} {\tiny \RIGHTarrow} 04:44(+1)}{\textsf{अश्विनी} {\tiny \RIGHTarrow} 18:38\hspace{2ex}}{10:59-12:16}{14:51-16:09} 
\\ \hline
%मकर
\caldata{24}{\sunmonth{तुला}{8}{}}{\sundata{08:25}{18:42}{10:28}}{\textsf{\krishna~द्वितीया} {\tiny \RIGHTarrow} 05:30(+1)}{\textsf{अपभरणी} {\tiny \RIGHTarrow} 20:08\hspace{2ex}}{17:24-18:42}{13:33-14:50} 
&
%मकर
\caldata{25}{\sunmonth{तुला}{9}{}}{\sundata{08:27}{18:41}{10:29}}{\textsf{\krishna~तृतीया} {\tiny \RIGHTarrow} 05:54(+1)}{\textsf{कृत्तिका} {\tiny \RIGHTarrow} 21:15\hspace{2ex}}{09:43-11:00}{12:17-13:34} 
&
%मकर
\caldata{26}{\sunmonth{तुला}{10}{}}{\sundata{08:29}{18:39}{10:31}}{\textsf{\krishna~चतुर्थी} {\tiny \RIGHTarrow} 05:56(+1)}{\textsf{रोहिणी} {\tiny \RIGHTarrow} 22:01\hspace{2ex}}{16:06-17:22}{11:01-12:17} 
&
%मकर
\caldata{27}{\sunmonth{तुला}{11}{}}{\sundata{08:30}{18:37}{10:31}}{\textsf{\krishna~पञ्चमी} {\tiny \RIGHTarrow} 05:33(+1)}{\textsf{मृगशीर्ष} {\tiny \RIGHTarrow} 22:24\hspace{2ex}}{13:33-14:49}{09:45-11:01} 
&
%मकर
\caldata{28}{\sunmonth{तुला}{12}{}}{\sundata{08:32}{18:35}{10:32}}{\textsf{\krishna~षष्ठी} {\tiny \RIGHTarrow} 04:45(+1)}{\textsf{आर्द्रा} {\tiny \RIGHTarrow} 22:23\hspace{2ex}}{14:48-16:04}{08:32-09:47} 
&
%मकर
\caldata{29}{\sunmonth{तुला}{13}{}}{\sundata{08:33}{18:34}{10:33}}{\textsf{\krishna~सप्तमी} {\tiny \RIGHTarrow} 03:30(+1)}{\textsf{पुनर्वसू} {\tiny \RIGHTarrow} 21:56\hspace{2ex}}{12:18-13:33}{16:03-17:18} 
&
%मकर
\caldata{30}{\sunmonth{तुला}{14}{}}{\sundata{08:35}{18:32}{10:34}}{\textsf{\krishna~अष्टमी} {\tiny \RIGHTarrow} 01:48(+1)}{\textsf{पुष्य} {\tiny \RIGHTarrow} 21:03\hspace{2ex}}{11:04-12:18}{14:48-16:02} 
\\ \hline
%मकर
\caldata{31}{\sunmonth{तुला}{15}{}}{\sundata{07:37}{17:30}{09:35}}{\textsf{\krishna~नवमी} {\tiny \RIGHTarrow} 22:41\hspace{2ex}}{\textsf{आश्रेषा} {\tiny \RIGHTarrow} 18:44\hspace{2ex}}{16:15-17:30}{12:33-13:47} 
&
%मकर
{}  &
{}  &
{}  &
{}  &
{}  &
\\ \hline
\end{tabular}


%\clearpage
\begin{tabular}{|c|c|c|c|c|c|c|}
\multicolumn{7}{c}{\Large \bfseries NOVEMBER 2010}\\[3mm]
\hline
\textbf{SUN} & \textbf{MON} & \textbf{TUE} & \textbf{WED} & \textbf{THU} & \textbf{FRI} & \textbf{SAT} \\ \hline
{}  &
\caldata{1}{\sunmonth{तुला}{16}{}}{\sundata{07:38}{17:29}{09:36}}{\textsf{\krishna~दशमी} {\tiny \RIGHTarrow} 20:12\hspace{2ex}}{\textsf{मघा} {\tiny \RIGHTarrow} 17:02\hspace{2ex}}{08:51-10:05}{11:19-12:33} 
&
%मकर
\caldata{2}{\sunmonth{तुला}{17}{}}{\sundata{07:40}{17:27}{09:37}}{\textsf{\krishna~एकादशी} {\tiny \RIGHTarrow} 17:26\hspace{2ex}}{\textsf{पूर्वफल्गुनी} {\tiny \RIGHTarrow} 15:02\hspace{2ex}}{15:00-16:13}{10:06-11:20} 
&
%मकर
\caldata{3}{\sunmonth{तुला}{18}{}}{\sundata{07:41}{17:25}{09:37}}{\textsf{\krishna~द्वादशी} {\tiny \RIGHTarrow} 14:29\hspace{2ex}}{\textsf{उत्तरफल्गुनी} {\tiny \RIGHTarrow} 12:50\hspace{2ex}}{12:33-13:46}{08:54-10:07} 
&
%मकर
\caldata{4}{\sunmonth{तुला}{19}{}}{\sundata{07:43}{17:24}{09:39}}{\textsf{\krishna~त्रयोदशी} {\tiny \RIGHTarrow} 11:28\hspace{2ex}}{\textsf{हस्त} {\tiny \RIGHTarrow} 10:34\hspace{2ex}}{13:46-14:58}{07:43-08:55} 
&
%मकर
\caldata{5}{\sunmonth{तुला}{20}{}}{\sundata{07:45}{17:22}{09:40}}{\textsf{\krishna~चतुर्दशी} {\tiny \RIGHTarrow} 08:32\hspace{2ex}}{\textsf{चित्रा} {\tiny \RIGHTarrow} 08:22\hspace{2ex}}{11:21-12:33}{14:57-16:09} 
&
%मकर
\caldata{6}{\sunmonth{तुला}{21}{}}{\sundata{07:46}{17:21}{09:41}}{\textsf{\shukla~प्रथमा} {\tiny \RIGHTarrow} 03:39(+1)}{\textsf{विशाखा} {\tiny \RIGHTarrow} 04:56(+1)}{10:09-11:21}{13:45-14:57} 
\\ \hline
%मकर
\caldata{7}{\sunmonth{तुला}{22}{}}{\sundata{07:48}{17:19}{09:42}}{\textsf{\shukla~द्वितीया} {\tiny \RIGHTarrow} 01:59(+1)}{\textsf{अनूराधा} {\tiny \RIGHTarrow} 04:00(+1)}{16:07-17:19}{12:33-13:44} 
&
%मकर
\caldata{8}{\sunmonth{तुला}{23}{}}{\sundata{07:49}{17:18}{09:42}}{\textsf{\shukla~तृतीया} {\tiny \RIGHTarrow} 01:00(+1)}{\textsf{ज्येष्ठा} {\tiny \RIGHTarrow} 03:45(+1)}{09:00-10:11}{11:22-12:33} 
&
%मकर
\caldata{9}{\sunmonth{तुला}{24}{}}{\sundata{07:51}{17:16}{09:44}}{\textsf{\shukla~चतुर्थी} {\tiny \RIGHTarrow} 00:47(+1)}{\textsf{मूला} {\tiny \RIGHTarrow} 04:14(+1)}{14:54-16:05}{10:12-11:22} 
&
%मकर
\caldata{10}{\sunmonth{तुला}{25}{}}{\sundata{07:53}{17:15}{09:45}}{\textsf{\shukla~पञ्चमी} {\tiny \RIGHTarrow} 01:20(+1)}{\textsf{पूर्वाषाढा} {\tiny \RIGHTarrow} 05:28(+1)}{12:34-13:44}{09:03-10:13} 
&
%मकर
\caldata{11}{\sunmonth{तुला}{26}{}}{\sundata{07:54}{17:14}{09:46}}{\textsf{\shukla~षष्ठी} {\tiny \RIGHTarrow} 02:37(+1)}{\textsf{उत्तराषाढा} {\tiny \RIGHTarrow} 07:24(+1)}{13:44-14:54}{07:54-09:04} 
&
%मकर
\caldata{12}{\sunmonth{तुला}{27}{}}{\sundata{07:56}{17:12}{09:47}}{\textsf{\shukla~सप्तमी} {\tiny \RIGHTarrow} 04:31(+1)}{\textsf{श्रवण} {\tiny \RIGHTarrow} \textsf{अहोरात्रम्}}{11:24-12:34}{14:53-16:02} 
&
%मकर
\caldata{13}{\sunmonth{तुला}{28}{}}{\sundata{07:57}{17:11}{09:47}}{\textsf{\shukla~अष्टमी} {\tiny \RIGHTarrow} 06:51(+1)}{\textsf{श्रवण} {\tiny \RIGHTarrow} 09:54\hspace{2ex}}{10:15-11:24}{13:43-14:52} 
\\ \hline
%मकर
\caldata{14}{\sunmonth{तुला}{29}{}}{\sundata{07:59}{17:10}{09:49}}{\textsf{\shukla~नवमी} {\tiny \RIGHTarrow} \textsf{अहोरात्रम्}}{\textsf{श्रविष्ठा} {\tiny \RIGHTarrow} 12:46\hspace{2ex}}{16:01-17:10}{12:34-13:43} 
&
%मकर
\caldata{15}{\sunmonth{तुला}{30}{}}{\sundata{08:00}{17:09}{09:49}}{\textsf{\shukla~नवमी} {\tiny \RIGHTarrow} 09:23\hspace{2ex}}{\textsf{शतभिषक्} {\tiny \RIGHTarrow} 15:44\hspace{2ex}}{09:08-10:17}{11:25-12:34} 
&
%मकर
\caldata{16}{\sunmonth{वृश्चिक}{1}{{\textsf{तुला} {\tiny \RIGHTarrow} 12:41\hspace{2ex}}}}{\sundata{08:02}{17:07}{09:51}}{\textsf{\shukla~दशमी} {\tiny \RIGHTarrow} 11:51\hspace{2ex}}{\textsf{प्रोष्ठपदा} {\tiny \RIGHTarrow} 18:36\hspace{2ex}}{14:50-15:58}{10:18-11:26} 
&
%मकर
\caldata{17}{\sunmonth{वृश्चिक}{2}{}}{\sundata{08:04}{17:06}{09:52}}{\textsf{\shukla~एकादशी} {\tiny \RIGHTarrow} 14:04\hspace{2ex}}{\textsf{उत्तरप्रोष्ठपदा} {\tiny \RIGHTarrow} 21:12\hspace{2ex}}{12:35-13:42}{09:11-10:19} 
&
%मकर
\caldata{18}{\sunmonth{वृश्चिक}{3}{}}{\sundata{08:05}{17:05}{09:53}}{\textsf{\shukla~द्वादशी} {\tiny \RIGHTarrow} 15:53\hspace{2ex}}{\textsf{रेवती} {\tiny \RIGHTarrow} 23:24\hspace{2ex}}{13:42-14:50}{08:05-09:12} 
&
%मकर
\caldata{19}{\sunmonth{वृश्चिक}{4}{}}{\sundata{08:07}{17:04}{09:54}}{\textsf{\shukla~त्रयोदशी} {\tiny \RIGHTarrow} 17:13\hspace{2ex}}{\textsf{अश्विनी} {\tiny \RIGHTarrow} 01:07(+1)}{11:28-12:35}{14:49-15:56} 
&
%मकर
\caldata{20}{\sunmonth{वृश्चिक}{5}{}}{\sundata{08:08}{17:03}{09:55}}{\textsf{\shukla~चतुर्दशी} {\tiny \RIGHTarrow} 18:03\hspace{2ex}}{\textsf{अपभरणी} {\tiny \RIGHTarrow} 02:21(+1)}{10:21-11:28}{13:42-14:49} 
\\ \hline
%मकर
\caldata{21}{\sunmonth{वृश्चिक}{6}{}}{\sundata{08:10}{17:02}{09:56}}{\textsf{\fullmoon~पूर्णिमा} {\tiny \RIGHTarrow} 18:23\hspace{2ex}}{\textsf{कृत्तिका} {\tiny \RIGHTarrow} 03:07(+1)}{15:55-17:02}{12:36-13:42} 
&
%मकर
\caldata{22}{\sunmonth{वृश्चिक}{7}{}}{\sundata{08:11}{17:01}{09:57}}{\textsf{\krishna~प्रथमा} {\tiny \RIGHTarrow} 18:16\hspace{2ex}}{\textsf{रोहिणी} {\tiny \RIGHTarrow} 03:27(+1)}{09:17-10:23}{11:29-12:36} 
&
%मकर
\caldata{23}{\sunmonth{वृश्चिक}{8}{}}{\sundata{08:13}{17:00}{09:58}}{\textsf{\krishna~द्वितीया} {\tiny \RIGHTarrow} 17:45\hspace{2ex}}{\textsf{मृगशीर्ष} {\tiny \RIGHTarrow} 03:24(+1)}{14:48-15:54}{10:24-11:30} 
&
%मकर
\caldata{24}{\sunmonth{वृश्चिक}{9}{}}{\sundata{08:14}{16:59}{09:59}}{\textsf{\krishna~तृतीया} {\tiny \RIGHTarrow} 16:52\hspace{2ex}}{\textsf{आर्द्रा} {\tiny \RIGHTarrow} 03:02(+1)}{12:36-13:42}{09:19-10:25} 
&
%मकर
\caldata{25}{\sunmonth{वृश्चिक}{10}{}}{\sundata{08:15}{16:58}{09:59}}{\textsf{\krishna~चतुर्थी} {\tiny \RIGHTarrow} 15:40\hspace{2ex}}{\textsf{पुनर्वसू} {\tiny \RIGHTarrow} 02:22(+1)}{13:41-14:47}{08:15-09:20} 
&
%मकर
\caldata{26}{\sunmonth{वृश्चिक}{11}{}}{\sundata{08:17}{16:58}{10:01}}{\textsf{\krishna~पञ्चमी} {\tiny \RIGHTarrow} 14:13\hspace{2ex}}{\textsf{पुष्य} {\tiny \RIGHTarrow} 01:27(+1)}{11:32-12:37}{14:47-15:52} 
&
%मकर
\caldata{27}{\sunmonth{वृश्चिक}{12}{}}{\sundata{08:18}{16:57}{10:01}}{\textsf{\krishna~षष्ठी} {\tiny \RIGHTarrow} 12:31\hspace{2ex}}{\textsf{आश्रेषा} {\tiny \RIGHTarrow} 00:19(+1)}{10:27-11:32}{13:42-14:47} 
\\ \hline
%मकर
\caldata{28}{\sunmonth{वृश्चिक}{13}{}}{\sundata{08:20}{16:56}{10:03}}{\textsf{\krishna~सप्तमी} {\tiny \RIGHTarrow} 10:37\hspace{2ex}}{\textsf{मघा} {\tiny \RIGHTarrow} 23:00\hspace{2ex}}{15:51-16:56}{12:38-13:42} 
&
%मकर
\caldata{29}{\sunmonth{वृश्चिक}{14}{}}{\sundata{08:21}{16:56}{10:04}}{\textsf{\krishna~अष्टमी} {\tiny \RIGHTarrow} 08:32\hspace{2ex}}{\textsf{पूर्वफल्गुनी} {\tiny \RIGHTarrow} 21:32\hspace{2ex}}{09:25-10:29}{11:34-12:38} 
&
%मकर
\caldata{30}{\sunmonth{वृश्चिक}{15}{}}{\sundata{08:22}{16:55}{10:04}}{\textsf{\krishna~दशमी} {\tiny \RIGHTarrow} 04:02(+1)}{\textsf{उत्तरफल्गुनी} {\tiny \RIGHTarrow} 19:59\hspace{2ex}}{14:46-15:50}{10:30-11:34} 
&
%मकर
{}  &
{}  &
{}  &
\\ \hline
\end{tabular}


%\clearpage
\begin{tabular}{|c|c|c|c|c|c|c|}
\multicolumn{7}{c}{\Large \bfseries DECEMBER 2010}\\[3mm]
\hline
\textbf{SUN} & \textbf{MON} & \textbf{TUE} & \textbf{WED} & \textbf{THU} & \textbf{FRI} & \textbf{SAT} \\ \hline
{}  &
{}  &
{}  &
\caldata{1}{\sunmonth{वृश्चिक}{16}{}}{\sundata{08:24}{16:54}{10:06}}{\textsf{\krishna~एकादशी} {\tiny \RIGHTarrow} 01:46(+1)}{\textsf{हस्त} {\tiny \RIGHTarrow} 18:24\hspace{2ex}}{12:39-13:42}{09:27-10:31} 
&
%मकर
\caldata{2}{\sunmonth{वृश्चिक}{17}{}}{\sundata{08:25}{16:54}{10:06}}{\textsf{\krishna~द्वादशी} {\tiny \RIGHTarrow} 23:35\hspace{2ex}}{\textsf{चित्रा} {\tiny \RIGHTarrow} 16:53\hspace{2ex}}{13:43-14:46}{08:25-09:28} 
&
%मकर
\caldata{3}{\sunmonth{वृश्चिक}{18}{}}{\sundata{08:26}{16:53}{10:07}}{\textsf{\krishna~त्रयोदशी} {\tiny \RIGHTarrow} 21:37\hspace{2ex}}{\textsf{स्वाति} {\tiny \RIGHTarrow} 15:31\hspace{2ex}}{11:36-12:39}{14:46-15:49} 
&
%मकर
\caldata{4}{\sunmonth{वृश्चिक}{19}{}}{\sundata{08:27}{16:53}{10:08}}{\textsf{\krishna~चतुर्दशी} {\tiny \RIGHTarrow} 19:56\hspace{2ex}}{\textsf{विशाखा} {\tiny \RIGHTarrow} 14:25\hspace{2ex}}{10:33-11:36}{13:43-14:46} 
\\ \hline
%मकर
\caldata{5}{\sunmonth{वृश्चिक}{20}{}}{\sundata{08:28}{16:53}{10:09}}{\textsf{\newmoon~अमावस्या} {\tiny \RIGHTarrow} 18:39\hspace{2ex}}{\textsf{अनूराधा} {\tiny \RIGHTarrow} 13:41\hspace{2ex}}{15:49-16:53}{12:40-13:43} 
&
%मकर
\caldata{6}{\sunmonth{वृश्चिक}{21}{}}{\sundata{08:30}{16:52}{10:10}}{\textsf{\shukla~प्रथमा} {\tiny \RIGHTarrow} 17:54\hspace{2ex}}{\textsf{ज्येष्ठा} {\tiny \RIGHTarrow} 13:27\hspace{2ex}}{09:32-10:35}{11:38-12:41} 
&
%मकर
\caldata{7}{\sunmonth{वृश्चिक}{22}{}}{\sundata{08:31}{16:52}{10:11}}{\textsf{\shukla~द्वितीया} {\tiny \RIGHTarrow} 17:44\hspace{2ex}}{\textsf{मूला} {\tiny \RIGHTarrow} 13:47\hspace{2ex}}{14:46-15:49}{10:36-11:38} 
&
%मकर
\caldata{8}{\sunmonth{वृश्चिक}{23}{}}{\sundata{08:32}{16:52}{10:12}}{\textsf{\shukla~तृतीया} {\tiny \RIGHTarrow} 18:13\hspace{2ex}}{\textsf{पूर्वाषाढा} {\tiny \RIGHTarrow} 14:44\hspace{2ex}}{12:42-13:44}{09:34-10:37} 
&
%मकर
\caldata{9}{\sunmonth{वृश्चिक}{24}{}}{\sundata{08:33}{16:52}{10:12}}{\textsf{\shukla~चतुर्थी} {\tiny \RIGHTarrow} 19:21\hspace{2ex}}{\textsf{उत्तराषाढा} {\tiny \RIGHTarrow} 16:19\hspace{2ex}}{13:44-14:47}{08:33-09:35} 
&
%मकर
\caldata{10}{\sunmonth{वृश्चिक}{25}{}}{\sundata{08:34}{16:52}{10:13}}{\textsf{\shukla~पञ्चमी} {\tiny \RIGHTarrow} 21:04\hspace{2ex}}{\textsf{श्रवण} {\tiny \RIGHTarrow} 18:27\hspace{2ex}}{11:40-12:43}{14:47-15:49} 
&
%मकर
\caldata{11}{\sunmonth{वृश्चिक}{26}{}}{\sundata{08:35}{16:52}{10:14}}{\textsf{\shukla~षष्ठी} {\tiny \RIGHTarrow} 23:15\hspace{2ex}}{\textsf{श्रविष्ठा} {\tiny \RIGHTarrow} 21:03\hspace{2ex}}{10:39-11:41}{13:45-14:47} 
\\ \hline
%मकर
\caldata{12}{\sunmonth{वृश्चिक}{27}{}}{\sundata{08:36}{16:52}{10:15}}{\textsf{\shukla~सप्तमी} {\tiny \RIGHTarrow} 01:42(+1)}{\textsf{शतभिषक्} {\tiny \RIGHTarrow} 23:56\hspace{2ex}}{15:50-16:52}{12:44-13:46} 
&
%मकर
\caldata{13}{\sunmonth{वृश्चिक}{28}{}}{\sundata{08:37}{16:52}{10:16}}{\textsf{\shukla~अष्टमी} {\tiny \RIGHTarrow} 04:13(+1)}{\textsf{प्रोष्ठपदा} {\tiny \RIGHTarrow} 02:52(+1)}{09:38-10:40}{11:42-12:44} 
&
%मकर
\caldata{14}{\sunmonth{वृश्चिक}{29}{}}{\sundata{08:38}{16:52}{10:16}}{\textsf{\shukla~नवमी} {\tiny \RIGHTarrow} 06:34(+1)}{\textsf{उत्तरप्रोष्ठपदा} {\tiny \RIGHTarrow} 05:39(+1)}{14:48-15:50}{10:41-11:43} 
&
%मकर
\caldata{15}{\sunmonth{वृश्चिक}{30}{{\textsf{वृश्चिक} {\tiny \RIGHTarrow} 03:22(+1)}}}{\sundata{08:38}{16:52}{10:16}}{\textsf{\shukla~दशमी} {\tiny \RIGHTarrow} 08:31(+1)}{\textsf{रेवती} {\tiny \RIGHTarrow} 08:05(+1)}{12:45-13:46}{09:39-10:41} 
&
%मकर
\caldata{16}{\sunmonth{धनुः}{1}{}}{\sundata{08:39}{16:52}{10:17}}{\textsf{\shukla~एकादशी} {\tiny \RIGHTarrow} \textsf{अहोरात्रम्}}{\textsf{अश्विनी} {\tiny \RIGHTarrow} \textsf{अहोरात्रम्}}{13:47-14:48}{08:39-09:40} 
&
%मकर
\caldata{17}{\sunmonth{धनुः}{2}{}}{\sundata{08:40}{16:52}{10:18}}{\textsf{\shukla~एकादशी} {\tiny \RIGHTarrow} 09:55\hspace{2ex}}{\textsf{अश्विनी} {\tiny \RIGHTarrow} 09:59\hspace{2ex}}{11:44-12:46}{14:49-15:50} 
&
%मकर
\caldata{18}{\sunmonth{धनुः}{3}{}}{\sundata{08:41}{16:53}{10:19}}{\textsf{\shukla~द्वादशी} {\tiny \RIGHTarrow} 10:40\hspace{2ex}}{\textsf{अपभरणी} {\tiny \RIGHTarrow} 11:16\hspace{2ex}}{10:44-11:45}{13:48-14:50} 
\\ \hline
%मकर
\caldata{19}{\sunmonth{धनुः}{4}{}}{\sundata{08:41}{16:53}{10:19}}{\textsf{\shukla~त्रयोदशी} {\tiny \RIGHTarrow} 10:47\hspace{2ex}}{\textsf{कृत्तिका} {\tiny \RIGHTarrow} 11:56\hspace{2ex}}{15:51-16:53}{12:47-13:48} 
&
%मकर
\caldata{20}{\sunmonth{धनुः}{5}{}}{\sundata{08:42}{16:53}{10:20}}{\textsf{\shukla~चतुर्दशी} {\tiny \RIGHTarrow} 10:17\hspace{2ex}}{\textsf{रोहिणी} {\tiny \RIGHTarrow} 12:00\hspace{2ex}}{09:43-10:44}{11:46-12:47} 
&
%मकर
\caldata{21}{\sunmonth{धनुः}{6}{}}{\sundata{08:42}{16:54}{10:20}}{\textsf{\fullmoon~पूर्णिमा} {\tiny \RIGHTarrow} 09:13\hspace{2ex}}{\textsf{मृगशीर्ष} {\tiny \RIGHTarrow} 11:31\hspace{2ex}}{14:51-15:52}{10:45-11:46} 
&
%मकर
\caldata{22}{\sunmonth{धनुः}{7}{}}{\sundata{08:43}{16:54}{10:21}}{\textsf{\krishna~द्वितीया} {\tiny \RIGHTarrow} 05:45(+1)}{\textsf{आर्द्रा} {\tiny \RIGHTarrow} 10:36\hspace{2ex}}{12:48-13:49}{09:44-10:45} 
&
%मकर
\caldata{23}{\sunmonth{धनुः}{8}{}}{\sundata{08:43}{16:55}{10:21}}{\textsf{\krishna~तृतीया} {\tiny \RIGHTarrow} 03:35(+1)}{\textsf{पुनर्वसू} {\tiny \RIGHTarrow} 09:20\hspace{2ex}}{13:50-14:52}{08:43-09:44} 
&
%मकर
\caldata{24}{\sunmonth{धनुः}{9}{}}{\sundata{08:44}{16:56}{10:22}}{\textsf{\krishna~चतुर्थी} {\tiny \RIGHTarrow} 01:17(+1)}{\textsf{आश्रेषा} {\tiny \RIGHTarrow} 06:11(+1)}{11:48-12:50}{14:53-15:54} 
&
%मकर
\caldata{25}{\sunmonth{धनुः}{10}{}}{\sundata{08:44}{16:56}{10:22}}{\textsf{\krishna~पञ्चमी} {\tiny \RIGHTarrow} 22:56\hspace{2ex}}{\textsf{मघा} {\tiny \RIGHTarrow} 04:32(+1)}{10:47-11:48}{13:51-14:53} 
\\ \hline
%मकर
\caldata{26}{\sunmonth{धनुः}{11}{}}{\sundata{08:44}{16:57}{10:22}}{\textsf{\krishna~षष्ठी} {\tiny \RIGHTarrow} 20:36\hspace{2ex}}{\textsf{पूर्वफल्गुनी} {\tiny \RIGHTarrow} 02:55(+1)}{15:55-16:57}{12:50-13:52} 
&
%मकर
\caldata{27}{\sunmonth{धनुः}{12}{}}{\sundata{08:45}{16:58}{10:23}}{\textsf{\krishna~सप्तमी} {\tiny \RIGHTarrow} 18:23\hspace{2ex}}{\textsf{उत्तरफल्गुनी} {\tiny \RIGHTarrow} 01:25(+1)}{09:46-10:48}{11:49-12:51} 
&
%मकर
\caldata{28}{\sunmonth{धनुः}{13}{}}{\sundata{08:45}{16:58}{10:23}}{\textsf{\krishna~अष्टमी} {\tiny \RIGHTarrow} 16:19\hspace{2ex}}{\textsf{हस्त} {\tiny \RIGHTarrow} 00:07(+1)}{14:54-15:56}{10:48-11:49} 
&
%मकर
\caldata{29}{\sunmonth{धनुः}{14}{}}{\sundata{08:45}{16:59}{10:23}}{\textsf{\krishna~नवमी} {\tiny \RIGHTarrow} 14:28\hspace{2ex}}{\textsf{चित्रा} {\tiny \RIGHTarrow} 23:01\hspace{2ex}}{12:52-13:53}{09:46-10:48} 
&
%मकर
\caldata{30}{\sunmonth{धनुः}{15}{}}{\sundata{08:45}{17:00}{10:24}}{\textsf{\krishna~दशमी} {\tiny \RIGHTarrow} 12:52\hspace{2ex}}{\textsf{स्वाति} {\tiny \RIGHTarrow} 22:12\hspace{2ex}}{13:54-14:56}{08:45-09:46} 
&
%मकर
\caldata{31}{\sunmonth{धनुः}{16}{}}{\sundata{08:45}{17:01}{10:24}}{\textsf{\krishna~एकादशी} {\tiny \RIGHTarrow} 11:33\hspace{2ex}}{\textsf{विशाखा} {\tiny \RIGHTarrow} 21:40\hspace{2ex}}{11:51-12:53}{14:57-15:59} 
&
%मकर
\\ \hline
\end{tabular}


%\clearpage

\end{center}
\end{document}
