\documentclass[a4paper,11pt,landscape]{article}
\usepackage[sort&compress,square,numbers]{natbib}

\usepackage[xetex]{graphicx}
%\usepackage{fullpage}
\usepackage{multirow}
\usepackage[normalsections]{savetrees}
\usepackage{euler}
\usepackage{fontspec}
\usepackage{xltxtra}
\usepackage{url}
\usepackage{multicol}
\usepackage{bbding}
% PDF SETUP
% ---- FILL IN HERE THE DOC TITLE AND AUTHOR
%\usepackage[bookmarks, colorlinks, breaklinks, pdftitle={Karthik Raman - vita},pdfauthor={Karthik Raman}]{hyperref} 
\usepackage[dvipsnames]{xcolor} 
\usepackage{wasysym} 
%\hypersetup{linkcolor=Sepia,citecolor=blue,filecolor=black,urlcolor=Blue} 


\defaultfontfeatures{Scale=MatchLowercase,Mapping=tex-text}
\setmainfont{Scala Sans LF}
\setsansfont{Sanskrit 2003:script=deva}

\newcommand{\sunmonth}[3]{%
}
\newcommand{\sundata}[3]{%
\mbox{{\sun\tiny\UParrow} \small #1}\\
\mbox{{\sun\tiny\DOWNarrow} \small  #2}\\
\scriptsize
\mbox{(\textsf{स} #3)}
}
\newcommand{\caldata}[7]{%
\begin{minipage}{2.6cm}
\begin{minipage}[t]{1.6cm}
\vspace{.2ex}
%\mbox{}\\
%\SunshineOpenCircled
#3\\
\mbox{#4}\\
\mbox{#5}\\
\mbox{\textsf{राहु~} #6}\\
\mbox{\textsf{यम~} #7}\\
\end{minipage}\begin{minipage}[c]{1.cm}
\vspace{.4ex}
\begin{flushright} \textcolor{blue}{\font\x="Plantin Std" at 24 pt\x #1}
\end{flushright}
\end{minipage}
\end{minipage}
}

\addtolength{\headsep}{-3ex}
\pagestyle{empty}
\newcommand{\ahoratram}{\textsf{अहोरात्रम्}}
\newcommand{\ashwini}{अश्विनी}
\newcommand{\apabharani}{अपभरणी}
\newcommand{\krittika}{कृत्तिका}
\newcommand{\rohini}{रोहिणी}
\newcommand{\mrigashirsha}{मृगशीर्ष}
\newcommand{\ardra}{आर्द्रा}
\newcommand{\punarvasu}{पुनर्वसू}
\newcommand{\pushya}{पुष्य}
\newcommand{\ashresha}{आश्रेषा}
\newcommand{\magha}{मघा}
\newcommand{\purvaphalguni}{पूर्वफल्गुनी}
\newcommand{\uttaraphalguni}{उत्तरफल्गुनी}
\newcommand{\hasta}{हस्त}
\newcommand{\chitra}{चित्रा}
\newcommand{\svati}{स्वाति}
\newcommand{\vishakha}{विशाखा}
\newcommand{\anuradha}{अनूराधा}
\newcommand{\jyeshtha}{ज्येष्ठा}
\newcommand{\mula}{मूला}
\newcommand{\purvashadha}{पूर्वाषाढा}
\newcommand{\uttarashadha}{उत्तराषाढा}
\newcommand{\shravana}{श्रवण}
\newcommand{\shravishtha}{श्रविष्ठा}
\newcommand{\shatabhishak}{शतभिषक्}
\newcommand{\proshthapada}{प्रोष्ठपदा}
\newcommand{\uttaraproshthapada}{उत्तरप्रोष्ठपदा}
\newcommand{\revati}{रेवती}

\newcommand{\spra}{शुक्ल प्रथमा}
\newcommand{\sdvi}{शुक्ल द्वितीया}
\newcommand{\stri}{शुक्ल तृतीया}
\newcommand{\scha}{शुक्ल चतुर्थी}
\newcommand{\spanc}{शुक्ल पञ्चमी}
\newcommand{\ssha}{शुक्ल षष्ठी}
\newcommand{\ssap}{शुक्ल सप्तमी}
\newcommand{\sasht}{शुक्ल अष्टमी}
\newcommand{\snav}{शुक्ल नवमी}
\newcommand{\sdas}{शुक्ल दशमी}
\newcommand{\seka}{शुक्ल एकादशी}
\newcommand{\sdva}{शुक्ल द्वादशी}
\newcommand{\stra}{शुक्ल त्रयोदशी}
\newcommand{\schaturdashi}{शुक्ल चतुर्दशी}
\newcommand{\purnima}{\fullmoon~पूर्णिमा}
\newcommand{\kpra}{कृष्ण प्रथमा}
\newcommand{\kdvi}{कृष्ण द्वितीया}
\newcommand{\ktri}{कृष्ण तृतीया}
\newcommand{\kcha}{कृष्ण चतुर्थी}
\newcommand{\kpanc}{कृष्ण पञ्चमी}
\newcommand{\ksha}{कृष्ण षष्ठी}
\newcommand{\ksap}{कृष्ण सप्तमी}
\newcommand{\kasht}{कृष्ण अष्टमी}
\newcommand{\knav}{कृष्ण नवमी}
\newcommand{\kdas}{कृष्ण दशमी}
\newcommand{\keka}{कृष्ण एकादशी}
\newcommand{\kdva}{कृष्ण द्वादशी}
\newcommand{\ktra}{कृष्ण त्रयोदशी}
\newcommand{\kchaturdashi}{कृष्ण चतुर्दशी}
\newcommand{\ama}{\newmoon~अमावस्या}
\begin{document}
\pagestyle{empty}
\begin{center}
\mbox{}\\[2.5in]
\hrule\mbox{}
\mbox{}\\[1ex]
\mbox{}
{\font\x="Warnock Pro" at 60 pt\x 2010\\[0.5cm]}
\mbox{}
{\font\x="Warnock Pro" at 48 pt\x \uppercase{Mumbai}\\[0.3cm]}
\hrule
\begin{tabular}{|c|c|c|c|c|c|c|}
\multicolumn{7}{c}{\Large \bfseries JANUARY 2010}\\
\hline
\textbf{SUN} & \textbf{MON} & \textbf{TUE} & \textbf{WED} & \textbf{THU} & \textbf{FRI} & \textbf{SAT} \\ \hline
{}  &
{}  &
{}  &
{}  &
{}  &
\caldata{1}{07:12}{09:23}{11:19-12:41}{15:26-16:48}{18:11}{\textsf{\kpra} {\tiny \RIGHTarrow} 21:13\hspace{2ex}}{\textsf{\punarvasu} {\tiny \RIGHTarrow} 03:55(+1)} 
&
\caldata{2}{07:13}{09:24}{09:57-11:20}{14:04-15:27}{18:12}{\textsf{\kdvi} {\tiny \RIGHTarrow} 17:44\hspace{2ex}}{\textsf{\pushya} {\tiny \RIGHTarrow} 01:12(+1)} 
\\ \hline
\caldata{3}{07:13}{09:24}{16:49-18:12}{12:42-14:04}{18:12}{\textsf{\ktri} {\tiny \RIGHTarrow} 14:22\hspace{2ex}}{\textsf{\ashresha} {\tiny \RIGHTarrow} 22:40\hspace{2ex}} 
&
\caldata{4}{07:13}{09:25}{08:35-09:58}{11:20-12:43}{18:13}{\textsf{\kcha} {\tiny \RIGHTarrow} 11:16\hspace{2ex}}{\textsf{\magha} {\tiny \RIGHTarrow} 20:27\hspace{2ex}} 
&
\caldata{5}{07:14}{09:26}{15:29-16:51}{09:59-11:21}{18:14}{\textsf{\kpanc} {\tiny \RIGHTarrow} 08:32\hspace{2ex}}{\textsf{\purvaphalguni} {\tiny \RIGHTarrow} 18:40\hspace{2ex}} 
&
\caldata{6}{07:14}{09:26}{12:44-14:06}{08:36-09:59}{18:14}{\textsf{\ksap} {\tiny \RIGHTarrow} 04:45(+1)}{\textsf{\uttaraphalguni} {\tiny \RIGHTarrow} 17:25\hspace{2ex}} 
&
\caldata{7}{07:14}{09:26}{14:07-15:29}{07:14-08:36}{18:15}{\textsf{\kasht} {\tiny \RIGHTarrow} 03:47(+1)}{\textsf{\hasta} {\tiny \RIGHTarrow} 16:47\hspace{2ex}} 
&
\caldata{8}{07:14}{09:26}{11:21-12:44}{15:29-16:52}{18:15}{\textsf{\knav} {\tiny \RIGHTarrow} 03:29(+1)}{\textsf{\chitra} {\tiny \RIGHTarrow} 16:47\hspace{2ex}} 
&
\caldata{9}{07:14}{09:26}{09:59-11:22}{14:07-15:30}{18:16}{\textsf{\kdas} {\tiny \RIGHTarrow} 03:48(+1)}{\textsf{\svati} {\tiny \RIGHTarrow} 17:24\hspace{2ex}} 
\\ \hline
\caldata{10}{07:15}{09:27}{16:54-18:17}{12:46-14:08}{18:17}{\textsf{\keka} {\tiny \RIGHTarrow} 04:42(+1)}{\textsf{\vishakha} {\tiny \RIGHTarrow} 18:36\hspace{2ex}} 
&
\caldata{11}{07:15}{09:27}{08:37-10:00}{11:23-12:46}{18:17}{\textsf{\kdva} {\tiny \RIGHTarrow} 06:07(+1)}{\textsf{\anuradha} {\tiny \RIGHTarrow} 20:20\hspace{2ex}} 
&
\caldata{12}{07:15}{09:27}{15:32-16:55}{10:00-11:23}{18:18}{\textsf{\ktra} {\tiny \RIGHTarrow} 07:58(+1)}{\textsf{\jyeshtha} {\tiny \RIGHTarrow} 22:31\hspace{2ex}} 
&
\caldata{13}{07:15}{09:27}{12:47-14:10}{08:38-10:01}{18:19}{\textsf{\ktra} {\tiny \RIGHTarrow} 07:59\hspace{2ex}}{\textsf{\mula} {\tiny \RIGHTarrow} 01:03(+1)} 
&
\caldata{14}{07:15}{09:27}{14:10-15:33}{07:15-08:38}{18:19}{\textsf{\kchaturdashi} {\tiny \RIGHTarrow} 10:12\hspace{2ex}}{\textsf{\purvashadha} {\tiny \RIGHTarrow} 03:52(+1)} 
&
\caldata{15}{07:15}{09:28}{11:24-12:47}{15:33-16:56}{18:20}{\textsf{\ama} {\tiny \RIGHTarrow} 12:42\hspace{2ex}}{\textsf{\uttarashadha} {\tiny \RIGHTarrow} 06:53(+1)} 
&
\caldata{16}{07:15}{09:28}{10:01-11:24}{14:11-15:34}{18:21}{\textsf{\spra} {\tiny \RIGHTarrow} 15:21\hspace{2ex}}{\textsf{\shravana} {\tiny \RIGHTarrow} 10:01(+1)} 
\\ \hline
\caldata{17}{07:15}{09:28}{16:57-18:21}{12:48-14:11}{18:21}{\textsf{\sdvi} {\tiny \RIGHTarrow} 18:04\hspace{2ex}}{\textsf{\shravana} {\tiny \RIGHTarrow} 10:01\hspace{2ex}} 
&
\caldata{18}{07:15}{09:28}{08:38-10:01}{11:25-12:48}{18:22}{\textsf{\stri} {\tiny \RIGHTarrow} 20:44\hspace{2ex}}{\textsf{\shravishtha} {\tiny \RIGHTarrow} 13:07\hspace{2ex}} 
&
\caldata{19}{07:15}{09:28}{15:36-16:59}{10:02-11:25}{18:23}{\textsf{\scha} {\tiny \RIGHTarrow} 23:12\hspace{2ex}}{\textsf{\shatabhishak} {\tiny \RIGHTarrow} 16:04\hspace{2ex}} 
&
\caldata{20}{07:15}{09:28}{12:49-14:12}{08:38-10:02}{18:23}{\textsf{\spanc} {\tiny \RIGHTarrow} 06:41\hspace{2ex}}{\textsf{\proshthapada} {\tiny \RIGHTarrow} 06:51\hspace{2ex}} 
&
\caldata{21}{07:15}{09:28}{14:13-15:36}{07:15-08:38}{18:24}{\textsf{\ssha} {\tiny \RIGHTarrow} 03:02(+1)}{\textsf{\uttaraproshthapada} {\tiny \RIGHTarrow} 21:01\hspace{2ex}} 
&
\caldata{22}{07:15}{09:28}{11:25-12:49}{15:36-17:00}{18:24}{\textsf{\ssap} {\tiny \RIGHTarrow} 04:05(+1)}{\textsf{\revati} {\tiny \RIGHTarrow} 22:43\hspace{2ex}} 
&
\caldata{23}{07:15}{09:29}{10:02-11:26}{14:13-15:37}{18:25}{\textsf{\sasht} {\tiny \RIGHTarrow} 04:25(+1)}{\textsf{\ashwini} {\tiny \RIGHTarrow} 23:45\hspace{2ex}} 
\\ \hline
\caldata{24}{07:15}{09:29}{17:02-18:26}{12:50-14:14}{18:26}{\textsf{\snav} {\tiny \RIGHTarrow} 03:57(+1)}{\textsf{\apabharani} {\tiny \RIGHTarrow} 00:03(+1)} 
&
\caldata{25}{07:15}{09:29}{08:38-10:02}{11:26-12:50}{18:26}{\textsf{\sdas} {\tiny \RIGHTarrow} 02:43(+1)}{\textsf{\krittika} {\tiny \RIGHTarrow} 23:34\hspace{2ex}} 
&
\caldata{26}{07:15}{09:29}{15:39-17:03}{10:03-11:27}{18:27}{\textsf{\seka} {\tiny \RIGHTarrow} 00:44(+1)}{\textsf{\rohini} {\tiny \RIGHTarrow} 22:22\hspace{2ex}} 
&
\caldata{27}{07:15}{09:29}{12:51-14:15}{08:39-10:03}{18:27}{\textsf{\sdva} {\tiny \RIGHTarrow} 22:07\hspace{2ex}}{\textsf{\mrigashirsha} {\tiny \RIGHTarrow} 20:31\hspace{2ex}} 
&
\caldata{28}{07:14}{09:28}{14:15-15:39}{07:14-08:38}{18:28}{\textsf{\stra} {\tiny \RIGHTarrow} 19:00\hspace{2ex}}{\textsf{\ardra} {\tiny \RIGHTarrow} 18:09\hspace{2ex}} 
&
\caldata{29}{07:14}{09:29}{11:27-12:51}{15:40-17:04}{18:29}{\textsf{\schaturdashi} {\tiny \RIGHTarrow} 15:30\hspace{2ex}}{\textsf{\punarvasu} {\tiny \RIGHTarrow} 15:25\hspace{2ex}} 
&
\caldata{30}{07:14}{09:29}{10:02-11:27}{14:15-15:40}{18:29}{\textsf{\purnima} {\tiny \RIGHTarrow} 11:47\hspace{2ex}}{\textsf{\pushya} {\tiny \RIGHTarrow} 12:27\hspace{2ex}} 
\\ \hline
\caldata{31}{07:14}{09:29}{17:05-18:30}{12:52-14:16}{18:30}{\textsf{\kpra} {\tiny \RIGHTarrow} 08:00\hspace{2ex}}{\textsf{\ashresha} {\tiny \RIGHTarrow} 09:25\hspace{2ex}} 
&
{}  &
{}  &
{}  &
{}  &
{}  &
\\ \hline
\end{tabular}


%\clearpage
\begin{tabular}{|c|c|c|c|c|c|c|}
\multicolumn{7}{c}{\Large \bfseries FEBRUARY 2010}\\
\hline
\textbf{SUN} & \textbf{MON} & \textbf{TUE} & \textbf{WED} & \textbf{THU} & \textbf{FRI} & \textbf{SAT} \\ \hline
{}  &
\caldata{1}{07:13}{09:28}{08:37-10:02}{11:26-12:51}{18:30}{\textsf{\ktri} {\tiny \RIGHTarrow} 01:00(+1)}{\textsf{\purvaphalguni} {\tiny \RIGHTarrow} 03:57(+1)} 
&
\caldata{2}{07:13}{09:28}{15:41-17:06}{10:02-11:27}{18:31}{\textsf{\kcha} {\tiny \RIGHTarrow} 22:07\hspace{2ex}}{\textsf{\uttaraphalguni} {\tiny \RIGHTarrow} 01:53(+1)} 
&
\caldata{3}{07:13}{09:28}{12:52-14:16}{08:37-10:02}{18:31}{\textsf{\kpanc} {\tiny \RIGHTarrow} 19:50\hspace{2ex}}{\textsf{\hasta} {\tiny \RIGHTarrow} 00:25(+1)} 
&
\caldata{4}{07:12}{09:28}{14:17-15:42}{07:12-08:37}{18:32}{\textsf{\ksha} {\tiny \RIGHTarrow} 18:16\hspace{2ex}}{\textsf{\chitra} {\tiny \RIGHTarrow} 23:41\hspace{2ex}} 
&
\caldata{5}{07:12}{09:28}{11:27-12:52}{15:42-17:07}{18:32}{\textsf{\ksap} {\tiny \RIGHTarrow} 17:29\hspace{2ex}}{\textsf{\svati} {\tiny \RIGHTarrow} 23:44\hspace{2ex}} 
&
\caldata{6}{07:12}{09:28}{10:02-11:27}{14:17-15:42}{18:33}{\textsf{\kasht} {\tiny \RIGHTarrow} 17:31\hspace{2ex}}{\textsf{\vishakha} {\tiny \RIGHTarrow} 00:33(+1)} 
\\ \hline
\caldata{7}{07:11}{09:27}{17:08-18:34}{12:52-14:17}{18:34}{\textsf{\knav} {\tiny \RIGHTarrow} 18:20\hspace{2ex}}{\textsf{\anuradha} {\tiny \RIGHTarrow} 02:07(+1)} 
&
\caldata{8}{07:11}{09:27}{08:36-10:01}{11:27-12:52}{18:34}{\textsf{\kdas} {\tiny \RIGHTarrow} 19:50\hspace{2ex}}{\textsf{\jyeshtha} {\tiny \RIGHTarrow} 04:17(+1)} 
&
\caldata{9}{07:10}{09:27}{15:43-17:09}{10:01-11:26}{18:35}{\textsf{\keka} {\tiny \RIGHTarrow} 21:53\hspace{2ex}}{\textsf{\mula} {\tiny \RIGHTarrow} 06:56(+1)} 
&
\caldata{10}{07:10}{09:27}{12:52-14:18}{08:35-10:01}{18:35}{\textsf{\kdva} {\tiny \RIGHTarrow} 00:18(+1)}{\textsf{\purvashadha} {\tiny \RIGHTarrow} 09:54(+1)} 
&
\caldata{11}{07:10}{09:27}{14:18-15:44}{07:10-08:35}{18:36}{\textsf{\ktra} {\tiny \RIGHTarrow} 02:57(+1)}{\textsf{\purvashadha} {\tiny \RIGHTarrow} 09:54\hspace{2ex}} 
&
\caldata{12}{07:09}{09:26}{11:26-12:52}{15:44-17:10}{18:36}{\textsf{\kchaturdashi} {\tiny \RIGHTarrow} 05:41(+1)}{\textsf{\uttarashadha} {\tiny \RIGHTarrow} 13:02\hspace{2ex}} 
&
\caldata{13}{07:09}{09:26}{10:00-11:26}{14:18-15:44}{18:36}{\textsf{\ama} {\tiny \RIGHTarrow} 08:21(+1)}{\textsf{\shravana} {\tiny \RIGHTarrow} 16:10\hspace{2ex}} 
\\ \hline
\caldata{14}{07:08}{09:25}{17:10-18:37}{12:52-14:18}{18:37}{\textsf{\ama} {\tiny \RIGHTarrow} 08:21\hspace{2ex}}{\textsf{\shravishtha} {\tiny \RIGHTarrow} 19:12\hspace{2ex}} 
&
\caldata{15}{07:08}{09:25}{08:34-10:00}{11:26-12:52}{18:37}{\textsf{\spra} {\tiny \RIGHTarrow} 10:51\hspace{2ex}}{\textsf{\shatabhishak} {\tiny \RIGHTarrow} 22:04\hspace{2ex}} 
&
\caldata{16}{07:07}{09:25}{15:45-17:11}{09:59-11:26}{18:38}{\textsf{\sdvi} {\tiny \RIGHTarrow} 06:56\hspace{2ex}}{\textsf{\proshthapada} {\tiny \RIGHTarrow} 06:30\hspace{2ex}} 
&
\caldata{17}{07:06}{09:24}{12:52-14:18}{08:32-09:59}{18:38}{\textsf{\stri} {\tiny \RIGHTarrow} 15:04\hspace{2ex}}{\textsf{\uttaraproshthapada} {\tiny \RIGHTarrow} 02:59(+1)} 
&
\caldata{18}{07:06}{09:24}{14:19-15:45}{07:06-08:32}{18:39}{\textsf{\scha} {\tiny \RIGHTarrow} 16:39\hspace{2ex}}{\textsf{\revati} {\tiny \RIGHTarrow} 04:53(+1)} 
&
\caldata{19}{07:05}{09:23}{11:25-12:52}{15:45-17:12}{18:39}{\textsf{\spanc} {\tiny \RIGHTarrow} 17:46\hspace{2ex}}{\textsf{\ashwini} {\tiny \RIGHTarrow} 06:19(+1)} 
&
\caldata{20}{07:05}{09:23}{09:58-11:25}{14:18-15:45}{18:39}{\textsf{\ssha} {\tiny \RIGHTarrow} 18:22\hspace{2ex}}{\textsf{\apabharani} {\tiny \RIGHTarrow} 07:13(+1)} 
\\ \hline
\caldata{21}{07:04}{09:23}{17:13-18:40}{12:52-14:19}{18:40}{\textsf{\ssap} {\tiny \RIGHTarrow} 18:21\hspace{2ex}}{\textsf{\apabharani} {\tiny \RIGHTarrow} 07:13\hspace{2ex}} 
&
\caldata{22}{07:03}{09:22}{08:30-09:57}{11:24-12:51}{18:40}{\textsf{\sasht} {\tiny \RIGHTarrow} 17:42\hspace{2ex}}{\textsf{\krittika} {\tiny \RIGHTarrow} 07:29\hspace{2ex}} 
&
\caldata{23}{07:03}{09:22}{15:46-17:13}{09:57-11:24}{18:41}{\textsf{\snav} {\tiny \RIGHTarrow} 16:24\hspace{2ex}}{\textsf{\rohini} {\tiny \RIGHTarrow} 07:07\hspace{2ex}} 
&
\caldata{24}{07:02}{09:21}{12:51-14:18}{08:29-09:56}{18:41}{\textsf{\sdas} {\tiny \RIGHTarrow} 14:27\hspace{2ex}}{\textsf{\ardra} {\tiny \RIGHTarrow} 04:27(+1)} 
&
\caldata{25}{07:01}{09:21}{14:18-15:46}{07:01-08:28}{18:41}{\textsf{\seka} {\tiny \RIGHTarrow} 11:55\hspace{2ex}}{\textsf{\punarvasu} {\tiny \RIGHTarrow} 02:16(+1)} 
&
\caldata{26}{07:01}{09:21}{11:23-12:51}{15:46-17:14}{18:42}{\textsf{\sdva} {\tiny \RIGHTarrow} 08:54\hspace{2ex}}{\textsf{\pushya} {\tiny \RIGHTarrow} 23:42\hspace{2ex}} 
&
\caldata{27}{07:00}{09:20}{09:55-11:23}{14:18-15:46}{18:42}{\textsf{\schaturdashi} {\tiny \RIGHTarrow} 01:50(+1)}{\textsf{\ashresha} {\tiny \RIGHTarrow} 20:52\hspace{2ex}} 
\\ \hline
\caldata{28}{06:59}{09:19}{17:14-18:42}{12:50-14:18}{18:42}{\textsf{\purnima} {\tiny \RIGHTarrow} 22:08\hspace{2ex}}{\textsf{\magha} {\tiny \RIGHTarrow} 17:57\hspace{2ex}} 
&
{}  &
{}  &
{}  &
{}  &
{}  &
\\ \hline
\end{tabular}


%\clearpage
\begin{tabular}{|c|c|c|c|c|c|c|}
\multicolumn{7}{c}{\Large \bfseries MARCH 2010}\\
\hline
\textbf{SUN} & \textbf{MON} & \textbf{TUE} & \textbf{WED} & \textbf{THU} & \textbf{FRI} & \textbf{SAT} \\ \hline
{}  &
\caldata{1}{06:59}{09:19}{08:27-09:55}{11:23-12:51}{18:43}{\textsf{\kpra} {\tiny \RIGHTarrow} 18:35\hspace{2ex}}{\textsf{\purvaphalguni} {\tiny \RIGHTarrow} 15:07\hspace{2ex}} 
&
\caldata{2}{06:58}{09:19}{15:46-17:14}{09:54-11:22}{18:43}{\textsf{\kdvi} {\tiny \RIGHTarrow} 15:18\hspace{2ex}}{\textsf{\uttaraphalguni} {\tiny \RIGHTarrow} 12:31\hspace{2ex}} 
&
\caldata{3}{06:57}{09:18}{12:50-14:18}{08:25-09:53}{18:43}{\textsf{\ktri} {\tiny \RIGHTarrow} 12:29\hspace{2ex}}{\textsf{\hasta} {\tiny \RIGHTarrow} 10:23\hspace{2ex}} 
&
\caldata{4}{06:56}{09:17}{14:18-15:47}{06:56-08:24}{18:44}{\textsf{\kcha} {\tiny \RIGHTarrow} 10:18\hspace{2ex}}{\textsf{\chitra} {\tiny \RIGHTarrow} 08:51\hspace{2ex}} 
&
\caldata{5}{06:56}{09:17}{11:21-12:50}{15:47-17:15}{18:44}{\textsf{\kpanc} {\tiny \RIGHTarrow} 08:53\hspace{2ex}}{\textsf{\svati} {\tiny \RIGHTarrow} 08:05\hspace{2ex}} 
&
\caldata{6}{06:55}{09:16}{09:52-11:20}{14:18-15:46}{18:44}{\textsf{\ksha} {\tiny \RIGHTarrow} 08:21\hspace{2ex}}{\textsf{\vishakha} {\tiny \RIGHTarrow} 08:11\hspace{2ex}} 
\\ \hline
\caldata{7}{06:54}{09:16}{17:16-18:45}{12:49-14:18}{18:45}{\textsf{\ksap} {\tiny \RIGHTarrow} 08:43\hspace{2ex}}{\textsf{\anuradha} {\tiny \RIGHTarrow} 09:09\hspace{2ex}} 
&
\caldata{8}{06:53}{09:15}{08:22-09:51}{11:20-12:49}{18:45}{\textsf{\kasht} {\tiny \RIGHTarrow} 09:55\hspace{2ex}}{\textsf{\jyeshtha} {\tiny \RIGHTarrow} 10:56\hspace{2ex}} 
&
\caldata{9}{06:53}{09:15}{15:47-17:16}{09:51-11:20}{18:45}{\textsf{\knav} {\tiny \RIGHTarrow} 11:49\hspace{2ex}}{\textsf{\mula} {\tiny \RIGHTarrow} 13:21\hspace{2ex}} 
&
\caldata{10}{06:52}{09:14}{12:49-14:18}{08:21-09:50}{18:46}{\textsf{\kdas} {\tiny \RIGHTarrow} 14:12\hspace{2ex}}{\textsf{\purvashadha} {\tiny \RIGHTarrow} 16:14\hspace{2ex}} 
&
\caldata{11}{06:51}{09:14}{14:17-15:47}{06:51-08:20}{18:46}{\textsf{\keka} {\tiny \RIGHTarrow} 16:52\hspace{2ex}}{\textsf{\uttarashadha} {\tiny \RIGHTarrow} 19:21\hspace{2ex}} 
&
\caldata{12}{06:50}{09:13}{11:18-12:48}{15:47-17:16}{18:46}{\textsf{\kdva} {\tiny \RIGHTarrow} 19:34\hspace{2ex}}{\textsf{\shravana} {\tiny \RIGHTarrow} 22:31\hspace{2ex}} 
&
\caldata{13}{06:49}{09:12}{09:48-11:17}{14:17-15:46}{18:46}{\textsf{\ktra} {\tiny \RIGHTarrow} 22:10\hspace{2ex}}{\textsf{\shravishtha} {\tiny \RIGHTarrow} 01:33(+1)} 
\\ \hline
\caldata{14}{06:49}{09:12}{17:17-18:47}{12:48-14:17}{18:47}{\textsf{\kchaturdashi} {\tiny \RIGHTarrow} 00:30(+1)}{\textsf{\shatabhishak} {\tiny \RIGHTarrow} 04:20(+1)} 
&
\caldata{15}{06:48}{09:11}{08:17-09:47}{11:17-12:47}{18:47}{\textsf{\ama} {\tiny \RIGHTarrow} 02:29(+1)}{\textsf{\proshthapada} {\tiny \RIGHTarrow} 06:47(+1)} 
&
\caldata{16}{06:47}{09:11}{15:47-17:17}{09:47-11:17}{18:47}{\textsf{\spra} {\tiny \RIGHTarrow} 06:06\hspace{2ex}}{\textsf{\uttaraproshthapada} {\tiny \RIGHTarrow} 05:52\hspace{2ex}} 
&
\caldata{17}{06:46}{09:10}{12:46-14:16}{08:16-09:46}{18:47}{\textsf{\sdvi} {\tiny \RIGHTarrow} 05:19(+1)}{\textsf{\uttaraproshthapada} {\tiny \RIGHTarrow} 08:50\hspace{2ex}} 
&
\caldata{18}{06:45}{09:09}{14:16-15:47}{06:45-08:15}{18:48}{\textsf{\stri} {\tiny \RIGHTarrow} 06:08(+1)}{\textsf{\revati} {\tiny \RIGHTarrow} 10:31\hspace{2ex}} 
&
\caldata{19}{06:44}{09:08}{11:15-12:46}{15:47-17:17}{18:48}{\textsf{\scha} {\tiny \RIGHTarrow} 06:33(+1)}{\textsf{\ashwini} {\tiny \RIGHTarrow} 11:49\hspace{2ex}} 
&
\caldata{20}{06:44}{09:08}{09:45-11:15}{14:16-15:47}{18:48}{\textsf{\spanc} {\tiny \RIGHTarrow} 07:30\hspace{2ex}}{\textsf{\apabharani} {\tiny \RIGHTarrow} 12:44\hspace{2ex}} 
\\ \hline
\caldata{21}{06:43}{09:08}{17:17-18:48}{12:45-14:16}{18:48}{\textsf{\ssha} {\tiny \RIGHTarrow} 06:04(+1)}{\textsf{\krittika} {\tiny \RIGHTarrow} 13:13\hspace{2ex}} 
&
\caldata{22}{06:42}{09:07}{08:12-09:43}{11:14-12:45}{18:49}{\textsf{\ssap} {\tiny \RIGHTarrow} 05:07(+1)}{\textsf{\rohini} {\tiny \RIGHTarrow} 13:17\hspace{2ex}} 
&
\caldata{23}{06:41}{09:06}{15:47-17:18}{09:43-11:14}{18:49}{\textsf{\sasht} {\tiny \RIGHTarrow} 03:41(+1)}{\textsf{\mrigashirsha} {\tiny \RIGHTarrow} 12:52\hspace{2ex}} 
&
\caldata{24}{06:40}{09:05}{12:44-14:15}{08:11-09:42}{18:49}{\textsf{\snav} {\tiny \RIGHTarrow} 01:46(+1)}{\textsf{\ardra} {\tiny \RIGHTarrow} 11:59\hspace{2ex}} 
&
\caldata{25}{06:39}{09:05}{14:15-15:46}{06:39-08:10}{18:49}{\textsf{\sdas} {\tiny \RIGHTarrow} 23:23\hspace{2ex}}{\textsf{\punarvasu} {\tiny \RIGHTarrow} 10:38\hspace{2ex}} 
&
\caldata{26}{06:39}{09:05}{11:13-12:44}{15:47-17:18}{18:50}{\textsf{\seka} {\tiny \RIGHTarrow} 20:38\hspace{2ex}}{\textsf{\pushya} {\tiny \RIGHTarrow} 08:51\hspace{2ex}} 
&
\caldata{27}{06:38}{09:04}{09:41-11:12}{14:15-15:47}{18:50}{\textsf{\sdva} {\tiny \RIGHTarrow} 17:34\hspace{2ex}}{\textsf{\ashresha} {\tiny \RIGHTarrow} 06:42\hspace{2ex}} 
\\ \hline
\caldata{28}{06:37}{09:03}{17:18-18:50}{12:43-14:15}{18:50}{\textsf{\stra} {\tiny \RIGHTarrow} 14:21\hspace{2ex}}{\textsf{\purvaphalguni} {\tiny \RIGHTarrow} 01:47(+1)} 
&
\caldata{29}{06:36}{09:02}{08:07-09:39}{11:11-12:43}{18:50}{\textsf{\schaturdashi} {\tiny \RIGHTarrow} 11:05\hspace{2ex}}{\textsf{\uttaraphalguni} {\tiny \RIGHTarrow} 23:21\hspace{2ex}} 
&
\caldata{30}{06:35}{09:02}{15:47-17:19}{09:39-11:11}{18:51}{\textsf{\purnima} {\tiny \RIGHTarrow} 07:56\hspace{2ex}}{\textsf{\hasta} {\tiny \RIGHTarrow} 21:10\hspace{2ex}} 
&
\caldata{31}{06:34}{09:01}{12:42-14:14}{08:06-09:38}{18:51}{\textsf{\kdvi} {\tiny \RIGHTarrow} 02:45(+1)}{\textsf{\chitra} {\tiny \RIGHTarrow} 19:23\hspace{2ex}} 
&
{}  &
{}  &
\\ \hline
\end{tabular}


%\clearpage
\begin{tabular}{|c|c|c|c|c|c|c|}
\multicolumn{7}{c}{\Large \bfseries APRIL 2010}\\
\hline
\textbf{SUN} & \textbf{MON} & \textbf{TUE} & \textbf{WED} & \textbf{THU} & \textbf{FRI} & \textbf{SAT} \\ \hline
{}  &
{}  &
{}  &
{}  &
\caldata{1}{06:34}{09:01}{14:14-15:46}{06:34-08:06}{18:51}{\textsf{\ktri} {\tiny \RIGHTarrow} 01:03(+1)}{\textsf{\svati} {\tiny \RIGHTarrow} 18:10\hspace{2ex}} 
&
\caldata{2}{06:33}{09:00}{11:09-12:42}{15:46-17:18}{18:51}{\textsf{\kcha} {\tiny \RIGHTarrow} 00:05(+1)}{\textsf{\vishakha} {\tiny \RIGHTarrow} 17:40\hspace{2ex}} 
&
\caldata{3}{06:32}{09:00}{09:37-11:09}{14:14-15:47}{18:52}{\textsf{\kpanc} {\tiny \RIGHTarrow} 23:58\hspace{2ex}}{\textsf{\anuradha} {\tiny \RIGHTarrow} 17:59\hspace{2ex}} 
\\ \hline
\caldata{4}{06:31}{08:59}{17:19-18:52}{12:41-14:14}{18:52}{\textsf{\ksha} {\tiny \RIGHTarrow} 00:41(+1)}{\textsf{\jyeshtha} {\tiny \RIGHTarrow} 19:07\hspace{2ex}} 
&
\caldata{5}{06:30}{08:58}{08:02-09:35}{11:08-12:41}{18:52}{\textsf{\ksap} {\tiny \RIGHTarrow} 02:09(+1)}{\textsf{\mula} {\tiny \RIGHTarrow} 21:01\hspace{2ex}} 
&
\caldata{6}{06:29}{08:57}{15:46-17:19}{09:34-11:07}{18:52}{\textsf{\kasht} {\tiny \RIGHTarrow} 04:15(+1)}{\textsf{\purvashadha} {\tiny \RIGHTarrow} 23:32\hspace{2ex}} 
&
\caldata{7}{06:29}{08:57}{12:41-14:14}{08:02-09:35}{18:53}{\textsf{\knav} {\tiny \RIGHTarrow} 06:44(+1)}{\textsf{\uttarashadha} {\tiny \RIGHTarrow} 02:28(+1)} 
&
\caldata{8}{06:28}{08:57}{14:13-15:46}{06:28-08:01}{18:53}{\textsf{\knav} {\tiny \RIGHTarrow} 06:44\hspace{2ex}}{\textsf{\shravana} {\tiny \RIGHTarrow} 05:35(+1)} 
&
\caldata{9}{06:27}{08:56}{11:06-12:40}{15:46-17:19}{18:53}{\textsf{\kdas} {\tiny \RIGHTarrow} 09:22\hspace{2ex}}{\textsf{\shravishtha} {\tiny \RIGHTarrow} 08:39(+1)} 
&
\caldata{10}{06:26}{08:55}{09:32-11:06}{14:12-15:46}{18:53}{\textsf{\keka} {\tiny \RIGHTarrow} 11:52\hspace{2ex}}{\textsf{\shravishtha} {\tiny \RIGHTarrow} 08:37\hspace{2ex}} 
\\ \hline
\caldata{11}{06:25}{08:54}{17:20-18:54}{12:39-14:13}{18:54}{\textsf{\kdva} {\tiny \RIGHTarrow} 14:04\hspace{2ex}}{\textsf{\shatabhishak} {\tiny \RIGHTarrow} 11:23\hspace{2ex}} 
&
\caldata{12}{06:25}{08:54}{07:58-09:32}{11:05-12:39}{18:54}{\textsf{\ktra} {\tiny \RIGHTarrow} 06:06\hspace{2ex}}{\textsf{\proshthapada} {\tiny \RIGHTarrow} 06:09\hspace{2ex}} 
&
\caldata{13}{06:24}{08:54}{15:46-17:20}{09:31-11:05}{18:54}{\textsf{\kchaturdashi} {\tiny \RIGHTarrow} 17:08\hspace{2ex}}{\textsf{\uttaraproshthapada} {\tiny \RIGHTarrow} 15:39\hspace{2ex}} 
&
\caldata{14}{06:23}{08:53}{12:38-14:12}{07:56-09:30}{18:54}{\textsf{\ama} {\tiny \RIGHTarrow} 17:55\hspace{2ex}}{\textsf{\revati} {\tiny \RIGHTarrow} 17:04\hspace{2ex}} 
&
\caldata{15}{06:22}{08:52}{14:12-15:46}{06:22-07:56}{18:55}{\textsf{\spra} {\tiny \RIGHTarrow} 18:15\hspace{2ex}}{\textsf{\ashwini} {\tiny \RIGHTarrow} 18:02\hspace{2ex}} 
&
\caldata{16}{06:22}{08:52}{11:04-12:38}{15:46-17:20}{18:55}{\textsf{\sdvi} {\tiny \RIGHTarrow} 18:09\hspace{2ex}}{\textsf{\apabharani} {\tiny \RIGHTarrow} 18:36\hspace{2ex}} 
&
\caldata{17}{06:21}{08:51}{09:29-11:03}{14:12-15:46}{18:55}{\textsf{\stri} {\tiny \RIGHTarrow} 17:41\hspace{2ex}}{\textsf{\krittika} {\tiny \RIGHTarrow} 18:49\hspace{2ex}} 
\\ \hline
\caldata{18}{06:20}{08:51}{17:20-18:55}{12:37-14:11}{18:55}{\textsf{\scha} {\tiny \RIGHTarrow} 16:54\hspace{2ex}}{\textsf{\rohini} {\tiny \RIGHTarrow} 18:41\hspace{2ex}} 
&
\caldata{19}{06:19}{08:50}{07:53-09:28}{11:02-12:37}{18:56}{\textsf{\spanc} {\tiny \RIGHTarrow} 15:48\hspace{2ex}}{\textsf{\mrigashirsha} {\tiny \RIGHTarrow} 18:17\hspace{2ex}} 
&
\caldata{20}{06:19}{08:50}{15:46-17:21}{09:28-11:02}{18:56}{\textsf{\ssha} {\tiny \RIGHTarrow} 14:25\hspace{2ex}}{\textsf{\ardra} {\tiny \RIGHTarrow} 17:34\hspace{2ex}} 
&
\caldata{21}{06:18}{08:49}{12:37-14:11}{07:52-09:27}{18:56}{\textsf{\ssap} {\tiny \RIGHTarrow} 12:44\hspace{2ex}}{\textsf{\punarvasu} {\tiny \RIGHTarrow} 16:35\hspace{2ex}} 
&
\caldata{22}{06:17}{08:49}{14:12-15:47}{06:17-07:52}{18:57}{\textsf{\sasht} {\tiny \RIGHTarrow} 10:47\hspace{2ex}}{\textsf{\pushya} {\tiny \RIGHTarrow} 15:20\hspace{2ex}} 
&
\caldata{23}{06:17}{08:49}{11:02-12:37}{15:47-17:22}{18:57}{\textsf{\snav} {\tiny \RIGHTarrow} 08:35\hspace{2ex}}{\textsf{\ashresha} {\tiny \RIGHTarrow} 13:49\hspace{2ex}} 
&
\caldata{24}{06:16}{08:48}{09:26-11:01}{14:11-15:46}{18:57}{\textsf{\seka} {\tiny \RIGHTarrow} 03:33(+1)}{\textsf{\magha} {\tiny \RIGHTarrow} 12:07\hspace{2ex}} 
\\ \hline
\caldata{25}{06:15}{08:47}{17:21-18:57}{12:36-14:11}{18:57}{\textsf{\sdva} {\tiny \RIGHTarrow} 00:54(+1)}{\textsf{\purvaphalguni} {\tiny \RIGHTarrow} 10:17\hspace{2ex}} 
&
\caldata{26}{06:15}{08:47}{07:50-09:25}{11:01-12:36}{18:58}{\textsf{\stra} {\tiny \RIGHTarrow} 22:20\hspace{2ex}}{\textsf{\uttaraphalguni} {\tiny \RIGHTarrow} 08:25\hspace{2ex}} 
&
\caldata{27}{06:14}{08:46}{15:47-17:22}{09:25-11:00}{18:58}{\textsf{\schaturdashi} {\tiny \RIGHTarrow} 19:56\hspace{2ex}}{\textsf{\hasta} {\tiny \RIGHTarrow} 06:38\hspace{2ex}} 
&
\caldata{28}{06:13}{08:46}{12:35-14:11}{07:48-09:24}{18:58}{\textsf{\purnima} {\tiny \RIGHTarrow} 17:52\hspace{2ex}}{\textsf{\svati} {\tiny \RIGHTarrow} 03:59(+1)} 
&
\caldata{29}{06:13}{08:46}{14:11-15:47}{06:13-07:48}{18:59}{\textsf{\kpra} {\tiny \RIGHTarrow} 16:16\hspace{2ex}}{\textsf{\vishakha} {\tiny \RIGHTarrow} 03:22(+1)} 
&
\caldata{30}{06:12}{08:45}{10:59-12:35}{15:47-17:23}{18:59}{\textsf{\kdvi} {\tiny \RIGHTarrow} 15:15\hspace{2ex}}{\textsf{\anuradha} {\tiny \RIGHTarrow} 03:24(+1)} 
&
\\ \hline
\end{tabular}


%\clearpage
\begin{tabular}{|c|c|c|c|c|c|c|}
\multicolumn{7}{c}{\Large \bfseries MAY 2010}\\
\hline
\textbf{SUN} & \textbf{MON} & \textbf{TUE} & \textbf{WED} & \textbf{THU} & \textbf{FRI} & \textbf{SAT} \\ \hline
{}  &
{}  &
{}  &
{}  &
{}  &
{}  &
\caldata{1}{06:12}{08:45}{09:23-10:59}{14:11-15:47}{18:59}{\textsf{\ktri} {\tiny \RIGHTarrow} 14:56\hspace{2ex}}{\textsf{\jyeshtha} {\tiny \RIGHTarrow} 04:08(+1)} 
\\ \hline
\caldata{2}{06:11}{08:44}{17:23-19:00}{12:35-14:11}{19:00}{\textsf{\kcha} {\tiny \RIGHTarrow} 15:22\hspace{2ex}}{\textsf{\mula} {\tiny \RIGHTarrow} 05:36(+1)} 
&
\caldata{3}{06:10}{08:44}{07:46-09:22}{10:58-12:35}{19:00}{\textsf{\kpanc} {\tiny \RIGHTarrow} 16:31\hspace{2ex}}{\textsf{\purvashadha} {\tiny \RIGHTarrow} 07:43(+1)} 
&
\caldata{4}{06:10}{08:44}{15:47-17:23}{09:22-10:58}{19:00}{\textsf{\ksha} {\tiny \RIGHTarrow} 18:18\hspace{2ex}}{\textsf{\purvashadha} {\tiny \RIGHTarrow} 07:45\hspace{2ex}} 
&
\caldata{5}{06:09}{08:43}{12:35-14:11}{07:45-09:22}{19:01}{\textsf{\ksap} {\tiny \RIGHTarrow} 20:32\hspace{2ex}}{\textsf{\uttarashadha} {\tiny \RIGHTarrow} 10:25\hspace{2ex}} 
&
\caldata{6}{06:09}{08:43}{14:11-15:48}{06:09-07:45}{19:01}{\textsf{\kasht} {\tiny \RIGHTarrow} 23:00\hspace{2ex}}{\textsf{\shravana} {\tiny \RIGHTarrow} 13:22\hspace{2ex}} 
&
\caldata{7}{06:08}{08:42}{10:57-12:34}{15:47-17:24}{19:01}{\textsf{\knav} {\tiny \RIGHTarrow} 01:27(+1)}{\textsf{\shravishtha} {\tiny \RIGHTarrow} 16:23\hspace{2ex}} 
&
\caldata{8}{06:08}{08:42}{09:21-10:58}{14:11-15:48}{19:02}{\textsf{\kdas} {\tiny \RIGHTarrow} 03:39(+1)}{\textsf{\shatabhishak} {\tiny \RIGHTarrow} 19:14\hspace{2ex}} 
\\ \hline
\caldata{9}{06:07}{08:42}{17:25-19:02}{12:34-14:11}{19:02}{\textsf{\keka} {\tiny \RIGHTarrow} 05:23\hspace{2ex}}{\textsf{\proshthapada} {\tiny \RIGHTarrow} 05:35\hspace{2ex}} 
&
\caldata{10}{06:07}{08:42}{07:43-09:20}{10:57-12:34}{19:02}{\textsf{\kdva} {\tiny \RIGHTarrow} 06:35(+1)}{\textsf{\uttaraproshthapada} {\tiny \RIGHTarrow} 23:39\hspace{2ex}} 
&
\caldata{11}{06:07}{08:42}{15:49-17:26}{09:21-10:58}{19:03}{\textsf{\kdva} {\tiny \RIGHTarrow} 06:35\hspace{2ex}}{\textsf{\revati} {\tiny \RIGHTarrow} 01:02(+1)} 
&
\caldata{12}{06:06}{08:41}{12:34-14:11}{07:43-09:20}{19:03}{\textsf{\ktra} {\tiny \RIGHTarrow} 07:08\hspace{2ex}}{\textsf{\ashwini} {\tiny \RIGHTarrow} 01:49(+1)} 
&
\caldata{13}{06:06}{08:41}{14:12-15:49}{06:06-07:43}{19:04}{\textsf{\kchaturdashi} {\tiny \RIGHTarrow} 07:07\hspace{2ex}}{\textsf{\apabharani} {\tiny \RIGHTarrow} 02:04(+1)} 
&
\caldata{14}{06:05}{08:40}{10:57-12:34}{15:49-17:26}{19:04}{\textsf{\ama} {\tiny \RIGHTarrow} 06:34\hspace{2ex}}{\textsf{\krittika} {\tiny \RIGHTarrow} 01:51(+1)} 
&
\caldata{15}{06:05}{08:40}{09:19-10:57}{14:11-15:49}{19:04}{\textsf{\sdvi} {\tiny \RIGHTarrow} 04:09(+1)}{\textsf{\rohini} {\tiny \RIGHTarrow} 01:16(+1)} 
\\ \hline
\caldata{16}{06:05}{08:41}{17:27-19:05}{12:35-14:12}{19:05}{\textsf{\stri} {\tiny \RIGHTarrow} 02:29(+1)}{\textsf{\mrigashirsha} {\tiny \RIGHTarrow} 00:23(+1)} 
&
\caldata{17}{06:04}{08:40}{07:41-09:19}{10:56-12:34}{19:05}{\textsf{\scha} {\tiny \RIGHTarrow} 00:37(+1)}{\textsf{\ardra} {\tiny \RIGHTarrow} 23:17\hspace{2ex}} 
&
\caldata{18}{06:04}{08:40}{15:50-17:28}{09:19-10:57}{19:06}{\textsf{\spanc} {\tiny \RIGHTarrow} 22:36\hspace{2ex}}{\textsf{\punarvasu} {\tiny \RIGHTarrow} 22:03\hspace{2ex}} 
&
\caldata{19}{06:04}{08:40}{12:35-14:12}{07:41-09:19}{19:06}{\textsf{\ssha} {\tiny \RIGHTarrow} 20:29\hspace{2ex}}{\textsf{\pushya} {\tiny \RIGHTarrow} 20:42\hspace{2ex}} 
&
\caldata{20}{06:03}{08:39}{14:12-15:50}{06:03-07:40}{19:06}{\textsf{\ssap} {\tiny \RIGHTarrow} 18:18\hspace{2ex}}{\textsf{\ashresha} {\tiny \RIGHTarrow} 19:18\hspace{2ex}} 
&
\caldata{21}{06:03}{08:39}{10:57-12:35}{15:51-17:29}{19:07}{\textsf{\sasht} {\tiny \RIGHTarrow} 16:06\hspace{2ex}}{\textsf{\magha} {\tiny \RIGHTarrow} 17:53\hspace{2ex}} 
&
\caldata{22}{06:03}{08:39}{09:19-10:57}{14:13-15:51}{19:07}{\textsf{\snav} {\tiny \RIGHTarrow} 13:55\hspace{2ex}}{\textsf{\purvaphalguni} {\tiny \RIGHTarrow} 16:29\hspace{2ex}} 
\\ \hline
\caldata{23}{06:03}{08:39}{17:29-19:07}{12:35-14:13}{19:07}{\textsf{\sdas} {\tiny \RIGHTarrow} 11:48\hspace{2ex}}{\textsf{\uttaraphalguni} {\tiny \RIGHTarrow} 15:09\hspace{2ex}} 
&
\caldata{24}{06:02}{08:39}{07:40-09:18}{10:56-12:35}{19:08}{\textsf{\seka} {\tiny \RIGHTarrow} 09:47\hspace{2ex}}{\textsf{\hasta} {\tiny \RIGHTarrow} 13:57\hspace{2ex}} 
&
\caldata{25}{06:02}{08:39}{15:51-17:29}{09:18-10:56}{19:08}{\textsf{\sdva} {\tiny \RIGHTarrow} 07:59\hspace{2ex}}{\textsf{\chitra} {\tiny \RIGHTarrow} 12:58\hspace{2ex}} 
&
\caldata{26}{06:02}{08:39}{12:35-14:13}{07:40-09:18}{19:09}{\textsf{\stra} {\tiny \RIGHTarrow} 06:27\hspace{2ex}}{\textsf{\svati} {\tiny \RIGHTarrow} 12:17\hspace{2ex}} 
&
\caldata{27}{06:02}{08:39}{14:13-15:52}{06:02-07:40}{19:09}{\textsf{\purnima} {\tiny \RIGHTarrow} 04:38(+1)}{\textsf{\vishakha} {\tiny \RIGHTarrow} 12:00\hspace{2ex}} 
&
\caldata{28}{06:02}{08:39}{10:57-12:35}{15:52-17:30}{19:09}{\textsf{\kpra} {\tiny \RIGHTarrow} 04:30(+1)}{\textsf{\anuradha} {\tiny \RIGHTarrow} 12:12\hspace{2ex}} 
&
\caldata{29}{06:02}{08:39}{09:19-10:57}{14:14-15:53}{19:10}{\textsf{\kdvi} {\tiny \RIGHTarrow} 04:57(+1)}{\textsf{\jyeshtha} {\tiny \RIGHTarrow} 12:57\hspace{2ex}} 
\\ \hline
\caldata{30}{06:01}{08:38}{17:31-19:10}{12:35-14:14}{19:10}{\textsf{\ktri} {\tiny \RIGHTarrow} 06:01(+1)}{\textsf{\mula} {\tiny \RIGHTarrow} 14:18\hspace{2ex}} 
&
\caldata{31}{06:01}{08:39}{07:39-09:18}{10:57-12:36}{19:11}{\textsf{\kcha} {\tiny \RIGHTarrow} 07:38(+1)}{\textsf{\purvashadha} {\tiny \RIGHTarrow} 16:13\hspace{2ex}} 
&
{}  &
{}  &
{}  &
{}  &
\\ \hline
\end{tabular}


%\clearpage
\begin{tabular}{|c|c|c|c|c|c|c|}
\multicolumn{7}{c}{\Large \bfseries JUNE 2010}\\
\hline
\textbf{SUN} & \textbf{MON} & \textbf{TUE} & \textbf{WED} & \textbf{THU} & \textbf{FRI} & \textbf{SAT} \\ \hline
{}  &
{}  &
\caldata{1}{06:01}{08:39}{15:53-17:32}{09:18-10:57}{19:11}{\textsf{\kcha} {\tiny \RIGHTarrow} 07:39\hspace{2ex}}{\textsf{\uttarashadha} {\tiny \RIGHTarrow} 18:39\hspace{2ex}} 
&
\caldata{2}{06:01}{08:39}{12:36-14:14}{07:39-09:18}{19:11}{\textsf{\kpanc} {\tiny \RIGHTarrow} 09:45\hspace{2ex}}{\textsf{\shravana} {\tiny \RIGHTarrow} 21:26\hspace{2ex}} 
&
\caldata{3}{06:01}{08:39}{14:15-15:54}{06:01-07:39}{19:12}{\textsf{\ksha} {\tiny \RIGHTarrow} 12:06\hspace{2ex}}{\textsf{\shravishtha} {\tiny \RIGHTarrow} 00:24(+1)} 
&
\caldata{4}{06:01}{08:39}{10:57-12:36}{15:54-17:33}{19:12}{\textsf{\ksap} {\tiny \RIGHTarrow} 14:31\hspace{2ex}}{\textsf{\shatabhishak} {\tiny \RIGHTarrow} 03:21(+1)} 
&
\caldata{5}{06:01}{08:39}{09:18-10:57}{14:15-15:54}{19:12}{\textsf{\kasht} {\tiny \RIGHTarrow} 16:46\hspace{2ex}}{\textsf{\proshthapada} {\tiny \RIGHTarrow} 06:02(+1)} 
\\ \hline
\caldata{6}{06:01}{08:39}{17:34-19:13}{12:37-14:16}{19:13}{\textsf{\knav} {\tiny \RIGHTarrow} 05:37\hspace{2ex}}{\textsf{\proshthapada} {\tiny \RIGHTarrow} 06:01\hspace{2ex}} 
&
\caldata{7}{06:01}{08:39}{07:40-09:19}{10:58-12:37}{19:13}{\textsf{\kdas} {\tiny \RIGHTarrow} 19:56\hspace{2ex}}{\textsf{\uttaraproshthapada} {\tiny \RIGHTarrow} 08:14\hspace{2ex}} 
&
\caldata{8}{06:01}{08:39}{15:55-17:34}{09:19-10:58}{19:13}{\textsf{\keka} {\tiny \RIGHTarrow} 20:35\hspace{2ex}}{\textsf{\revati} {\tiny \RIGHTarrow} 09:50\hspace{2ex}} 
&
\caldata{9}{06:01}{08:39}{12:37-14:16}{07:40-09:19}{19:14}{\textsf{\kdva} {\tiny \RIGHTarrow} 20:32\hspace{2ex}}{\textsf{\ashwini} {\tiny \RIGHTarrow} 10:45\hspace{2ex}} 
&
\caldata{10}{06:01}{08:39}{14:16-15:55}{06:01-07:40}{19:14}{\textsf{\ktra} {\tiny \RIGHTarrow} 19:49\hspace{2ex}}{\textsf{\apabharani} {\tiny \RIGHTarrow} 11:01\hspace{2ex}} 
&
\caldata{11}{06:01}{08:39}{10:58-12:37}{15:55-17:34}{19:14}{\textsf{\kchaturdashi} {\tiny \RIGHTarrow} 18:30\hspace{2ex}}{\textsf{\krittika} {\tiny \RIGHTarrow} 10:40\hspace{2ex}} 
&
\caldata{12}{06:01}{08:39}{09:19-10:58}{14:17-15:56}{19:15}{\textsf{\ama} {\tiny \RIGHTarrow} 16:41\hspace{2ex}}{\textsf{\rohini} {\tiny \RIGHTarrow} 09:46\hspace{2ex}} 
\\ \hline
\caldata{13}{06:01}{08:39}{17:35-19:15}{12:38-14:17}{19:15}{\textsf{\spra} {\tiny \RIGHTarrow} 14:28\hspace{2ex}}{\textsf{\mrigashirsha} {\tiny \RIGHTarrow} 08:27\hspace{2ex}} 
&
\caldata{14}{06:02}{08:40}{07:41-09:20}{10:59-12:38}{19:15}{\textsf{\sdvi} {\tiny \RIGHTarrow} 11:59\hspace{2ex}}{\textsf{\ardra} {\tiny \RIGHTarrow} 06:49\hspace{2ex}} 
&
\caldata{15}{06:02}{08:40}{15:57-17:36}{09:20-10:59}{19:16}{\textsf{\stri} {\tiny \RIGHTarrow} 09:20\hspace{2ex}}{\textsf{\pushya} {\tiny \RIGHTarrow} 03:04(+1)} 
&
\caldata{16}{06:02}{08:40}{12:39-14:18}{07:41-09:20}{19:16}{\textsf{\scha} {\tiny \RIGHTarrow} 06:37\hspace{2ex}}{\textsf{\ashresha} {\tiny \RIGHTarrow} 01:12(+1)} 
&
\caldata{17}{06:02}{08:40}{14:18-15:57}{06:02-07:41}{19:16}{\textsf{\ssha} {\tiny \RIGHTarrow} 01:25(+1)}{\textsf{\magha} {\tiny \RIGHTarrow} 23:27\hspace{2ex}} 
&
\caldata{18}{06:02}{08:40}{10:59-12:39}{15:57-17:36}{19:16}{\textsf{\ssap} {\tiny \RIGHTarrow} 23:06\hspace{2ex}}{\textsf{\purvaphalguni} {\tiny \RIGHTarrow} 21:54\hspace{2ex}} 
&
\caldata{19}{06:02}{08:41}{09:20-11:00}{14:18-15:58}{19:17}{\textsf{\sasht} {\tiny \RIGHTarrow} 21:01\hspace{2ex}}{\textsf{\uttaraphalguni} {\tiny \RIGHTarrow} 20:36\hspace{2ex}} 
\\ \hline
\caldata{20}{06:03}{08:41}{17:37-19:17}{12:40-14:19}{19:17}{\textsf{\snav} {\tiny \RIGHTarrow} 19:15\hspace{2ex}}{\textsf{\hasta} {\tiny \RIGHTarrow} 19:36\hspace{2ex}} 
&
\caldata{21}{06:03}{08:41}{07:42-09:21}{11:00-12:40}{19:17}{\textsf{\sdas} {\tiny \RIGHTarrow} 17:50\hspace{2ex}}{\textsf{\chitra} {\tiny \RIGHTarrow} 18:57\hspace{2ex}} 
&
\caldata{22}{06:03}{08:41}{15:58-17:37}{09:21-11:00}{19:17}{\textsf{\seka} {\tiny \RIGHTarrow} 16:47\hspace{2ex}}{\textsf{\svati} {\tiny \RIGHTarrow} 18:41\hspace{2ex}} 
&
\caldata{23}{06:03}{08:42}{12:40-14:19}{07:42-09:21}{19:18}{\textsf{\sdva} {\tiny \RIGHTarrow} 16:10\hspace{2ex}}{\textsf{\vishakha} {\tiny \RIGHTarrow} 18:50\hspace{2ex}} 
&
\caldata{24}{06:03}{08:42}{14:19-15:59}{06:03-07:42}{19:18}{\textsf{\stra} {\tiny \RIGHTarrow} 15:59\hspace{2ex}}{\textsf{\anuradha} {\tiny \RIGHTarrow} 19:26\hspace{2ex}} 
&
\caldata{25}{06:04}{08:42}{11:01-12:41}{15:59-17:38}{19:18}{\textsf{\schaturdashi} {\tiny \RIGHTarrow} 16:17\hspace{2ex}}{\textsf{\jyeshtha} {\tiny \RIGHTarrow} 20:29\hspace{2ex}} 
&
\caldata{26}{06:04}{08:42}{09:22-11:01}{14:20-15:59}{19:18}{\textsf{\purnima} {\tiny \RIGHTarrow} 17:03\hspace{2ex}}{\textsf{\mula} {\tiny \RIGHTarrow} 22:00\hspace{2ex}} 
\\ \hline
\caldata{27}{06:04}{08:42}{17:38-19:18}{12:41-14:20}{19:18}{\textsf{\kpra} {\tiny \RIGHTarrow} 18:18\hspace{2ex}}{\textsf{\purvashadha} {\tiny \RIGHTarrow} 23:58\hspace{2ex}} 
&
\caldata{28}{06:05}{08:43}{07:44-09:23}{11:02-12:41}{19:18}{\textsf{\kdvi} {\tiny \RIGHTarrow} 19:58\hspace{2ex}}{\textsf{\uttarashadha} {\tiny \RIGHTarrow} 02:20(+1)} 
&
\caldata{29}{06:05}{08:43}{15:59-17:38}{09:23-11:02}{19:18}{\textsf{\ktri} {\tiny \RIGHTarrow} 22:01\hspace{2ex}}{\textsf{\shravana} {\tiny \RIGHTarrow} 05:04(+1)} 
&
\caldata{30}{06:05}{08:43}{12:42-14:21}{07:44-09:23}{19:19}{\textsf{\kcha} {\tiny \RIGHTarrow} 00:20(+1)}{\textsf{\shravishtha} {\tiny \RIGHTarrow} 08:00(+1)} 
&
{}  &
{}  &
\\ \hline
\end{tabular}


%\clearpage
\begin{tabular}{|c|c|c|c|c|c|c|}
\multicolumn{7}{c}{\Large \bfseries JULY 2010}\\
\hline
\textbf{SUN} & \textbf{MON} & \textbf{TUE} & \textbf{WED} & \textbf{THU} & \textbf{FRI} & \textbf{SAT} \\ \hline
{}  &
{}  &
{}  &
{}  &
\caldata{1}{06:05}{08:43}{14:21-16:00}{06:05-07:44}{19:19}{\textsf{\kpanc} {\tiny \RIGHTarrow} 02:45(+1)}{\textsf{\shravishtha} {\tiny \RIGHTarrow} 08:01\hspace{2ex}} 
&
\caldata{2}{06:06}{08:44}{11:03-12:42}{16:00-17:39}{19:19}{\textsf{\ksha} {\tiny \RIGHTarrow} 05:07(+1)}{\textsf{\shatabhishak} {\tiny \RIGHTarrow} 11:00\hspace{2ex}} 
&
\caldata{3}{06:06}{08:44}{09:24-11:03}{14:21-16:00}{19:19}{\textsf{\ksap} {\tiny \RIGHTarrow} 05:18\hspace{2ex}}{\textsf{\proshthapada} {\tiny \RIGHTarrow} 05:50\hspace{2ex}} 
\\ \hline
\caldata{4}{06:06}{08:44}{17:39-19:19}{12:42-14:21}{19:19}{\textsf{\ksap} {\tiny \RIGHTarrow} 07:11\hspace{2ex}}{\textsf{\uttaraproshthapada} {\tiny \RIGHTarrow} 16:23\hspace{2ex}} 
&
\caldata{5}{06:07}{08:45}{07:46-09:25}{11:04-12:43}{19:19}{\textsf{\kasht} {\tiny \RIGHTarrow} 08:47\hspace{2ex}}{\textsf{\revati} {\tiny \RIGHTarrow} 18:24\hspace{2ex}} 
&
\caldata{6}{06:07}{08:45}{16:01-17:40}{09:25-11:04}{19:19}{\textsf{\knav} {\tiny \RIGHTarrow} 09:46\hspace{2ex}}{\textsf{\ashwini} {\tiny \RIGHTarrow} 19:47\hspace{2ex}} 
&
\caldata{7}{06:07}{08:45}{12:43-14:22}{07:46-09:25}{19:19}{\textsf{\kdas} {\tiny \RIGHTarrow} 10:02\hspace{2ex}}{\textsf{\apabharani} {\tiny \RIGHTarrow} 20:26\hspace{2ex}} 
&
\caldata{8}{06:08}{08:46}{14:22-16:01}{06:08-07:46}{19:19}{\textsf{\keka} {\tiny \RIGHTarrow} 09:34\hspace{2ex}}{\textsf{\krittika} {\tiny \RIGHTarrow} 20:21\hspace{2ex}} 
&
\caldata{9}{06:08}{08:46}{11:04-12:43}{16:01-17:40}{19:19}{\textsf{\kdva} {\tiny \RIGHTarrow} 08:22\hspace{2ex}}{\textsf{\rohini} {\tiny \RIGHTarrow} 19:34\hspace{2ex}} 
&
\caldata{10}{06:08}{08:46}{09:25-11:04}{14:22-16:01}{19:19}{\textsf{\ktra} {\tiny \RIGHTarrow} 06:30\hspace{2ex}}{\textsf{\mrigashirsha} {\tiny \RIGHTarrow} 18:11\hspace{2ex}} 
\\ \hline
\caldata{11}{06:09}{08:47}{17:40-19:19}{12:44-14:22}{19:19}{\textsf{\ama} {\tiny \RIGHTarrow} 01:08(+1)}{\textsf{\ardra} {\tiny \RIGHTarrow} 16:17\hspace{2ex}} 
&
\caldata{12}{06:09}{08:46}{07:47-09:26}{11:04-12:43}{19:18}{\textsf{\spra} {\tiny \RIGHTarrow} 21:57\hspace{2ex}}{\textsf{\punarvasu} {\tiny \RIGHTarrow} 14:03\hspace{2ex}} 
&
\caldata{13}{06:09}{08:46}{16:00-17:39}{09:26-11:04}{19:18}{\textsf{\sdvi} {\tiny \RIGHTarrow} 18:38\hspace{2ex}}{\textsf{\pushya} {\tiny \RIGHTarrow} 11:36\hspace{2ex}} 
&
\caldata{14}{06:10}{08:47}{12:44-14:22}{07:48-09:27}{19:18}{\textsf{\stri} {\tiny \RIGHTarrow} 15:19\hspace{2ex}}{\textsf{\ashresha} {\tiny \RIGHTarrow} 09:04\hspace{2ex}} 
&
\caldata{15}{06:10}{08:47}{14:22-16:01}{06:10-07:48}{19:18}{\textsf{\scha} {\tiny \RIGHTarrow} 12:07\hspace{2ex}}{\textsf{\magha} {\tiny \RIGHTarrow} 06:38\hspace{2ex}} 
&
\caldata{16}{06:10}{08:47}{11:05-12:44}{16:01-17:39}{19:18}{\textsf{\spanc} {\tiny \RIGHTarrow} 09:11\hspace{2ex}}{\textsf{\uttaraphalguni} {\tiny \RIGHTarrow} 02:38(+1)} 
&
\caldata{17}{06:11}{08:48}{09:27-11:06}{14:22-16:01}{19:18}{\textsf{\ssha} {\tiny \RIGHTarrow} 06:36\hspace{2ex}}{\textsf{\hasta} {\tiny \RIGHTarrow} 01:16(+1)} 
\\ \hline
\caldata{18}{06:11}{08:48}{17:39-19:18}{12:44-14:22}{19:18}{\textsf{\sasht} {\tiny \RIGHTarrow} 03:00(+1)}{\textsf{\chitra} {\tiny \RIGHTarrow} 00:25(+1)} 
&
\caldata{19}{06:12}{08:49}{07:50-09:28}{11:06-12:44}{19:17}{\textsf{\snav} {\tiny \RIGHTarrow} 02:02(+1)}{\textsf{\svati} {\tiny \RIGHTarrow} 00:08(+1)} 
&
\caldata{20}{06:12}{08:49}{16:00-17:38}{09:28-11:06}{19:17}{\textsf{\sdas} {\tiny \RIGHTarrow} 01:39(+1)}{\textsf{\vishakha} {\tiny \RIGHTarrow} 00:26(+1)} 
&
\caldata{21}{06:12}{08:49}{12:44-14:22}{07:50-09:28}{19:17}{\textsf{\seka} {\tiny \RIGHTarrow} 01:49(+1)}{\textsf{\anuradha} {\tiny \RIGHTarrow} 01:16(+1)} 
&
\caldata{22}{06:13}{08:49}{14:23-16:01}{06:13-07:51}{19:17}{\textsf{\sdva} {\tiny \RIGHTarrow} 02:30(+1)}{\textsf{\jyeshtha} {\tiny \RIGHTarrow} 02:37(+1)} 
&
\caldata{23}{06:13}{08:49}{11:06-12:44}{16:00-17:38}{19:16}{\textsf{\stra} {\tiny \RIGHTarrow} 03:39(+1)}{\textsf{\mula} {\tiny \RIGHTarrow} 04:24(+1)} 
&
\caldata{24}{06:13}{08:49}{09:28-11:06}{14:22-16:00}{19:16}{\textsf{\schaturdashi} {\tiny \RIGHTarrow} 05:12(+1)}{\textsf{\purvashadha} {\tiny \RIGHTarrow} 06:36(+1)} 
\\ \hline
\caldata{25}{06:14}{08:50}{17:38-19:16}{12:45-14:22}{19:16}{\textsf{\purnima} {\tiny \RIGHTarrow} 07:06(+1)}{\textsf{\purvashadha} {\tiny \RIGHTarrow} 06:36\hspace{2ex}} 
&
\caldata{26}{06:14}{08:50}{07:51-09:29}{11:06-12:44}{19:15}{\textsf{\purnima} {\tiny \RIGHTarrow} 07:06\hspace{2ex}}{\textsf{\uttarashadha} {\tiny \RIGHTarrow} 09:08\hspace{2ex}} 
&
\caldata{27}{06:14}{08:50}{15:59-17:37}{09:29-11:06}{19:15}{\textsf{\kpra} {\tiny \RIGHTarrow} 09:17\hspace{2ex}}{\textsf{\shravana} {\tiny \RIGHTarrow} 11:55\hspace{2ex}} 
&
\caldata{28}{06:15}{08:51}{12:45-14:22}{07:52-09:30}{19:15}{\textsf{\kdvi} {\tiny \RIGHTarrow} 11:38\hspace{2ex}}{\textsf{\shravishtha} {\tiny \RIGHTarrow} 14:51\hspace{2ex}} 
&
\caldata{29}{06:15}{08:50}{14:21-15:59}{06:15-07:52}{19:14}{\textsf{\ktri} {\tiny \RIGHTarrow} 14:04\hspace{2ex}}{\textsf{\shatabhishak} {\tiny \RIGHTarrow} 17:51\hspace{2ex}} 
&
\caldata{30}{06:15}{08:50}{11:07-12:44}{15:59-17:36}{19:14}{\textsf{\kcha} {\tiny \RIGHTarrow} 05:56\hspace{2ex}}{\textsf{\proshthapada} {\tiny \RIGHTarrow} 05:46\hspace{2ex}} 
&
\caldata{31}{06:16}{08:51}{09:30-11:07}{14:21-15:58}{19:13}{\textsf{\kpanc} {\tiny \RIGHTarrow} 18:41\hspace{2ex}}{\textsf{\uttaraproshthapada} {\tiny \RIGHTarrow} 23:33\hspace{2ex}} 
\\ \hline
\end{tabular}


%\clearpage
\begin{tabular}{|c|c|c|c|c|c|c|}
\multicolumn{7}{c}{\Large \bfseries AUGUST 2010}\\
\hline
\textbf{SUN} & \textbf{MON} & \textbf{TUE} & \textbf{WED} & \textbf{THU} & \textbf{FRI} & \textbf{SAT} \\ \hline
\caldata{1}{06:16}{08:51}{17:35-19:13}{12:44-14:21}{19:13}{\textsf{\ksha} {\tiny \RIGHTarrow} 20:33\hspace{2ex}}{\textsf{\revati} {\tiny \RIGHTarrow} 01:58(+1)} 
&
\caldata{2}{06:16}{08:51}{07:53-09:30}{11:07-12:44}{19:12}{\textsf{\ksap} {\tiny \RIGHTarrow} 21:56\hspace{2ex}}{\textsf{\ashwini} {\tiny \RIGHTarrow} 03:52(+1)} 
&
\caldata{3}{06:17}{08:52}{15:58-17:35}{09:30-11:07}{19:12}{\textsf{\kasht} {\tiny \RIGHTarrow} 22:41\hspace{2ex}}{\textsf{\apabharani} {\tiny \RIGHTarrow} 05:08(+1)} 
&
\caldata{4}{06:17}{08:51}{12:44-14:20}{07:53-09:30}{19:11}{\textsf{\knav} {\tiny \RIGHTarrow} 22:43\hspace{2ex}}{\textsf{\krittika} {\tiny \RIGHTarrow} 05:40(+1)} 
&
\caldata{5}{06:17}{08:51}{14:20-15:57}{06:17-07:53}{19:11}{\textsf{\kdas} {\tiny \RIGHTarrow} 21:59\hspace{2ex}}{\textsf{\rohini} {\tiny \RIGHTarrow} 05:26(+1)} 
&
\caldata{6}{06:18}{08:52}{11:07-12:44}{15:57-17:33}{19:10}{\textsf{\keka} {\tiny \RIGHTarrow} 20:28\hspace{2ex}}{\textsf{\mrigashirsha} {\tiny \RIGHTarrow} 04:26(+1)} 
&
\caldata{7}{06:18}{08:52}{09:31-11:07}{14:20-15:57}{19:10}{\textsf{\kdva} {\tiny \RIGHTarrow} 18:16\hspace{2ex}}{\textsf{\ardra} {\tiny \RIGHTarrow} 02:46(+1)} 
\\ \hline
\caldata{8}{06:18}{08:52}{17:32-19:09}{12:43-14:19}{19:09}{\textsf{\ktra} {\tiny \RIGHTarrow} 15:28\hspace{2ex}}{\textsf{\punarvasu} {\tiny \RIGHTarrow} 00:32(+1)} 
&
\caldata{9}{06:19}{08:53}{07:55-09:31}{11:07-12:44}{19:09}{\textsf{\kchaturdashi} {\tiny \RIGHTarrow} 12:13\hspace{2ex}}{\textsf{\pushya} {\tiny \RIGHTarrow} 21:54\hspace{2ex}} 
&
\caldata{10}{06:19}{08:52}{15:55-17:31}{09:31-11:07}{19:08}{\textsf{\ama} {\tiny \RIGHTarrow} 08:37\hspace{2ex}}{\textsf{\ashresha} {\tiny \RIGHTarrow} 19:02\hspace{2ex}} 
&
\caldata{11}{06:19}{08:52}{12:43-14:19}{07:55-09:31}{19:08}{\textsf{\sdvi} {\tiny \RIGHTarrow} 01:06(+1)}{\textsf{\magha} {\tiny \RIGHTarrow} 16:06\hspace{2ex}} 
&
\caldata{12}{06:19}{08:52}{14:19-15:55}{06:19-07:55}{19:07}{\textsf{\stri} {\tiny \RIGHTarrow} 21:34\hspace{2ex}}{\textsf{\purvaphalguni} {\tiny \RIGHTarrow} 13:17\hspace{2ex}} 
&
\caldata{13}{06:20}{08:53}{11:07-12:43}{15:54-17:30}{19:06}{\textsf{\scha} {\tiny \RIGHTarrow} 18:23\hspace{2ex}}{\textsf{\uttaraphalguni} {\tiny \RIGHTarrow} 10:44\hspace{2ex}} 
&
\caldata{14}{06:20}{08:53}{09:31-11:07}{14:18-15:54}{19:06}{\textsf{\spanc} {\tiny \RIGHTarrow} 15:41\hspace{2ex}}{\textsf{\hasta} {\tiny \RIGHTarrow} 08:37\hspace{2ex}} 
\\ \hline
\caldata{15}{06:20}{08:53}{17:29-19:05}{12:42-14:18}{19:05}{\textsf{\ssha} {\tiny \RIGHTarrow} 13:35\hspace{2ex}}{\textsf{\chitra} {\tiny \RIGHTarrow} 07:06\hspace{2ex}} 
&
\caldata{16}{06:21}{08:53}{07:56-09:31}{11:07-12:42}{19:04}{\textsf{\ssap} {\tiny \RIGHTarrow} 12:12\hspace{2ex}}{\textsf{\vishakha} {\tiny \RIGHTarrow} 06:12(+1)} 
&
\caldata{17}{06:21}{08:53}{15:53-17:28}{09:31-11:07}{19:04}{\textsf{\sasht} {\tiny \RIGHTarrow} 11:34\hspace{2ex}}{\textsf{\anuradha} {\tiny \RIGHTarrow} 06:51(+1)} 
&
\caldata{18}{06:21}{08:53}{12:42-14:17}{07:56-09:31}{19:03}{\textsf{\snav} {\tiny \RIGHTarrow} 11:41\hspace{2ex}}{\textsf{\anuradha} {\tiny \RIGHTarrow} 06:51\hspace{2ex}} 
&
\caldata{19}{06:21}{08:53}{14:16-15:51}{06:21-07:56}{19:02}{\textsf{\sdas} {\tiny \RIGHTarrow} 12:29\hspace{2ex}}{\textsf{\jyeshtha} {\tiny \RIGHTarrow} 08:12\hspace{2ex}} 
&
\caldata{20}{06:22}{08:54}{11:07-12:42}{15:52-17:27}{19:02}{\textsf{\seka} {\tiny \RIGHTarrow} 13:51\hspace{2ex}}{\textsf{\mula} {\tiny \RIGHTarrow} 10:08\hspace{2ex}} 
&
\caldata{21}{06:22}{08:53}{09:31-11:06}{14:16-15:51}{19:01}{\textsf{\sdva} {\tiny \RIGHTarrow} 15:40\hspace{2ex}}{\textsf{\purvashadha} {\tiny \RIGHTarrow} 12:30\hspace{2ex}} 
\\ \hline
\caldata{22}{06:22}{08:53}{17:25-19:00}{12:41-14:15}{19:00}{\textsf{\stra} {\tiny \RIGHTarrow} 17:48\hspace{2ex}}{\textsf{\uttarashadha} {\tiny \RIGHTarrow} 15:11\hspace{2ex}} 
&
\caldata{23}{06:22}{08:53}{07:56-09:31}{11:05-12:40}{18:59}{\textsf{\schaturdashi} {\tiny \RIGHTarrow} 20:08\hspace{2ex}}{\textsf{\shravana} {\tiny \RIGHTarrow} 18:03\hspace{2ex}} 
&
\caldata{24}{06:22}{08:53}{15:49-17:24}{09:31-11:05}{18:59}{\textsf{\purnima} {\tiny \RIGHTarrow} 22:34\hspace{2ex}}{\textsf{\shravishtha} {\tiny \RIGHTarrow} 21:02\hspace{2ex}} 
&
\caldata{25}{06:23}{08:54}{12:40-14:14}{07:57-09:31}{18:58}{\textsf{\kpra} {\tiny \RIGHTarrow} 01:01(+1)}{\textsf{\shatabhishak} {\tiny \RIGHTarrow} 00:01(+1)} 
&
\caldata{26}{06:23}{08:53}{14:14-15:48}{06:23-07:57}{18:57}{\textsf{\kdvi} {\tiny \RIGHTarrow} 03:23(+1)}{\textsf{\proshthapada} {\tiny \RIGHTarrow} 02:57(+1)} 
&
\caldata{27}{06:23}{08:53}{11:05-12:39}{15:47-17:21}{18:56}{\textsf{\ktri} {\tiny \RIGHTarrow} 05:39\hspace{2ex}}{\textsf{\uttaraproshthapada} {\tiny \RIGHTarrow} 05:35\hspace{2ex}} 
&
\caldata{28}{06:23}{08:53}{09:31-11:05}{14:13-15:47}{18:55}{\textsf{\kcha} {\tiny \RIGHTarrow} 07:34(+1)}{\textsf{\revati} {\tiny \RIGHTarrow} 08:16(+1)} 
\\ \hline
\caldata{29}{06:24}{08:54}{17:21-18:55}{12:39-14:13}{18:55}{\textsf{\kcha} {\tiny \RIGHTarrow} 07:33\hspace{2ex}}{\textsf{\revati} {\tiny \RIGHTarrow} 08:15\hspace{2ex}} 
&
\caldata{30}{06:24}{08:54}{07:57-09:31}{11:05-12:39}{18:54}{\textsf{\kpanc} {\tiny \RIGHTarrow} 09:08\hspace{2ex}}{\textsf{\ashwini} {\tiny \RIGHTarrow} 10:25\hspace{2ex}} 
&
\caldata{31}{06:24}{08:53}{15:45-17:19}{09:31-11:04}{18:53}{\textsf{\ksha} {\tiny \RIGHTarrow} 10:15\hspace{2ex}}{\textsf{\apabharani} {\tiny \RIGHTarrow} 12:07\hspace{2ex}} 
&
{}  &
{}  &
{}  &
\\ \hline
\end{tabular}


%\clearpage
\begin{tabular}{|c|c|c|c|c|c|c|}
\multicolumn{7}{c}{\Large \bfseries SEPTEMBER 2010}\\
\hline
\textbf{SUN} & \textbf{MON} & \textbf{TUE} & \textbf{WED} & \textbf{THU} & \textbf{FRI} & \textbf{SAT} \\ \hline
{}  &
{}  &
{}  &
\caldata{1}{06:24}{08:53}{12:38-14:11}{07:57-09:31}{18:52}{\textsf{\ksap} {\tiny \RIGHTarrow} 10:47\hspace{2ex}}{\textsf{\krittika} {\tiny \RIGHTarrow} 13:14\hspace{2ex}} 
&
\caldata{2}{06:24}{08:53}{14:10-15:44}{06:24-07:57}{18:51}{\textsf{\kasht} {\tiny \RIGHTarrow} 10:38\hspace{2ex}}{\textsf{\rohini} {\tiny \RIGHTarrow} 13:41\hspace{2ex}} 
&
\caldata{3}{06:25}{08:54}{11:04-12:38}{15:44-17:17}{18:51}{\textsf{\knav} {\tiny \RIGHTarrow} 09:47\hspace{2ex}}{\textsf{\mrigashirsha} {\tiny \RIGHTarrow} 13:26\hspace{2ex}} 
&
\caldata{4}{06:25}{08:54}{09:31-11:04}{14:10-15:43}{18:50}{\textsf{\kdas} {\tiny \RIGHTarrow} 08:12\hspace{2ex}}{\textsf{\ardra} {\tiny \RIGHTarrow} 12:27\hspace{2ex}} 
\\ \hline
\caldata{5}{06:25}{08:53}{17:16-18:49}{12:37-14:10}{18:49}{\textsf{\kdva} {\tiny \RIGHTarrow} 02:58(+1)}{\textsf{\punarvasu} {\tiny \RIGHTarrow} 10:49\hspace{2ex}} 
&
\caldata{6}{06:25}{08:53}{07:57-09:30}{11:03-12:36}{18:48}{\textsf{\ktra} {\tiny \RIGHTarrow} 23:34\hspace{2ex}}{\textsf{\pushya} {\tiny \RIGHTarrow} 08:35\hspace{2ex}} 
&
\caldata{7}{06:25}{08:53}{15:41-17:14}{09:30-11:03}{18:47}{\textsf{\kchaturdashi} {\tiny \RIGHTarrow} 19:51\hspace{2ex}}{\textsf{\magha} {\tiny \RIGHTarrow} 02:54(+1)} 
&
\caldata{8}{06:25}{08:53}{12:35-14:08}{07:57-09:30}{18:46}{\textsf{\ama} {\tiny \RIGHTarrow} 16:00\hspace{2ex}}{\textsf{\purvaphalguni} {\tiny \RIGHTarrow} 23:50\hspace{2ex}} 
&
\caldata{9}{06:26}{08:53}{14:07-15:40}{06:26-07:58}{18:45}{\textsf{\spra} {\tiny \RIGHTarrow} 12:09\hspace{2ex}}{\textsf{\uttaraphalguni} {\tiny \RIGHTarrow} 20:53\hspace{2ex}} 
&
\caldata{10}{06:26}{08:53}{11:02-12:35}{15:39-17:11}{18:44}{\textsf{\sdvi} {\tiny \RIGHTarrow} 08:30\hspace{2ex}}{\textsf{\hasta} {\tiny \RIGHTarrow} 18:14\hspace{2ex}} 
&
\caldata{11}{06:26}{08:53}{09:30-11:02}{14:07-15:39}{18:44}{\textsf{\scha} {\tiny \RIGHTarrow} 02:37(+1)}{\textsf{\chitra} {\tiny \RIGHTarrow} 16:03\hspace{2ex}} 
\\ \hline
\caldata{12}{06:26}{08:53}{17:10-18:43}{12:34-14:06}{18:43}{\textsf{\spanc} {\tiny \RIGHTarrow} 00:41(+1)}{\textsf{\svati} {\tiny \RIGHTarrow} 14:30\hspace{2ex}} 
&
\caldata{13}{06:26}{08:53}{07:58-09:30}{11:02-12:34}{18:42}{\textsf{\ssha} {\tiny \RIGHTarrow} 23:32\hspace{2ex}}{\textsf{\vishakha} {\tiny \RIGHTarrow} 13:43\hspace{2ex}} 
&
\caldata{14}{06:27}{08:53}{15:37-17:09}{09:30-11:02}{18:41}{\textsf{\ssap} {\tiny \RIGHTarrow} 23:14\hspace{2ex}}{\textsf{\anuradha} {\tiny \RIGHTarrow} 13:45\hspace{2ex}} 
&
\caldata{15}{06:27}{08:53}{12:33-14:05}{07:58-09:30}{18:40}{\textsf{\sasht} {\tiny \RIGHTarrow} 23:46\hspace{2ex}}{\textsf{\jyeshtha} {\tiny \RIGHTarrow} 14:38\hspace{2ex}} 
&
\caldata{16}{06:27}{08:53}{14:04-15:36}{06:27-07:58}{18:39}{\textsf{\snav} {\tiny \RIGHTarrow} 01:00(+1)}{\textsf{\mula} {\tiny \RIGHTarrow} 16:15\hspace{2ex}} 
&
\caldata{17}{06:27}{08:53}{11:01-12:32}{15:35-17:06}{18:38}{\textsf{\sdas} {\tiny \RIGHTarrow} 02:50(+1)}{\textsf{\purvashadha} {\tiny \RIGHTarrow} 18:29\hspace{2ex}} 
&
\caldata{18}{06:27}{08:53}{09:29-11:00}{14:03-15:34}{18:37}{\textsf{\seka} {\tiny \RIGHTarrow} 05:04(+1)}{\textsf{\uttarashadha} {\tiny \RIGHTarrow} 21:09\hspace{2ex}} 
\\ \hline
\caldata{19}{06:27}{08:53}{17:05-18:37}{12:32-14:03}{18:37}{\textsf{\sdva} {\tiny \RIGHTarrow} 07:32(+1)}{\textsf{\shravana} {\tiny \RIGHTarrow} 00:05(+1)} 
&
\caldata{20}{06:28}{08:53}{07:59-09:30}{11:01-12:32}{18:36}{\textsf{\sdva} {\tiny \RIGHTarrow} 07:32\hspace{2ex}}{\textsf{\shravishtha} {\tiny \RIGHTarrow} 03:06(+1)} 
&
\caldata{21}{06:28}{08:53}{15:33-17:04}{09:29-11:00}{18:35}{\textsf{\stra} {\tiny \RIGHTarrow} 10:03\hspace{2ex}}{\textsf{\shatabhishak} {\tiny \RIGHTarrow} 06:06(+1)} 
&
\caldata{22}{06:28}{08:53}{12:31-14:01}{07:58-09:29}{18:34}{\textsf{\schaturdashi} {\tiny \RIGHTarrow} 12:29\hspace{2ex}}{\textsf{\proshthapada} {\tiny \RIGHTarrow} 08:57(+1)} 
&
\caldata{23}{06:28}{08:53}{14:01-15:31}{06:28-07:58}{18:33}{\textsf{\purnima} {\tiny \RIGHTarrow} 06:12\hspace{2ex}}{\textsf{\proshthapada} {\tiny \RIGHTarrow} 06:23\hspace{2ex}} 
&
\caldata{24}{06:28}{08:52}{10:59-12:30}{15:31-17:01}{18:32}{\textsf{\kpra} {\tiny \RIGHTarrow} 16:49\hspace{2ex}}{\textsf{\uttaraproshthapada} {\tiny \RIGHTarrow} 11:34\hspace{2ex}} 
&
\caldata{25}{06:29}{08:53}{09:29-10:59}{14:00-15:30}{18:31}{\textsf{\kdvi} {\tiny \RIGHTarrow} 18:36\hspace{2ex}}{\textsf{\revati} {\tiny \RIGHTarrow} 13:58\hspace{2ex}} 
\\ \hline
\caldata{26}{06:29}{08:53}{16:59-18:30}{12:29-13:59}{18:30}{\textsf{\ktri} {\tiny \RIGHTarrow} 20:05\hspace{2ex}}{\textsf{\ashwini} {\tiny \RIGHTarrow} 16:04\hspace{2ex}} 
&
\caldata{27}{06:29}{08:53}{07:59-09:29}{10:59-12:29}{18:29}{\textsf{\kcha} {\tiny \RIGHTarrow} 21:12\hspace{2ex}}{\textsf{\apabharani} {\tiny \RIGHTarrow} 17:50\hspace{2ex}} 
&
\caldata{28}{06:29}{08:53}{15:29-16:59}{09:29-10:59}{18:29}{\textsf{\kpanc} {\tiny \RIGHTarrow} 21:53\hspace{2ex}}{\textsf{\krittika} {\tiny \RIGHTarrow} 19:11\hspace{2ex}} 
&
\caldata{29}{06:29}{08:52}{12:28-13:58}{07:58-09:28}{18:28}{\textsf{\ksha} {\tiny \RIGHTarrow} 22:05\hspace{2ex}}{\textsf{\rohini} {\tiny \RIGHTarrow} 20:04\hspace{2ex}} 
&
\caldata{30}{06:29}{08:52}{13:57-15:27}{06:29-07:58}{18:27}{\textsf{\ksap} {\tiny \RIGHTarrow} 21:42\hspace{2ex}}{\textsf{\mrigashirsha} {\tiny \RIGHTarrow} 20:24\hspace{2ex}} 
&
{}  &
\\ \hline
\end{tabular}


%\clearpage
\begin{tabular}{|c|c|c|c|c|c|c|}
\multicolumn{7}{c}{\Large \bfseries OCTOBER 2010}\\
\hline
\textbf{SUN} & \textbf{MON} & \textbf{TUE} & \textbf{WED} & \textbf{THU} & \textbf{FRI} & \textbf{SAT} \\ \hline
{}  &
{}  &
{}  &
{}  &
{}  &
\caldata{1}{06:30}{08:53}{10:58-12:28}{15:27-16:56}{18:26}{\textsf{\kasht} {\tiny \RIGHTarrow} 20:42\hspace{2ex}}{\textsf{\ardra} {\tiny \RIGHTarrow} 20:07\hspace{2ex}} 
&
\caldata{2}{06:30}{08:53}{09:28-10:58}{13:56-15:26}{18:25}{\textsf{\knav} {\tiny \RIGHTarrow} 19:04\hspace{2ex}}{\textsf{\punarvasu} {\tiny \RIGHTarrow} 19:13\hspace{2ex}} 
\\ \hline
\caldata{3}{06:30}{08:52}{16:54-18:24}{12:27-13:56}{18:24}{\textsf{\kdas} {\tiny \RIGHTarrow} 16:51\hspace{2ex}}{\textsf{\pushya} {\tiny \RIGHTarrow} 17:44\hspace{2ex}} 
&
\caldata{4}{06:30}{08:52}{07:59-09:28}{10:57-12:26}{18:23}{\textsf{\keka} {\tiny \RIGHTarrow} 14:06\hspace{2ex}}{\textsf{\ashresha} {\tiny \RIGHTarrow} 15:42\hspace{2ex}} 
&
\caldata{5}{06:31}{08:53}{15:25-16:54}{09:29-10:58}{18:23}{\textsf{\kdva} {\tiny \RIGHTarrow} 10:56\hspace{2ex}}{\textsf{\magha} {\tiny \RIGHTarrow} 13:16\hspace{2ex}} 
&
\caldata{6}{06:31}{08:53}{12:26-13:55}{07:59-09:28}{18:22}{\textsf{\ktra} {\tiny \RIGHTarrow} 07:27\hspace{2ex}}{\textsf{\purvaphalguni} {\tiny \RIGHTarrow} 10:33\hspace{2ex}} 
&
\caldata{7}{06:31}{08:53}{13:54-15:23}{06:31-07:59}{18:21}{\textsf{\ama} {\tiny \RIGHTarrow} 00:16(+1)}{\textsf{\uttaraphalguni} {\tiny \RIGHTarrow} 07:43\hspace{2ex}} 
&
\caldata{8}{06:31}{08:52}{10:56-12:25}{15:22-16:51}{18:20}{\textsf{\spra} {\tiny \RIGHTarrow} 20:57\hspace{2ex}}{\textsf{\chitra} {\tiny \RIGHTarrow} 02:29(+1)} 
&
\caldata{9}{06:32}{08:53}{09:28-10:57}{13:53-15:22}{18:19}{\textsf{\sdvi} {\tiny \RIGHTarrow} 18:03\hspace{2ex}}{\textsf{\svati} {\tiny \RIGHTarrow} 00:30(+1)} 
\\ \hline
\caldata{10}{06:32}{08:53}{16:50-18:19}{12:25-13:53}{18:19}{\textsf{\stri} {\tiny \RIGHTarrow} 15:43\hspace{2ex}}{\textsf{\vishakha} {\tiny \RIGHTarrow} 23:08\hspace{2ex}} 
&
\caldata{11}{06:32}{08:53}{08:00-09:28}{10:56-12:25}{18:18}{\textsf{\scha} {\tiny \RIGHTarrow} 14:06\hspace{2ex}}{\textsf{\anuradha} {\tiny \RIGHTarrow} 22:30\hspace{2ex}} 
&
\caldata{12}{06:32}{08:53}{15:20-16:48}{09:28-10:56}{18:17}{\textsf{\spanc} {\tiny \RIGHTarrow} 13:18\hspace{2ex}}{\textsf{\jyeshtha} {\tiny \RIGHTarrow} 22:42\hspace{2ex}} 
&
\caldata{13}{06:33}{08:53}{12:24-13:52}{08:00-09:28}{18:16}{\textsf{\ssha} {\tiny \RIGHTarrow} 13:23\hspace{2ex}}{\textsf{\mula} {\tiny \RIGHTarrow} 23:44\hspace{2ex}} 
&
\caldata{14}{06:33}{08:53}{13:51-15:19}{06:33-08:00}{18:15}{\textsf{\ssap} {\tiny \RIGHTarrow} 14:17\hspace{2ex}}{\textsf{\purvashadha} {\tiny \RIGHTarrow} 01:31(+1)} 
&
\caldata{15}{06:33}{08:53}{10:56-12:24}{15:19-16:47}{18:15}{\textsf{\sasht} {\tiny \RIGHTarrow} 15:54\hspace{2ex}}{\textsf{\uttarashadha} {\tiny \RIGHTarrow} 03:54(+1)} 
&
\caldata{16}{06:33}{08:53}{09:28-10:55}{13:51-15:18}{18:14}{\textsf{\snav} {\tiny \RIGHTarrow} 18:02\hspace{2ex}}{\textsf{\shravana} {\tiny \RIGHTarrow} 06:42(+1)} 
\\ \hline
\caldata{17}{06:34}{08:53}{16:45-18:13}{12:23-13:50}{18:13}{\textsf{\sdas} {\tiny \RIGHTarrow} 20:28\hspace{2ex}}{\textsf{\shravana} {\tiny \RIGHTarrow} 06:42\hspace{2ex}} 
&
\caldata{18}{06:34}{08:53}{08:01-09:28}{10:56-12:23}{18:13}{\textsf{\seka} {\tiny \RIGHTarrow} 23:01\hspace{2ex}}{\textsf{\shravishtha} {\tiny \RIGHTarrow} 09:43\hspace{2ex}} 
&
\caldata{19}{06:34}{08:53}{15:17-16:44}{09:28-10:55}{18:12}{\textsf{\sdva} {\tiny \RIGHTarrow} 01:28(+1)}{\textsf{\shatabhishak} {\tiny \RIGHTarrow} 12:43\hspace{2ex}} 
&
\caldata{20}{06:35}{08:54}{12:23-13:50}{08:02-09:29}{18:11}{\textsf{\stra} {\tiny \RIGHTarrow} 05:55\hspace{2ex}}{\textsf{\proshthapada} {\tiny \RIGHTarrow} 06:16\hspace{2ex}} 
&
\caldata{21}{06:35}{08:54}{13:50-15:17}{06:35-08:02}{18:11}{\textsf{\schaturdashi} {\tiny \RIGHTarrow} 05:34(+1)}{\textsf{\uttaraproshthapada} {\tiny \RIGHTarrow} 18:04\hspace{2ex}} 
&
\caldata{22}{06:35}{08:54}{10:55-12:22}{15:16-16:43}{18:10}{\textsf{\purnima} {\tiny \RIGHTarrow} 07:06(+1)}{\textsf{\revati} {\tiny \RIGHTarrow} 20:17\hspace{2ex}} 
&
\caldata{23}{06:36}{08:54}{09:29-10:55}{13:49-15:15}{18:09}{\textsf{\purnima} {\tiny \RIGHTarrow} 07:06\hspace{2ex}}{\textsf{\ashwini} {\tiny \RIGHTarrow} 22:08\hspace{2ex}} 
\\ \hline
\caldata{24}{06:36}{08:54}{16:42-18:09}{12:22-13:49}{18:09}{\textsf{\kpra} {\tiny \RIGHTarrow} 08:14\hspace{2ex}}{\textsf{\apabharani} {\tiny \RIGHTarrow} 23:38\hspace{2ex}} 
&
\caldata{25}{06:36}{08:54}{08:02-09:29}{10:55-12:22}{18:08}{\textsf{\kdvi} {\tiny \RIGHTarrow} 09:00\hspace{2ex}}{\textsf{\krittika} {\tiny \RIGHTarrow} 00:46(+1)} 
&
\caldata{26}{06:37}{08:55}{15:14-16:40}{09:29-10:55}{18:07}{\textsf{\ktri} {\tiny \RIGHTarrow} 09:24\hspace{2ex}}{\textsf{\rohini} {\tiny \RIGHTarrow} 01:32(+1)} 
&
\caldata{27}{06:37}{08:55}{12:22-13:48}{08:03-09:29}{18:07}{\textsf{\kcha} {\tiny \RIGHTarrow} 09:26\hspace{2ex}}{\textsf{\mrigashirsha} {\tiny \RIGHTarrow} 01:55(+1)} 
&
\caldata{28}{06:38}{08:55}{13:48-15:14}{06:38-08:04}{18:06}{\textsf{\kpanc} {\tiny \RIGHTarrow} 09:03\hspace{2ex}}{\textsf{\ardra} {\tiny \RIGHTarrow} 01:54(+1)} 
&
\caldata{29}{06:38}{08:55}{10:56-12:22}{15:14-16:40}{18:06}{\textsf{\ksha} {\tiny \RIGHTarrow} 08:16\hspace{2ex}}{\textsf{\punarvasu} {\tiny \RIGHTarrow} 01:27(+1)} 
&
\caldata{30}{06:38}{08:55}{09:29-10:55}{13:47-15:13}{18:05}{\textsf{\ksap} {\tiny \RIGHTarrow} 07:02\hspace{2ex}}{\textsf{\pushya} {\tiny \RIGHTarrow} 00:33(+1)} 
\\ \hline
\caldata{31}{06:39}{08:56}{16:39-18:05}{12:22-13:47}{18:05}{\textsf{\knav} {\tiny \RIGHTarrow} 03:13(+1)}{\textsf{\ashresha} {\tiny \RIGHTarrow} 23:14\hspace{2ex}} 
&
{}  &
{}  &
{}  &
{}  &
{}  &
\\ \hline
\end{tabular}


%\clearpage
\begin{tabular}{|c|c|c|c|c|c|c|}
\multicolumn{7}{c}{\Large \bfseries NOVEMBER 2010}\\
\hline
\textbf{SUN} & \textbf{MON} & \textbf{TUE} & \textbf{WED} & \textbf{THU} & \textbf{FRI} & \textbf{SAT} \\ \hline
{}  &
\caldata{1}{06:39}{08:56}{08:04-09:30}{10:55-12:21}{18:04}{\textsf{\kdas} {\tiny \RIGHTarrow} 00:43(+1)}{\textsf{\magha} {\tiny \RIGHTarrow} 21:32\hspace{2ex}} 
&
\caldata{2}{06:40}{08:56}{15:13-16:38}{09:31-10:56}{18:04}{\textsf{\keka} {\tiny \RIGHTarrow} 21:56\hspace{2ex}}{\textsf{\purvaphalguni} {\tiny \RIGHTarrow} 19:32\hspace{2ex}} 
&
\caldata{3}{06:40}{08:56}{12:21-13:46}{08:05-09:30}{18:03}{\textsf{\kdva} {\tiny \RIGHTarrow} 18:59\hspace{2ex}}{\textsf{\uttaraphalguni} {\tiny \RIGHTarrow} 17:20\hspace{2ex}} 
&
\caldata{4}{06:41}{08:57}{13:47-15:12}{06:41-08:06}{18:03}{\textsf{\ktra} {\tiny \RIGHTarrow} 15:59\hspace{2ex}}{\textsf{\hasta} {\tiny \RIGHTarrow} 15:04\hspace{2ex}} 
&
\caldata{5}{06:41}{08:57}{10:56-12:21}{15:11-16:36}{18:02}{\textsf{\kchaturdashi} {\tiny \RIGHTarrow} 13:04\hspace{2ex}}{\textsf{\chitra} {\tiny \RIGHTarrow} 12:54\hspace{2ex}} 
&
\caldata{6}{06:42}{08:58}{09:32-10:57}{13:47-15:12}{18:02}{\textsf{\ama} {\tiny \RIGHTarrow} 10:23\hspace{2ex}}{\textsf{\svati} {\tiny \RIGHTarrow} 10:58\hspace{2ex}} 
\\ \hline
\caldata{7}{06:42}{08:57}{16:36-18:01}{12:21-13:46}{18:01}{\textsf{\spra} {\tiny \RIGHTarrow} 08:07\hspace{2ex}}{\textsf{\vishakha} {\tiny \RIGHTarrow} 09:26\hspace{2ex}} 
&
\caldata{8}{06:43}{08:58}{08:07-09:32}{10:57-12:22}{18:01}{\textsf{\stri} {\tiny \RIGHTarrow} 05:26(+1)}{\textsf{\anuradha} {\tiny \RIGHTarrow} 08:29\hspace{2ex}} 
&
\caldata{9}{06:43}{08:58}{15:11-16:36}{09:32-10:57}{18:01}{\textsf{\scha} {\tiny \RIGHTarrow} 05:13(+1)}{\textsf{\jyeshtha} {\tiny \RIGHTarrow} 08:13\hspace{2ex}} 
&
\caldata{10}{06:44}{08:59}{12:22-13:46}{08:08-09:33}{18:00}{\textsf{\spanc} {\tiny \RIGHTarrow} 05:47(+1)}{\textsf{\mula} {\tiny \RIGHTarrow} 08:42\hspace{2ex}} 
&
\caldata{11}{06:44}{08:59}{13:46-15:11}{06:44-08:08}{18:00}{\textsf{\ssha} {\tiny \RIGHTarrow} 07:04(+1)}{\textsf{\purvashadha} {\tiny \RIGHTarrow} 09:58\hspace{2ex}} 
&
\caldata{12}{06:45}{09:00}{10:58-12:22}{15:11-16:35}{18:00}{\textsf{\ssha} {\tiny \RIGHTarrow} 07:05\hspace{2ex}}{\textsf{\uttarashadha} {\tiny \RIGHTarrow} 11:56\hspace{2ex}} 
&
\caldata{13}{06:45}{09:00}{09:33-10:58}{13:46-15:11}{18:00}{\textsf{\ssap} {\tiny \RIGHTarrow} 09:01\hspace{2ex}}{\textsf{\shravana} {\tiny \RIGHTarrow} 14:26\hspace{2ex}} 
\\ \hline
\caldata{14}{06:46}{09:00}{16:34-17:59}{12:22-13:46}{17:59}{\textsf{\sasht} {\tiny \RIGHTarrow} 11:21\hspace{2ex}}{\textsf{\shravishtha} {\tiny \RIGHTarrow} 17:16\hspace{2ex}} 
&
\caldata{15}{06:46}{09:00}{08:10-09:34}{10:58-12:22}{17:59}{\textsf{\snav} {\tiny \RIGHTarrow} 13:52\hspace{2ex}}{\textsf{\shatabhishak} {\tiny \RIGHTarrow} 20:14\hspace{2ex}} 
&
\caldata{16}{06:47}{09:01}{15:11-16:35}{09:35-10:59}{17:59}{\textsf{\sdas} {\tiny \RIGHTarrow} 06:29\hspace{2ex}}{\textsf{\proshthapada} {\tiny \RIGHTarrow} 06:13\hspace{2ex}} 
&
\caldata{17}{06:47}{09:01}{12:23-13:47}{08:11-09:35}{17:59}{\textsf{\seka} {\tiny \RIGHTarrow} 18:33\hspace{2ex}}{\textsf{\uttaraproshthapada} {\tiny \RIGHTarrow} 01:43(+1)} 
&
\caldata{18}{06:48}{09:02}{13:47-15:11}{06:48-08:11}{17:59}{\textsf{\sdva} {\tiny \RIGHTarrow} 20:23\hspace{2ex}}{\textsf{\revati} {\tiny \RIGHTarrow} 03:55(+1)} 
&
\caldata{19}{06:49}{09:02}{10:59-12:23}{15:10-16:34}{17:58}{\textsf{\stra} {\tiny \RIGHTarrow} 21:43\hspace{2ex}}{\textsf{\ashwini} {\tiny \RIGHTarrow} 05:39(+1)} 
&
\caldata{20}{06:49}{09:02}{09:36-10:59}{13:47-15:10}{17:58}{\textsf{\schaturdashi} {\tiny \RIGHTarrow} 22:33\hspace{2ex}}{\textsf{\apabharani} {\tiny \RIGHTarrow} 06:54(+1)} 
\\ \hline
\caldata{21}{06:50}{09:03}{16:34-17:58}{12:24-13:47}{17:58}{\textsf{\purnima} {\tiny \RIGHTarrow} 22:54\hspace{2ex}}{\textsf{\apabharani} {\tiny \RIGHTarrow} 06:54\hspace{2ex}} 
&
\caldata{22}{06:50}{09:03}{08:13-09:37}{11:00-12:24}{17:58}{\textsf{\kpra} {\tiny \RIGHTarrow} 22:46\hspace{2ex}}{\textsf{\krittika} {\tiny \RIGHTarrow} 07:39\hspace{2ex}} 
&
\caldata{23}{06:51}{09:04}{15:11-16:34}{09:37-11:01}{17:58}{\textsf{\kdvi} {\tiny \RIGHTarrow} 22:15\hspace{2ex}}{\textsf{\rohini} {\tiny \RIGHTarrow} 07:58\hspace{2ex}} 
&
\caldata{24}{06:51}{09:04}{12:24-13:47}{08:14-09:37}{17:58}{\textsf{\ktri} {\tiny \RIGHTarrow} 21:22\hspace{2ex}}{\textsf{\mrigashirsha} {\tiny \RIGHTarrow} 07:55\hspace{2ex}} 
&
\caldata{25}{06:52}{09:05}{13:48-15:11}{06:52-08:15}{17:58}{\textsf{\kcha} {\tiny \RIGHTarrow} 20:10\hspace{2ex}}{\textsf{\ardra} {\tiny \RIGHTarrow} 07:33\hspace{2ex}} 
&
\caldata{26}{06:53}{09:06}{11:02-12:25}{15:11-16:34}{17:58}{\textsf{\kpanc} {\tiny \RIGHTarrow} 18:42\hspace{2ex}}{\textsf{\punarvasu} {\tiny \RIGHTarrow} 06:53\hspace{2ex}} 
&
\caldata{27}{06:53}{09:06}{09:39-11:02}{13:48-15:11}{17:58}{\textsf{\ksha} {\tiny \RIGHTarrow} 17:00\hspace{2ex}}{\textsf{\ashresha} {\tiny \RIGHTarrow} 04:50(+1)} 
\\ \hline
\caldata{28}{06:54}{09:06}{16:35-17:58}{12:26-13:49}{17:58}{\textsf{\ksap} {\tiny \RIGHTarrow} 15:06\hspace{2ex}}{\textsf{\magha} {\tiny \RIGHTarrow} 03:31(+1)} 
&
\caldata{29}{06:55}{09:07}{08:17-09:40}{11:03-12:26}{17:58}{\textsf{\kasht} {\tiny \RIGHTarrow} 13:01\hspace{2ex}}{\textsf{\purvaphalguni} {\tiny \RIGHTarrow} 02:02(+1)} 
&
\caldata{30}{06:55}{09:07}{15:12-16:35}{09:40-11:03}{17:58}{\textsf{\knav} {\tiny \RIGHTarrow} 10:49\hspace{2ex}}{\textsf{\uttaraphalguni} {\tiny \RIGHTarrow} 00:29(+1)} 
&
{}  &
{}  &
{}  &
\\ \hline
\end{tabular}


%\clearpage
\begin{tabular}{|c|c|c|c|c|c|c|}
\multicolumn{7}{c}{\Large \bfseries DECEMBER 2010}\\
\hline
\textbf{SUN} & \textbf{MON} & \textbf{TUE} & \textbf{WED} & \textbf{THU} & \textbf{FRI} & \textbf{SAT} \\ \hline
{}  &
{}  &
{}  &
\caldata{1}{06:56}{09:08}{12:27-13:49}{08:18-09:41}{17:58}{\textsf{\kdas} {\tiny \RIGHTarrow} 08:32\hspace{2ex}}{\textsf{\hasta} {\tiny \RIGHTarrow} 22:54\hspace{2ex}} 
&
\caldata{2}{06:56}{09:08}{13:50-15:13}{06:56-08:18}{17:59}{\textsf{\kdva} {\tiny \RIGHTarrow} 04:05(+1)}{\textsf{\chitra} {\tiny \RIGHTarrow} 21:23\hspace{2ex}} 
&
\caldata{3}{06:57}{09:09}{11:05-12:28}{15:13-16:36}{17:59}{\textsf{\ktra} {\tiny \RIGHTarrow} 02:06(+1)}{\textsf{\svati} {\tiny \RIGHTarrow} 20:01\hspace{2ex}} 
&
\caldata{4}{06:58}{09:10}{09:43-11:05}{13:51-15:13}{17:59}{\textsf{\kchaturdashi} {\tiny \RIGHTarrow} 00:25(+1)}{\textsf{\vishakha} {\tiny \RIGHTarrow} 18:55\hspace{2ex}} 
\\ \hline
\caldata{5}{06:58}{09:10}{16:36-17:59}{12:28-13:51}{17:59}{\textsf{\ama} {\tiny \RIGHTarrow} 23:09\hspace{2ex}}{\textsf{\anuradha} {\tiny \RIGHTarrow} 18:12\hspace{2ex}} 
&
\caldata{6}{06:59}{09:11}{08:21-09:44}{11:06-12:29}{17:59}{\textsf{\spra} {\tiny \RIGHTarrow} 22:24\hspace{2ex}}{\textsf{\jyeshtha} {\tiny \RIGHTarrow} 17:58\hspace{2ex}} 
&
\caldata{7}{06:59}{09:11}{15:14-16:37}{09:44-11:06}{18:00}{\textsf{\sdvi} {\tiny \RIGHTarrow} 22:14\hspace{2ex}}{\textsf{\mula} {\tiny \RIGHTarrow} 18:18\hspace{2ex}} 
&
\caldata{8}{07:00}{09:12}{12:30-13:52}{08:22-09:45}{18:00}{\textsf{\stri} {\tiny \RIGHTarrow} 22:43\hspace{2ex}}{\textsf{\purvashadha} {\tiny \RIGHTarrow} 19:15\hspace{2ex}} 
&
\caldata{9}{07:01}{09:12}{13:52-15:15}{07:01-08:23}{18:00}{\textsf{\scha} {\tiny \RIGHTarrow} 23:50\hspace{2ex}}{\textsf{\uttarashadha} {\tiny \RIGHTarrow} 20:49\hspace{2ex}} 
&
\caldata{10}{07:01}{09:12}{11:08-12:30}{15:15-16:37}{18:00}{\textsf{\spanc} {\tiny \RIGHTarrow} 01:33(+1)}{\textsf{\shravana} {\tiny \RIGHTarrow} 22:57\hspace{2ex}} 
&
\caldata{11}{07:02}{09:13}{09:46-11:09}{13:53-15:16}{18:01}{\textsf{\ssha} {\tiny \RIGHTarrow} 03:44(+1)}{\textsf{\shravishtha} {\tiny \RIGHTarrow} 01:33(+1)} 
\\ \hline
\caldata{12}{07:02}{09:13}{16:38-18:01}{12:31-13:53}{18:01}{\textsf{\ssap} {\tiny \RIGHTarrow} 06:12(+1)}{\textsf{\shatabhishak} {\tiny \RIGHTarrow} 04:25(+1)} 
&
\caldata{13}{07:03}{09:14}{08:25-09:47}{11:09-12:32}{18:01}{\textsf{\sasht} {\tiny \RIGHTarrow} 08:44(+1)}{\textsf{\proshthapada} {\tiny \RIGHTarrow} 07:23(+1)} 
&
\caldata{14}{07:04}{09:15}{15:17-16:39}{09:48-11:10}{18:02}{\textsf{\sasht} {\tiny \RIGHTarrow} 07:01\hspace{2ex}}{\textsf{\proshthapada} {\tiny \RIGHTarrow} 07:03\hspace{2ex}} 
&
\caldata{15}{07:04}{09:15}{12:33-13:55}{08:26-09:48}{18:02}{\textsf{\snav} {\tiny \RIGHTarrow} 11:03\hspace{2ex}}{\textsf{\uttaraproshthapada} {\tiny \RIGHTarrow} 10:09\hspace{2ex}} 
&
\caldata{16}{07:05}{09:16}{13:56-15:18}{07:05-08:27}{18:03}{\textsf{\sdas} {\tiny \RIGHTarrow} 12:59\hspace{2ex}}{\textsf{\revati} {\tiny \RIGHTarrow} 12:33\hspace{2ex}} 
&
\caldata{17}{07:05}{09:16}{11:11-12:34}{15:18-16:40}{18:03}{\textsf{\seka} {\tiny \RIGHTarrow} 14:22\hspace{2ex}}{\textsf{\ashwini} {\tiny \RIGHTarrow} 14:27\hspace{2ex}} 
&
\caldata{18}{07:06}{09:17}{09:50-11:12}{13:56-15:18}{18:03}{\textsf{\sdva} {\tiny \RIGHTarrow} 15:08\hspace{2ex}}{\textsf{\apabharani} {\tiny \RIGHTarrow} 15:44\hspace{2ex}} 
\\ \hline
\caldata{19}{07:06}{09:17}{16:41-18:04}{12:35-13:57}{18:04}{\textsf{\stra} {\tiny \RIGHTarrow} 15:14\hspace{2ex}}{\textsf{\krittika} {\tiny \RIGHTarrow} 16:24\hspace{2ex}} 
&
\caldata{20}{07:07}{09:18}{08:29-09:51}{11:13-12:35}{18:04}{\textsf{\schaturdashi} {\tiny \RIGHTarrow} 14:44\hspace{2ex}}{\textsf{\rohini} {\tiny \RIGHTarrow} 16:28\hspace{2ex}} 
&
\caldata{21}{07:07}{09:18}{15:20-16:42}{09:51-11:13}{18:05}{\textsf{\purnima} {\tiny \RIGHTarrow} 13:40\hspace{2ex}}{\textsf{\mrigashirsha} {\tiny \RIGHTarrow} 15:59\hspace{2ex}} 
&
\caldata{22}{07:08}{09:19}{12:36-13:58}{08:30-09:52}{18:05}{\textsf{\kpra} {\tiny \RIGHTarrow} 12:08\hspace{2ex}}{\textsf{\ardra} {\tiny \RIGHTarrow} 15:04\hspace{2ex}} 
&
\caldata{23}{07:08}{09:19}{13:59-15:21}{07:08-08:30}{18:06}{\textsf{\kdvi} {\tiny \RIGHTarrow} 10:15\hspace{2ex}}{\textsf{\punarvasu} {\tiny \RIGHTarrow} 13:49\hspace{2ex}} 
&
\caldata{24}{07:09}{09:20}{11:15-12:37}{15:21-16:43}{18:06}{\textsf{\ktri} {\tiny \RIGHTarrow} 08:06\hspace{2ex}}{\textsf{\pushya} {\tiny \RIGHTarrow} 12:19\hspace{2ex}} 
&
\caldata{25}{07:09}{09:20}{09:53-11:15}{14:00-15:22}{18:07}{\textsf{\kpanc} {\tiny \RIGHTarrow} 03:25(+1)}{\textsf{\ashresha} {\tiny \RIGHTarrow} 10:42\hspace{2ex}} 
\\ \hline
\caldata{26}{07:10}{09:21}{16:44-18:07}{12:38-14:00}{18:07}{\textsf{\ksha} {\tiny \RIGHTarrow} 01:06(+1)}{\textsf{\magha} {\tiny \RIGHTarrow} 09:02\hspace{2ex}} 
&
\caldata{27}{07:10}{09:21}{08:32-09:54}{11:16-12:39}{18:08}{\textsf{\ksap} {\tiny \RIGHTarrow} 22:53\hspace{2ex}}{\textsf{\purvaphalguni} {\tiny \RIGHTarrow} 07:24\hspace{2ex}} 
&
\caldata{28}{07:11}{09:22}{15:24-16:46}{09:55-11:17}{18:09}{\textsf{\kasht} {\tiny \RIGHTarrow} 20:49\hspace{2ex}}{\textsf{\hasta} {\tiny \RIGHTarrow} 04:36(+1)} 
&
\caldata{29}{07:11}{09:22}{12:40-14:02}{08:33-09:55}{18:09}{\textsf{\knav} {\tiny \RIGHTarrow} 18:58\hspace{2ex}}{\textsf{\chitra} {\tiny \RIGHTarrow} 03:30(+1)} 
&
\caldata{30}{07:11}{09:22}{14:02-15:25}{07:11-08:33}{18:10}{\textsf{\kdas} {\tiny \RIGHTarrow} 17:23\hspace{2ex}}{\textsf{\svati} {\tiny \RIGHTarrow} 02:41(+1)} 
&
\caldata{31}{07:12}{09:23}{11:18-12:41}{15:25-16:47}{18:10}{\textsf{\keka} {\tiny \RIGHTarrow} 16:04\hspace{2ex}}{\textsf{\vishakha} {\tiny \RIGHTarrow} 02:09(+1)} 
&
\\ \hline
\end{tabular}


%\clearpage

\end{center}
\end{document}
