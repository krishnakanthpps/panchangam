\documentclass[a4paper,11pt,landscape]{article}
\usepackage[sort&compress,square,numbers]{natbib}

\usepackage[xetex]{graphicx}
%\usepackage{fullpage}
\usepackage{multirow}
\usepackage[normalsections]{savetrees}
\usepackage{euler}
\usepackage{fontspec}
\usepackage{xltxtra}
\usepackage{url}
\usepackage{multicol}
\usepackage{bbding}
% PDF SETUP
% ---- FILL IN HERE THE DOC TITLE AND AUTHOR
%\usepackage[bookmarks, colorlinks, breaklinks, pdftitle={Karthik Raman - vita},pdfauthor={Karthik Raman}]{hyperref} 
\usepackage[dvipsnames]{xcolor} 
\usepackage{wasysym} 
%\hypersetup{linkcolor=Sepia,citecolor=blue,filecolor=black,urlcolor=Blue} 


\defaultfontfeatures{Scale=MatchLowercase,Mapping=tex-text}
\setmainfont{Scala Sans LF}
\setsansfont{Sanskrit 2003:script=deva}

%%%%%%% Numbers and counters %%%%%%%
\newcount\num
\newcount\tempone \newcount\temptwo
\newcommand{\devanumber}[1]{%
\num=#1\devanumberrecurse}

\newcommand{\devanumberrecurse}{%
{\tempone=\num
%  \showthe\tempone\ %
\ifnum\num > 0 
	\divide \num by 10%
	\temptwo=\num \multiply\temptwo by -10%
	\devanumberrecurse%
% 	\\stage 2\ %
% 	\showthe\temptwo\ %
% 	temp1=\number\tempone\ %
% 	num=\number\num\ %
	\advance\tempone by \temptwo%
	\devadigit
\fi
}}

\newcommand{\devadigit}{%
\ifcase\tempone०\or१\or२\or३\or४\or५\or६\or७\or८\or९\fi%\number\tempone%
}

%%%%%%%% Calendar display stuff %%%%%%%%%%%
\newcommand{\sunmonth}[3]{%
\footnotesize\textsf{#1~\devanumber{#2}}\\
\mbox{\scriptsize #3}
}
\newcommand{\sundata}[3]{%
\mbox{{\sun\tiny\UParrow} \scriptsize #1}\\
\mbox{{\sun\tiny\DOWNarrow} \scriptsize  #2}~\scriptsize{\mbox{(\textsf{स} #3)}}
}
\newcommand{\caldata}[7]{%
\begin{minipage}{3.15cm}
\begin{minipage}[t]{1.95cm}
#2\\
%\mbox{}\\
%\SunshineOpenCircled
#3\\
\mbox{#4}\\
\mbox{#5}\\
\mbox{\textsf{राहु~} #6}\\
\mbox{\textsf{यम~} #7}\\
\end{minipage}\begin{minipage}[c]{1.2cm}
\vspace{.4ex}
\hfill \mbox{\textcolor{blue}{\font\x="Minion Pro" at 32 pt\x #1}\hspace{-2mm}}
\end{minipage}
\end{minipage}
}

\addtolength{\headsep}{-3ex}
\pagestyle{empty}
\newcommand{\ahoratram}{\textsf{अहोरात्रम्}}
\newcommand{\ashwini}{अश्विनी}
\newcommand{\apabharani}{अपभरणी}
\newcommand{\krittika}{कृत्तिका}
\newcommand{\rohini}{रोहिणी}
\newcommand{\mrigashirsha}{मृगशीर्ष}
\newcommand{\ardra}{आर्द्रा}
\newcommand{\punarvasu}{पुनर्वसू}
\newcommand{\pushya}{पुष्य}
\newcommand{\ashresha}{आश्रेषा}
\newcommand{\magha}{मघा}
\newcommand{\purvaphalguni}{पूर्वफल्गुनी}
\newcommand{\uttaraphalguni}{उत्तरफल्गुनी}
\newcommand{\hasta}{हस्त}
\newcommand{\chitra}{चित्रा}
\newcommand{\svati}{स्वाति}
\newcommand{\vishakha}{विशाखा}
\newcommand{\anuradha}{अनूराधा}
\newcommand{\jyeshtha}{ज्येष्ठा}
\newcommand{\mula}{मूला}
\newcommand{\purvashadha}{पूर्वाषाढा}
\newcommand{\uttarashadha}{उत्तराषाढा}
\newcommand{\shravana}{श्रवण}
\newcommand{\shravishtha}{श्रविष्ठा}
\newcommand{\shatabhishak}{शतभिषक्}
\newcommand{\proshthapada}{प्रोष्ठपदा}
\newcommand{\uttaraproshthapada}{उत्तरप्रोष्ठपदा}
\newcommand{\revati}{रेवती}

\newcommand{\shukla}{शुक्ल}
\newcommand{\spra}{शुक्ल प्रथमा}
\newcommand{\sdvi}{\shukla~द्वितीया}
\newcommand{\stri}{\shukla~तृतीया}
\newcommand{\scha}{\shukla~चतुर्थी}
\newcommand{\spanc}{\shukla~पञ्चमी}
\newcommand{\ssha}{\shukla~षष्ठी}
\newcommand{\ssap}{\shukla~सप्तमी}
\newcommand{\sasht}{\shukla~अष्टमी}
\newcommand{\snav}{\shukla~नवमी}
\newcommand{\sdas}{\shukla~दशमी}
\newcommand{\seka}{\shukla~एकादशी}
\newcommand{\sdva}{\shukla~द्वादशी}
\newcommand{\stra}{\shukla~त्रयोदशी}
\newcommand{\schaturdashi}{\shukla~चतुर्दशी}

\newcommand{\purnima}{\fullmoon~पूर्णिमा}

\newcommand{\krishna}{कृष्ण}
\newcommand{\kpra}{\krishna~प्रथमा}
\newcommand{\kdvi}{\krishna~द्वितीया}
\newcommand{\ktri}{\krishna~तृतीया}
\newcommand{\kcha}{\krishna~चतुर्थी}
\newcommand{\kpanc}{\krishna~पञ्चमी}
\newcommand{\ksha}{\krishna~षष्ठी}
\newcommand{\ksap}{\krishna~सप्तमी}
\newcommand{\kasht}{\krishna~अष्टमी}
\newcommand{\knav}{\krishna~नवमी}
\newcommand{\kdas}{\krishna~दशमी}
\newcommand{\keka}{\krishna~एकादशी}
\newcommand{\kdva}{\krishna~द्वादशी}
\newcommand{\ktra}{\krishna~त्रयोदशी}
\newcommand{\kchaturdashi}{\krishna~चतुर्दशी}
\newcommand{\ama}{\newmoon~अमावस्या}

\newcommand{\mesha}{मेष}
\newcommand{\vrishabha}{वृषभ}
\newcommand{\mithuna}{मिथुन}
\newcommand{\karkataka}{कर्कटक}
\newcommand{\simha}{सिंह}
\newcommand{\kanya}{कन्या}
\newcommand{\tula}{तुला}
\newcommand{\vrishchika}{वृश्चिक}
\newcommand{\dhanur}{धनुः}
\newcommand{\makara}{मकर}
\newcommand{\kumbha}{कुम्भ}
\newcommand{\mina}{मीन}

\newcommand{\SUNDAY}{SUN}
\newcommand{\MONDAY}{MON}
\newcommand{\TUESDAY}{TUE}
\newcommand{\WEDNESDAY}{WED}
\newcommand{\THURSDAY}{THU}
\newcommand{\FRIDAY}{FRI}
\newcommand{\SATURDAY}{SAT}

\begin{document}
\pagestyle{empty}
\begin{center}
\mbox{}\\[2.5in]
\hrule\mbox{}
\mbox{}\\[1ex]
\mbox{}
{\font\x="Warnock Pro" at 60 pt\x 2010\\[0.5cm]}
\mbox{}
{\font\x="Warnock Pro" at 48 pt\x \uppercase{Zurich}\\[0.3cm]}
\hrule
%\sunmonth{\dhanur}{17}{}

\begin{tabular}{|c|c|c|c|c|c|c|}
\multicolumn{7}{c}{\Large \bfseries JANUARY 2010}\\
\hline
\textbf{SUN} & \textbf{MON} & \textbf{TUE} & \textbf{WED} & \textbf{THU} & \textbf{FRI} & \textbf{SAT} \\ \hline
{}  &
{}  &
{}  &
{}  &
{}  &
\caldata{1}{08:14}{09:56}{11:25-12:29}{14:36-15:40}{16:44}{\textsf{\kpra} {\tiny \RIGHTarrow} 16:43\hspace{2ex}}{\textsf{\punarvasu} {\tiny \RIGHTarrow} 23:26\hspace{2ex}} 
&
%\sunmonth{\dhanur}{18}{}

\caldata{2}{08:14}{09:56}{10:21-11:25}{13:33-14:37}{16:45}{\textsf{\kdvi} {\tiny \RIGHTarrow} 13:14\hspace{2ex}}{\textsf{\pushya} {\tiny \RIGHTarrow} 20:43\hspace{2ex}} 
\\ \hline
%\sunmonth{\dhanur}{19}{}

\caldata{3}{08:14}{09:56}{15:42-16:46}{12:30-13:34}{16:46}{\textsf{\ktri} {\tiny \RIGHTarrow} 09:51\hspace{2ex}}{\textsf{\ashresha} {\tiny \RIGHTarrow} 18:10\hspace{2ex}} 
&
%\sunmonth{\dhanur}{20}{}

\caldata{4}{08:14}{09:56}{09:18-10:22}{11:26-12:30}{16:47}{\textsf{\kpanc} {\tiny \RIGHTarrow} 04:04(+1)}{\textsf{\magha} {\tiny \RIGHTarrow} 15:56\hspace{2ex}} 
&
%\sunmonth{\dhanur}{21}{}

\caldata{5}{08:14}{09:56}{14:39-15:43}{10:22-11:26}{16:48}{\textsf{\ksha} {\tiny \RIGHTarrow} 01:53(+1)}{\textsf{\purvaphalguni} {\tiny \RIGHTarrow} 14:09\hspace{2ex}} 
&
%\sunmonth{\dhanur}{22}{}

\caldata{6}{08:14}{09:57}{12:31-13:35}{09:18-10:22}{16:49}{\textsf{\ksap} {\tiny \RIGHTarrow} 00:18(+1)}{\textsf{\uttaraphalguni} {\tiny \RIGHTarrow} 12:54\hspace{2ex}} 
&
%\sunmonth{\dhanur}{23}{}

\caldata{7}{08:13}{09:56}{13:36-14:40}{08:13-09:17}{16:50}{\textsf{\kasht} {\tiny \RIGHTarrow} 23:20\hspace{2ex}}{\textsf{\hasta} {\tiny \RIGHTarrow} 12:15\hspace{2ex}} 
&
%\sunmonth{\dhanur}{24}{}

\caldata{8}{08:13}{09:56}{11:27-12:32}{14:41-15:46}{16:51}{\textsf{\knav} {\tiny \RIGHTarrow} 23:01\hspace{2ex}}{\textsf{\chitra} {\tiny \RIGHTarrow} 12:15\hspace{2ex}} 
&
%\sunmonth{\dhanur}{25}{}

\caldata{9}{08:13}{09:56}{10:22-11:27}{13:37-14:42}{16:52}{\textsf{\kdas} {\tiny \RIGHTarrow} 23:20\hspace{2ex}}{\textsf{\svati} {\tiny \RIGHTarrow} 12:52\hspace{2ex}} 
\\ \hline
%\sunmonth{\dhanur}{26}{}

\caldata{10}{08:12}{09:56}{15:48-16:54}{12:33-13:38}{16:54}{\textsf{\keka} {\tiny \RIGHTarrow} 00:14(+1)}{\textsf{\vishakha} {\tiny \RIGHTarrow} 14:05\hspace{2ex}} 
&
%\sunmonth{\dhanur}{27}{}

\caldata{11}{08:12}{09:56}{09:17-10:22}{11:28-12:33}{16:55}{\textsf{\kdva} {\tiny \RIGHTarrow} 01:38(+1)}{\textsf{\anuradha} {\tiny \RIGHTarrow} 15:50\hspace{2ex}} 
&
%\sunmonth{\dhanur}{28}{}

\caldata{12}{08:11}{09:56}{14:44-15:50}{10:22-11:27}{16:56}{\textsf{\ktra} {\tiny \RIGHTarrow} 03:30(+1)}{\textsf{\jyeshtha} {\tiny \RIGHTarrow} 18:01\hspace{2ex}} 
&
%\sunmonth{\makara}{1}{(08:01:(+1))}

\caldata{13}{08:11}{09:56}{12:34-13:39}{09:16-10:22}{16:57}{\textsf{\kchaturdashi} {\tiny \RIGHTarrow} 05:42(+1)}{\textsf{\mula} {\tiny \RIGHTarrow} 20:33\hspace{2ex}} 
&
%\sunmonth{\makara}{2}{}

\caldata{14}{08:10}{09:55}{13:40-14:46}{08:10-09:16}{16:59}{\textsf{\ama} {\tiny \RIGHTarrow} \ahoratram}{\textsf{\purvashadha} {\tiny \RIGHTarrow} 23:23\hspace{2ex}} 
&
%\sunmonth{\makara}{3}{}

\caldata{15}{08:10}{09:56}{11:28-12:35}{14:47-15:53}{17:00}{\textsf{\ama} {\tiny \RIGHTarrow} 08:11\hspace{2ex}}{\textsf{\uttarashadha} {\tiny \RIGHTarrow} 02:24(+1)} 
&
%\sunmonth{\makara}{4}{}

\caldata{16}{08:09}{09:55}{10:22-11:28}{13:41-14:48}{17:01}{\textsf{\spra} {\tiny \RIGHTarrow} 10:51\hspace{2ex}}{\textsf{\shravana} {\tiny \RIGHTarrow} 05:31(+1)} 
\\ \hline
%\sunmonth{\makara}{5}{}

\caldata{17}{08:08}{09:55}{15:56-17:03}{12:35-13:42}{17:03}{\textsf{\sdvi} {\tiny \RIGHTarrow} 13:34\hspace{2ex}}{\textsf{\shravishtha} {\tiny \RIGHTarrow} \ahoratram} 
&
%\sunmonth{\makara}{6}{}

\caldata{18}{08:08}{09:55}{09:15-10:22}{11:29-12:36}{17:04}{\textsf{\stri} {\tiny \RIGHTarrow} 16:14\hspace{2ex}}{\textsf{\shravishtha} {\tiny \RIGHTarrow} 08:37\hspace{2ex}} 
&
%\sunmonth{\makara}{7}{}

\caldata{19}{08:07}{09:54}{14:51-15:58}{10:21-11:29}{17:06}{\textsf{\scha} {\tiny \RIGHTarrow} 18:42\hspace{2ex}}{\textsf{\shatabhishak} {\tiny \RIGHTarrow} 11:35\hspace{2ex}} 
&
%\sunmonth{\makara}{8}{}

\caldata{20}{08:06}{09:54}{12:36-13:44}{09:13-10:21}{17:07}{\textsf{\spanc} {\tiny \RIGHTarrow} 20:51\hspace{2ex}}{\textsf{\proshthapada} {\tiny \RIGHTarrow} 14:16\hspace{2ex}} 
&
%\sunmonth{\makara}{9}{}

\caldata{21}{08:05}{09:53}{13:45-14:53}{08:05-09:13}{17:09}{\textsf{\ssha} {\tiny \RIGHTarrow} 22:30\hspace{2ex}}{\textsf{\uttaraproshthapada} {\tiny \RIGHTarrow} 16:31\hspace{2ex}} 
&
%\sunmonth{\makara}{10}{}

\caldata{22}{08:04}{09:53}{11:28-12:37}{14:53-16:01}{17:10}{\textsf{\ssap} {\tiny \RIGHTarrow} 23:33\hspace{2ex}}{\textsf{\revati} {\tiny \RIGHTarrow} 18:13\hspace{2ex}} 
&
%\sunmonth{\makara}{11}{}

\caldata{23}{08:03}{09:52}{10:20-11:28}{13:46-14:54}{17:12}{\textsf{\sasht} {\tiny \RIGHTarrow} 23:52\hspace{2ex}}{\textsf{\ashwini} {\tiny \RIGHTarrow} 19:14\hspace{2ex}} 
\\ \hline
%\sunmonth{\makara}{12}{}

\caldata{24}{08:02}{09:52}{16:04-17:13}{12:37-13:46}{17:13}{\textsf{\snav} {\tiny \RIGHTarrow} 23:25\hspace{2ex}}{\textsf{\apabharani} {\tiny \RIGHTarrow} 19:32\hspace{2ex}} 
&
%\sunmonth{\makara}{13}{}

\caldata{25}{08:01}{09:51}{09:10-10:19}{11:28-12:38}{17:15}{\textsf{\sdas} {\tiny \RIGHTarrow} 22:11\hspace{2ex}}{\textsf{\krittika} {\tiny \RIGHTarrow} 19:04\hspace{2ex}} 
&
%\sunmonth{\makara}{14}{}

\caldata{26}{08:00}{09:51}{14:57-16:06}{10:19-11:28}{17:16}{\textsf{\seka} {\tiny \RIGHTarrow} 20:13\hspace{2ex}}{\textsf{\rohini} {\tiny \RIGHTarrow} 17:52\hspace{2ex}} 
&
%\sunmonth{\makara}{15}{}

\caldata{27}{07:59}{09:50}{12:38-13:48}{09:08-10:18}{17:18}{\textsf{\sdva} {\tiny \RIGHTarrow} 17:37\hspace{2ex}}{\textsf{\mrigashirsha} {\tiny \RIGHTarrow} 16:02\hspace{2ex}} 
&
%\sunmonth{\makara}{16}{}

\caldata{28}{07:58}{09:50}{13:48-14:58}{07:58-09:08}{17:19}{\textsf{\stra} {\tiny \RIGHTarrow} 14:30\hspace{2ex}}{\textsf{\ardra} {\tiny \RIGHTarrow} 13:41\hspace{2ex}} 
&
%\sunmonth{\makara}{17}{}

\caldata{29}{07:57}{09:49}{11:28-12:39}{15:00-16:10}{17:21}{\textsf{\schaturdashi} {\tiny \RIGHTarrow} 11:01\hspace{2ex}}{\textsf{\punarvasu} {\tiny \RIGHTarrow} 10:56\hspace{2ex}} 
&
%\sunmonth{\makara}{18}{}

\caldata{30}{07:56}{09:49}{10:17-11:28}{13:49-15:00}{17:22}{\textsf{\kpra} {\tiny \RIGHTarrow} 03:30(+1)}{\textsf{\pushya} {\tiny \RIGHTarrow} 07:57\hspace{2ex}} 
\\ \hline
%\sunmonth{\makara}{19}{}

\caldata{31}{07:54}{09:48}{16:12-17:24}{12:39-13:50}{17:24}{\textsf{\kdvi} {\tiny \RIGHTarrow} 23:52\hspace{2ex}}{\textsf{\magha} {\tiny \RIGHTarrow} 02:02(+1)} 
&
%\sunmonth{\makara}{20}{}

{}  &
{}  &
{}  &
{}  &
{}  &
\\ \hline
\end{tabular}


%\clearpage
\begin{tabular}{|c|c|c|c|c|c|c|}
\multicolumn{7}{c}{\Large \bfseries FEBRUARY 2010}\\
\hline
\textbf{SUN} & \textbf{MON} & \textbf{TUE} & \textbf{WED} & \textbf{THU} & \textbf{FRI} & \textbf{SAT} \\ \hline
{}  &
\caldata{1}{07:53}{09:47}{09:04-10:16}{11:27-12:39}{17:25}{\textsf{\ktri} {\tiny \RIGHTarrow} 20:31\hspace{2ex}}{\textsf{\purvaphalguni} {\tiny \RIGHTarrow} 23:29\hspace{2ex}} 
&
%\sunmonth{\makara}{21}{}

\caldata{2}{07:52}{09:47}{15:03-16:15}{10:15-11:27}{17:27}{\textsf{\kcha} {\tiny \RIGHTarrow} 17:38\hspace{2ex}}{\textsf{\uttaraphalguni} {\tiny \RIGHTarrow} 21:24\hspace{2ex}} 
&
%\sunmonth{\makara}{22}{}

\caldata{3}{07:50}{09:45}{12:39-13:51}{09:02-10:14}{17:28}{\textsf{\kpanc} {\tiny \RIGHTarrow} 15:19\hspace{2ex}}{\textsf{\hasta} {\tiny \RIGHTarrow} 19:56\hspace{2ex}} 
&
%\sunmonth{\makara}{23}{}

\caldata{4}{07:49}{09:45}{13:52-15:04}{07:49-09:01}{17:30}{\textsf{\ksha} {\tiny \RIGHTarrow} 13:44\hspace{2ex}}{\textsf{\chitra} {\tiny \RIGHTarrow} 19:12\hspace{2ex}} 
&
%\sunmonth{\makara}{24}{}

\caldata{5}{07:48}{09:44}{11:27-12:40}{15:06-16:19}{17:32}{\textsf{\ksap} {\tiny \RIGHTarrow} 12:57\hspace{2ex}}{\textsf{\svati} {\tiny \RIGHTarrow} 19:14\hspace{2ex}} 
&
%\sunmonth{\makara}{25}{}

\caldata{6}{07:46}{09:43}{10:12-11:26}{13:52-15:06}{17:33}{\textsf{\kasht} {\tiny \RIGHTarrow} 12:59\hspace{2ex}}{\textsf{\vishakha} {\tiny \RIGHTarrow} 20:04\hspace{2ex}} 
\\ \hline
%\sunmonth{\makara}{26}{}

\caldata{7}{07:45}{09:43}{16:21-17:35}{12:40-13:53}{17:35}{\textsf{\knav} {\tiny \RIGHTarrow} 13:49\hspace{2ex}}{\textsf{\anuradha} {\tiny \RIGHTarrow} 21:38\hspace{2ex}} 
&
%\sunmonth{\makara}{27}{}

\caldata{8}{07:43}{09:41}{08:57-10:11}{11:25-12:39}{17:36}{\textsf{\kdas} {\tiny \RIGHTarrow} 15:19\hspace{2ex}}{\textsf{\jyeshtha} {\tiny \RIGHTarrow} 23:48\hspace{2ex}} 
&
%\sunmonth{\makara}{28}{}

\caldata{9}{07:42}{09:41}{15:09-16:23}{10:11-11:25}{17:38}{\textsf{\keka} {\tiny \RIGHTarrow} 17:23\hspace{2ex}}{\textsf{\mula} {\tiny \RIGHTarrow} 02:27(+1)} 
&
%\sunmonth{\makara}{29}{}

\caldata{10}{07:40}{09:39}{12:39-13:54}{08:54-10:09}{17:39}{\textsf{\kdva} {\tiny \RIGHTarrow} 19:48\hspace{2ex}}{\textsf{\purvashadha} {\tiny \RIGHTarrow} 05:25(+1)} 
&
%\sunmonth{\makara}{30}{}

\caldata{11}{07:39}{09:39}{13:55-15:10}{07:39-08:54}{17:41}{\textsf{\ktra} {\tiny \RIGHTarrow} 22:27\hspace{2ex}}{\textsf{\uttarashadha} {\tiny \RIGHTarrow} \ahoratram} 
&
%\sunmonth{\kumbha}{1}{(21:00:\hspace{2ex})}

\caldata{12}{07:37}{09:38}{11:24-12:40}{15:11-16:27}{17:43}{\textsf{\kchaturdashi} {\tiny \RIGHTarrow} 01:10(+1)}{\textsf{\uttarashadha} {\tiny \RIGHTarrow} 08:32\hspace{2ex}} 
&
%\sunmonth{\kumbha}{2}{}

\caldata{13}{07:36}{09:37}{10:08-11:24}{13:56-15:12}{17:44}{\textsf{\ama} {\tiny \RIGHTarrow} 03:50(+1)}{\textsf{\shravana} {\tiny \RIGHTarrow} 11:40\hspace{2ex}} 
\\ \hline
%\sunmonth{\kumbha}{3}{}

\caldata{14}{07:34}{09:36}{16:29-17:46}{12:40-13:56}{17:46}{\textsf{\spra} {\tiny \RIGHTarrow} 06:21(+1)}{\textsf{\shravishtha} {\tiny \RIGHTarrow} 14:42\hspace{2ex}} 
&
%\sunmonth{\kumbha}{4}{}

\caldata{15}{07:32}{09:35}{08:48-10:05}{11:22-12:39}{17:47}{\textsf{\sdvi} {\tiny \RIGHTarrow} \ahoratram}{\textsf{\shatabhishak} {\tiny \RIGHTarrow} 17:34\hspace{2ex}} 
&
%\sunmonth{\kumbha}{5}{}

\caldata{16}{07:31}{09:34}{15:14-16:31}{10:05-11:22}{17:49}{\textsf{\sdvi} {\tiny \RIGHTarrow} 08:38\hspace{2ex}}{\textsf{\proshthapada} {\tiny \RIGHTarrow} 20:10\hspace{2ex}} 
&
%\sunmonth{\kumbha}{6}{}

\caldata{17}{07:29}{09:33}{12:39-13:57}{08:46-10:04}{17:50}{\textsf{\stri} {\tiny \RIGHTarrow} 10:35\hspace{2ex}}{\textsf{\uttaraproshthapada} {\tiny \RIGHTarrow} 22:28\hspace{2ex}} 
&
%\sunmonth{\kumbha}{7}{}

\caldata{18}{07:27}{09:32}{13:57-15:15}{07:27-08:45}{17:52}{\textsf{\scha} {\tiny \RIGHTarrow} 12:10\hspace{2ex}}{\textsf{\revati} {\tiny \RIGHTarrow} 00:22(+1)} 
&
%\sunmonth{\kumbha}{8}{}

\caldata{19}{07:26}{09:31}{11:21-12:39}{15:16-16:34}{17:53}{\textsf{\spanc} {\tiny \RIGHTarrow} 13:17\hspace{2ex}}{\textsf{\ashwini} {\tiny \RIGHTarrow} 01:47(+1)} 
&
%\sunmonth{\kumbha}{9}{}

\caldata{20}{07:24}{09:30}{10:01-11:20}{13:58-15:17}{17:55}{\textsf{\ssha} {\tiny \RIGHTarrow} 13:52\hspace{2ex}}{\textsf{\apabharani} {\tiny \RIGHTarrow} 02:40(+1)} 
\\ \hline
%\sunmonth{\kumbha}{10}{}

\caldata{21}{07:22}{09:29}{16:37-17:57}{12:39-13:58}{17:57}{\textsf{\ssap} {\tiny \RIGHTarrow} 13:52\hspace{2ex}}{\textsf{\krittika} {\tiny \RIGHTarrow} 02:56(+1)} 
&
%\sunmonth{\kumbha}{11}{}

\caldata{22}{07:20}{09:27}{08:39-09:59}{11:19-12:39}{17:58}{\textsf{\sasht} {\tiny \RIGHTarrow} 13:13\hspace{2ex}}{\textsf{\rohini} {\tiny \RIGHTarrow} 02:34(+1)} 
&
%\sunmonth{\kumbha}{12}{}

\caldata{23}{07:19}{09:27}{15:19-16:39}{09:59-11:19}{18:00}{\textsf{\snav} {\tiny \RIGHTarrow} 11:55\hspace{2ex}}{\textsf{\mrigashirsha} {\tiny \RIGHTarrow} 01:32(+1)} 
&
%\sunmonth{\kumbha}{13}{}

\caldata{24}{07:17}{09:25}{12:39-13:59}{08:37-09:58}{18:01}{\textsf{\sdas} {\tiny \RIGHTarrow} 09:59\hspace{2ex}}{\textsf{\ardra} {\tiny \RIGHTarrow} 23:54\hspace{2ex}} 
&
%\sunmonth{\kumbha}{14}{}

\caldata{25}{07:15}{09:24}{14:00-15:21}{07:15-08:36}{18:03}{\textsf{\seka} {\tiny \RIGHTarrow} 07:28\hspace{2ex}}{\textsf{\punarvasu} {\tiny \RIGHTarrow} 21:45\hspace{2ex}} 
&
%\sunmonth{\kumbha}{15}{}

\caldata{26}{07:13}{09:23}{11:17-12:38}{15:21-16:42}{18:04}{\textsf{\stra} {\tiny \RIGHTarrow} 00:58(+1)}{\textsf{\pushya} {\tiny \RIGHTarrow} 19:12\hspace{2ex}} 
&
%\sunmonth{\kumbha}{16}{}

\caldata{27}{07:11}{09:22}{09:54-11:16}{14:00-15:22}{18:06}{\textsf{\schaturdashi} {\tiny \RIGHTarrow} 21:20\hspace{2ex}}{\textsf{\ashresha} {\tiny \RIGHTarrow} 16:22\hspace{2ex}} 
\\ \hline
%\sunmonth{\kumbha}{17}{}

\caldata{28}{07:09}{09:20}{16:44-18:07}{12:38-14:00}{18:07}{\textsf{\purnima} {\tiny \RIGHTarrow} 17:39\hspace{2ex}}{\textsf{\magha} {\tiny \RIGHTarrow} 13:27\hspace{2ex}} 
&
%\sunmonth{\kumbha}{18}{}

{}  &
{}  &
{}  &
{}  &
{}  &
\\ \hline
\end{tabular}


%\clearpage
\begin{tabular}{|c|c|c|c|c|c|c|}
\multicolumn{7}{c}{\Large \bfseries MARCH 2010}\\
\hline
\textbf{SUN} & \textbf{MON} & \textbf{TUE} & \textbf{WED} & \textbf{THU} & \textbf{FRI} & \textbf{SAT} \\ \hline
{}  &
\caldata{1}{07:07}{09:19}{08:29-09:52}{11:15-12:38}{18:09}{\textsf{\kpra} {\tiny \RIGHTarrow} 14:04\hspace{2ex}}{\textsf{\purvaphalguni} {\tiny \RIGHTarrow} 10:36\hspace{2ex}} 
&
%\sunmonth{\kumbha}{19}{}

\caldata{2}{07:06}{09:18}{15:24-16:47}{09:52-11:15}{18:10}{\textsf{\kdvi} {\tiny \RIGHTarrow} 10:46\hspace{2ex}}{\textsf{\uttaraphalguni} {\tiny \RIGHTarrow} 07:59\hspace{2ex}} 
&
%\sunmonth{\kumbha}{20}{}

\caldata{3}{07:04}{09:17}{12:38-14:01}{08:27-09:51}{18:12}{\textsf{\ktri} {\tiny \RIGHTarrow} 07:56\hspace{2ex}}{\textsf{\chitra} {\tiny \RIGHTarrow} 04:21(+1)} 
&
%\sunmonth{\kumbha}{21}{}

\caldata{4}{07:02}{09:16}{14:01-15:25}{07:02-08:25}{18:13}{\textsf{\kpanc} {\tiny \RIGHTarrow} 04:24(+1)}{\textsf{\svati} {\tiny \RIGHTarrow} 03:37(+1)} 
&
%\sunmonth{\kumbha}{22}{}

\caldata{5}{07:00}{09:15}{11:13-12:37}{15:26-16:50}{18:15}{\textsf{\ksha} {\tiny \RIGHTarrow} 03:53(+1)}{\textsf{\vishakha} {\tiny \RIGHTarrow} 03:43(+1)} 
&
%\sunmonth{\kumbha}{23}{}

\caldata{6}{06:58}{09:13}{09:47-11:12}{14:01-15:26}{18:16}{\textsf{\ksap} {\tiny \RIGHTarrow} 04:14(+1)}{\textsf{\anuradha} {\tiny \RIGHTarrow} 04:39(+1)} 
\\ \hline
%\sunmonth{\kumbha}{24}{}

\caldata{7}{06:56}{09:12}{16:52-18:18}{12:37-14:02}{18:18}{\textsf{\kasht} {\tiny \RIGHTarrow} 05:24(+1)}{\textsf{\jyeshtha} {\tiny \RIGHTarrow} 06:24(+1)} 
&
%\sunmonth{\kumbha}{25}{}

\caldata{8}{06:54}{09:11}{08:19-09:45}{11:10-12:36}{18:19}{\textsf{\knav} {\tiny \RIGHTarrow} \ahoratram}{\textsf{\mula} {\tiny \RIGHTarrow} \ahoratram} 
&
%\sunmonth{\kumbha}{26}{}

\caldata{9}{06:52}{09:09}{15:28-16:54}{09:44-11:10}{18:21}{\textsf{\knav} {\tiny \RIGHTarrow} 07:17\hspace{2ex}}{\textsf{\mula} {\tiny \RIGHTarrow} 08:50\hspace{2ex}} 
&
%\sunmonth{\kumbha}{27}{}

\caldata{10}{06:50}{09:08}{12:36-14:02}{08:16-09:43}{18:22}{\textsf{\kdas} {\tiny \RIGHTarrow} 09:41\hspace{2ex}}{\textsf{\purvashadha} {\tiny \RIGHTarrow} 11:43\hspace{2ex}} 
&
%\sunmonth{\kumbha}{28}{}

\caldata{11}{06:48}{09:07}{14:02-15:29}{06:48-08:14}{18:23}{\textsf{\keka} {\tiny \RIGHTarrow} 12:21\hspace{2ex}}{\textsf{\uttarashadha} {\tiny \RIGHTarrow} 14:51\hspace{2ex}} 
&
%\sunmonth{\kumbha}{29}{}

\caldata{12}{06:46}{09:05}{11:08-12:35}{15:30-16:57}{18:25}{\textsf{\kdva} {\tiny \RIGHTarrow} 15:04\hspace{2ex}}{\textsf{\shravana} {\tiny \RIGHTarrow} 18:01\hspace{2ex}} 
&
%\sunmonth{\kumbha}{30}{}

\caldata{13}{06:44}{09:04}{09:39-11:07}{14:02-15:30}{18:26}{\textsf{\ktra} {\tiny \RIGHTarrow} 17:39\hspace{2ex}}{\textsf{\shravishtha} {\tiny \RIGHTarrow} 21:02\hspace{2ex}} 
\\ \hline
%\sunmonth{\mina}{1}{(17:52:\hspace{2ex})}

\caldata{14}{06:42}{09:03}{16:59-18:28}{12:35-14:03}{18:28}{\textsf{\kchaturdashi} {\tiny \RIGHTarrow} 19:59\hspace{2ex}}{\textsf{\shatabhishak} {\tiny \RIGHTarrow} 23:49\hspace{2ex}} 
&
%\sunmonth{\mina}{2}{}

\caldata{15}{06:40}{09:01}{08:08-09:37}{11:05-12:34}{18:29}{\textsf{\ama} {\tiny \RIGHTarrow} 21:58\hspace{2ex}}{\textsf{\proshthapada} {\tiny \RIGHTarrow} 02:15(+1)} 
&
%\sunmonth{\mina}{3}{}

\caldata{16}{06:38}{09:00}{15:32-17:01}{09:36-11:05}{18:31}{\textsf{\spra} {\tiny \RIGHTarrow} 23:35\hspace{2ex}}{\textsf{\uttaraproshthapada} {\tiny \RIGHTarrow} 04:20(+1)} 
&
%\sunmonth{\mina}{4}{}

\caldata{17}{06:36}{08:59}{12:34-14:03}{08:05-09:35}{18:32}{\textsf{\sdvi} {\tiny \RIGHTarrow} 00:48(+1)}{\textsf{\revati} {\tiny \RIGHTarrow} 06:02(+1)} 
&
%\sunmonth{\mina}{5}{}

\caldata{18}{06:34}{08:58}{14:04-15:34}{06:34-08:04}{18:34}{\textsf{\stri} {\tiny \RIGHTarrow} 01:36(+1)}{\textsf{\ashwini} {\tiny \RIGHTarrow} \ahoratram} 
&
%\sunmonth{\mina}{6}{}

\caldata{19}{06:32}{08:56}{11:03-12:33}{15:34-17:04}{18:35}{\textsf{\scha} {\tiny \RIGHTarrow} 02:01(+1)}{\textsf{\ashwini} {\tiny \RIGHTarrow} 07:21\hspace{2ex}} 
&
%\sunmonth{\mina}{7}{}

\caldata{20}{06:30}{08:55}{09:31-11:02}{14:03-15:34}{18:36}{\textsf{\spanc} {\tiny \RIGHTarrow} 02:00(+1)}{\textsf{\apabharani} {\tiny \RIGHTarrow} 08:15\hspace{2ex}} 
\\ \hline
%\sunmonth{\mina}{8}{}

\caldata{21}{06:28}{08:54}{17:06-18:38}{12:33-14:04}{18:38}{\textsf{\ssha} {\tiny \RIGHTarrow} 01:32(+1)}{\textsf{\krittika} {\tiny \RIGHTarrow} 08:45\hspace{2ex}} 
&
%\sunmonth{\mina}{9}{}

\caldata{22}{06:26}{08:52}{07:57-09:29}{11:00-12:32}{18:39}{\textsf{\ssap} {\tiny \RIGHTarrow} 00:35(+1)}{\textsf{\rohini} {\tiny \RIGHTarrow} 08:48\hspace{2ex}} 
&
%\sunmonth{\mina}{10}{}

\caldata{23}{06:24}{08:51}{15:36-17:08}{09:28-11:00}{18:41}{\textsf{\sasht} {\tiny \RIGHTarrow} 23:10\hspace{2ex}}{\textsf{\mrigashirsha} {\tiny \RIGHTarrow} 08:24\hspace{2ex}} 
&
%\sunmonth{\mina}{11}{}

\caldata{24}{06:22}{08:50}{12:32-14:04}{07:54-09:27}{18:42}{\textsf{\snav} {\tiny \RIGHTarrow} 21:15\hspace{2ex}}{\textsf{\ardra} {\tiny \RIGHTarrow} 07:31\hspace{2ex}} 
&
%\sunmonth{\mina}{12}{}

\caldata{25}{06:20}{08:48}{14:04-15:37}{06:20-07:52}{18:43}{\textsf{\sdas} {\tiny \RIGHTarrow} 18:53\hspace{2ex}}{\textsf{\pushya} {\tiny \RIGHTarrow} 04:21(+1)} 
&
%\sunmonth{\mina}{13}{}

\caldata{26}{06:18}{08:47}{10:58-12:31}{15:38-17:11}{18:45}{\textsf{\seka} {\tiny \RIGHTarrow} 16:08\hspace{2ex}}{\textsf{\ashresha} {\tiny \RIGHTarrow} 02:11(+1)} 
&
%\sunmonth{\mina}{14}{}

\caldata{27}{06:16}{08:46}{09:23-10:57}{14:04-15:38}{18:46}{\textsf{\sdva} {\tiny \RIGHTarrow} 13:05\hspace{2ex}}{\textsf{\magha} {\tiny \RIGHTarrow} 23:47\hspace{2ex}} 
\\ \hline
%\sunmonth{\mina}{15}{}

\caldata{28}{07:14}{09:44}{18:13-19:48}{13:31-15:05}{19:48}{\textsf{\stra} {\tiny \RIGHTarrow} 10:51\hspace{2ex}}{\textsf{\purvaphalguni} {\tiny \RIGHTarrow} 22:17\hspace{2ex}} 
&
%\sunmonth{\mina}{16}{}

\caldata{29}{07:12}{09:43}{08:46-10:21}{11:55-13:30}{19:49}{\textsf{\schaturdashi} {\tiny \RIGHTarrow} 07:34\hspace{2ex}}{\textsf{\uttaraphalguni} {\tiny \RIGHTarrow} 19:52\hspace{2ex}} 
&
%\sunmonth{\mina}{17}{}

\caldata{30}{07:10}{09:42}{16:40-18:15}{10:20-11:55}{19:50}{\textsf{\kpra} {\tiny \RIGHTarrow} 01:37(+1)}{\textsf{\hasta} {\tiny \RIGHTarrow} 17:40\hspace{2ex}} 
&
%\sunmonth{\mina}{18}{}

\caldata{31}{07:08}{09:40}{13:30-15:05}{08:43-10:19}{19:52}{\textsf{\kdvi} {\tiny \RIGHTarrow} 23:17\hspace{2ex}}{\textsf{\chitra} {\tiny \RIGHTarrow} 15:52\hspace{2ex}} 
&
%\sunmonth{\mina}{19}{}

{}  &
{}  &
\\ \hline
\end{tabular}


%\clearpage
\begin{tabular}{|c|c|c|c|c|c|c|}
\multicolumn{7}{c}{\Large \bfseries APRIL 2010}\\
\hline
\textbf{SUN} & \textbf{MON} & \textbf{TUE} & \textbf{WED} & \textbf{THU} & \textbf{FRI} & \textbf{SAT} \\ \hline
{}  &
{}  &
{}  &
{}  &
\caldata{1}{07:06}{09:39}{15:05-16:41}{07:06-08:41}{19:53}{\textsf{\ktri} {\tiny \RIGHTarrow} 21:34\hspace{2ex}}{\textsf{\svati} {\tiny \RIGHTarrow} 14:39\hspace{2ex}} 
&
%\sunmonth{\mina}{20}{}

\caldata{2}{07:04}{09:38}{11:53-13:29}{16:42-18:18}{19:55}{\textsf{\kcha} {\tiny \RIGHTarrow} 20:37\hspace{2ex}}{\textsf{\vishakha} {\tiny \RIGHTarrow} 14:09\hspace{2ex}} 
&
%\sunmonth{\mina}{21}{}

\caldata{3}{07:02}{09:36}{10:15-11:52}{15:05-16:42}{19:56}{\textsf{\kpanc} {\tiny \RIGHTarrow} 20:29\hspace{2ex}}{\textsf{\anuradha} {\tiny \RIGHTarrow} 14:28\hspace{2ex}} 
\\ \hline
%\sunmonth{\mina}{22}{}

\caldata{4}{07:00}{09:35}{18:20-19:58}{13:29-15:06}{19:58}{\textsf{\ksha} {\tiny \RIGHTarrow} 21:12\hspace{2ex}}{\textsf{\jyeshtha} {\tiny \RIGHTarrow} 15:36\hspace{2ex}} 
&
%\sunmonth{\mina}{23}{}

\caldata{5}{06:58}{09:34}{08:35-10:13}{11:50-13:28}{19:59}{\textsf{\ksap} {\tiny \RIGHTarrow} 22:40\hspace{2ex}}{\textsf{\mula} {\tiny \RIGHTarrow} 17:31\hspace{2ex}} 
&
%\sunmonth{\mina}{24}{}

\caldata{6}{06:56}{09:32}{16:44-18:22}{10:12-11:50}{20:00}{\textsf{\kasht} {\tiny \RIGHTarrow} 00:46(+1)}{\textsf{\purvashadha} {\tiny \RIGHTarrow} 20:02\hspace{2ex}} 
&
%\sunmonth{\mina}{25}{}

\caldata{7}{06:54}{09:31}{13:28-15:06}{08:32-10:11}{20:02}{\textsf{\knav} {\tiny \RIGHTarrow} 03:15(+1)}{\textsf{\uttarashadha} {\tiny \RIGHTarrow} 22:58\hspace{2ex}} 
&
%\sunmonth{\mina}{26}{}

\caldata{8}{06:52}{09:30}{15:06-16:45}{06:52-08:30}{20:03}{\textsf{\kdas} {\tiny \RIGHTarrow} 05:52(+1)}{\textsf{\shravana} {\tiny \RIGHTarrow} 02:05(+1)} 
&
%\sunmonth{\mina}{27}{}

\caldata{9}{06:50}{09:28}{11:47-13:27}{16:45-18:24}{20:04}{\textsf{\keka} {\tiny \RIGHTarrow} \ahoratram}{\textsf{\shravishtha} {\tiny \RIGHTarrow} 05:08(+1)} 
&
%\sunmonth{\mina}{28}{}

\caldata{10}{06:48}{09:27}{10:07-11:47}{15:06-16:46}{20:06}{\textsf{\keka} {\tiny \RIGHTarrow} 08:23\hspace{2ex}}{\textsf{\shatabhishak} {\tiny \RIGHTarrow} \ahoratram} 
\\ \hline
%\sunmonth{\mina}{29}{}

\caldata{11}{06:46}{09:26}{18:26-20:07}{13:26-15:06}{20:07}{\textsf{\kdva} {\tiny \RIGHTarrow} 10:35\hspace{2ex}}{\textsf{\shatabhishak} {\tiny \RIGHTarrow} 07:54\hspace{2ex}} 
&
%\sunmonth{\mina}{30}{}

\caldata{12}{06:45}{09:25}{08:25-10:06}{11:46-13:27}{20:09}{\textsf{\ktra} {\tiny \RIGHTarrow} 12:21\hspace{2ex}}{\textsf{\proshthapada} {\tiny \RIGHTarrow} 10:16\hspace{2ex}} 
&
%\sunmonth{\mesha}{1}{(03:21:(+1))}

\caldata{13}{06:43}{09:24}{16:48-18:29}{10:04-11:45}{20:10}{\textsf{\kchaturdashi} {\tiny \RIGHTarrow} 13:38\hspace{2ex}}{\textsf{\uttaraproshthapada} {\tiny \RIGHTarrow} 12:09\hspace{2ex}} 
&
%\sunmonth{\mesha}{2}{}

\caldata{14}{06:41}{09:23}{13:26-15:07}{08:22-10:03}{20:11}{\textsf{\ama} {\tiny \RIGHTarrow} 14:25\hspace{2ex}}{\textsf{\revati} {\tiny \RIGHTarrow} 13:34\hspace{2ex}} 
&
%\sunmonth{\mesha}{3}{}

\caldata{15}{06:39}{09:21}{15:07-16:49}{06:39-08:20}{20:13}{\textsf{\spra} {\tiny \RIGHTarrow} 14:45\hspace{2ex}}{\textsf{\ashwini} {\tiny \RIGHTarrow} 14:33\hspace{2ex}} 
&
%\sunmonth{\mesha}{4}{}

\caldata{16}{06:37}{09:20}{11:43-13:25}{16:49-18:31}{20:14}{\textsf{\sdvi} {\tiny \RIGHTarrow} 14:39\hspace{2ex}}{\textsf{\apabharani} {\tiny \RIGHTarrow} 15:06\hspace{2ex}} 
&
%\sunmonth{\mesha}{5}{}

\caldata{17}{06:35}{09:19}{10:00-11:42}{15:08-16:50}{20:16}{\textsf{\stri} {\tiny \RIGHTarrow} 14:12\hspace{2ex}}{\textsf{\krittika} {\tiny \RIGHTarrow} 15:19\hspace{2ex}} 
\\ \hline
%\sunmonth{\mesha}{6}{}

\caldata{18}{06:33}{09:17}{18:34-20:17}{13:25-15:08}{20:17}{\textsf{\scha} {\tiny \RIGHTarrow} 13:24\hspace{2ex}}{\textsf{\rohini} {\tiny \RIGHTarrow} 15:12\hspace{2ex}} 
&
%\sunmonth{\mesha}{7}{}

\caldata{19}{06:31}{09:16}{08:14-09:57}{11:41-13:24}{20:18}{\textsf{\spanc} {\tiny \RIGHTarrow} 12:19\hspace{2ex}}{\textsf{\mrigashirsha} {\tiny \RIGHTarrow} 14:47\hspace{2ex}} 
&
%\sunmonth{\mesha}{8}{}

\caldata{20}{06:30}{09:16}{16:52-18:36}{09:57-11:41}{20:20}{\textsf{\ssha} {\tiny \RIGHTarrow} 10:55\hspace{2ex}}{\textsf{\ardra} {\tiny \RIGHTarrow} 14:05\hspace{2ex}} 
&
%\sunmonth{\mesha}{9}{}

\caldata{21}{06:28}{09:14}{13:24-15:08}{08:12-09:56}{20:21}{\textsf{\ssap} {\tiny \RIGHTarrow} 09:15\hspace{2ex}}{\textsf{\punarvasu} {\tiny \RIGHTarrow} 13:05\hspace{2ex}} 
&
%\sunmonth{\mesha}{10}{}

\caldata{22}{06:26}{09:13}{15:09-16:53}{06:26-08:10}{20:23}{\textsf{\sasht} {\tiny \RIGHTarrow} 07:18\hspace{2ex}}{\textsf{\pushya} {\tiny \RIGHTarrow} 11:50\hspace{2ex}} 
&
%\sunmonth{\mesha}{11}{}

\caldata{23}{06:24}{09:12}{11:39-13:24}{16:54-18:39}{20:24}{\textsf{\sdas} {\tiny \RIGHTarrow} 02:38(+1)}{\textsf{\ashresha} {\tiny \RIGHTarrow} 10:20\hspace{2ex}} 
&
%\sunmonth{\mesha}{12}{}

\caldata{24}{06:22}{09:10}{09:52-11:38}{15:08-16:54}{20:25}{\textsf{\seka} {\tiny \RIGHTarrow} 00:03(+1)}{\textsf{\magha} {\tiny \RIGHTarrow} 08:37\hspace{2ex}} 
\\ \hline
%\sunmonth{\mesha}{13}{}

\caldata{25}{06:21}{09:10}{18:41-20:27}{13:24-15:09}{20:27}{\textsf{\sdva} {\tiny \RIGHTarrow} 21:25\hspace{2ex}}{\textsf{\purvaphalguni} {\tiny \RIGHTarrow} 06:47\hspace{2ex}} 
&
%\sunmonth{\mesha}{14}{}

\caldata{26}{06:19}{09:08}{08:05-09:51}{11:37-13:23}{20:28}{\textsf{\stra} {\tiny \RIGHTarrow} 18:50\hspace{2ex}}{\textsf{\hasta} {\tiny \RIGHTarrow} 03:09(+1)} 
&
%\sunmonth{\mesha}{15}{}

\caldata{27}{06:17}{09:07}{16:56-18:43}{09:50-11:36}{20:30}{\textsf{\schaturdashi} {\tiny \RIGHTarrow} 16:26\hspace{2ex}}{\textsf{\chitra} {\tiny \RIGHTarrow} 01:38(+1)} 
&
%\sunmonth{\mesha}{16}{}

\caldata{28}{06:16}{09:07}{13:23-15:10}{08:02-09:49}{20:31}{\textsf{\purnima} {\tiny \RIGHTarrow} 14:22\hspace{2ex}}{\textsf{\svati} {\tiny \RIGHTarrow} 00:30(+1)} 
&
%\sunmonth{\mesha}{17}{}

\caldata{29}{06:14}{09:05}{15:10-16:57}{06:14-08:01}{20:32}{\textsf{\kpra} {\tiny \RIGHTarrow} 12:45\hspace{2ex}}{\textsf{\vishakha} {\tiny \RIGHTarrow} 23:54\hspace{2ex}} 
&
%\sunmonth{\mesha}{18}{}

\caldata{30}{06:12}{09:04}{11:35-13:23}{16:58-18:46}{20:34}{\textsf{\kdvi} {\tiny \RIGHTarrow} 11:44\hspace{2ex}}{\textsf{\anuradha} {\tiny \RIGHTarrow} 23:56\hspace{2ex}} 
&
%\sunmonth{\mesha}{19}{}

\\ \hline
\end{tabular}


%\clearpage
\begin{tabular}{|c|c|c|c|c|c|c|}
\multicolumn{7}{c}{\Large \bfseries MAY 2010}\\
\hline
\textbf{SUN} & \textbf{MON} & \textbf{TUE} & \textbf{WED} & \textbf{THU} & \textbf{FRI} & \textbf{SAT} \\ \hline
{}  &
{}  &
{}  &
{}  &
{}  &
{}  &
\caldata{1}{06:11}{09:03}{09:47-11:35}{15:11-16:59}{20:35}{\textsf{\ktri} {\tiny \RIGHTarrow} 11:25\hspace{2ex}}{\textsf{\jyeshtha} {\tiny \RIGHTarrow} 00:40(+1)} 
\\ \hline
%\sunmonth{\mesha}{20}{}

\caldata{2}{06:09}{09:02}{18:47-20:36}{13:22-15:10}{20:36}{\textsf{\kcha} {\tiny \RIGHTarrow} 11:51\hspace{2ex}}{\textsf{\mula} {\tiny \RIGHTarrow} 02:08(+1)} 
&
%\sunmonth{\mesha}{21}{}

\caldata{3}{06:08}{09:02}{07:56-09:45}{11:34-13:23}{20:38}{\textsf{\kpanc} {\tiny \RIGHTarrow} 13:00\hspace{2ex}}{\textsf{\purvashadha} {\tiny \RIGHTarrow} 04:15(+1)} 
&
%\sunmonth{\mesha}{22}{}

\caldata{4}{06:06}{09:00}{17:00-18:49}{09:44-11:33}{20:39}{\textsf{\ksha} {\tiny \RIGHTarrow} 14:47\hspace{2ex}}{\textsf{\uttarashadha} {\tiny \RIGHTarrow} \ahoratram} 
&
%\sunmonth{\mesha}{23}{}

\caldata{5}{06:04}{08:59}{13:22-15:12}{07:53-09:43}{20:41}{\textsf{\ksap} {\tiny \RIGHTarrow} 17:02\hspace{2ex}}{\textsf{\uttarashadha} {\tiny \RIGHTarrow} 06:54\hspace{2ex}} 
&
%\sunmonth{\mesha}{24}{}

\caldata{6}{06:03}{08:58}{15:12-17:02}{06:03-07:52}{20:42}{\textsf{\kasht} {\tiny \RIGHTarrow} 19:30\hspace{2ex}}{\textsf{\shravana} {\tiny \RIGHTarrow} 09:52\hspace{2ex}} 
&
%\sunmonth{\mesha}{25}{}

\caldata{7}{06:01}{08:57}{11:31-13:22}{17:02-18:52}{20:43}{\textsf{\knav} {\tiny \RIGHTarrow} 21:56\hspace{2ex}}{\textsf{\shravishtha} {\tiny \RIGHTarrow} 12:53\hspace{2ex}} 
&
%\sunmonth{\mesha}{26}{}

\caldata{8}{06:00}{08:57}{09:41-11:31}{15:13-17:03}{20:45}{\textsf{\kdas} {\tiny \RIGHTarrow} 00:08(+1)}{\textsf{\shatabhishak} {\tiny \RIGHTarrow} 15:44\hspace{2ex}} 
\\ \hline
%\sunmonth{\mesha}{27}{}

\caldata{9}{05:59}{08:56}{18:55-20:46}{13:22-15:13}{20:46}{\textsf{\keka} {\tiny \RIGHTarrow} 01:52(+1)}{\textsf{\proshthapada} {\tiny \RIGHTarrow} 18:11\hspace{2ex}} 
&
%\sunmonth{\mesha}{28}{}

\caldata{10}{05:57}{08:55}{07:48-09:39}{11:30-13:22}{20:47}{\textsf{\kdva} {\tiny \RIGHTarrow} 03:03(+1)}{\textsf{\uttaraproshthapada} {\tiny \RIGHTarrow} 20:08\hspace{2ex}} 
&
%\sunmonth{\mesha}{29}{}

\caldata{11}{05:56}{08:54}{17:05-18:56}{09:39-11:30}{20:48}{\textsf{\ktra} {\tiny \RIGHTarrow} 03:38(+1)}{\textsf{\revati} {\tiny \RIGHTarrow} 21:31\hspace{2ex}} 
&
%\sunmonth{\mesha}{30}{}

\caldata{12}{05:54}{08:53}{13:22-15:14}{07:46-09:38}{20:50}{\textsf{\kchaturdashi} {\tiny \RIGHTarrow} 03:36(+1)}{\textsf{\ashwini} {\tiny \RIGHTarrow} 22:18\hspace{2ex}} 
&
%\sunmonth{\mesha}{31}{}

\caldata{13}{05:53}{08:52}{15:14-17:06}{05:53-07:45}{20:51}{\textsf{\ama} {\tiny \RIGHTarrow} 03:02(+1)}{\textsf{\apabharani} {\tiny \RIGHTarrow} 22:33\hspace{2ex}} 
&
%\sunmonth{\vrishabha}{1}{(00:10:(+1))}

\caldata{14}{05:52}{08:52}{11:29-13:22}{17:07-18:59}{20:52}{\textsf{\spra} {\tiny \RIGHTarrow} 02:01(+1)}{\textsf{\krittika} {\tiny \RIGHTarrow} 22:20\hspace{2ex}} 
&
%\sunmonth{\vrishabha}{2}{}

\caldata{15}{05:51}{08:51}{09:36-11:29}{15:15-17:08}{20:54}{\textsf{\sdvi} {\tiny \RIGHTarrow} 00:38(+1)}{\textsf{\rohini} {\tiny \RIGHTarrow} 21:45\hspace{2ex}} 
\\ \hline
%\sunmonth{\vrishabha}{3}{}

\caldata{16}{05:49}{08:50}{19:01-20:55}{13:22-15:15}{20:55}{\textsf{\stri} {\tiny \RIGHTarrow} 22:58\hspace{2ex}}{\textsf{\mrigashirsha} {\tiny \RIGHTarrow} 20:53\hspace{2ex}} 
&
%\sunmonth{\vrishabha}{4}{}

\caldata{17}{05:48}{08:49}{07:41-09:35}{11:28-13:22}{20:56}{\textsf{\scha} {\tiny \RIGHTarrow} 21:06\hspace{2ex}}{\textsf{\ardra} {\tiny \RIGHTarrow} 19:47\hspace{2ex}} 
&
%\sunmonth{\vrishabha}{5}{}

\caldata{18}{05:47}{08:49}{17:09-19:03}{09:34-11:28}{20:57}{\textsf{\spanc} {\tiny \RIGHTarrow} 19:05\hspace{2ex}}{\textsf{\punarvasu} {\tiny \RIGHTarrow} 18:33\hspace{2ex}} 
&
%\sunmonth{\vrishabha}{6}{}

\caldata{19}{05:46}{08:48}{13:22-15:16}{07:40-09:34}{20:58}{\textsf{\ssha} {\tiny \RIGHTarrow} 16:59\hspace{2ex}}{\textsf{\pushya} {\tiny \RIGHTarrow} 17:12\hspace{2ex}} 
&
%\sunmonth{\vrishabha}{7}{}

\caldata{20}{05:45}{08:48}{15:16-17:11}{05:45-07:39}{21:00}{\textsf{\ssap} {\tiny \RIGHTarrow} 14:48\hspace{2ex}}{\textsf{\ashresha} {\tiny \RIGHTarrow} 15:48\hspace{2ex}} 
&
%\sunmonth{\vrishabha}{8}{}

\caldata{21}{05:44}{08:47}{11:27-13:22}{17:11-19:06}{21:01}{\textsf{\sasht} {\tiny \RIGHTarrow} 12:36\hspace{2ex}}{\textsf{\magha} {\tiny \RIGHTarrow} 14:23\hspace{2ex}} 
&
%\sunmonth{\vrishabha}{9}{}

\caldata{22}{05:43}{08:46}{09:32-11:27}{15:17-17:12}{21:02}{\textsf{\snav} {\tiny \RIGHTarrow} 10:25\hspace{2ex}}{\textsf{\purvaphalguni} {\tiny \RIGHTarrow} 12:59\hspace{2ex}} 
\\ \hline
%\sunmonth{\vrishabha}{10}{}

\caldata{23}{05:42}{08:46}{19:07-21:03}{13:22-15:17}{21:03}{\textsf{\sdas} {\tiny \RIGHTarrow} 08:17\hspace{2ex}}{\textsf{\uttaraphalguni} {\tiny \RIGHTarrow} 11:39\hspace{2ex}} 
&
%\sunmonth{\vrishabha}{11}{}

\caldata{24}{05:41}{08:45}{07:36-09:31}{11:27-13:22}{21:04}{\textsf{\seka} {\tiny \RIGHTarrow} 06:17\hspace{2ex}}{\textsf{\hasta} {\tiny \RIGHTarrow} 10:27\hspace{2ex}} 
&
%\sunmonth{\vrishabha}{12}{}

\caldata{25}{05:40}{08:45}{17:13-19:09}{09:31-11:26}{21:05}{\textsf{\stra} {\tiny \RIGHTarrow} 02:58(+1)}{\textsf{\chitra} {\tiny \RIGHTarrow} 09:27\hspace{2ex}} 
&
%\sunmonth{\vrishabha}{13}{}

\caldata{26}{05:39}{08:44}{13:22-15:18}{07:34-09:30}{21:06}{\textsf{\schaturdashi} {\tiny \RIGHTarrow} 01:50(+1)}{\textsf{\svati} {\tiny \RIGHTarrow} 08:46\hspace{2ex}} 
&
%\sunmonth{\vrishabha}{14}{}

\caldata{27}{05:38}{08:43}{15:18-17:14}{05:38-07:34}{21:07}{\textsf{\purnima} {\tiny \RIGHTarrow} 01:09(+1)}{\textsf{\vishakha} {\tiny \RIGHTarrow} 08:29\hspace{2ex}} 
&
%\sunmonth{\vrishabha}{15}{}

\caldata{28}{05:37}{08:43}{11:26-13:23}{17:16-19:12}{21:09}{\textsf{\kpra} {\tiny \RIGHTarrow} 01:01(+1)}{\textsf{\anuradha} {\tiny \RIGHTarrow} 08:41\hspace{2ex}} 
&
%\sunmonth{\vrishabha}{16}{}

\caldata{29}{05:36}{08:42}{09:29-11:26}{15:19-17:16}{21:10}{\textsf{\kdvi} {\tiny \RIGHTarrow} 01:29(+1)}{\textsf{\jyeshtha} {\tiny \RIGHTarrow} 09:26\hspace{2ex}} 
\\ \hline
%\sunmonth{\vrishabha}{17}{}

\caldata{30}{05:36}{08:43}{19:14-21:11}{13:23-15:20}{21:11}{\textsf{\ktri} {\tiny \RIGHTarrow} 02:33(+1)}{\textsf{\mula} {\tiny \RIGHTarrow} 10:47\hspace{2ex}} 
&
%\sunmonth{\vrishabha}{18}{}

\caldata{31}{05:35}{08:42}{07:32-09:29}{11:26-13:23}{21:11}{\textsf{\kcha} {\tiny \RIGHTarrow} 04:10(+1)}{\textsf{\purvashadha} {\tiny \RIGHTarrow} 12:43\hspace{2ex}} 
&
%\sunmonth{\vrishabha}{19}{}

{}  &
{}  &
{}  &
{}  &
\\ \hline
\end{tabular}


%\clearpage
\begin{tabular}{|c|c|c|c|c|c|c|}
\multicolumn{7}{c}{\Large \bfseries JUNE 2010}\\
\hline
\textbf{SUN} & \textbf{MON} & \textbf{TUE} & \textbf{WED} & \textbf{THU} & \textbf{FRI} & \textbf{SAT} \\ \hline
{}  &
{}  &
\caldata{1}{05:34}{08:41}{17:17-19:14}{09:28-11:25}{21:12}{\textsf{\kpanc} {\tiny \RIGHTarrow} \ahoratram}{\textsf{\uttarashadha} {\tiny \RIGHTarrow} 15:08\hspace{2ex}} 
&
%\sunmonth{\vrishabha}{20}{}

\caldata{2}{05:34}{08:41}{13:23-15:20}{07:31-09:28}{21:13}{\textsf{\kpanc} {\tiny \RIGHTarrow} 06:14\hspace{2ex}}{\textsf{\shravana} {\tiny \RIGHTarrow} 17:56\hspace{2ex}} 
&
%\sunmonth{\vrishabha}{21}{}

\caldata{3}{05:33}{08:41}{15:21-17:18}{05:33-07:30}{21:14}{\textsf{\ksha} {\tiny \RIGHTarrow} 08:36\hspace{2ex}}{\textsf{\shravishtha} {\tiny \RIGHTarrow} 20:54\hspace{2ex}} 
&
%\sunmonth{\vrishabha}{22}{}

\caldata{4}{05:33}{08:41}{11:26-13:24}{17:19-19:17}{21:15}{\textsf{\ksap} {\tiny \RIGHTarrow} 11:02\hspace{2ex}}{\textsf{\shatabhishak} {\tiny \RIGHTarrow} 23:50\hspace{2ex}} 
&
%\sunmonth{\vrishabha}{23}{}

\caldata{5}{05:32}{08:40}{09:28-11:26}{15:22-17:20}{21:16}{\textsf{\kasht} {\tiny \RIGHTarrow} 13:16\hspace{2ex}}{\textsf{\proshthapada} {\tiny \RIGHTarrow} 02:31(+1)} 
\\ \hline
%\sunmonth{\vrishabha}{24}{}

\caldata{6}{05:32}{08:41}{19:18-21:17}{13:24-15:22}{21:17}{\textsf{\knav} {\tiny \RIGHTarrow} 15:07\hspace{2ex}}{\textsf{\uttaraproshthapada} {\tiny \RIGHTarrow} 04:45(+1)} 
&
%\sunmonth{\vrishabha}{25}{}

\caldata{7}{05:31}{08:40}{07:29-09:27}{11:25-13:24}{21:17}{\textsf{\kdas} {\tiny \RIGHTarrow} 16:25\hspace{2ex}}{\textsf{\revati} {\tiny \RIGHTarrow} \ahoratram} 
&
%\sunmonth{\vrishabha}{26}{}

\caldata{8}{05:31}{08:40}{17:21-19:19}{09:27-11:26}{21:18}{\textsf{\keka} {\tiny \RIGHTarrow} 17:04\hspace{2ex}}{\textsf{\revati} {\tiny \RIGHTarrow} 06:21\hspace{2ex}} 
&
%\sunmonth{\vrishabha}{27}{}

\caldata{9}{05:31}{08:40}{13:25-15:23}{07:29-09:28}{21:19}{\textsf{\kdva} {\tiny \RIGHTarrow} 17:01\hspace{2ex}}{\textsf{\ashwini} {\tiny \RIGHTarrow} 07:17\hspace{2ex}} 
&
%\sunmonth{\vrishabha}{28}{}

\caldata{10}{05:30}{08:39}{15:23-17:21}{05:30-07:28}{21:19}{\textsf{\ktra} {\tiny \RIGHTarrow} 16:19\hspace{2ex}}{\textsf{\apabharani} {\tiny \RIGHTarrow} 07:33\hspace{2ex}} 
&
%\sunmonth{\vrishabha}{29}{}

\caldata{11}{05:30}{08:40}{11:26-13:25}{17:22-19:21}{21:20}{\textsf{\kchaturdashi} {\tiny \RIGHTarrow} 15:00\hspace{2ex}}{\textsf{\krittika} {\tiny \RIGHTarrow} 07:11\hspace{2ex}} 
&
%\sunmonth{\vrishabha}{30}{}

\caldata{12}{05:30}{08:40}{09:27-11:26}{15:24-17:23}{21:21}{\textsf{\ama} {\tiny \RIGHTarrow} 13:11\hspace{2ex}}{\textsf{\rohini} {\tiny \RIGHTarrow} 06:17\hspace{2ex}} 
\\ \hline
%\sunmonth{\vrishabha}{31}{}

\caldata{13}{05:30}{08:40}{19:22-21:21}{13:25-15:24}{21:21}{\textsf{\spra} {\tiny \RIGHTarrow} 10:59\hspace{2ex}}{\textsf{\ardra} {\tiny \RIGHTarrow} 03:18(+1)} 
&
%\sunmonth{\vrishabha}{32}{}

\caldata{14}{05:30}{08:40}{07:29-09:28}{11:27-13:26}{21:22}{\textsf{\sdvi} {\tiny \RIGHTarrow} 08:30\hspace{2ex}}{\textsf{\punarvasu} {\tiny \RIGHTarrow} 01:28(+1)} 
&
%\sunmonth{\mithuna}{1}{(06:43:\hspace{2ex})}

\caldata{15}{05:30}{08:40}{17:24-19:23}{09:28-11:27}{21:22}{\textsf{\stri} {\tiny \RIGHTarrow} 05:50\hspace{2ex}}{\textsf{\pushya} {\tiny \RIGHTarrow} 23:34\hspace{2ex}} 
&
%\sunmonth{\mithuna}{2}{}

\caldata{16}{05:30}{08:40}{13:26-15:25}{07:29-09:28}{21:23}{\textsf{\spanc} {\tiny \RIGHTarrow} 00:28(+1)}{\textsf{\ashresha} {\tiny \RIGHTarrow} 21:42\hspace{2ex}} 
&
%\sunmonth{\mithuna}{3}{}

\caldata{17}{05:30}{08:40}{15:25-17:24}{05:30-07:29}{21:23}{\textsf{\ssha} {\tiny \RIGHTarrow} 21:56\hspace{2ex}}{\textsf{\magha} {\tiny \RIGHTarrow} 19:57\hspace{2ex}} 
&
%\sunmonth{\mithuna}{4}{}

\caldata{18}{05:30}{08:40}{11:27-13:26}{17:24-19:23}{21:23}{\textsf{\ssap} {\tiny \RIGHTarrow} 19:36\hspace{2ex}}{\textsf{\purvaphalguni} {\tiny \RIGHTarrow} 18:24\hspace{2ex}} 
&
%\sunmonth{\mithuna}{5}{}

\caldata{19}{05:30}{08:40}{09:28-11:27}{15:26-17:25}{21:24}{\textsf{\sasht} {\tiny \RIGHTarrow} 17:31\hspace{2ex}}{\textsf{\uttaraphalguni} {\tiny \RIGHTarrow} 17:06\hspace{2ex}} 
\\ \hline
%\sunmonth{\mithuna}{6}{}

\caldata{20}{05:30}{08:40}{19:24-21:24}{13:27-15:26}{21:24}{\textsf{\snav} {\tiny \RIGHTarrow} 15:45\hspace{2ex}}{\textsf{\hasta} {\tiny \RIGHTarrow} 16:06\hspace{2ex}} 
&
%\sunmonth{\mithuna}{7}{}

\caldata{21}{05:30}{08:40}{07:29-09:28}{11:27-13:27}{21:24}{\textsf{\sdas} {\tiny \RIGHTarrow} 14:19\hspace{2ex}}{\textsf{\chitra} {\tiny \RIGHTarrow} 15:27\hspace{2ex}} 
&
%\sunmonth{\mithuna}{8}{}

\caldata{22}{05:30}{08:40}{17:25-19:24}{09:28-11:27}{21:24}{\textsf{\seka} {\tiny \RIGHTarrow} 13:17\hspace{2ex}}{\textsf{\svati} {\tiny \RIGHTarrow} 15:11\hspace{2ex}} 
&
%\sunmonth{\mithuna}{9}{}

\caldata{23}{05:31}{08:41}{13:27-15:26}{07:30-09:29}{21:24}{\textsf{\sdva} {\tiny \RIGHTarrow} 12:40\hspace{2ex}}{\textsf{\vishakha} {\tiny \RIGHTarrow} 15:20\hspace{2ex}} 
&
%\sunmonth{\mithuna}{10}{}

\caldata{24}{05:31}{08:41}{15:27-17:26}{05:31-07:30}{21:25}{\textsf{\stra} {\tiny \RIGHTarrow} 12:29\hspace{2ex}}{\textsf{\anuradha} {\tiny \RIGHTarrow} 15:56\hspace{2ex}} 
&
%\sunmonth{\mithuna}{11}{}

\caldata{25}{05:31}{08:41}{11:28-13:28}{17:26-19:25}{21:25}{\textsf{\schaturdashi} {\tiny \RIGHTarrow} 12:46\hspace{2ex}}{\textsf{\jyeshtha} {\tiny \RIGHTarrow} 16:59\hspace{2ex}} 
&
%\sunmonth{\mithuna}{12}{}

\caldata{26}{05:32}{08:42}{09:30-11:29}{15:27-17:26}{21:25}{\textsf{\purnima} {\tiny \RIGHTarrow} 13:33\hspace{2ex}}{\textsf{\mula} {\tiny \RIGHTarrow} 18:30\hspace{2ex}} 
\\ \hline
%\sunmonth{\mithuna}{13}{}

\caldata{27}{05:32}{08:42}{19:25-21:25}{13:28-15:27}{21:25}{\textsf{\kpra} {\tiny \RIGHTarrow} 14:47\hspace{2ex}}{\textsf{\purvashadha} {\tiny \RIGHTarrow} 20:28\hspace{2ex}} 
&
%\sunmonth{\mithuna}{14}{}

\caldata{28}{05:33}{08:43}{07:31-09:30}{11:29-13:28}{21:24}{\textsf{\kdvi} {\tiny \RIGHTarrow} 16:28\hspace{2ex}}{\textsf{\uttarashadha} {\tiny \RIGHTarrow} 22:51\hspace{2ex}} 
&
%\sunmonth{\mithuna}{15}{}

\caldata{29}{05:33}{08:43}{17:26-19:25}{09:30-11:29}{21:24}{\textsf{\ktri} {\tiny \RIGHTarrow} 18:31\hspace{2ex}}{\textsf{\shravana} {\tiny \RIGHTarrow} 01:34(+1)} 
&
%\sunmonth{\mithuna}{16}{}

\caldata{30}{05:34}{08:44}{13:29-15:27}{07:32-09:31}{21:24}{\textsf{\kcha} {\tiny \RIGHTarrow} 20:50\hspace{2ex}}{\textsf{\shravishtha} {\tiny \RIGHTarrow} 04:31(+1)} 
&
%\sunmonth{\mithuna}{17}{}

{}  &
{}  &
\\ \hline
\end{tabular}


%\clearpage
\begin{tabular}{|c|c|c|c|c|c|c|}
\multicolumn{7}{c}{\Large \bfseries JULY 2010}\\
\hline
\textbf{SUN} & \textbf{MON} & \textbf{TUE} & \textbf{WED} & \textbf{THU} & \textbf{FRI} & \textbf{SAT} \\ \hline
{}  &
{}  &
{}  &
{}  &
\caldata{1}{05:34}{08:44}{15:27-17:26}{05:34-07:32}{21:24}{\textsf{\kpanc} {\tiny \RIGHTarrow} 23:15\hspace{2ex}}{\textsf{\shatabhishak} {\tiny \RIGHTarrow} \ahoratram} 
&
%\sunmonth{\mithuna}{18}{}

\caldata{2}{05:35}{08:44}{11:30-13:29}{17:26-19:25}{21:24}{\textsf{\ksha} {\tiny \RIGHTarrow} 01:36(+1)}{\textsf{\shatabhishak} {\tiny \RIGHTarrow} 07:31\hspace{2ex}} 
&
%\sunmonth{\mithuna}{19}{}

\caldata{3}{05:35}{08:44}{09:32-11:30}{15:27-17:26}{21:23}{\textsf{\ksap} {\tiny \RIGHTarrow} 03:41(+1)}{\textsf{\proshthapada} {\tiny \RIGHTarrow} 10:22\hspace{2ex}} 
\\ \hline
%\sunmonth{\mithuna}{20}{}

\caldata{4}{05:36}{08:45}{19:24-21:23}{13:29-15:27}{21:23}{\textsf{\kasht} {\tiny \RIGHTarrow} 05:19(+1)}{\textsf{\uttaraproshthapada} {\tiny \RIGHTarrow} 12:53\hspace{2ex}} 
&
%\sunmonth{\mithuna}{21}{}

\caldata{5}{05:37}{08:46}{07:35-09:33}{11:31-13:30}{21:23}{\textsf{\knav} {\tiny \RIGHTarrow} \ahoratram}{\textsf{\revati} {\tiny \RIGHTarrow} 14:54\hspace{2ex}} 
&
%\sunmonth{\mithuna}{22}{}

\caldata{6}{05:37}{08:46}{17:25-19:23}{09:33-11:31}{21:22}{\textsf{\knav} {\tiny \RIGHTarrow} 06:18\hspace{2ex}}{\textsf{\ashwini} {\tiny \RIGHTarrow} 16:17\hspace{2ex}} 
&
%\sunmonth{\mithuna}{23}{}

\caldata{7}{05:38}{08:46}{13:30-15:28}{07:36-09:34}{21:22}{\textsf{\kdas} {\tiny \RIGHTarrow} 06:35\hspace{2ex}}{\textsf{\apabharani} {\tiny \RIGHTarrow} 16:56\hspace{2ex}} 
&
%\sunmonth{\mithuna}{24}{}

\caldata{8}{05:39}{08:47}{15:27-17:25}{05:39-07:36}{21:21}{\textsf{\keka} {\tiny \RIGHTarrow} 06:06\hspace{2ex}}{\textsf{\krittika} {\tiny \RIGHTarrow} 16:51\hspace{2ex}} 
&
%\sunmonth{\mithuna}{25}{}

\caldata{9}{05:40}{08:48}{11:32-13:30}{17:25-19:23}{21:21}{\textsf{\ktra} {\tiny \RIGHTarrow} 02:59(+1)}{\textsf{\rohini} {\tiny \RIGHTarrow} 16:04\hspace{2ex}} 
&
%\sunmonth{\mithuna}{26}{}

\caldata{10}{05:41}{08:48}{09:35-11:33}{15:27-17:25}{21:20}{\textsf{\kchaturdashi} {\tiny \RIGHTarrow} 00:30(+1)}{\textsf{\mrigashirsha} {\tiny \RIGHTarrow} 14:41\hspace{2ex}} 
\\ \hline
%\sunmonth{\mithuna}{27}{}

\caldata{11}{05:42}{08:49}{19:22-21:20}{13:31-15:28}{21:20}{\textsf{\ama} {\tiny \RIGHTarrow} 21:37\hspace{2ex}}{\textsf{\ardra} {\tiny \RIGHTarrow} 12:48\hspace{2ex}} 
&
%\sunmonth{\mithuna}{28}{}

\caldata{12}{05:42}{08:49}{07:39-09:36}{11:33-13:30}{21:19}{\textsf{\spra} {\tiny \RIGHTarrow} 18:27\hspace{2ex}}{\textsf{\punarvasu} {\tiny \RIGHTarrow} 10:33\hspace{2ex}} 
&
%\sunmonth{\mithuna}{29}{}

\caldata{13}{05:43}{08:50}{17:24-19:21}{09:36-11:33}{21:18}{\textsf{\sdvi} {\tiny \RIGHTarrow} 15:08\hspace{2ex}}{\textsf{\pushya} {\tiny \RIGHTarrow} 08:06\hspace{2ex}} 
&
%\sunmonth{\mithuna}{30}{}

\caldata{14}{05:44}{08:50}{13:31-15:27}{07:40-09:37}{21:18}{\textsf{\stri} {\tiny \RIGHTarrow} 11:49\hspace{2ex}}{\textsf{\magha} {\tiny \RIGHTarrow} 03:08(+1)} 
&
%\sunmonth{\mithuna}{31}{}

\caldata{15}{05:45}{08:51}{15:27-17:24}{05:45-07:41}{21:17}{\textsf{\scha} {\tiny \RIGHTarrow} 08:37\hspace{2ex}}{\textsf{\purvaphalguni} {\tiny \RIGHTarrow} 00:58(+1)} 
&
%\sunmonth{\karkataka}{1}{(17:33:\hspace{2ex})}

\caldata{16}{05:46}{08:52}{11:34-13:31}{17:23-19:19}{21:16}{\textsf{\ssha} {\tiny \RIGHTarrow} 03:07(+1)}{\textsf{\uttaraphalguni} {\tiny \RIGHTarrow} 23:09\hspace{2ex}} 
&
%\sunmonth{\karkataka}{2}{}

\caldata{17}{05:47}{08:52}{09:39-11:35}{15:27-17:23}{21:15}{\textsf{\ssap} {\tiny \RIGHTarrow} 01:04(+1)}{\textsf{\hasta} {\tiny \RIGHTarrow} 21:47\hspace{2ex}} 
\\ \hline
%\sunmonth{\karkataka}{3}{}

\caldata{18}{05:48}{08:53}{19:18-21:14}{13:31-15:26}{21:14}{\textsf{\sasht} {\tiny \RIGHTarrow} 23:32\hspace{2ex}}{\textsf{\chitra} {\tiny \RIGHTarrow} 20:56\hspace{2ex}} 
&
%\sunmonth{\karkataka}{4}{}

\caldata{19}{05:49}{08:53}{07:44-09:40}{11:35-13:31}{21:13}{\textsf{\snav} {\tiny \RIGHTarrow} 22:34\hspace{2ex}}{\textsf{\svati} {\tiny \RIGHTarrow} 20:39\hspace{2ex}} 
&
%\sunmonth{\karkataka}{5}{}

\caldata{20}{05:51}{08:55}{17:21-19:16}{09:41-11:36}{21:12}{\textsf{\sdas} {\tiny \RIGHTarrow} 22:10\hspace{2ex}}{\textsf{\vishakha} {\tiny \RIGHTarrow} 20:57\hspace{2ex}} 
&
%\sunmonth{\karkataka}{6}{}

\caldata{21}{05:52}{08:55}{13:31-15:26}{07:46-09:41}{21:11}{\textsf{\seka} {\tiny \RIGHTarrow} 22:20\hspace{2ex}}{\textsf{\anuradha} {\tiny \RIGHTarrow} 21:47\hspace{2ex}} 
&
%\sunmonth{\karkataka}{7}{}

\caldata{22}{05:53}{08:56}{15:26-17:20}{05:53-07:47}{21:10}{\textsf{\sdva} {\tiny \RIGHTarrow} 23:01\hspace{2ex}}{\textsf{\jyeshtha} {\tiny \RIGHTarrow} 23:07\hspace{2ex}} 
&
%\sunmonth{\karkataka}{8}{}

\caldata{23}{05:54}{08:57}{11:37-13:31}{17:20-19:14}{21:09}{\textsf{\stra} {\tiny \RIGHTarrow} 00:10(+1)}{\textsf{\mula} {\tiny \RIGHTarrow} 00:55(+1)} 
&
%\sunmonth{\karkataka}{9}{}

\caldata{24}{05:55}{08:57}{09:43-11:37}{15:25-17:19}{21:08}{\textsf{\schaturdashi} {\tiny \RIGHTarrow} 01:43(+1)}{\textsf{\purvashadha} {\tiny \RIGHTarrow} 03:07(+1)} 
\\ \hline
%\sunmonth{\karkataka}{10}{}

\caldata{25}{05:56}{08:58}{19:13-21:07}{13:31-15:25}{21:07}{\textsf{\purnima} {\tiny \RIGHTarrow} 03:37(+1)}{\textsf{\uttarashadha} {\tiny \RIGHTarrow} 05:37(+1)} 
&
%\sunmonth{\karkataka}{11}{}

\caldata{26}{05:57}{08:58}{07:50-09:44}{11:37-13:31}{21:06}{\textsf{\kpra} {\tiny \RIGHTarrow} 05:46(+1)}{\textsf{\shravana} {\tiny \RIGHTarrow} \ahoratram} 
&
%\sunmonth{\karkataka}{12}{}

\caldata{27}{05:59}{09:00}{17:17-19:10}{09:45-11:38}{21:04}{\textsf{\kdvi} {\tiny \RIGHTarrow} \ahoratram}{\textsf{\shravana} {\tiny \RIGHTarrow} 08:24\hspace{2ex}} 
&
%\sunmonth{\karkataka}{13}{}

\caldata{28}{06:00}{09:00}{13:31-15:24}{07:52-09:45}{21:03}{\textsf{\kdvi} {\tiny \RIGHTarrow} 08:08\hspace{2ex}}{\textsf{\shravishtha} {\tiny \RIGHTarrow} 11:21\hspace{2ex}} 
&
%\sunmonth{\karkataka}{14}{}

\caldata{29}{06:01}{09:01}{15:24-17:16}{06:01-07:53}{21:02}{\textsf{\ktri} {\tiny \RIGHTarrow} 10:34\hspace{2ex}}{\textsf{\shatabhishak} {\tiny \RIGHTarrow} 14:21\hspace{2ex}} 
&
%\sunmonth{\karkataka}{15}{}

\caldata{30}{06:02}{09:01}{11:39-13:31}{17:16-19:08}{21:01}{\textsf{\kcha} {\tiny \RIGHTarrow} 12:58\hspace{2ex}}{\textsf{\proshthapada} {\tiny \RIGHTarrow} 17:18\hspace{2ex}} 
&
%\sunmonth{\karkataka}{16}{}

\caldata{31}{06:03}{09:02}{09:47-11:39}{15:23-17:15}{20:59}{\textsf{\kpanc} {\tiny \RIGHTarrow} 15:11\hspace{2ex}}{\textsf{\uttaraproshthapada} {\tiny \RIGHTarrow} 20:03\hspace{2ex}} 
\\ \hline
%\sunmonth{\karkataka}{17}{}

\end{tabular}


%\clearpage
\begin{tabular}{|c|c|c|c|c|c|c|}
\multicolumn{7}{c}{\Large \bfseries AUGUST 2010}\\
\hline
\textbf{SUN} & \textbf{MON} & \textbf{TUE} & \textbf{WED} & \textbf{THU} & \textbf{FRI} & \textbf{SAT} \\ \hline
\caldata{1}{06:05}{09:03}{19:06-20:58}{13:31-15:23}{20:58}{\textsf{\ksha} {\tiny \RIGHTarrow} 17:03\hspace{2ex}}{\textsf{\revati} {\tiny \RIGHTarrow} 22:27\hspace{2ex}} 
&
%\sunmonth{\karkataka}{18}{}

\caldata{2}{06:06}{09:04}{07:57-09:48}{11:39-13:31}{20:56}{\textsf{\ksap} {\tiny \RIGHTarrow} 18:26\hspace{2ex}}{\textsf{\ashwini} {\tiny \RIGHTarrow} 00:20(+1)} 
&
%\sunmonth{\karkataka}{19}{}

\caldata{3}{06:07}{09:04}{17:13-19:04}{09:49-11:40}{20:55}{\textsf{\kasht} {\tiny \RIGHTarrow} 19:11\hspace{2ex}}{\textsf{\apabharani} {\tiny \RIGHTarrow} 01:36(+1)} 
&
%\sunmonth{\karkataka}{20}{}

\caldata{4}{06:08}{09:05}{13:31-15:21}{07:58-09:49}{20:54}{\textsf{\knav} {\tiny \RIGHTarrow} 19:12\hspace{2ex}}{\textsf{\krittika} {\tiny \RIGHTarrow} 02:08(+1)} 
&
%\sunmonth{\karkataka}{21}{}

\caldata{5}{06:10}{09:06}{15:21-17:11}{06:10-08:00}{20:52}{\textsf{\kdas} {\tiny \RIGHTarrow} 18:28\hspace{2ex}}{\textsf{\rohini} {\tiny \RIGHTarrow} 01:54(+1)} 
&
%\sunmonth{\karkataka}{22}{}

\caldata{6}{06:11}{09:07}{11:41-13:31}{17:11-19:01}{20:51}{\textsf{\keka} {\tiny \RIGHTarrow} 16:58\hspace{2ex}}{\textsf{\mrigashirsha} {\tiny \RIGHTarrow} 00:54(+1)} 
&
%\sunmonth{\karkataka}{23}{}

\caldata{7}{06:12}{09:07}{09:51-11:40}{15:20-17:09}{20:49}{\textsf{\kdva} {\tiny \RIGHTarrow} 14:47\hspace{2ex}}{\textsf{\ardra} {\tiny \RIGHTarrow} 23:14\hspace{2ex}} 
\\ \hline
%\sunmonth{\karkataka}{24}{}

\caldata{8}{06:14}{09:08}{18:58-20:48}{13:31-15:20}{20:48}{\textsf{\ktra} {\tiny \RIGHTarrow} 11:59\hspace{2ex}}{\textsf{\punarvasu} {\tiny \RIGHTarrow} 21:01\hspace{2ex}} 
&
%\sunmonth{\karkataka}{25}{}

\caldata{9}{06:15}{09:09}{08:03-09:52}{11:41-13:30}{20:46}{\textsf{\kchaturdashi} {\tiny \RIGHTarrow} 08:44\hspace{2ex}}{\textsf{\pushya} {\tiny \RIGHTarrow} 18:24\hspace{2ex}} 
&
%\sunmonth{\karkataka}{26}{}

\caldata{10}{06:16}{09:09}{17:07-18:55}{09:53-11:41}{20:44}{\textsf{\spra} {\tiny \RIGHTarrow} 01:21(+1)}{\textsf{\ashresha} {\tiny \RIGHTarrow} 15:32\hspace{2ex}} 
&
%\sunmonth{\karkataka}{27}{}

\caldata{11}{06:17}{09:10}{13:30-15:18}{08:05-09:53}{20:43}{\textsf{\sdvi} {\tiny \RIGHTarrow} 21:37\hspace{2ex}}{\textsf{\magha} {\tiny \RIGHTarrow} 12:36\hspace{2ex}} 
&
%\sunmonth{\karkataka}{28}{}

\caldata{12}{06:19}{09:11}{15:17-17:05}{06:19-08:06}{20:41}{\textsf{\stri} {\tiny \RIGHTarrow} 18:05\hspace{2ex}}{\textsf{\purvaphalguni} {\tiny \RIGHTarrow} 09:46\hspace{2ex}} 
&
%\sunmonth{\karkataka}{29}{}

\caldata{13}{06:20}{09:11}{11:42-13:29}{17:04-18:51}{20:39}{\textsf{\scha} {\tiny \RIGHTarrow} 14:53\hspace{2ex}}{\textsf{\uttaraphalguni} {\tiny \RIGHTarrow} 07:12\hspace{2ex}} 
&
%\sunmonth{\karkataka}{30}{}

\caldata{14}{06:21}{09:12}{09:55-11:42}{15:16-17:03}{20:38}{\textsf{\spanc} {\tiny \RIGHTarrow} 12:09\hspace{2ex}}{\textsf{\chitra} {\tiny \RIGHTarrow} 03:37(+1)} 
\\ \hline
%\sunmonth{\karkataka}{31}{}

\caldata{15}{06:23}{09:13}{18:49-20:36}{13:29-15:16}{20:36}{\textsf{\ssha} {\tiny \RIGHTarrow} 10:03\hspace{2ex}}{\textsf{\svati} {\tiny \RIGHTarrow} 02:49(+1)} 
&
%\sunmonth{\simha}{1}{(01:56:(+1))}

\caldata{16}{06:24}{09:14}{08:10-09:56}{11:42-13:29}{20:34}{\textsf{\ssap} {\tiny \RIGHTarrow} 08:40\hspace{2ex}}{\textsf{\vishakha} {\tiny \RIGHTarrow} 02:44(+1)} 
&
%\sunmonth{\simha}{2}{}

\caldata{17}{06:25}{09:14}{17:01-18:47}{09:57-11:43}{20:33}{\textsf{\sasht} {\tiny \RIGHTarrow} 08:02\hspace{2ex}}{\textsf{\anuradha} {\tiny \RIGHTarrow} 03:23(+1)} 
&
%\sunmonth{\simha}{3}{}

\caldata{18}{06:27}{09:15}{13:29-15:14}{08:12-09:58}{20:31}{\textsf{\snav} {\tiny \RIGHTarrow} 08:09\hspace{2ex}}{\textsf{\jyeshtha} {\tiny \RIGHTarrow} 04:42(+1)} 
&
%\sunmonth{\simha}{4}{}

\caldata{19}{06:28}{09:16}{15:13-16:58}{06:28-08:13}{20:29}{\textsf{\sdas} {\tiny \RIGHTarrow} 08:58\hspace{2ex}}{\textsf{\mula} {\tiny \RIGHTarrow} \ahoratram} 
&
%\sunmonth{\simha}{5}{}

\caldata{20}{06:29}{09:16}{11:43-13:28}{16:57-18:42}{20:27}{\textsf{\seka} {\tiny \RIGHTarrow} 10:20\hspace{2ex}}{\textsf{\mula} {\tiny \RIGHTarrow} 06:36\hspace{2ex}} 
&
%\sunmonth{\simha}{6}{}

\caldata{21}{06:31}{09:17}{09:59-11:43}{15:12-16:56}{20:25}{\textsf{\sdva} {\tiny \RIGHTarrow} 12:10\hspace{2ex}}{\textsf{\purvashadha} {\tiny \RIGHTarrow} 08:59\hspace{2ex}} 
\\ \hline
%\sunmonth{\simha}{7}{}

\caldata{22}{06:32}{09:18}{18:40-20:24}{13:28-15:12}{20:24}{\textsf{\stra} {\tiny \RIGHTarrow} 14:18\hspace{2ex}}{\textsf{\uttarashadha} {\tiny \RIGHTarrow} 11:40\hspace{2ex}} 
&
%\sunmonth{\simha}{8}{}

\caldata{23}{06:33}{09:18}{08:16-10:00}{11:43-13:27}{20:22}{\textsf{\schaturdashi} {\tiny \RIGHTarrow} 16:38\hspace{2ex}}{\textsf{\shravana} {\tiny \RIGHTarrow} 14:33\hspace{2ex}} 
&
%\sunmonth{\simha}{9}{}

\caldata{24}{06:35}{09:20}{16:53-18:36}{10:01-11:44}{20:20}{\textsf{\purnima} {\tiny \RIGHTarrow} 19:04\hspace{2ex}}{\textsf{\shravishtha} {\tiny \RIGHTarrow} 17:32\hspace{2ex}} 
&
%\sunmonth{\simha}{10}{}

\caldata{25}{06:36}{09:20}{13:27-15:09}{08:18-10:01}{20:18}{\textsf{\kpra} {\tiny \RIGHTarrow} 21:30\hspace{2ex}}{\textsf{\shatabhishak} {\tiny \RIGHTarrow} 20:31\hspace{2ex}} 
&
%\sunmonth{\simha}{11}{}

\caldata{26}{06:37}{09:20}{15:08-16:51}{06:37-08:19}{20:16}{\textsf{\kdvi} {\tiny \RIGHTarrow} 23:52\hspace{2ex}}{\textsf{\proshthapada} {\tiny \RIGHTarrow} 23:26\hspace{2ex}} 
&
%\sunmonth{\simha}{12}{}

\caldata{27}{06:39}{09:22}{11:44-13:26}{16:50-18:32}{20:14}{\textsf{\ktri} {\tiny \RIGHTarrow} 02:05(+1)}{\textsf{\uttaraproshthapada} {\tiny \RIGHTarrow} 02:13(+1)} 
&
%\sunmonth{\simha}{13}{}

\caldata{28}{06:40}{09:22}{10:03-11:44}{15:07-16:49}{20:12}{\textsf{\kcha} {\tiny \RIGHTarrow} 04:03(+1)}{\textsf{\revati} {\tiny \RIGHTarrow} 04:45(+1)} 
\\ \hline
%\sunmonth{\simha}{14}{}

\caldata{29}{06:41}{09:22}{18:28-20:10}{13:25-15:06}{20:10}{\textsf{\kpanc} {\tiny \RIGHTarrow} 05:39(+1)}{\textsf{\ashwini} {\tiny \RIGHTarrow} \ahoratram} 
&
%\sunmonth{\simha}{15}{}

\caldata{30}{06:43}{09:24}{08:23-10:04}{11:45-13:26}{20:09}{\textsf{\ksha} {\tiny \RIGHTarrow} \ahoratram}{\textsf{\ashwini} {\tiny \RIGHTarrow} 06:56\hspace{2ex}} 
&
%\sunmonth{\simha}{16}{}

\caldata{31}{06:44}{09:24}{16:46-18:26}{10:04-11:45}{20:07}{\textsf{\ksha} {\tiny \RIGHTarrow} 06:47\hspace{2ex}}{\textsf{\apabharani} {\tiny \RIGHTarrow} 08:38\hspace{2ex}} 
&
%\sunmonth{\simha}{17}{}

{}  &
{}  &
{}  &
\\ \hline
\end{tabular}


%\clearpage
\begin{tabular}{|c|c|c|c|c|c|c|}
\multicolumn{7}{c}{\Large \bfseries SEPTEMBER 2010}\\
\hline
\textbf{SUN} & \textbf{MON} & \textbf{TUE} & \textbf{WED} & \textbf{THU} & \textbf{FRI} & \textbf{SAT} \\ \hline
{}  &
{}  &
{}  &
\caldata{1}{06:45}{09:25}{13:25-15:05}{08:25-10:05}{20:05}{\textsf{\ksap} {\tiny \RIGHTarrow} 07:19\hspace{2ex}}{\textsf{\krittika} {\tiny \RIGHTarrow} 09:46\hspace{2ex}} 
&
%\sunmonth{\simha}{18}{}

\caldata{2}{06:46}{09:25}{15:04-16:43}{06:46-08:25}{20:03}{\textsf{\kasht} {\tiny \RIGHTarrow} 07:11\hspace{2ex}}{\textsf{\rohini} {\tiny \RIGHTarrow} 10:13\hspace{2ex}} 
&
%\sunmonth{\simha}{19}{}

\caldata{3}{06:48}{09:26}{11:45-13:24}{16:42-18:21}{20:01}{\textsf{\kdas} {\tiny \RIGHTarrow} 04:42(+1)}{\textsf{\mrigashirsha} {\tiny \RIGHTarrow} 09:58\hspace{2ex}} 
&
%\sunmonth{\simha}{20}{}

\caldata{4}{06:49}{09:27}{10:06-11:45}{15:02-16:41}{19:59}{\textsf{\keka} {\tiny \RIGHTarrow} 02:22(+1)}{\textsf{\ardra} {\tiny \RIGHTarrow} 09:00\hspace{2ex}} 
\\ \hline
%\sunmonth{\simha}{21}{}

\caldata{5}{06:50}{09:27}{18:18-19:57}{13:23-15:01}{19:57}{\textsf{\kdva} {\tiny \RIGHTarrow} 23:26\hspace{2ex}}{\textsf{\punarvasu} {\tiny \RIGHTarrow} 07:21\hspace{2ex}} 
&
%\sunmonth{\simha}{22}{}

\caldata{6}{06:52}{09:28}{08:29-10:07}{11:45-13:23}{19:55}{\textsf{\ktra} {\tiny \RIGHTarrow} 20:03\hspace{2ex}}{\textsf{\ashresha} {\tiny \RIGHTarrow} 02:22(+1)} 
&
%\sunmonth{\simha}{23}{}

\caldata{7}{06:53}{09:29}{16:38-18:15}{10:08-11:45}{19:53}{\textsf{\kchaturdashi} {\tiny \RIGHTarrow} 16:21\hspace{2ex}}{\textsf{\magha} {\tiny \RIGHTarrow} 23:24\hspace{2ex}} 
&
%\sunmonth{\simha}{24}{}

\caldata{8}{06:54}{09:29}{13:22-14:59}{08:31-10:08}{19:51}{\textsf{\ama} {\tiny \RIGHTarrow} 12:30\hspace{2ex}}{\textsf{\purvaphalguni} {\tiny \RIGHTarrow} 20:20\hspace{2ex}} 
&
%\sunmonth{\simha}{25}{}

\caldata{9}{06:56}{09:30}{14:59-16:35}{06:56-08:32}{19:49}{\textsf{\spra} {\tiny \RIGHTarrow} 08:38\hspace{2ex}}{\textsf{\uttaraphalguni} {\tiny \RIGHTarrow} 17:23\hspace{2ex}} 
&
%\sunmonth{\simha}{26}{}

\caldata{10}{06:57}{09:31}{11:45-13:22}{16:34-18:10}{19:47}{\textsf{\stri} {\tiny \RIGHTarrow} 01:48(+1)}{\textsf{\hasta} {\tiny \RIGHTarrow} 14:43\hspace{2ex}} 
&
%\sunmonth{\simha}{27}{}

\caldata{11}{06:58}{09:31}{10:09-11:45}{14:57-16:33}{19:45}{\textsf{\scha} {\tiny \RIGHTarrow} 23:09\hspace{2ex}}{\textsf{\chitra} {\tiny \RIGHTarrow} 12:32\hspace{2ex}} 
\\ \hline
%\sunmonth{\simha}{28}{}

\caldata{12}{07:00}{09:32}{18:07-19:43}{13:21-14:56}{19:43}{\textsf{\spanc} {\tiny \RIGHTarrow} 21:12\hspace{2ex}}{\textsf{\svati} {\tiny \RIGHTarrow} 10:58\hspace{2ex}} 
&
%\sunmonth{\simha}{29}{}

\caldata{13}{07:01}{09:33}{08:36-10:11}{11:46-13:21}{19:41}{\textsf{\ssha} {\tiny \RIGHTarrow} 20:04\hspace{2ex}}{\textsf{\vishakha} {\tiny \RIGHTarrow} 10:10\hspace{2ex}} 
&
%\sunmonth{\simha}{30}{}

\caldata{14}{07:02}{09:33}{16:29-18:04}{10:11-11:45}{19:39}{\textsf{\ssap} {\tiny \RIGHTarrow} 19:45\hspace{2ex}}{\textsf{\anuradha} {\tiny \RIGHTarrow} 10:13\hspace{2ex}} 
&
%\sunmonth{\simha}{31}{}

\caldata{15}{07:04}{09:34}{13:20-14:54}{08:38-10:12}{19:37}{\textsf{\sasht} {\tiny \RIGHTarrow} 20:16\hspace{2ex}}{\textsf{\jyeshtha} {\tiny \RIGHTarrow} 11:06\hspace{2ex}} 
&
%\sunmonth{\kanya}{1}{(01:52:(+1))}

\caldata{16}{07:05}{09:35}{14:53-16:27}{07:05-08:38}{19:35}{\textsf{\snav} {\tiny \RIGHTarrow} 21:31\hspace{2ex}}{\textsf{\mula} {\tiny \RIGHTarrow} 12:44\hspace{2ex}} 
&
%\sunmonth{\kanya}{2}{}

\caldata{17}{07:06}{09:35}{11:46-13:19}{16:26-17:59}{19:33}{\textsf{\sdas} {\tiny \RIGHTarrow} 23:21\hspace{2ex}}{\textsf{\purvashadha} {\tiny \RIGHTarrow} 14:58\hspace{2ex}} 
&
%\sunmonth{\kanya}{3}{}

\caldata{18}{07:08}{09:36}{10:13-11:46}{14:51-16:24}{19:30}{\textsf{\seka} {\tiny \RIGHTarrow} 01:35(+1)}{\textsf{\uttarashadha} {\tiny \RIGHTarrow} 17:39\hspace{2ex}} 
\\ \hline
%\sunmonth{\kanya}{4}{}

\caldata{19}{07:09}{09:36}{17:55-19:28}{13:18-14:50}{19:28}{\textsf{\sdva} {\tiny \RIGHTarrow} 04:02(+1)}{\textsf{\shravana} {\tiny \RIGHTarrow} 20:35\hspace{2ex}} 
&
%\sunmonth{\kanya}{5}{}

\caldata{20}{07:10}{09:37}{08:42-10:14}{11:46-13:18}{19:26}{\textsf{\stra} {\tiny \RIGHTarrow} 06:33(+1)}{\textsf{\shravishtha} {\tiny \RIGHTarrow} 23:36\hspace{2ex}} 
&
%\sunmonth{\kanya}{6}{}

\caldata{21}{07:12}{09:38}{16:21-17:52}{10:15-11:46}{19:24}{\textsf{\schaturdashi} {\tiny \RIGHTarrow} \ahoratram}{\textsf{\shatabhishak} {\tiny \RIGHTarrow} 02:35(+1)} 
&
%\sunmonth{\kanya}{7}{}

\caldata{22}{07:13}{09:38}{13:17-14:48}{08:44-10:15}{19:22}{\textsf{\schaturdashi} {\tiny \RIGHTarrow} 09:00\hspace{2ex}}{\textsf{\proshthapada} {\tiny \RIGHTarrow} 05:26(+1)} 
&
%\sunmonth{\kanya}{8}{}

\caldata{23}{07:14}{09:39}{14:47-16:18}{07:14-08:44}{19:20}{\textsf{\purnima} {\tiny \RIGHTarrow} 11:16\hspace{2ex}}{\textsf{\uttaraproshthapada} {\tiny \RIGHTarrow} \ahoratram} 
&
%\sunmonth{\kanya}{9}{}

\caldata{24}{07:16}{09:40}{11:46-13:17}{16:17-17:47}{19:18}{\textsf{\kpra} {\tiny \RIGHTarrow} 13:19\hspace{2ex}}{\textsf{\uttaraproshthapada} {\tiny \RIGHTarrow} 08:05\hspace{2ex}} 
&
%\sunmonth{\kanya}{10}{}

\caldata{25}{07:17}{09:40}{10:16-11:46}{14:46-16:16}{19:16}{\textsf{\kdvi} {\tiny \RIGHTarrow} 15:06\hspace{2ex}}{\textsf{\revati} {\tiny \RIGHTarrow} 10:29\hspace{2ex}} 
\\ \hline
%\sunmonth{\kanya}{11}{}

\caldata{26}{07:18}{09:41}{17:44-19:14}{13:16-14:45}{19:14}{\textsf{\ktri} {\tiny \RIGHTarrow} 16:35\hspace{2ex}}{\textsf{\ashwini} {\tiny \RIGHTarrow} 12:35\hspace{2ex}} 
&
%\sunmonth{\kanya}{12}{}

\caldata{27}{07:20}{09:42}{08:49-10:18}{11:47-13:16}{19:12}{\textsf{\kcha} {\tiny \RIGHTarrow} 17:42\hspace{2ex}}{\textsf{\apabharani} {\tiny \RIGHTarrow} 14:20\hspace{2ex}} 
&
%\sunmonth{\kanya}{13}{}

\caldata{28}{07:21}{09:42}{16:12-17:41}{10:18-11:46}{19:10}{\textsf{\kpanc} {\tiny \RIGHTarrow} 18:23\hspace{2ex}}{\textsf{\krittika} {\tiny \RIGHTarrow} 15:42\hspace{2ex}} 
&
%\sunmonth{\kanya}{14}{}

\caldata{29}{07:22}{09:43}{13:15-14:43}{08:50-10:18}{19:08}{\textsf{\ksha} {\tiny \RIGHTarrow} 18:34\hspace{2ex}}{\textsf{\rohini} {\tiny \RIGHTarrow} 16:34\hspace{2ex}} 
&
%\sunmonth{\kanya}{15}{}

\caldata{30}{07:24}{09:44}{14:42-16:10}{07:24-08:51}{19:06}{\textsf{\ksap} {\tiny \RIGHTarrow} 18:12\hspace{2ex}}{\textsf{\mrigashirsha} {\tiny \RIGHTarrow} 16:54\hspace{2ex}} 
&
%\sunmonth{\kanya}{16}{}

{}  &
\\ \hline
\end{tabular}


%\clearpage
\begin{tabular}{|c|c|c|c|c|c|c|}
\multicolumn{7}{c}{\Large \bfseries OCTOBER 2010}\\
\hline
\textbf{SUN} & \textbf{MON} & \textbf{TUE} & \textbf{WED} & \textbf{THU} & \textbf{FRI} & \textbf{SAT} \\ \hline
{}  &
{}  &
{}  &
{}  &
{}  &
\caldata{1}{07:25}{09:44}{11:47-13:14}{16:09-17:36}{19:04}{\textsf{\kasht} {\tiny \RIGHTarrow} 17:12\hspace{2ex}}{\textsf{\ardra} {\tiny \RIGHTarrow} 16:37\hspace{2ex}} 
&
%\sunmonth{\kanya}{17}{}

\caldata{2}{07:27}{09:46}{10:20-11:47}{14:41-16:08}{19:02}{\textsf{\knav} {\tiny \RIGHTarrow} 15:35\hspace{2ex}}{\textsf{\punarvasu} {\tiny \RIGHTarrow} 15:44\hspace{2ex}} 
\\ \hline
%\sunmonth{\kanya}{18}{}

\caldata{3}{07:28}{09:46}{17:33-19:00}{13:14-14:40}{19:00}{\textsf{\kdas} {\tiny \RIGHTarrow} 13:22\hspace{2ex}}{\textsf{\pushya} {\tiny \RIGHTarrow} 14:14\hspace{2ex}} 
&
%\sunmonth{\kanya}{19}{}

\caldata{4}{07:29}{09:46}{08:55-10:21}{11:47-13:13}{18:58}{\textsf{\keka} {\tiny \RIGHTarrow} 10:38\hspace{2ex}}{\textsf{\ashresha} {\tiny \RIGHTarrow} 12:14\hspace{2ex}} 
&
%\sunmonth{\kanya}{20}{}

\caldata{5}{07:31}{09:48}{16:04-17:30}{10:22-11:47}{18:56}{\textsf{\ktra} {\tiny \RIGHTarrow} 03:56(+1)}{\textsf{\magha} {\tiny \RIGHTarrow} 09:47\hspace{2ex}} 
&
%\sunmonth{\kanya}{21}{}

\caldata{6}{07:32}{09:48}{13:13-14:38}{08:57-10:22}{18:54}{\textsf{\kchaturdashi} {\tiny \RIGHTarrow} 00:19(+1)}{\textsf{\uttaraphalguni} {\tiny \RIGHTarrow} 04:13(+1)} 
&
%\sunmonth{\kanya}{22}{}

\caldata{7}{07:34}{09:49}{14:37-16:02}{07:34-08:58}{18:52}{\textsf{\ama} {\tiny \RIGHTarrow} 20:46\hspace{2ex}}{\textsf{\hasta} {\tiny \RIGHTarrow} 01:28(+1)} 
&
%\sunmonth{\kanya}{23}{}

\caldata{8}{07:35}{09:50}{11:48-13:12}{16:01-17:25}{18:50}{\textsf{\spra} {\tiny \RIGHTarrow} 17:27\hspace{2ex}}{\textsf{\chitra} {\tiny \RIGHTarrow} 23:01\hspace{2ex}} 
&
%\sunmonth{\kanya}{24}{}

\caldata{9}{07:36}{09:50}{10:24-11:48}{14:36-16:00}{18:48}{\textsf{\sdvi} {\tiny \RIGHTarrow} 14:32\hspace{2ex}}{\textsf{\svati} {\tiny \RIGHTarrow} 21:01\hspace{2ex}} 
\\ \hline
%\sunmonth{\kanya}{25}{}

\caldata{10}{07:38}{09:51}{17:22-18:46}{13:12-14:35}{18:46}{\textsf{\stri} {\tiny \RIGHTarrow} 12:10\hspace{2ex}}{\textsf{\vishakha} {\tiny \RIGHTarrow} 19:39\hspace{2ex}} 
&
%\sunmonth{\kanya}{26}{}

\caldata{11}{07:39}{09:52}{09:02-10:25}{11:48-13:11}{18:44}{\textsf{\scha} {\tiny \RIGHTarrow} 10:33\hspace{2ex}}{\textsf{\anuradha} {\tiny \RIGHTarrow} 19:01\hspace{2ex}} 
&
%\sunmonth{\kanya}{27}{}

\caldata{12}{07:41}{09:53}{15:56-17:19}{10:26-11:48}{18:42}{\textsf{\spanc} {\tiny \RIGHTarrow} 09:45\hspace{2ex}}{\textsf{\jyeshtha} {\tiny \RIGHTarrow} 19:13\hspace{2ex}} 
&
%\sunmonth{\kanya}{28}{}

\caldata{13}{07:42}{09:53}{13:11-14:33}{09:04-10:26}{18:41}{\textsf{\ssha} {\tiny \RIGHTarrow} 09:50\hspace{2ex}}{\textsf{\mula} {\tiny \RIGHTarrow} 20:15\hspace{2ex}} 
&
%\sunmonth{\kanya}{29}{}

\caldata{14}{07:43}{09:54}{14:33-15:55}{07:43-09:05}{18:39}{\textsf{\ssap} {\tiny \RIGHTarrow} 10:45\hspace{2ex}}{\textsf{\purvashadha} {\tiny \RIGHTarrow} 22:02\hspace{2ex}} 
&
%\sunmonth{\kanya}{30}{}

\caldata{15}{07:45}{09:55}{11:49-13:11}{15:54-17:15}{18:37}{\textsf{\sasht} {\tiny \RIGHTarrow} 12:22\hspace{2ex}}{\textsf{\uttarashadha} {\tiny \RIGHTarrow} 00:25(+1)} 
&
%\sunmonth{\kanya}{31}{}

\caldata{16}{07:46}{09:55}{10:28-11:49}{14:31-15:52}{18:35}{\textsf{\snav} {\tiny \RIGHTarrow} 14:31\hspace{2ex}}{\textsf{\shravana} {\tiny \RIGHTarrow} 03:13(+1)} 
\\ \hline
%\sunmonth{\tula}{1}{(13:50:\hspace{2ex})}

\caldata{17}{07:48}{09:57}{17:12-18:33}{13:10-14:31}{18:33}{\textsf{\sdas} {\tiny \RIGHTarrow} 16:58\hspace{2ex}}{\textsf{\shravishtha} {\tiny \RIGHTarrow} 06:13(+1)} 
&
%\sunmonth{\tula}{2}{}

\caldata{18}{07:49}{09:57}{09:09-10:29}{11:49-13:10}{18:31}{\textsf{\seka} {\tiny \RIGHTarrow} 19:31\hspace{2ex}}{\textsf{\shatabhishak} {\tiny \RIGHTarrow} \ahoratram} 
&
%\sunmonth{\tula}{3}{}

\caldata{19}{07:51}{09:58}{15:49-17:09}{10:30-11:50}{18:29}{\textsf{\sdva} {\tiny \RIGHTarrow} 21:57\hspace{2ex}}{\textsf{\shatabhishak} {\tiny \RIGHTarrow} 09:13\hspace{2ex}} 
&
%\sunmonth{\tula}{4}{}

\caldata{20}{07:52}{09:59}{13:10-14:29}{09:11-10:31}{18:28}{\textsf{\stra} {\tiny \RIGHTarrow} 00:10(+1)}{\textsf{\proshthapada} {\tiny \RIGHTarrow} 12:02\hspace{2ex}} 
&
%\sunmonth{\tula}{5}{}

\caldata{21}{07:54}{10:00}{14:29-15:48}{07:54-09:13}{18:26}{\textsf{\schaturdashi} {\tiny \RIGHTarrow} 02:03(+1)}{\textsf{\uttaraproshthapada} {\tiny \RIGHTarrow} 14:35\hspace{2ex}} 
&
%\sunmonth{\tula}{6}{}

\caldata{22}{07:55}{10:00}{11:50-13:09}{15:46-17:05}{18:24}{\textsf{\purnima} {\tiny \RIGHTarrow} 03:34(+1)}{\textsf{\revati} {\tiny \RIGHTarrow} 16:47\hspace{2ex}} 
&
%\sunmonth{\tula}{7}{}

\caldata{23}{07:56}{10:01}{10:32-11:50}{14:27-15:45}{18:22}{\textsf{\kpra} {\tiny \RIGHTarrow} 04:44(+1)}{\textsf{\ashwini} {\tiny \RIGHTarrow} 18:38\hspace{2ex}} 
\\ \hline
%\sunmonth{\tula}{8}{}

\caldata{24}{07:58}{10:02}{17:02-18:20}{13:09-14:26}{18:20}{\textsf{\kdvi} {\tiny \RIGHTarrow} 05:30(+1)}{\textsf{\apabharani} {\tiny \RIGHTarrow} 20:08\hspace{2ex}} 
&
%\sunmonth{\tula}{9}{}

\caldata{25}{07:59}{10:03}{09:16-10:34}{11:51-13:09}{18:19}{\textsf{\ktri} {\tiny \RIGHTarrow} 05:55(+1)}{\textsf{\krittika} {\tiny \RIGHTarrow} 21:15\hspace{2ex}} 
&
%\sunmonth{\tula}{10}{}

\caldata{26}{08:01}{10:04}{15:43-17:00}{10:35-11:52}{18:17}{\textsf{\kcha} {\tiny \RIGHTarrow} 05:56(+1)}{\textsf{\rohini} {\tiny \RIGHTarrow} 22:01\hspace{2ex}} 
&
%\sunmonth{\tula}{11}{}

\caldata{27}{08:02}{10:04}{13:08-14:25}{09:18-10:35}{18:15}{\textsf{\kpanc} {\tiny \RIGHTarrow} 05:33(+1)}{\textsf{\mrigashirsha} {\tiny \RIGHTarrow} 22:24\hspace{2ex}} 
&
%\sunmonth{\tula}{12}{}

\caldata{28}{08:04}{10:06}{14:25-15:41}{08:04-09:20}{18:14}{\textsf{\ksha} {\tiny \RIGHTarrow} 04:45(+1)}{\textsf{\ardra} {\tiny \RIGHTarrow} 22:23\hspace{2ex}} 
&
%\sunmonth{\tula}{13}{}

\caldata{29}{08:05}{10:06}{11:52-13:08}{15:40-16:56}{18:12}{\textsf{\ksap} {\tiny \RIGHTarrow} 03:30(+1)}{\textsf{\punarvasu} {\tiny \RIGHTarrow} 21:56\hspace{2ex}} 
&
%\sunmonth{\tula}{14}{}

\caldata{30}{08:07}{10:07}{10:37-11:53}{14:23-15:39}{18:10}{\textsf{\kasht} {\tiny \RIGHTarrow} 01:49(+1)}{\textsf{\pushya} {\tiny \RIGHTarrow} 21:03\hspace{2ex}} 
\\ \hline
%\sunmonth{\tula}{15}{}

\caldata{31}{07:08}{09:08}{15:53-17:09}{12:08-13:23}{17:09}{\textsf{\knav} {\tiny \RIGHTarrow} 22:41\hspace{2ex}}{\textsf{\ashresha} {\tiny \RIGHTarrow} 18:44\hspace{2ex}} 
&
%\sunmonth{\tula}{16}{}

{}  &
{}  &
{}  &
{}  &
{}  &
\\ \hline
\end{tabular}


%\clearpage
\begin{tabular}{|c|c|c|c|c|c|c|}
\multicolumn{7}{c}{\Large \bfseries NOVEMBER 2010}\\
\hline
\textbf{SUN} & \textbf{MON} & \textbf{TUE} & \textbf{WED} & \textbf{THU} & \textbf{FRI} & \textbf{SAT} \\ \hline
{}  &
\caldata{1}{07:10}{09:09}{08:24-09:39}{10:53-12:08}{17:07}{\textsf{\kdas} {\tiny \RIGHTarrow} 20:12\hspace{2ex}}{\textsf{\magha} {\tiny \RIGHTarrow} 17:02\hspace{2ex}} 
&
%\sunmonth{\tula}{17}{}

\caldata{2}{07:11}{09:10}{14:37-15:51}{09:39-10:54}{17:06}{\textsf{\keka} {\tiny \RIGHTarrow} 17:26\hspace{2ex}}{\textsf{\purvaphalguni} {\tiny \RIGHTarrow} 15:02\hspace{2ex}} 
&
%\sunmonth{\tula}{18}{}

\caldata{3}{07:13}{09:11}{12:08-13:22}{08:26-09:40}{17:04}{\textsf{\kdva} {\tiny \RIGHTarrow} 14:29\hspace{2ex}}{\textsf{\uttaraphalguni} {\tiny \RIGHTarrow} 12:50\hspace{2ex}} 
&
%\sunmonth{\tula}{19}{}

\caldata{4}{07:14}{09:11}{13:22-14:35}{07:14-08:27}{17:03}{\textsf{\ktra} {\tiny \RIGHTarrow} 11:28\hspace{2ex}}{\textsf{\hasta} {\tiny \RIGHTarrow} 10:34\hspace{2ex}} 
&
%\sunmonth{\tula}{20}{}

\caldata{5}{07:16}{09:13}{10:55-12:08}{14:34-15:47}{17:01}{\textsf{\kchaturdashi} {\tiny \RIGHTarrow} 08:32\hspace{2ex}}{\textsf{\chitra} {\tiny \RIGHTarrow} 08:22\hspace{2ex}} 
&
%\sunmonth{\tula}{21}{}

\caldata{6}{07:17}{09:13}{09:42-10:55}{13:21-14:34}{17:00}{\textsf{\spra} {\tiny \RIGHTarrow} 03:39(+1)}{\textsf{\vishakha} {\tiny \RIGHTarrow} 04:56(+1)} 
\\ \hline
%\sunmonth{\tula}{22}{}

\caldata{7}{07:19}{09:14}{15:45-16:58}{12:08-13:20}{16:58}{\textsf{\sdvi} {\tiny \RIGHTarrow} 01:59(+1)}{\textsf{\anuradha} {\tiny \RIGHTarrow} 04:00(+1)} 
&
%\sunmonth{\tula}{23}{}

\caldata{8}{07:20}{09:15}{08:32-09:44}{10:56-12:08}{16:57}{\textsf{\stri} {\tiny \RIGHTarrow} 01:00(+1)}{\textsf{\jyeshtha} {\tiny \RIGHTarrow} 03:44(+1)} 
&
%\sunmonth{\tula}{24}{}

\caldata{9}{07:22}{09:16}{14:32-15:44}{09:45-10:57}{16:56}{\textsf{\scha} {\tiny \RIGHTarrow} 00:46(+1)}{\textsf{\mula} {\tiny \RIGHTarrow} 04:13(+1)} 
&
%\sunmonth{\tula}{25}{}

\caldata{10}{07:23}{09:17}{12:08-13:19}{08:34-09:45}{16:54}{\textsf{\spanc} {\tiny \RIGHTarrow} 01:20(+1)}{\textsf{\purvashadha} {\tiny \RIGHTarrow} 05:28(+1)} 
&
%\sunmonth{\tula}{26}{}

\caldata{11}{07:25}{09:18}{13:20-14:31}{07:25-08:36}{16:53}{\textsf{\ssha} {\tiny \RIGHTarrow} 02:37(+1)}{\textsf{\uttarashadha} {\tiny \RIGHTarrow} 07:23(+1)} 
&
%\sunmonth{\tula}{27}{}

\caldata{12}{07:26}{09:19}{10:58-12:09}{14:30-15:41}{16:52}{\textsf{\ssap} {\tiny \RIGHTarrow} 04:31(+1)}{\textsf{\shravana} {\tiny \RIGHTarrow} \ahoratram} 
&
%\sunmonth{\tula}{28}{}

\caldata{13}{07:28}{09:20}{09:48-10:59}{13:19-14:30}{16:51}{\textsf{\sasht} {\tiny \RIGHTarrow} 06:51(+1)}{\textsf{\shravana} {\tiny \RIGHTarrow} 09:54\hspace{2ex}} 
\\ \hline
%\sunmonth{\tula}{29}{}

\caldata{14}{07:29}{09:21}{15:39-16:50}{12:09-13:19}{16:50}{\textsf{\snav} {\tiny \RIGHTarrow} \ahoratram}{\textsf{\shravishtha} {\tiny \RIGHTarrow} 12:46\hspace{2ex}} 
&
%\sunmonth{\tula}{30}{}

\caldata{15}{07:31}{09:22}{08:40-09:50}{10:59-12:09}{16:48}{\textsf{\snav} {\tiny \RIGHTarrow} 09:23\hspace{2ex}}{\textsf{\shatabhishak} {\tiny \RIGHTarrow} 15:44\hspace{2ex}} 
&
%\sunmonth{\vrishchika}{1}{(12:41:\hspace{2ex})}

\caldata{16}{07:32}{09:23}{14:28-15:37}{09:50-11:00}{16:47}{\textsf{\sdas} {\tiny \RIGHTarrow} 11:51\hspace{2ex}}{\textsf{\proshthapada} {\tiny \RIGHTarrow} 18:36\hspace{2ex}} 
&
%\sunmonth{\vrishchika}{2}{}

\caldata{17}{07:34}{09:24}{12:10-13:19}{08:43-09:52}{16:46}{\textsf{\seka} {\tiny \RIGHTarrow} 14:04\hspace{2ex}}{\textsf{\uttaraproshthapada} {\tiny \RIGHTarrow} 21:12\hspace{2ex}} 
&
%\sunmonth{\vrishchika}{3}{}

\caldata{18}{07:35}{09:25}{13:18-14:27}{07:35-08:43}{16:45}{\textsf{\sdva} {\tiny \RIGHTarrow} 15:53\hspace{2ex}}{\textsf{\revati} {\tiny \RIGHTarrow} 23:24\hspace{2ex}} 
&
%\sunmonth{\vrishchika}{4}{}

\caldata{19}{07:37}{09:26}{11:02-12:10}{14:27-15:35}{16:44}{\textsf{\stra} {\tiny \RIGHTarrow} 17:13\hspace{2ex}}{\textsf{\ashwini} {\tiny \RIGHTarrow} 01:07(+1)} 
&
%\sunmonth{\vrishchika}{5}{}

\caldata{20}{07:38}{09:27}{09:54-11:02}{13:18-14:26}{16:43}{\textsf{\schaturdashi} {\tiny \RIGHTarrow} 18:03\hspace{2ex}}{\textsf{\apabharani} {\tiny \RIGHTarrow} 02:21(+1)} 
\\ \hline
%\sunmonth{\vrishchika}{6}{}

\caldata{21}{07:40}{09:28}{15:34-16:42}{12:11-13:18}{16:42}{\textsf{\purnima} {\tiny \RIGHTarrow} 18:23\hspace{2ex}}{\textsf{\krittika} {\tiny \RIGHTarrow} 03:07(+1)} 
&
%\sunmonth{\vrishchika}{7}{}

\caldata{22}{07:41}{09:29}{08:48-09:56}{11:03-12:11}{16:41}{\textsf{\kpra} {\tiny \RIGHTarrow} 18:16\hspace{2ex}}{\textsf{\rohini} {\tiny \RIGHTarrow} 03:27(+1)} 
&
%\sunmonth{\vrishchika}{8}{}

\caldata{23}{07:42}{09:29}{14:26-15:33}{09:56-11:04}{16:41}{\textsf{\kdvi} {\tiny \RIGHTarrow} 17:45\hspace{2ex}}{\textsf{\mrigashirsha} {\tiny \RIGHTarrow} 03:24(+1)} 
&
%\sunmonth{\vrishchika}{9}{}

\caldata{24}{07:44}{09:31}{12:12-13:19}{08:51-09:58}{16:40}{\textsf{\ktri} {\tiny \RIGHTarrow} 16:52\hspace{2ex}}{\textsf{\ardra} {\tiny \RIGHTarrow} 03:02(+1)} 
&
%\sunmonth{\vrishchika}{10}{}

\caldata{25}{07:45}{09:31}{13:18-14:25}{07:45-08:51}{16:39}{\textsf{\kcha} {\tiny \RIGHTarrow} 15:40\hspace{2ex}}{\textsf{\punarvasu} {\tiny \RIGHTarrow} 02:22(+1)} 
&
%\sunmonth{\vrishchika}{11}{}

\caldata{26}{07:47}{09:33}{11:06-12:12}{14:25-15:31}{16:38}{\textsf{\kpanc} {\tiny \RIGHTarrow} 14:13\hspace{2ex}}{\textsf{\pushya} {\tiny \RIGHTarrow} 01:27(+1)} 
&
%\sunmonth{\vrishchika}{12}{}

\caldata{27}{07:48}{09:34}{10:00-11:06}{13:19-14:25}{16:38}{\textsf{\ksha} {\tiny \RIGHTarrow} 12:31\hspace{2ex}}{\textsf{\ashresha} {\tiny \RIGHTarrow} 00:19(+1)} 
\\ \hline
%\sunmonth{\vrishchika}{13}{}

\caldata{28}{07:49}{09:34}{15:31-16:37}{12:13-13:19}{16:37}{\textsf{\ksap} {\tiny \RIGHTarrow} 10:37\hspace{2ex}}{\textsf{\magha} {\tiny \RIGHTarrow} 23:00\hspace{2ex}} 
&
%\sunmonth{\vrishchika}{14}{}

\caldata{29}{07:50}{09:35}{08:55-10:01}{11:07-12:13}{16:36}{\textsf{\kasht} {\tiny \RIGHTarrow} 08:32\hspace{2ex}}{\textsf{\purvaphalguni} {\tiny \RIGHTarrow} 21:32\hspace{2ex}} 
&
%\sunmonth{\vrishchika}{15}{}

\caldata{30}{07:52}{09:36}{14:25-15:30}{10:03-11:08}{16:36}{\textsf{\kdas} {\tiny \RIGHTarrow} 04:02(+1)}{\textsf{\uttaraphalguni} {\tiny \RIGHTarrow} 19:59\hspace{2ex}} 
&
%\sunmonth{\vrishchika}{16}{}

{}  &
{}  &
{}  &
\\ \hline
\end{tabular}


%\clearpage
\begin{tabular}{|c|c|c|c|c|c|c|}
\multicolumn{7}{c}{\Large \bfseries DECEMBER 2010}\\
\hline
\textbf{SUN} & \textbf{MON} & \textbf{TUE} & \textbf{WED} & \textbf{THU} & \textbf{FRI} & \textbf{SAT} \\ \hline
{}  &
{}  &
{}  &
\caldata{1}{07:53}{09:37}{12:14-13:19}{08:58-10:03}{16:35}{\textsf{\keka} {\tiny \RIGHTarrow} 01:46(+1)}{\textsf{\hasta} {\tiny \RIGHTarrow} 18:24\hspace{2ex}} 
&
%\sunmonth{\vrishchika}{17}{}

\caldata{2}{07:54}{09:38}{13:19-14:24}{07:54-08:59}{16:35}{\textsf{\kdva} {\tiny \RIGHTarrow} 23:35\hspace{2ex}}{\textsf{\chitra} {\tiny \RIGHTarrow} 16:53\hspace{2ex}} 
&
%\sunmonth{\vrishchika}{18}{}

\caldata{3}{07:55}{09:39}{11:10-12:15}{14:25-15:30}{16:35}{\textsf{\ktra} {\tiny \RIGHTarrow} 21:37\hspace{2ex}}{\textsf{\svati} {\tiny \RIGHTarrow} 15:31\hspace{2ex}} 
&
%\sunmonth{\vrishchika}{19}{}

\caldata{4}{07:57}{09:40}{10:06-11:10}{13:20-14:24}{16:34}{\textsf{\kchaturdashi} {\tiny \RIGHTarrow} 19:56\hspace{2ex}}{\textsf{\vishakha} {\tiny \RIGHTarrow} 14:25\hspace{2ex}} 
\\ \hline
%\sunmonth{\vrishchika}{20}{}

\caldata{5}{07:58}{09:41}{15:29-16:34}{12:16-13:20}{16:34}{\textsf{\ama} {\tiny \RIGHTarrow} 18:39\hspace{2ex}}{\textsf{\anuradha} {\tiny \RIGHTarrow} 13:41\hspace{2ex}} 
&
%\sunmonth{\vrishchika}{21}{}

\caldata{6}{07:59}{09:42}{09:03-10:07}{11:12-12:16}{16:34}{\textsf{\spra} {\tiny \RIGHTarrow} 17:54\hspace{2ex}}{\textsf{\jyeshtha} {\tiny \RIGHTarrow} 13:27\hspace{2ex}} 
&
%\sunmonth{\vrishchika}{22}{}

\caldata{7}{08:00}{09:42}{14:24-15:28}{10:08-11:12}{16:33}{\textsf{\sdvi} {\tiny \RIGHTarrow} 17:44\hspace{2ex}}{\textsf{\mula} {\tiny \RIGHTarrow} 13:47\hspace{2ex}} 
&
%\sunmonth{\vrishchika}{23}{}

\caldata{8}{08:01}{09:43}{12:17-13:21}{09:05-10:09}{16:33}{\textsf{\stri} {\tiny \RIGHTarrow} 18:13\hspace{2ex}}{\textsf{\purvashadha} {\tiny \RIGHTarrow} 14:44\hspace{2ex}} 
&
%\sunmonth{\vrishchika}{24}{}

\caldata{9}{08:02}{09:44}{13:21-14:25}{08:02-09:05}{16:33}{\textsf{\scha} {\tiny \RIGHTarrow} 19:21\hspace{2ex}}{\textsf{\uttarashadha} {\tiny \RIGHTarrow} 16:19\hspace{2ex}} 
&
%\sunmonth{\vrishchika}{25}{}

\caldata{10}{08:03}{09:45}{11:14-12:18}{14:25-15:29}{16:33}{\textsf{\spanc} {\tiny \RIGHTarrow} 21:04\hspace{2ex}}{\textsf{\shravana} {\tiny \RIGHTarrow} 18:28\hspace{2ex}} 
&
%\sunmonth{\vrishchika}{26}{}

\caldata{11}{08:04}{09:45}{10:11-11:14}{13:22-14:25}{16:33}{\textsf{\ssha} {\tiny \RIGHTarrow} 23:15\hspace{2ex}}{\textsf{\shravishtha} {\tiny \RIGHTarrow} 21:03\hspace{2ex}} 
\\ \hline
%\sunmonth{\vrishchika}{27}{}

\caldata{12}{08:05}{09:46}{15:29-16:33}{12:19-13:22}{16:33}{\textsf{\ssap} {\tiny \RIGHTarrow} 01:42(+1)}{\textsf{\shatabhishak} {\tiny \RIGHTarrow} 23:56\hspace{2ex}} 
&
%\sunmonth{\vrishchika}{28}{}

\caldata{13}{08:06}{09:47}{09:09-10:12}{11:16-12:19}{16:33}{\textsf{\sasht} {\tiny \RIGHTarrow} 04:13(+1)}{\textsf{\proshthapada} {\tiny \RIGHTarrow} 02:52(+1)} 
&
%\sunmonth{\vrishchika}{29}{}

\caldata{14}{08:07}{09:48}{14:26-15:29}{10:13-11:16}{16:33}{\textsf{\snav} {\tiny \RIGHTarrow} 06:34(+1)}{\textsf{\uttaraproshthapada} {\tiny \RIGHTarrow} 05:39(+1)} 
&
%\sunmonth{\dhanur}{1}{(03:22:(+1))}

\caldata{15}{08:07}{09:48}{12:20-13:23}{09:10-10:13}{16:33}{\textsf{\sdas} {\tiny \RIGHTarrow} \ahoratram}{\textsf{\revati} {\tiny \RIGHTarrow} 08:06(+1)} 
&
%\sunmonth{\dhanur}{2}{}

\caldata{16}{08:08}{09:49}{13:24-14:27}{08:08-09:11}{16:34}{\textsf{\sdas} {\tiny \RIGHTarrow} 08:31\hspace{2ex}}{\textsf{\ashwini} {\tiny \RIGHTarrow} \ahoratram} 
&
%\sunmonth{\dhanur}{3}{}

\caldata{17}{08:09}{09:50}{11:18-12:21}{14:27-15:30}{16:34}{\textsf{\seka} {\tiny \RIGHTarrow} 09:54\hspace{2ex}}{\textsf{\ashwini} {\tiny \RIGHTarrow} 09:59\hspace{2ex}} 
&
%\sunmonth{\dhanur}{4}{}

\caldata{18}{08:09}{09:50}{10:15-11:18}{13:24-14:27}{16:34}{\textsf{\sdva} {\tiny \RIGHTarrow} 10:40\hspace{2ex}}{\textsf{\apabharani} {\tiny \RIGHTarrow} 11:16\hspace{2ex}} 
\\ \hline
%\sunmonth{\dhanur}{5}{}

\caldata{19}{08:10}{09:51}{15:31-16:35}{12:22-13:25}{16:35}{\textsf{\stra} {\tiny \RIGHTarrow} 10:47\hspace{2ex}}{\textsf{\krittika} {\tiny \RIGHTarrow} 11:56\hspace{2ex}} 
&
%\sunmonth{\dhanur}{6}{}

\caldata{20}{08:11}{09:51}{09:14-10:17}{11:20-12:23}{16:35}{\textsf{\schaturdashi} {\tiny \RIGHTarrow} 10:16\hspace{2ex}}{\textsf{\rohini} {\tiny \RIGHTarrow} 11:59\hspace{2ex}} 
&
%\sunmonth{\dhanur}{7}{}

\caldata{21}{08:11}{09:51}{14:29-15:32}{10:17-11:20}{16:35}{\textsf{\purnima} {\tiny \RIGHTarrow} 09:12\hspace{2ex}}{\textsf{\mrigashirsha} {\tiny \RIGHTarrow} 11:31\hspace{2ex}} 
&
%\sunmonth{\dhanur}{8}{}

\caldata{22}{08:12}{09:52}{12:24-13:27}{09:15-10:18}{16:36}{\textsf{\kdvi} {\tiny \RIGHTarrow} 05:45(+1)}{\textsf{\ardra} {\tiny \RIGHTarrow} 10:36\hspace{2ex}} 
&
%\sunmonth{\dhanur}{9}{}

\caldata{23}{08:12}{09:53}{13:27-14:30}{08:12-09:15}{16:37}{\textsf{\ktri} {\tiny \RIGHTarrow} 03:35(+1)}{\textsf{\punarvasu} {\tiny \RIGHTarrow} 09:20\hspace{2ex}} 
&
%\sunmonth{\dhanur}{10}{}

\caldata{24}{08:13}{09:53}{11:22-12:25}{14:31-15:34}{16:37}{\textsf{\kcha} {\tiny \RIGHTarrow} 01:17(+1)}{\textsf{\ashresha} {\tiny \RIGHTarrow} 06:11(+1)} 
&
%\sunmonth{\dhanur}{11}{}

\caldata{25}{08:13}{09:54}{10:19-11:22}{13:28-14:31}{16:38}{\textsf{\kpanc} {\tiny \RIGHTarrow} 22:56\hspace{2ex}}{\textsf{\magha} {\tiny \RIGHTarrow} 04:32(+1)} 
\\ \hline
%\sunmonth{\dhanur}{12}{}

\caldata{26}{08:13}{09:54}{15:34-16:38}{12:25-13:28}{16:38}{\textsf{\ksha} {\tiny \RIGHTarrow} 20:36\hspace{2ex}}{\textsf{\purvaphalguni} {\tiny \RIGHTarrow} 02:55(+1)} 
&
%\sunmonth{\dhanur}{13}{}

\caldata{27}{08:14}{09:55}{09:17-10:20}{11:23-12:26}{16:39}{\textsf{\ksap} {\tiny \RIGHTarrow} 18:23\hspace{2ex}}{\textsf{\uttaraphalguni} {\tiny \RIGHTarrow} 01:25(+1)} 
&
%\sunmonth{\dhanur}{14}{}

\caldata{28}{08:14}{09:55}{14:33-15:36}{10:20-11:23}{16:40}{\textsf{\kasht} {\tiny \RIGHTarrow} 16:19\hspace{2ex}}{\textsf{\hasta} {\tiny \RIGHTarrow} 00:07(+1)} 
&
%\sunmonth{\dhanur}{15}{}

\caldata{29}{08:14}{09:55}{12:27-13:30}{09:17-10:20}{16:41}{\textsf{\knav} {\tiny \RIGHTarrow} 14:28\hspace{2ex}}{\textsf{\chitra} {\tiny \RIGHTarrow} 23:01\hspace{2ex}} 
&
%\sunmonth{\dhanur}{16}{}

\caldata{30}{08:14}{09:55}{13:31-14:35}{08:14-09:17}{16:42}{\textsf{\kdas} {\tiny \RIGHTarrow} 12:52\hspace{2ex}}{\textsf{\svati} {\tiny \RIGHTarrow} 22:12\hspace{2ex}} 
&
%\sunmonth{\dhanur}{17}{}

\caldata{31}{08:14}{09:55}{11:24-12:28}{14:35-15:38}{16:42}{\textsf{\keka} {\tiny \RIGHTarrow} 11:33\hspace{2ex}}{\textsf{\vishakha} {\tiny \RIGHTarrow} 21:40\hspace{2ex}} 
&
\\ \hline
\end{tabular}


%\clearpage

\end{center}
\end{document}
