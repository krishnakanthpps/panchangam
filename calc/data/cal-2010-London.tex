\documentclass[a4paper,11pt,landscape]{article}
\usepackage[sort&compress,square,numbers]{natbib}

\usepackage[xetex]{graphicx}
%\usepackage{fullpage}
\usepackage{multirow}
\usepackage[normalsections]{savetrees}
\usepackage{euler}
\usepackage{fontspec}
\usepackage{xltxtra}
\usepackage{url}
\usepackage{multicol}
\usepackage{bbding}
% PDF SETUP
% ---- FILL IN HERE THE DOC TITLE AND AUTHOR
%\usepackage[bookmarks, colorlinks, breaklinks, pdftitle={Karthik Raman - vita},pdfauthor={Karthik Raman}]{hyperref} 
\usepackage[dvipsnames]{xcolor} 
\usepackage{wasysym} 
%\hypersetup{linkcolor=Sepia,citecolor=blue,filecolor=black,urlcolor=Blue} 


\defaultfontfeatures{Scale=MatchLowercase,Mapping=tex-text}
\setmainfont{Scala Sans LF}
\setsansfont{Sanskrit 2003:script=deva}

\newcommand{\caldata}[8]{%
\begin{minipage}{2.6cm}
\begin{minipage}[t]{1.6cm}
\vspace{.2ex}
%\mbox{}\\
%\SunshineOpenCircled
\mbox{{\sun\tiny\UParrow} \small #2}\\
\mbox{{\sun\tiny\DOWNarrow} \small  #6}\\
\scriptsize
\mbox{(\textsf{स} #3)}\\
\mbox{#7}\\
\mbox{#8}\\
\mbox{\textsf{राहु~} #4}\\
\mbox{\textsf{यम~} #5}\\
\end{minipage}\begin{minipage}[c]{1.cm}
\vspace{.4ex}
\begin{flushright} \textcolor{blue}{\font\x="Plantin Std" at 24 pt\x #1}
\end{flushright}
\end{minipage}
\end{minipage}
}

\addtolength{\headsep}{-3ex}
\pagestyle{empty}
\newcommand{\ashwini}{अश्विनी}
\newcommand{\apabharani}{अपभरणी}
\newcommand{\krittika}{कृत्तिका}
\newcommand{\rohini}{रोहिणी}
\newcommand{\mrigashirsha}{मृगशीर्ष}
\newcommand{\ardra}{आर्द्रा}
\newcommand{\punarvasu}{पुनर्वसू}
\newcommand{\pushya}{पुष्य}
\newcommand{\ashresha}{आश्रेषा}
\newcommand{\magha}{मघा}
\newcommand{\purvaphalguni}{पूर्वफल्गुनी}
\newcommand{\uttaraphalguni}{उत्तरफल्गुनी}
\newcommand{\hasta}{हस्त}
\newcommand{\chitra}{चित्रा}
\newcommand{\svati}{स्वाति}
\newcommand{\vishakha}{विशाखा}
\newcommand{\anuradha}{अनूराधा}
\newcommand{\jyeshtha}{ज्येष्ठा}
\newcommand{\mula}{मूला}
\newcommand{\purvashadha}{पूर्वाषाढा}
\newcommand{\uttarashadha}{उत्तराषाढा}
\newcommand{\shravana}{श्रवण}
\newcommand{\shravishtha}{श्रविष्ठा}
\newcommand{\shatabhishak}{शतभिषक्}
\newcommand{\proshthapada}{प्रोष्ठपदा}
\newcommand{\uttaraproshthapada}{उत्तरप्रोष्ठपदा}
\newcommand{\revati}{रेवती}

\newcommand{\spra}{शुक्ल प्रथमा}
\newcommand{\sdvi}{शुक्ल द्वितीया}
\newcommand{\stri}{शुक्ल तृतीया}
\newcommand{\scha}{शुक्ल चतुर्थी}
\newcommand{\spanc}{शुक्ल पञ्चमी}
\newcommand{\ssha}{शुक्ल षष्ठी}
\newcommand{\ssap}{शुक्ल सप्तमी}
\newcommand{\sasht}{शुक्ल अष्टमी}
\newcommand{\snav}{शुक्ल नवमी}
\newcommand{\sdas}{शुक्ल दशमी}
\newcommand{\seka}{शुक्ल एकादशी}
\newcommand{\sdva}{शुक्ल द्वादशी}
\newcommand{\stra}{शुक्ल त्रयोदशी}
\newcommand{\schaturdashi}{शुक्ल चतुर्दशी}
\newcommand{\purnima}{पूर्णिमा}
\newcommand{\kpra}{कृष्ण प्रथमा}
\newcommand{\kdvi}{कृष्ण द्वितीया}
\newcommand{\ktri}{कृष्ण तृतीया}
\newcommand{\kcha}{कृष्ण चतुर्थी}
\newcommand{\kpanc}{कृष्ण पञ्चमी}
\newcommand{\ksha}{कृष्ण षष्ठी}
\newcommand{\ksap}{कृष्ण सप्तमी}
\newcommand{\kasht}{कृष्ण अष्टमी}
\newcommand{\knav}{कृष्ण नवमी}
\newcommand{\kdas}{कृष्ण दशमी}
\newcommand{\keka}{कृष्ण एकादशी}
\newcommand{\kdva}{कृष्ण द्वादशी}
\newcommand{\ktra}{कृष्ण त्रयोदशी}
\newcommand{\kchaturdashi}{कृष्ण चतुर्दशी}
\newcommand{\ama}{अमावस्या}
\begin{document}
\pagestyle{empty}
\begin{center}
\mbox{}\\[2.5in]
\hrule\mbox{}
\mbox{}\\[1ex]
\mbox{}
{\font\x="Warnock Pro" at 60 pt\x 2010\\[0.5cm]}
\mbox{}
{\font\x="Warnock Pro" at 48 pt\x \uppercase{London}\\[0.3cm]}
\hrule
\begin{tabular}{|c|c|c|c|c|c|c|}
\multicolumn{7}{c}{\Large \bfseries JANUARY 2010}\\
\hline
\textbf{SUN} & \textbf{MON} & \textbf{TUE} & \textbf{WED} & \textbf{THU} & \textbf{FRI} & \textbf{SAT} \\ \hline
{}  &
{}  &
{}  &
{}  &
{}  &
\caldata{1}{\sunmonth{\dhanur}{17}{}}{\sundata{08:07}{15:59}{09:41}}{\textsf{\kpra} {\tiny \RIGHTarrow} 15:43\hspace{2ex}}{\textsf{\punarvasu} {\tiny \RIGHTarrow} 22:26\hspace{2ex}}{11:04-12:03}{14:01-15:00} 
&
\caldata{2}{\sunmonth{\dhanur}{18}{}}{\sundata{08:06}{16:00}{09:40}}{\textsf{\kdvi} {\tiny \RIGHTarrow} 12:14\hspace{2ex}}{\textsf{\pushya} {\tiny \RIGHTarrow} 19:43\hspace{2ex}}{10:04-11:03}{13:02-14:01} 
\\ \hline
\caldata{3}{\sunmonth{\dhanur}{19}{}}{\sundata{08:06}{16:01}{09:41}}{\textsf{\ktri} {\tiny \RIGHTarrow} 08:50\hspace{2ex}}{\textsf{\ashresha} {\tiny \RIGHTarrow} 17:10\hspace{2ex}}{15:01-16:01}{12:03-13:02} 
&
\caldata{4}{\sunmonth{\dhanur}{20}{}}{\sundata{08:06}{16:02}{09:41}}{\textsf{\kpanc} {\tiny \RIGHTarrow} 03:04(+1)}{\textsf{\magha} {\tiny \RIGHTarrow} 14:56\hspace{2ex}}{09:05-10:05}{11:04-12:04} 
&
\caldata{5}{\sunmonth{\dhanur}{21}{}}{\sundata{08:06}{16:03}{09:41}}{\textsf{\ksha} {\tiny \RIGHTarrow} 00:54(+1)}{\textsf{\purvaphalguni} {\tiny \RIGHTarrow} 13:08\hspace{2ex}}{14:03-15:03}{10:05-11:04} 
&
\caldata{6}{\sunmonth{\dhanur}{22}{}}{\sundata{08:05}{16:05}{09:41}}{\textsf{\ksap} {\tiny \RIGHTarrow} 23:18\hspace{2ex}}{\textsf{\uttaraphalguni} {\tiny \RIGHTarrow} 11:53\hspace{2ex}}{12:05-13:05}{09:05-10:05} 
&
\caldata{7}{\sunmonth{\dhanur}{23}{}}{\sundata{08:05}{16:06}{09:41}}{\textsf{\kasht} {\tiny \RIGHTarrow} 22:20\hspace{2ex}}{\textsf{\hasta} {\tiny \RIGHTarrow} 11:14\hspace{2ex}}{13:05-14:05}{08:05-09:05} 
&
\caldata{8}{\sunmonth{\dhanur}{24}{}}{\sundata{08:04}{16:07}{09:40}}{\textsf{\knav} {\tiny \RIGHTarrow} 22:01\hspace{2ex}}{\textsf{\chitra} {\tiny \RIGHTarrow} 11:14\hspace{2ex}}{11:05-12:05}{14:06-15:06} 
&
\caldata{9}{\sunmonth{\dhanur}{25}{}}{\sundata{08:04}{16:09}{09:41}}{\textsf{\kdas} {\tiny \RIGHTarrow} 22:20\hspace{2ex}}{\textsf{\svati} {\tiny \RIGHTarrow} 11:52\hspace{2ex}}{10:05-11:05}{13:07-14:07} 
\\ \hline
\caldata{10}{\sunmonth{\dhanur}{26}{}}{\sundata{08:03}{16:10}{09:40}}{\textsf{\keka} {\tiny \RIGHTarrow} 23:14\hspace{2ex}}{\textsf{\vishakha} {\tiny \RIGHTarrow} 13:05\hspace{2ex}}{15:09-16:10}{12:06-13:07} 
&
\caldata{11}{\sunmonth{\dhanur}{27}{}}{\sundata{08:03}{16:11}{09:40}}{\textsf{\kdva} {\tiny \RIGHTarrow} 00:39(+1)}{\textsf{\anuradha} {\tiny \RIGHTarrow} 14:49\hspace{2ex}}{09:04-10:05}{11:06-12:07} 
&
\caldata{12}{\sunmonth{\dhanur}{28}{}}{\sundata{08:02}{16:13}{09:40}}{\textsf{\ktra} {\tiny \RIGHTarrow} 02:30(+1)}{\textsf{\jyeshtha} {\tiny \RIGHTarrow} 17:00\hspace{2ex}}{14:10-15:11}{10:04-11:06} 
&
\caldata{13}{\sunmonth{\dhanur}{29}{{\textsf{\dhanur} {\tiny \RIGHTarrow} (07:01(+1))}}}{\sundata{08:01}{16:14}{09:39}}{\textsf{\kchaturdashi} {\tiny \RIGHTarrow} 04:42(+1)}{\textsf{\mula} {\tiny \RIGHTarrow} 19:33\hspace{2ex}}{12:07-13:09}{09:02-10:04} 
&
\caldata{14}{\sunmonth{\makara}{1}{}}{\sundata{08:01}{16:16}{09:40}}{\textsf{\ama} {\tiny \RIGHTarrow} 07:11(+1)}{\textsf{\purvashadha} {\tiny \RIGHTarrow} 22:23\hspace{2ex}}{13:10-14:12}{08:01-09:02} 
&
\caldata{15}{\sunmonth{\makara}{2}{}}{\sundata{08:00}{16:17}{09:39}}{\textsf{\spra} {\tiny \RIGHTarrow} \ahoratram}{\textsf{\uttarashadha} {\tiny \RIGHTarrow} 01:24(+1)}{11:06-12:08}{14:12-15:14} 
&
\caldata{16}{\sunmonth{\makara}{3}{}}{\sundata{07:59}{16:19}{09:39}}{\textsf{\spra} {\tiny \RIGHTarrow} 09:50\hspace{2ex}}{\textsf{\shravana} {\tiny \RIGHTarrow} 04:31(+1)}{10:04-11:06}{13:11-14:14} 
\\ \hline
\caldata{17}{\sunmonth{\makara}{4}{}}{\sundata{07:58}{16:21}{09:38}}{\textsf{\sdvi} {\tiny \RIGHTarrow} 12:34\hspace{2ex}}{\textsf{\shravishtha} {\tiny \RIGHTarrow} 07:37(+1)}{15:18-16:21}{12:09-13:12} 
&
\caldata{18}{\sunmonth{\makara}{5}{}}{\sundata{07:57}{16:22}{09:38}}{\textsf{\stri} {\tiny \RIGHTarrow} 15:14\hspace{2ex}}{\textsf{\shatabhishak} {\tiny \RIGHTarrow} \ahoratram}{09:00-10:03}{11:06-12:09} 
&
\caldata{19}{\sunmonth{\makara}{6}{}}{\sundata{07:56}{16:24}{09:37}}{\textsf{\scha} {\tiny \RIGHTarrow} 17:42\hspace{2ex}}{\textsf{\shatabhishak} {\tiny \RIGHTarrow} 10:35\hspace{2ex}}{14:17-15:20}{10:03-11:06} 
&
\caldata{20}{\sunmonth{\makara}{7}{}}{\sundata{07:55}{16:26}{09:37}}{\textsf{\spanc} {\tiny \RIGHTarrow} 19:50\hspace{2ex}}{\textsf{\proshthapada} {\tiny \RIGHTarrow} 13:16\hspace{2ex}}{12:10-13:14}{08:58-10:02} 
&
\caldata{21}{\sunmonth{\makara}{8}{}}{\sundata{07:54}{16:27}{09:36}}{\textsf{\ssha} {\tiny \RIGHTarrow} 21:30\hspace{2ex}}{\textsf{\uttaraproshthapada} {\tiny \RIGHTarrow} 15:31\hspace{2ex}}{13:14-14:18}{07:54-08:58} 
&
\caldata{22}{\sunmonth{\makara}{9}{}}{\sundata{07:53}{16:29}{09:36}}{\textsf{\ssap} {\tiny \RIGHTarrow} 22:32\hspace{2ex}}{\textsf{\revati} {\tiny \RIGHTarrow} 17:13\hspace{2ex}}{11:06-12:11}{14:20-15:24} 
&
\caldata{23}{\sunmonth{\makara}{10}{}}{\sundata{07:52}{16:31}{09:35}}{\textsf{\sasht} {\tiny \RIGHTarrow} 22:52\hspace{2ex}}{\textsf{\ashwini} {\tiny \RIGHTarrow} 18:14\hspace{2ex}}{10:01-11:06}{13:16-14:21} 
\\ \hline
\caldata{24}{\sunmonth{\makara}{11}{}}{\sundata{07:50}{16:32}{09:34}}{\textsf{\snav} {\tiny \RIGHTarrow} 22:24\hspace{2ex}}{\textsf{\apabharani} {\tiny \RIGHTarrow} 18:32\hspace{2ex}}{15:26-16:32}{12:11-13:16} 
&
\caldata{25}{\sunmonth{\makara}{12}{}}{\sundata{07:49}{16:34}{09:34}}{\textsf{\sdas} {\tiny \RIGHTarrow} 21:10\hspace{2ex}}{\textsf{\krittika} {\tiny \RIGHTarrow} 18:04\hspace{2ex}}{08:54-10:00}{11:05-12:11} 
&
\caldata{26}{\sunmonth{\makara}{13}{}}{\sundata{07:48}{16:36}{09:33}}{\textsf{\seka} {\tiny \RIGHTarrow} 19:13\hspace{2ex}}{\textsf{\rohini} {\tiny \RIGHTarrow} 16:52\hspace{2ex}}{14:24-15:30}{10:00-11:06} 
&
\caldata{27}{\sunmonth{\makara}{14}{}}{\sundata{07:46}{16:38}{09:32}}{\textsf{\sdva} {\tiny \RIGHTarrow} 16:37\hspace{2ex}}{\textsf{\mrigashirsha} {\tiny \RIGHTarrow} 15:02\hspace{2ex}}{12:12-13:18}{08:52-09:59} 
&
\caldata{28}{\sunmonth{\makara}{15}{}}{\sundata{07:45}{16:39}{09:31}}{\textsf{\stra} {\tiny \RIGHTarrow} 13:31\hspace{2ex}}{\textsf{\ardra} {\tiny \RIGHTarrow} 12:41\hspace{2ex}}{13:18-14:25}{07:45-08:51} 
&
\caldata{29}{\sunmonth{\makara}{16}{}}{\sundata{07:44}{16:41}{09:31}}{\textsf{\schaturdashi} {\tiny \RIGHTarrow} 10:01\hspace{2ex}}{\textsf{\punarvasu} {\tiny \RIGHTarrow} 09:56\hspace{2ex}}{11:05-12:12}{14:26-15:33} 
&
\caldata{30}{\sunmonth{\makara}{17}{}}{\sundata{07:42}{16:43}{09:30}}{\textsf{\kpra} {\tiny \RIGHTarrow} 02:30(+1)}{\textsf{\ashresha} {\tiny \RIGHTarrow} 03:55(+1)}{09:57-11:04}{13:20-14:27} 
\\ \hline
\caldata{31}{\sunmonth{\makara}{18}{}}{\sundata{07:41}{16:45}{09:29}}{\textsf{\kdvi} {\tiny \RIGHTarrow} 22:52\hspace{2ex}}{\textsf{\magha} {\tiny \RIGHTarrow} 01:02(+1)}{15:37-16:45}{12:13-13:21} 
&
{}  &
{}  &
{}  &
{}  &
{}  &
\\ \hline
\end{tabular}


%\clearpage
\begin{tabular}{|c|c|c|c|c|c|c|}
\multicolumn{7}{c}{\Large \bfseries FEBRUARY 2010}\\
\hline
\textbf{SUN} & \textbf{MON} & \textbf{TUE} & \textbf{WED} & \textbf{THU} & \textbf{FRI} & \textbf{SAT} \\ \hline
{}  &
\caldata{1}{\sunmonth{\makara}{19}{}}{\sundata{07:39}{16:47}{09:28}}{\textsf{\ktri} {\tiny \RIGHTarrow} 19:31\hspace{2ex}}{\textsf{\purvaphalguni} {\tiny \RIGHTarrow} 22:29\hspace{2ex}}{08:47-09:56}{11:04-12:13} 
&
\caldata{2}{\sunmonth{\makara}{20}{}}{\sundata{07:38}{16:48}{09:28}}{\textsf{\kcha} {\tiny \RIGHTarrow} 16:37\hspace{2ex}}{\textsf{\uttaraphalguni} {\tiny \RIGHTarrow} 20:24\hspace{2ex}}{14:30-15:39}{09:55-11:04} 
&
\caldata{3}{\sunmonth{\makara}{21}{}}{\sundata{07:36}{16:50}{09:26}}{\textsf{\kpanc} {\tiny \RIGHTarrow} 14:19\hspace{2ex}}{\textsf{\hasta} {\tiny \RIGHTarrow} 18:56\hspace{2ex}}{12:13-13:22}{08:45-09:54} 
&
\caldata{4}{\sunmonth{\makara}{22}{}}{\sundata{07:34}{16:52}{09:25}}{\textsf{\ksha} {\tiny \RIGHTarrow} 12:44\hspace{2ex}}{\textsf{\chitra} {\tiny \RIGHTarrow} 18:12\hspace{2ex}}{13:22-14:32}{07:34-08:43} 
&
\caldata{5}{\sunmonth{\makara}{23}{}}{\sundata{07:33}{16:54}{09:25}}{\textsf{\ksap} {\tiny \RIGHTarrow} 11:56\hspace{2ex}}{\textsf{\svati} {\tiny \RIGHTarrow} 18:14\hspace{2ex}}{11:03-12:13}{14:33-15:43} 
&
\caldata{6}{\sunmonth{\makara}{24}{}}{\sundata{07:31}{16:56}{09:24}}{\textsf{\kasht} {\tiny \RIGHTarrow} 11:59\hspace{2ex}}{\textsf{\vishakha} {\tiny \RIGHTarrow} 19:04\hspace{2ex}}{09:52-11:02}{13:24-14:34} 
\\ \hline
\caldata{7}{\sunmonth{\makara}{25}{}}{\sundata{07:29}{16:58}{09:22}}{\textsf{\knav} {\tiny \RIGHTarrow} 12:48\hspace{2ex}}{\textsf{\anuradha} {\tiny \RIGHTarrow} 20:38\hspace{2ex}}{15:46-16:58}{12:13-13:24} 
&
\caldata{8}{\sunmonth{\makara}{26}{}}{\sundata{07:28}{16:59}{09:22}}{\textsf{\kdas} {\tiny \RIGHTarrow} 14:19\hspace{2ex}}{\textsf{\jyeshtha} {\tiny \RIGHTarrow} 22:48\hspace{2ex}}{08:39-09:50}{11:02-12:13} 
&
\caldata{9}{\sunmonth{\makara}{27}{}}{\sundata{07:26}{17:01}{09:21}}{\textsf{\keka} {\tiny \RIGHTarrow} 16:22\hspace{2ex}}{\textsf{\mula} {\tiny \RIGHTarrow} 01:27(+1)}{14:37-15:49}{09:49-11:01} 
&
\caldata{10}{\sunmonth{\makara}{28}{}}{\sundata{07:24}{17:03}{09:19}}{\textsf{\kdva} {\tiny \RIGHTarrow} 18:48\hspace{2ex}}{\textsf{\purvashadha} {\tiny \RIGHTarrow} 04:25(+1)}{12:13-13:25}{08:36-09:48} 
&
\caldata{11}{\sunmonth{\makara}{29}{}}{\sundata{07:22}{17:05}{09:18}}{\textsf{\ktra} {\tiny \RIGHTarrow} 21:27\hspace{2ex}}{\textsf{\uttarashadha} {\tiny \RIGHTarrow} \ahoratram}{13:26-14:39}{07:22-08:34} 
&
\caldata{12}{\sunmonth{\makara}{30}{{\textsf{\makara} {\tiny \RIGHTarrow} (20:00\hspace{2ex})}}}{\sundata{07:20}{17:07}{09:17}}{\textsf{\kchaturdashi} {\tiny \RIGHTarrow} 00:10(+1)}{\textsf{\uttarashadha} {\tiny \RIGHTarrow} 07:32\hspace{2ex}}{11:00-12:13}{14:40-15:53} 
&
\caldata{13}{\sunmonth{\kumbha}{1}{}}{\sundata{07:19}{17:09}{09:17}}{\textsf{\ama} {\tiny \RIGHTarrow} 02:50(+1)}{\textsf{\shravana} {\tiny \RIGHTarrow} 10:40\hspace{2ex}}{09:46-11:00}{13:27-14:41} 
\\ \hline
\caldata{14}{\sunmonth{\kumbha}{2}{}}{\sundata{07:17}{17:11}{09:15}}{\textsf{\spra} {\tiny \RIGHTarrow} 05:21(+1)}{\textsf{\shravishtha} {\tiny \RIGHTarrow} 13:42\hspace{2ex}}{15:56-17:11}{12:14-13:28} 
&
\caldata{15}{\sunmonth{\kumbha}{3}{}}{\sundata{07:15}{17:12}{09:14}}{\textsf{\sdvi} {\tiny \RIGHTarrow} \ahoratram}{\textsf{\shatabhishak} {\tiny \RIGHTarrow} 16:34\hspace{2ex}}{08:29-09:44}{10:58-12:13} 
&
\caldata{16}{\sunmonth{\kumbha}{4}{}}{\sundata{07:13}{17:14}{09:13}}{\textsf{\sdvi} {\tiny \RIGHTarrow} 07:38\hspace{2ex}}{\textsf{\proshthapada} {\tiny \RIGHTarrow} 19:10\hspace{2ex}}{14:43-15:58}{09:43-10:58} 
&
\caldata{17}{\sunmonth{\kumbha}{5}{}}{\sundata{07:11}{17:16}{09:12}}{\textsf{\stri} {\tiny \RIGHTarrow} 09:36\hspace{2ex}}{\textsf{\uttaraproshthapada} {\tiny \RIGHTarrow} 21:28\hspace{2ex}}{12:13-13:29}{08:26-09:42} 
&
\caldata{18}{\sunmonth{\kumbha}{6}{}}{\sundata{07:09}{17:18}{09:10}}{\textsf{\scha} {\tiny \RIGHTarrow} 11:10\hspace{2ex}}{\textsf{\revati} {\tiny \RIGHTarrow} 23:21\hspace{2ex}}{13:29-14:45}{07:09-08:25} 
&
\caldata{19}{\sunmonth{\kumbha}{7}{}}{\sundata{07:07}{17:20}{09:09}}{\textsf{\spanc} {\tiny \RIGHTarrow} 12:17\hspace{2ex}}{\textsf{\ashwini} {\tiny \RIGHTarrow} 00:47(+1)}{10:56-12:13}{14:46-16:03} 
&
\caldata{20}{\sunmonth{\kumbha}{8}{}}{\sundata{07:05}{17:21}{09:08}}{\textsf{\ssha} {\tiny \RIGHTarrow} 12:53\hspace{2ex}}{\textsf{\apabharani} {\tiny \RIGHTarrow} 01:40(+1)}{09:39-10:56}{13:30-14:47} 
\\ \hline
\caldata{21}{\sunmonth{\kumbha}{9}{}}{\sundata{07:03}{17:23}{09:07}}{\textsf{\ssap} {\tiny \RIGHTarrow} 12:52\hspace{2ex}}{\textsf{\krittika} {\tiny \RIGHTarrow} 01:56(+1)}{16:05-17:23}{12:13-13:30} 
&
\caldata{22}{\sunmonth{\kumbha}{10}{}}{\sundata{07:01}{17:25}{09:05}}{\textsf{\sasht} {\tiny \RIGHTarrow} 12:14\hspace{2ex}}{\textsf{\rohini} {\tiny \RIGHTarrow} 01:34(+1)}{08:19-09:37}{10:55-12:13} 
&
\caldata{23}{\sunmonth{\kumbha}{11}{}}{\sundata{06:59}{17:27}{09:04}}{\textsf{\snav} {\tiny \RIGHTarrow} 10:56\hspace{2ex}}{\textsf{\mrigashirsha} {\tiny \RIGHTarrow} 00:32(+1)}{14:50-16:08}{09:36-10:54} 
&
\caldata{24}{\sunmonth{\kumbha}{12}{}}{\sundata{06:57}{17:29}{09:03}}{\textsf{\sdas} {\tiny \RIGHTarrow} 09:00\hspace{2ex}}{\textsf{\ardra} {\tiny \RIGHTarrow} 22:54\hspace{2ex}}{12:13-13:32}{08:16-09:35} 
&
\caldata{25}{\sunmonth{\kumbha}{13}{}}{\sundata{06:55}{17:30}{09:02}}{\textsf{\sdva} {\tiny \RIGHTarrow} 03:23(+1)}{\textsf{\punarvasu} {\tiny \RIGHTarrow} 20:45\hspace{2ex}}{13:31-14:51}{06:55-08:14} 
&
\caldata{26}{\sunmonth{\kumbha}{14}{}}{\sundata{06:52}{17:32}{09:00}}{\textsf{\stra} {\tiny \RIGHTarrow} 23:57\hspace{2ex}}{\textsf{\pushya} {\tiny \RIGHTarrow} 18:12\hspace{2ex}}{10:52-12:12}{14:52-16:12} 
&
\caldata{27}{\sunmonth{\kumbha}{15}{}}{\sundata{06:50}{17:34}{08:58}}{\textsf{\schaturdashi} {\tiny \RIGHTarrow} 20:20\hspace{2ex}}{\textsf{\ashresha} {\tiny \RIGHTarrow} 15:23\hspace{2ex}}{09:31-10:51}{13:32-14:53} 
\\ \hline
\caldata{28}{\sunmonth{\kumbha}{16}{}}{\sundata{06:48}{17:36}{08:57}}{\textsf{\purnima} {\tiny \RIGHTarrow} 16:39\hspace{2ex}}{\textsf{\magha} {\tiny \RIGHTarrow} 12:27\hspace{2ex}}{16:15-17:36}{12:12-13:33} 
&
{}  &
{}  &
{}  &
{}  &
{}  &
\\ \hline
\end{tabular}


%\clearpage
\begin{tabular}{|c|c|c|c|c|c|c|}
\multicolumn{7}{c}{\Large \bfseries MARCH 2010}\\
\hline
\textbf{SUN} & \textbf{MON} & \textbf{TUE} & \textbf{WED} & \textbf{THU} & \textbf{FRI} & \textbf{SAT} \\ \hline
{}  &
\caldata{1}{\sunmonth{\kumbha}{17}{}}{\sundata{06:46}{17:38}{08:56}}{\textsf{\kpra} {\tiny \RIGHTarrow} 13:04\hspace{2ex}}{\textsf{\purvaphalguni} {\tiny \RIGHTarrow} 09:35\hspace{2ex}}{08:07-09:29}{10:50-12:12} 
&
\caldata{2}{\sunmonth{\kumbha}{18}{}}{\sundata{06:44}{17:39}{08:55}}{\textsf{\kdvi} {\tiny \RIGHTarrow} 09:46\hspace{2ex}}{\textsf{\uttaraphalguni} {\tiny \RIGHTarrow} 06:58\hspace{2ex}}{14:55-16:17}{09:27-10:49} 
&
\caldata{3}{\sunmonth{\kumbha}{19}{}}{\sundata{06:42}{17:41}{08:53}}{\textsf{\ktri} {\tiny \RIGHTarrow} 06:55\hspace{2ex}}{\textsf{\chitra} {\tiny \RIGHTarrow} 03:22(+1)}{12:11-13:33}{08:04-09:26} 
&
\caldata{4}{\sunmonth{\kumbha}{20}{}}{\sundata{06:40}{17:43}{08:52}}{\textsf{\kpanc} {\tiny \RIGHTarrow} 03:24(+1)}{\textsf{\svati} {\tiny \RIGHTarrow} 02:38(+1)}{13:34-14:57}{06:40-08:02} 
&
\caldata{5}{\sunmonth{\kumbha}{21}{}}{\sundata{06:37}{17:45}{08:50}}{\textsf{\ksha} {\tiny \RIGHTarrow} 02:53(+1)}{\textsf{\vishakha} {\tiny \RIGHTarrow} 02:43(+1)}{10:47-12:11}{14:58-16:21} 
&
\caldata{6}{\sunmonth{\kumbha}{22}{}}{\sundata{06:35}{17:46}{08:49}}{\textsf{\ksap} {\tiny \RIGHTarrow} 03:14(+1)}{\textsf{\anuradha} {\tiny \RIGHTarrow} 03:40(+1)}{09:22-10:46}{13:34-14:58} 
\\ \hline
\caldata{7}{\sunmonth{\kumbha}{23}{}}{\sundata{06:33}{17:48}{08:48}}{\textsf{\kasht} {\tiny \RIGHTarrow} 04:25(+1)}{\textsf{\jyeshtha} {\tiny \RIGHTarrow} 05:24(+1)}{16:23-17:48}{12:10-13:34} 
&
\caldata{8}{\sunmonth{\kumbha}{24}{}}{\sundata{06:31}{17:50}{08:46}}{\textsf{\knav} {\tiny \RIGHTarrow} 06:17(+1)}{\textsf{\mula} {\tiny \RIGHTarrow} \ahoratram}{07:55-09:20}{10:45-12:10} 
&
\caldata{9}{\sunmonth{\kumbha}{25}{}}{\sundata{06:28}{17:52}{08:44}}{\textsf{\kdas} {\tiny \RIGHTarrow} \ahoratram}{\textsf{\mula} {\tiny \RIGHTarrow} 07:49\hspace{2ex}}{15:01-16:26}{09:19-10:44} 
&
\caldata{10}{\sunmonth{\kumbha}{26}{}}{\sundata{06:26}{17:53}{08:43}}{\textsf{\kdas} {\tiny \RIGHTarrow} 08:41\hspace{2ex}}{\textsf{\purvashadha} {\tiny \RIGHTarrow} 10:43\hspace{2ex}}{12:09-13:35}{07:51-09:17} 
&
\caldata{11}{\sunmonth{\kumbha}{27}{}}{\sundata{06:24}{17:55}{08:42}}{\textsf{\keka} {\tiny \RIGHTarrow} 11:21\hspace{2ex}}{\textsf{\uttarashadha} {\tiny \RIGHTarrow} 13:51\hspace{2ex}}{13:35-15:02}{06:24-07:50} 
&
\caldata{12}{\sunmonth{\kumbha}{28}{}}{\sundata{06:22}{17:57}{08:41}}{\textsf{\kdva} {\tiny \RIGHTarrow} 14:04\hspace{2ex}}{\textsf{\shravana} {\tiny \RIGHTarrow} 17:01\hspace{2ex}}{10:42-12:09}{15:03-16:30} 
&
\caldata{13}{\sunmonth{\kumbha}{29}{}}{\sundata{06:20}{17:58}{08:39}}{\textsf{\ktra} {\tiny \RIGHTarrow} 16:39\hspace{2ex}}{\textsf{\shravishtha} {\tiny \RIGHTarrow} 20:02\hspace{2ex}}{09:14-10:41}{13:36-15:03} 
\\ \hline
\caldata{14}{\sunmonth{\mina}{1}{{\textsf{\kumbha} {\tiny \RIGHTarrow} (16:52\hspace{2ex})}}}{\sundata{06:17}{18:00}{08:37}}{\textsf{\kchaturdashi} {\tiny \RIGHTarrow} 18:59\hspace{2ex}}{\textsf{\shatabhishak} {\tiny \RIGHTarrow} 22:49\hspace{2ex}}{16:32-18:00}{12:08-13:36} 
&
\caldata{15}{\sunmonth{\mina}{2}{}}{\sundata{06:15}{18:02}{08:36}}{\textsf{\ama} {\tiny \RIGHTarrow} 20:58\hspace{2ex}}{\textsf{\proshthapada} {\tiny \RIGHTarrow} 01:15(+1)}{07:43-09:11}{10:40-12:08} 
&
\caldata{16}{\sunmonth{\mina}{3}{}}{\sundata{06:13}{18:04}{08:35}}{\textsf{\spra} {\tiny \RIGHTarrow} 22:35\hspace{2ex}}{\textsf{\uttaraproshthapada} {\tiny \RIGHTarrow} 03:20(+1)}{15:06-16:35}{09:10-10:39} 
&
\caldata{17}{\sunmonth{\mina}{4}{}}{\sundata{06:10}{18:05}{08:33}}{\textsf{\sdvi} {\tiny \RIGHTarrow} 23:48\hspace{2ex}}{\textsf{\revati} {\tiny \RIGHTarrow} 05:02(+1)}{12:07-13:36}{07:39-09:08} 
&
\caldata{18}{\sunmonth{\mina}{5}{}}{\sundata{06:08}{18:07}{08:31}}{\textsf{\stri} {\tiny \RIGHTarrow} 00:36(+1)}{\textsf{\ashwini} {\tiny \RIGHTarrow} \ahoratram}{13:37-15:07}{06:08-07:37} 
&
\caldata{19}{\sunmonth{\mina}{6}{}}{\sundata{06:06}{18:09}{08:30}}{\textsf{\scha} {\tiny \RIGHTarrow} 01:01(+1)}{\textsf{\ashwini} {\tiny \RIGHTarrow} 06:21\hspace{2ex}}{10:37-12:07}{15:08-16:38} 
&
\caldata{20}{\sunmonth{\mina}{7}{}}{\sundata{06:04}{18:10}{08:29}}{\textsf{\spanc} {\tiny \RIGHTarrow} 01:00(+1)}{\textsf{\apabharani} {\tiny \RIGHTarrow} 07:15\hspace{2ex}}{09:05-10:36}{13:37-15:08} 
\\ \hline
\caldata{21}{\sunmonth{\mina}{8}{}}{\sundata{06:01}{18:12}{08:27}}{\textsf{\ssha} {\tiny \RIGHTarrow} 00:32(+1)}{\textsf{\krittika} {\tiny \RIGHTarrow} 07:45\hspace{2ex}}{16:40-18:12}{12:06-13:37} 
&
\caldata{22}{\sunmonth{\mina}{9}{}}{\sundata{05:59}{18:14}{08:26}}{\textsf{\ssap} {\tiny \RIGHTarrow} 23:35\hspace{2ex}}{\textsf{\rohini} {\tiny \RIGHTarrow} 07:49\hspace{2ex}}{07:30-09:02}{10:34-12:06} 
&
\caldata{23}{\sunmonth{\mina}{10}{}}{\sundata{05:57}{18:15}{08:24}}{\textsf{\sasht} {\tiny \RIGHTarrow} 22:10\hspace{2ex}}{\textsf{\mrigashirsha} {\tiny \RIGHTarrow} 07:24\hspace{2ex}}{15:10-16:42}{09:01-10:33} 
&
\caldata{24}{\sunmonth{\mina}{11}{}}{\sundata{05:55}{18:17}{08:23}}{\textsf{\snav} {\tiny \RIGHTarrow} 20:15\hspace{2ex}}{\textsf{\ardra} {\tiny \RIGHTarrow} 06:31\hspace{2ex}}{12:06-13:38}{07:27-09:00} 
&
\caldata{25}{\sunmonth{\mina}{12}{}}{\sundata{05:52}{18:19}{08:21}}{\textsf{\sdas} {\tiny \RIGHTarrow} 17:53\hspace{2ex}}{\textsf{\pushya} {\tiny \RIGHTarrow} 03:21(+1)}{13:38-15:12}{05:52-07:25} 
&
\caldata{26}{\sunmonth{\mina}{13}{}}{\sundata{05:50}{18:21}{08:20}}{\textsf{\seka} {\tiny \RIGHTarrow} 15:08\hspace{2ex}}{\textsf{\ashresha} {\tiny \RIGHTarrow} 01:11(+1)}{10:31-12:05}{15:13-16:47} 
&
\caldata{27}{\sunmonth{\mina}{14}{}}{\sundata{05:48}{18:22}{08:18}}{\textsf{\sdva} {\tiny \RIGHTarrow} 12:05\hspace{2ex}}{\textsf{\magha} {\tiny \RIGHTarrow} 22:47\hspace{2ex}}{08:56-10:30}{13:39-15:13} 
\\ \hline
\caldata{28}{\sunmonth{\mina}{15}{}}{\sundata{06:45}{19:24}{09:16}}{\textsf{\stra} {\tiny \RIGHTarrow} 09:51\hspace{2ex}}{\textsf{\purvaphalguni} {\tiny \RIGHTarrow} 21:17\hspace{2ex}}{17:49-19:24}{13:04-14:39} 
&
\caldata{29}{\sunmonth{\mina}{16}{}}{\sundata{06:43}{19:26}{09:15}}{\textsf{\purnima} {\tiny \RIGHTarrow} 03:26(+1)}{\textsf{\uttaraphalguni} {\tiny \RIGHTarrow} 18:52\hspace{2ex}}{08:18-09:53}{11:29-13:04} 
&
\caldata{30}{\sunmonth{\mina}{17}{}}{\sundata{06:41}{19:27}{09:14}}{\textsf{\kpra} {\tiny \RIGHTarrow} 00:37(+1)}{\textsf{\hasta} {\tiny \RIGHTarrow} 16:40\hspace{2ex}}{16:15-17:51}{09:52-11:28} 
&
\caldata{31}{\sunmonth{\mina}{18}{}}{\sundata{06:39}{19:29}{09:13}}{\textsf{\kdvi} {\tiny \RIGHTarrow} 22:17\hspace{2ex}}{\textsf{\chitra} {\tiny \RIGHTarrow} 14:52\hspace{2ex}}{13:04-14:40}{08:15-09:51} 
&
{}  &
{}  &
\\ \hline
\end{tabular}


%\clearpage
\begin{tabular}{|c|c|c|c|c|c|c|}
\multicolumn{7}{c}{\Large \bfseries APRIL 2010}\\
\hline
\textbf{SUN} & \textbf{MON} & \textbf{TUE} & \textbf{WED} & \textbf{THU} & \textbf{FRI} & \textbf{SAT} \\ \hline
{}  &
{}  &
{}  &
{}  &
\caldata{1}{\sunmonth{\mina}{19}{}}{\sundata{06:36}{19:31}{09:11}}{\textsf{\ktri} {\tiny \RIGHTarrow} 20:34\hspace{2ex}}{\textsf{\svati} {\tiny \RIGHTarrow} 13:39\hspace{2ex}}{14:40-16:17}{06:36-08:12} 
&
\caldata{2}{\sunmonth{\mina}{20}{}}{\sundata{06:34}{19:32}{09:09}}{\textsf{\kcha} {\tiny \RIGHTarrow} 19:37\hspace{2ex}}{\textsf{\vishakha} {\tiny \RIGHTarrow} 13:09\hspace{2ex}}{11:25-13:03}{16:17-17:54} 
&
\caldata{3}{\sunmonth{\mina}{21}{}}{\sundata{06:32}{19:34}{09:08}}{\textsf{\kpanc} {\tiny \RIGHTarrow} 19:29\hspace{2ex}}{\textsf{\anuradha} {\tiny \RIGHTarrow} 13:28\hspace{2ex}}{09:47-11:25}{14:40-16:18} 
\\ \hline
\caldata{4}{\sunmonth{\mina}{22}{}}{\sundata{06:30}{19:36}{09:07}}{\textsf{\ksha} {\tiny \RIGHTarrow} 20:12\hspace{2ex}}{\textsf{\jyeshtha} {\tiny \RIGHTarrow} 14:36\hspace{2ex}}{17:57-19:36}{13:03-14:41} 
&
\caldata{5}{\sunmonth{\mina}{23}{}}{\sundata{06:27}{19:37}{09:05}}{\textsf{\ksap} {\tiny \RIGHTarrow} 21:40\hspace{2ex}}{\textsf{\mula} {\tiny \RIGHTarrow} 16:31\hspace{2ex}}{08:05-09:44}{11:23-13:02} 
&
\caldata{6}{\sunmonth{\mina}{24}{}}{\sundata{06:25}{19:39}{09:03}}{\textsf{\kasht} {\tiny \RIGHTarrow} 23:46\hspace{2ex}}{\textsf{\purvashadha} {\tiny \RIGHTarrow} 19:02\hspace{2ex}}{16:20-17:59}{09:43-11:22} 
&
\caldata{7}{\sunmonth{\mina}{25}{}}{\sundata{06:23}{19:41}{09:02}}{\textsf{\knav} {\tiny \RIGHTarrow} 02:15(+1)}{\textsf{\uttarashadha} {\tiny \RIGHTarrow} 21:58\hspace{2ex}}{13:02-14:41}{08:02-09:42} 
&
\caldata{8}{\sunmonth{\mina}{26}{}}{\sundata{06:21}{19:42}{09:01}}{\textsf{\kdas} {\tiny \RIGHTarrow} 04:52(+1)}{\textsf{\shravana} {\tiny \RIGHTarrow} 01:05(+1)}{14:41-16:21}{06:21-08:01} 
&
\caldata{9}{\sunmonth{\mina}{27}{}}{\sundata{06:18}{19:44}{08:59}}{\textsf{\keka} {\tiny \RIGHTarrow} \ahoratram}{\textsf{\shravishtha} {\tiny \RIGHTarrow} 04:08(+1)}{11:20-13:01}{16:22-18:03} 
&
\caldata{10}{\sunmonth{\mina}{28}{}}{\sundata{06:16}{19:46}{08:58}}{\textsf{\keka} {\tiny \RIGHTarrow} 07:23\hspace{2ex}}{\textsf{\shatabhishak} {\tiny \RIGHTarrow} \ahoratram}{09:38-11:19}{14:42-16:23} 
\\ \hline
\caldata{11}{\sunmonth{\mina}{29}{}}{\sundata{06:14}{19:47}{08:56}}{\textsf{\kdva} {\tiny \RIGHTarrow} 09:35\hspace{2ex}}{\textsf{\shatabhishak} {\tiny \RIGHTarrow} 06:55\hspace{2ex}}{18:05-19:47}{13:00-14:42} 
&
\caldata{12}{\sunmonth{\mina}{30}{}}{\sundata{06:12}{19:49}{08:55}}{\textsf{\ktra} {\tiny \RIGHTarrow} 11:21\hspace{2ex}}{\textsf{\proshthapada} {\tiny \RIGHTarrow} 09:16\hspace{2ex}}{07:54-09:36}{11:18-13:00} 
&
\caldata{13}{\sunmonth{\mina}{31}{{\textsf{\mina} {\tiny \RIGHTarrow} (02:21(+1))}}}{\sundata{06:10}{19:51}{08:54}}{\textsf{\kchaturdashi} {\tiny \RIGHTarrow} 12:38\hspace{2ex}}{\textsf{\uttaraproshthapada} {\tiny \RIGHTarrow} 11:09\hspace{2ex}}{16:25-18:08}{09:35-11:17} 
&
\caldata{14}{\sunmonth{\mesha}{1}{}}{\sundata{06:07}{19:52}{08:52}}{\textsf{\ama} {\tiny \RIGHTarrow} 13:26\hspace{2ex}}{\textsf{\revati} {\tiny \RIGHTarrow} 12:34\hspace{2ex}}{12:59-14:42}{07:50-09:33} 
&
\caldata{15}{\sunmonth{\mesha}{2}{}}{\sundata{06:05}{19:54}{08:50}}{\textsf{\spra} {\tiny \RIGHTarrow} 13:45\hspace{2ex}}{\textsf{\ashwini} {\tiny \RIGHTarrow} 13:33\hspace{2ex}}{14:43-16:26}{06:05-07:48} 
&
\caldata{16}{\sunmonth{\mesha}{3}{}}{\sundata{06:03}{19:56}{08:49}}{\textsf{\sdvi} {\tiny \RIGHTarrow} 13:40\hspace{2ex}}{\textsf{\apabharani} {\tiny \RIGHTarrow} 14:06\hspace{2ex}}{11:15-12:59}{16:27-18:11} 
&
\caldata{17}{\sunmonth{\mesha}{4}{}}{\sundata{06:01}{19:57}{08:48}}{\textsf{\stri} {\tiny \RIGHTarrow} 13:12\hspace{2ex}}{\textsf{\krittika} {\tiny \RIGHTarrow} 14:19\hspace{2ex}}{09:30-11:14}{14:43-16:28} 
\\ \hline
\caldata{18}{\sunmonth{\mesha}{5}{}}{\sundata{05:59}{19:59}{08:47}}{\textsf{\scha} {\tiny \RIGHTarrow} 12:24\hspace{2ex}}{\textsf{\rohini} {\tiny \RIGHTarrow} 14:12\hspace{2ex}}{18:14-19:59}{12:59-14:44} 
&
\caldata{19}{\sunmonth{\mesha}{6}{}}{\sundata{05:57}{20:01}{08:45}}{\textsf{\spanc} {\tiny \RIGHTarrow} 11:19\hspace{2ex}}{\textsf{\mrigashirsha} {\tiny \RIGHTarrow} 13:47\hspace{2ex}}{07:42-09:28}{11:13-12:59} 
&
\caldata{20}{\sunmonth{\mesha}{7}{}}{\sundata{05:55}{20:02}{08:44}}{\textsf{\ssha} {\tiny \RIGHTarrow} 09:55\hspace{2ex}}{\textsf{\ardra} {\tiny \RIGHTarrow} 13:05\hspace{2ex}}{16:30-18:16}{09:26-11:12} 
&
\caldata{21}{\sunmonth{\mesha}{8}{}}{\sundata{05:53}{20:04}{08:43}}{\textsf{\ssap} {\tiny \RIGHTarrow} 08:15\hspace{2ex}}{\textsf{\punarvasu} {\tiny \RIGHTarrow} 12:06\hspace{2ex}}{12:58-14:44}{07:39-09:25} 
&
\caldata{22}{\sunmonth{\mesha}{9}{}}{\sundata{05:51}{20:06}{08:42}}{\textsf{\sasht} {\tiny \RIGHTarrow} 06:18\hspace{2ex}}{\textsf{\pushya} {\tiny \RIGHTarrow} 10:50\hspace{2ex}}{14:45-16:32}{05:51-07:37} 
&
\caldata{23}{\sunmonth{\mesha}{10}{}}{\sundata{05:49}{20:07}{08:40}}{\textsf{\sdas} {\tiny \RIGHTarrow} 01:38(+1)}{\textsf{\ashresha} {\tiny \RIGHTarrow} 09:20\hspace{2ex}}{11:10-12:58}{16:32-18:19} 
&
\caldata{24}{\sunmonth{\mesha}{11}{}}{\sundata{05:47}{20:09}{08:39}}{\textsf{\seka} {\tiny \RIGHTarrow} 23:03\hspace{2ex}}{\textsf{\magha} {\tiny \RIGHTarrow} 07:38\hspace{2ex}}{09:22-11:10}{14:45-16:33} 
\\ \hline
\caldata{25}{\sunmonth{\mesha}{12}{}}{\sundata{05:45}{20:11}{08:38}}{\textsf{\sdva} {\tiny \RIGHTarrow} 20:25\hspace{2ex}}{\textsf{\purvaphalguni} {\tiny \RIGHTarrow} 05:47\hspace{2ex}}{18:22-20:11}{12:58-14:46} 
&
\caldata{26}{\sunmonth{\mesha}{13}{}}{\sundata{05:43}{20:12}{08:36}}{\textsf{\stra} {\tiny \RIGHTarrow} 17:50\hspace{2ex}}{\textsf{\hasta} {\tiny \RIGHTarrow} 02:09(+1)}{07:31-09:20}{11:08-12:57} 
&
\caldata{27}{\sunmonth{\mesha}{14}{}}{\sundata{05:41}{20:14}{08:35}}{\textsf{\schaturdashi} {\tiny \RIGHTarrow} 15:26\hspace{2ex}}{\textsf{\chitra} {\tiny \RIGHTarrow} 00:38(+1)}{16:35-18:24}{09:19-11:08} 
&
\caldata{28}{\sunmonth{\mesha}{15}{}}{\sundata{05:39}{20:16}{08:34}}{\textsf{\purnima} {\tiny \RIGHTarrow} 13:22\hspace{2ex}}{\textsf{\svati} {\tiny \RIGHTarrow} 23:30\hspace{2ex}}{12:57-14:47}{07:28-09:18} 
&
\caldata{29}{\sunmonth{\mesha}{16}{}}{\sundata{05:37}{20:17}{08:33}}{\textsf{\kpra} {\tiny \RIGHTarrow} 11:45\hspace{2ex}}{\textsf{\vishakha} {\tiny \RIGHTarrow} 22:54\hspace{2ex}}{14:47-16:37}{05:37-07:27} 
&
\caldata{30}{\sunmonth{\mesha}{17}{}}{\sundata{05:35}{20:19}{08:31}}{\textsf{\kdvi} {\tiny \RIGHTarrow} 10:44\hspace{2ex}}{\textsf{\anuradha} {\tiny \RIGHTarrow} 22:56\hspace{2ex}}{11:06-12:57}{16:38-18:28} 
&
\\ \hline
\end{tabular}


%\clearpage
\begin{tabular}{|c|c|c|c|c|c|c|}
\multicolumn{7}{c}{\Large \bfseries MAY 2010}\\
\hline
\textbf{SUN} & \textbf{MON} & \textbf{TUE} & \textbf{WED} & \textbf{THU} & \textbf{FRI} & \textbf{SAT} \\ \hline
{}  &
{}  &
{}  &
{}  &
{}  &
{}  &
\caldata{1}{\sunmonth{\mesha}{18}{}}{\sundata{05:33}{20:20}{08:30}}{\textsf{\ktri} {\tiny \RIGHTarrow} 10:25\hspace{2ex}}{\textsf{\jyeshtha} {\tiny \RIGHTarrow} 23:41\hspace{2ex}}{09:14-11:05}{14:47-16:38} 
\\ \hline
\caldata{2}{\sunmonth{\mesha}{19}{}}{\sundata{05:31}{20:22}{08:29}}{\textsf{\kcha} {\tiny \RIGHTarrow} 10:50\hspace{2ex}}{\textsf{\mula} {\tiny \RIGHTarrow} 01:08(+1)}{18:30-20:22}{12:56-14:47} 
&
\caldata{3}{\sunmonth{\mesha}{20}{}}{\sundata{05:29}{20:24}{08:28}}{\textsf{\kpanc} {\tiny \RIGHTarrow} 12:00\hspace{2ex}}{\textsf{\purvashadha} {\tiny \RIGHTarrow} 03:15(+1)}{07:20-09:12}{11:04-12:56} 
&
\caldata{4}{\sunmonth{\mesha}{21}{}}{\sundata{05:27}{20:25}{08:26}}{\textsf{\ksha} {\tiny \RIGHTarrow} 13:47\hspace{2ex}}{\textsf{\uttarashadha} {\tiny \RIGHTarrow} \ahoratram}{16:40-18:32}{09:11-11:03} 
&
\caldata{5}{\sunmonth{\mesha}{22}{}}{\sundata{05:26}{20:27}{08:26}}{\textsf{\ksap} {\tiny \RIGHTarrow} 16:02\hspace{2ex}}{\textsf{\uttarashadha} {\tiny \RIGHTarrow} 05:54\hspace{2ex}}{12:56-14:49}{07:18-09:11} 
&
\caldata{6}{\sunmonth{\mesha}{23}{}}{\sundata{05:24}{20:29}{08:25}}{\textsf{\kasht} {\tiny \RIGHTarrow} 18:30\hspace{2ex}}{\textsf{\shravana} {\tiny \RIGHTarrow} 08:52\hspace{2ex}}{14:49-16:42}{05:24-07:17} 
&
\caldata{7}{\sunmonth{\mesha}{24}{}}{\sundata{05:22}{20:30}{08:23}}{\textsf{\knav} {\tiny \RIGHTarrow} 20:56\hspace{2ex}}{\textsf{\shravishtha} {\tiny \RIGHTarrow} 11:53\hspace{2ex}}{11:02-12:56}{16:43-18:36} 
&
\caldata{8}{\sunmonth{\mesha}{25}{}}{\sundata{05:20}{20:32}{08:22}}{\textsf{\kdas} {\tiny \RIGHTarrow} 23:08\hspace{2ex}}{\textsf{\shatabhishak} {\tiny \RIGHTarrow} 14:44\hspace{2ex}}{09:08-11:02}{14:50-16:44} 
\\ \hline
\caldata{9}{\sunmonth{\mesha}{26}{}}{\sundata{05:19}{20:33}{08:21}}{\textsf{\keka} {\tiny \RIGHTarrow} 00:52(+1)}{\textsf{\proshthapada} {\tiny \RIGHTarrow} 17:11\hspace{2ex}}{18:38-20:33}{12:56-14:50} 
&
\caldata{10}{\sunmonth{\mesha}{27}{}}{\sundata{05:17}{20:35}{08:20}}{\textsf{\kdva} {\tiny \RIGHTarrow} 02:03(+1)}{\textsf{\uttaraproshthapada} {\tiny \RIGHTarrow} 19:08\hspace{2ex}}{07:11-09:06}{11:01-12:56} 
&
\caldata{11}{\sunmonth{\mesha}{28}{}}{\sundata{05:15}{20:36}{08:19}}{\textsf{\ktra} {\tiny \RIGHTarrow} 02:37(+1)}{\textsf{\revati} {\tiny \RIGHTarrow} 20:31\hspace{2ex}}{16:45-18:40}{09:05-11:00} 
&
\caldata{12}{\sunmonth{\mesha}{29}{}}{\sundata{05:14}{20:38}{08:18}}{\textsf{\kchaturdashi} {\tiny \RIGHTarrow} 02:36(+1)}{\textsf{\ashwini} {\tiny \RIGHTarrow} 21:18\hspace{2ex}}{12:56-14:51}{07:09-09:05} 
&
\caldata{13}{\sunmonth{\mesha}{30}{}}{\sundata{05:12}{20:39}{08:17}}{\textsf{\ama} {\tiny \RIGHTarrow} 02:02(+1)}{\textsf{\apabharani} {\tiny \RIGHTarrow} 21:33\hspace{2ex}}{14:51-16:47}{05:12-07:07} 
&
\caldata{14}{\sunmonth{\mesha}{31}{{\textsf{\mesha} {\tiny \RIGHTarrow} (23:10\hspace{2ex})}}}{\sundata{05:11}{20:41}{08:17}}{\textsf{\spra} {\tiny \RIGHTarrow} 01:01(+1)}{\textsf{\krittika} {\tiny \RIGHTarrow} 21:20\hspace{2ex}}{10:59-12:56}{16:48-18:44} 
&
\caldata{15}{\sunmonth{\vrishabha}{1}{}}{\sundata{05:09}{20:42}{08:15}}{\textsf{\sdvi} {\tiny \RIGHTarrow} 23:38\hspace{2ex}}{\textsf{\rohini} {\tiny \RIGHTarrow} 20:45\hspace{2ex}}{09:02-10:58}{14:52-16:48} 
\\ \hline
\caldata{16}{\sunmonth{\vrishabha}{2}{}}{\sundata{05:08}{20:44}{08:15}}{\textsf{\stri} {\tiny \RIGHTarrow} 21:58\hspace{2ex}}{\textsf{\mrigashirsha} {\tiny \RIGHTarrow} 19:53\hspace{2ex}}{18:47-20:44}{12:56-14:53} 
&
\caldata{17}{\sunmonth{\vrishabha}{3}{}}{\sundata{05:06}{20:45}{08:13}}{\textsf{\scha} {\tiny \RIGHTarrow} 20:06\hspace{2ex}}{\textsf{\ardra} {\tiny \RIGHTarrow} 18:47\hspace{2ex}}{07:03-09:00}{10:58-12:55} 
&
\caldata{18}{\sunmonth{\vrishabha}{4}{}}{\sundata{05:05}{20:47}{08:13}}{\textsf{\spanc} {\tiny \RIGHTarrow} 18:05\hspace{2ex}}{\textsf{\punarvasu} {\tiny \RIGHTarrow} 17:33\hspace{2ex}}{16:51-18:49}{09:00-10:58} 
&
\caldata{19}{\sunmonth{\vrishabha}{5}{}}{\sundata{05:04}{20:48}{08:12}}{\textsf{\ssha} {\tiny \RIGHTarrow} 15:59\hspace{2ex}}{\textsf{\pushya} {\tiny \RIGHTarrow} 16:12\hspace{2ex}}{12:56-14:54}{07:02-09:00} 
&
\caldata{20}{\sunmonth{\vrishabha}{6}{}}{\sundata{05:02}{20:50}{08:11}}{\textsf{\ssap} {\tiny \RIGHTarrow} 13:48\hspace{2ex}}{\textsf{\ashresha} {\tiny \RIGHTarrow} 14:48\hspace{2ex}}{14:54-16:53}{05:02-07:00} 
&
\caldata{21}{\sunmonth{\vrishabha}{7}{}}{\sundata{05:01}{20:51}{08:11}}{\textsf{\sasht} {\tiny \RIGHTarrow} 11:36\hspace{2ex}}{\textsf{\magha} {\tiny \RIGHTarrow} 13:23\hspace{2ex}}{10:57-12:56}{16:53-18:52} 
&
\caldata{22}{\sunmonth{\vrishabha}{8}{}}{\sundata{05:00}{20:53}{08:10}}{\textsf{\snav} {\tiny \RIGHTarrow} 09:25\hspace{2ex}}{\textsf{\purvaphalguni} {\tiny \RIGHTarrow} 11:59\hspace{2ex}}{08:58-10:57}{14:55-16:54} 
\\ \hline
\caldata{23}{\sunmonth{\vrishabha}{9}{}}{\sundata{04:59}{20:54}{08:10}}{\textsf{\sdas} {\tiny \RIGHTarrow} 07:17\hspace{2ex}}{\textsf{\uttaraphalguni} {\tiny \RIGHTarrow} 10:39\hspace{2ex}}{18:54-20:54}{12:56-14:55} 
&
\caldata{24}{\sunmonth{\vrishabha}{10}{}}{\sundata{04:57}{20:55}{08:08}}{\textsf{\seka} {\tiny \RIGHTarrow} 05:16\hspace{2ex}}{\textsf{\hasta} {\tiny \RIGHTarrow} 09:27\hspace{2ex}}{06:56-08:56}{10:56-12:56} 
&
\caldata{25}{\sunmonth{\vrishabha}{11}{}}{\sundata{04:56}{20:56}{08:08}}{\textsf{\stra} {\tiny \RIGHTarrow} 01:58(+1)}{\textsf{\chitra} {\tiny \RIGHTarrow} 08:27\hspace{2ex}}{16:56-18:56}{08:56-10:56} 
&
\caldata{26}{\sunmonth{\vrishabha}{12}{}}{\sundata{04:55}{20:58}{08:07}}{\textsf{\schaturdashi} {\tiny \RIGHTarrow} 00:50(+1)}{\textsf{\svati} {\tiny \RIGHTarrow} 07:46\hspace{2ex}}{12:56-14:56}{06:55-08:55} 
&
\caldata{27}{\sunmonth{\vrishabha}{13}{}}{\sundata{04:54}{20:59}{08:07}}{\textsf{\purnima} {\tiny \RIGHTarrow} 00:09(+1)}{\textsf{\vishakha} {\tiny \RIGHTarrow} 07:29\hspace{2ex}}{14:57-16:57}{04:54-06:54} 
&
\caldata{28}{\sunmonth{\vrishabha}{14}{}}{\sundata{04:53}{21:00}{08:06}}{\textsf{\kpra} {\tiny \RIGHTarrow} 00:02(+1)}{\textsf{\anuradha} {\tiny \RIGHTarrow} 07:41\hspace{2ex}}{10:55-12:56}{16:58-18:59} 
&
\caldata{29}{\sunmonth{\vrishabha}{15}{}}{\sundata{04:52}{21:01}{08:05}}{\textsf{\kdvi} {\tiny \RIGHTarrow} 00:29(+1)}{\textsf{\jyeshtha} {\tiny \RIGHTarrow} 08:26\hspace{2ex}}{08:54-10:55}{14:57-16:58} 
\\ \hline
\caldata{30}{\sunmonth{\vrishabha}{16}{}}{\sundata{04:51}{21:03}{08:05}}{\textsf{\ktri} {\tiny \RIGHTarrow} 01:33(+1)}{\textsf{\mula} {\tiny \RIGHTarrow} 09:47\hspace{2ex}}{19:01-21:03}{12:57-14:58} 
&
\caldata{31}{\sunmonth{\vrishabha}{17}{}}{\sundata{04:50}{21:04}{08:04}}{\textsf{\kcha} {\tiny \RIGHTarrow} 03:10(+1)}{\textsf{\purvashadha} {\tiny \RIGHTarrow} 11:43\hspace{2ex}}{06:51-08:53}{10:55-12:57} 
&
{}  &
{}  &
{}  &
{}  &
\\ \hline
\end{tabular}


%\clearpage
\begin{tabular}{|c|c|c|c|c|c|c|}
\multicolumn{7}{c}{\Large \bfseries JUNE 2010}\\
\hline
\textbf{SUN} & \textbf{MON} & \textbf{TUE} & \textbf{WED} & \textbf{THU} & \textbf{FRI} & \textbf{SAT} \\ \hline
{}  &
{}  &
\caldata{1}{\sunmonth{\vrishabha}{18}{}}{\sundata{04:50}{21:05}{08:05}}{\textsf{\kpanc} {\tiny \RIGHTarrow} \ahoratram}{\textsf{\uttarashadha} {\tiny \RIGHTarrow} 14:08\hspace{2ex}}{17:01-19:03}{08:53-10:55} 
&
\caldata{2}{\sunmonth{\vrishabha}{19}{}}{\sundata{04:49}{21:06}{08:04}}{\textsf{\kpanc} {\tiny \RIGHTarrow} 05:14\hspace{2ex}}{\textsf{\shravana} {\tiny \RIGHTarrow} 16:56\hspace{2ex}}{12:57-14:59}{06:51-08:53} 
&
\caldata{3}{\sunmonth{\vrishabha}{20}{}}{\sundata{04:48}{21:07}{08:03}}{\textsf{\ksha} {\tiny \RIGHTarrow} 07:36\hspace{2ex}}{\textsf{\shravishtha} {\tiny \RIGHTarrow} 19:54\hspace{2ex}}{14:59-17:02}{04:48-06:50} 
&
\caldata{4}{\sunmonth{\vrishabha}{21}{}}{\sundata{04:47}{21:08}{08:03}}{\textsf{\ksap} {\tiny \RIGHTarrow} 10:02\hspace{2ex}}{\textsf{\shatabhishak} {\tiny \RIGHTarrow} 22:50\hspace{2ex}}{10:54-12:57}{17:02-19:05} 
&
\caldata{5}{\sunmonth{\vrishabha}{22}{}}{\sundata{04:47}{21:09}{08:03}}{\textsf{\kasht} {\tiny \RIGHTarrow} 12:16\hspace{2ex}}{\textsf{\proshthapada} {\tiny \RIGHTarrow} 01:31(+1)}{08:52-10:55}{15:00-17:03} 
\\ \hline
\caldata{6}{\sunmonth{\vrishabha}{23}{}}{\sundata{04:46}{21:10}{08:02}}{\textsf{\knav} {\tiny \RIGHTarrow} 14:07\hspace{2ex}}{\textsf{\uttaraproshthapada} {\tiny \RIGHTarrow} 03:45(+1)}{19:07-21:10}{12:58-15:01} 
&
\caldata{7}{\sunmonth{\vrishabha}{24}{}}{\sundata{04:46}{21:11}{08:03}}{\textsf{\kdas} {\tiny \RIGHTarrow} 15:25\hspace{2ex}}{\textsf{\revati} {\tiny \RIGHTarrow} \ahoratram}{06:49-08:52}{10:55-12:58} 
&
\caldata{8}{\sunmonth{\vrishabha}{25}{}}{\sundata{04:45}{21:11}{08:02}}{\textsf{\keka} {\tiny \RIGHTarrow} 16:04\hspace{2ex}}{\textsf{\revati} {\tiny \RIGHTarrow} 05:22\hspace{2ex}}{17:04-19:07}{08:51-10:54} 
&
\caldata{9}{\sunmonth{\vrishabha}{26}{}}{\sundata{04:45}{21:12}{08:02}}{\textsf{\kdva} {\tiny \RIGHTarrow} 16:02\hspace{2ex}}{\textsf{\ashwini} {\tiny \RIGHTarrow} 06:17\hspace{2ex}}{12:58-15:01}{06:48-08:51} 
&
\caldata{10}{\sunmonth{\vrishabha}{27}{}}{\sundata{04:44}{21:13}{08:01}}{\textsf{\ktra} {\tiny \RIGHTarrow} 15:19\hspace{2ex}}{\textsf{\apabharani} {\tiny \RIGHTarrow} 06:33\hspace{2ex}}{15:02-17:05}{04:44-06:47} 
&
\caldata{11}{\sunmonth{\vrishabha}{28}{}}{\sundata{04:44}{21:14}{08:02}}{\textsf{\kchaturdashi} {\tiny \RIGHTarrow} 14:00\hspace{2ex}}{\textsf{\krittika} {\tiny \RIGHTarrow} 06:11\hspace{2ex}}{10:55-12:59}{17:06-19:10} 
&
\caldata{12}{\sunmonth{\vrishabha}{29}{}}{\sundata{04:44}{21:14}{08:02}}{\textsf{\ama} {\tiny \RIGHTarrow} 12:11\hspace{2ex}}{\textsf{\rohini} {\tiny \RIGHTarrow} 05:17\hspace{2ex}}{08:51-10:55}{15:02-17:06} 
\\ \hline
\caldata{13}{\sunmonth{\vrishabha}{30}{}}{\sundata{04:43}{21:15}{08:01}}{\textsf{\spra} {\tiny \RIGHTarrow} 09:59\hspace{2ex}}{\textsf{\ardra} {\tiny \RIGHTarrow} 02:18(+1)}{19:11-21:15}{12:59-15:03} 
&
\caldata{14}{\sunmonth{\vrishabha}{31}{}}{\sundata{04:43}{21:15}{08:01}}{\textsf{\sdvi} {\tiny \RIGHTarrow} 07:30\hspace{2ex}}{\textsf{\punarvasu} {\tiny \RIGHTarrow} 00:28(+1)}{06:47-08:51}{10:55-12:59} 
&
\caldata{15}{\sunmonth{\mithuna}{1}{{\textsf{\vrishabha} {\tiny \RIGHTarrow} (05:43\hspace{2ex})}}}{\sundata{04:43}{21:16}{08:01}}{\textsf{\stri} {\tiny \RIGHTarrow} 04:50\hspace{2ex}}{\textsf{\pushya} {\tiny \RIGHTarrow} 22:34\hspace{2ex}}{17:07-19:11}{08:51-10:55} 
&
\caldata{16}{\sunmonth{\mithuna}{2}{}}{\sundata{04:43}{21:16}{08:01}}{\textsf{\spanc} {\tiny \RIGHTarrow} 23:28\hspace{2ex}}{\textsf{\ashresha} {\tiny \RIGHTarrow} 20:42\hspace{2ex}}{12:59-15:03}{06:47-08:51} 
&
\caldata{17}{\sunmonth{\mithuna}{3}{}}{\sundata{04:43}{21:17}{08:01}}{\textsf{\ssha} {\tiny \RIGHTarrow} 20:56\hspace{2ex}}{\textsf{\magha} {\tiny \RIGHTarrow} 18:57\hspace{2ex}}{15:04-17:08}{04:43-06:47} 
&
\caldata{18}{\sunmonth{\mithuna}{4}{}}{\sundata{04:43}{21:17}{08:01}}{\textsf{\ssap} {\tiny \RIGHTarrow} 18:36\hspace{2ex}}{\textsf{\purvaphalguni} {\tiny \RIGHTarrow} 17:24\hspace{2ex}}{10:55-13:00}{17:08-19:12} 
&
\caldata{19}{\sunmonth{\mithuna}{5}{}}{\sundata{04:43}{21:18}{08:02}}{\textsf{\sasht} {\tiny \RIGHTarrow} 16:31\hspace{2ex}}{\textsf{\uttaraphalguni} {\tiny \RIGHTarrow} 16:06\hspace{2ex}}{08:51-10:56}{15:04-17:09} 
\\ \hline
\caldata{20}{\sunmonth{\mithuna}{6}{}}{\sundata{04:43}{21:18}{08:02}}{\textsf{\snav} {\tiny \RIGHTarrow} 14:45\hspace{2ex}}{\textsf{\hasta} {\tiny \RIGHTarrow} 15:06\hspace{2ex}}{19:13-21:18}{13:00-15:04} 
&
\caldata{21}{\sunmonth{\mithuna}{7}{}}{\sundata{04:43}{21:18}{08:02}}{\textsf{\sdas} {\tiny \RIGHTarrow} 13:19\hspace{2ex}}{\textsf{\chitra} {\tiny \RIGHTarrow} 14:27\hspace{2ex}}{06:47-08:51}{10:56-13:00} 
&
\caldata{22}{\sunmonth{\mithuna}{8}{}}{\sundata{04:44}{21:18}{08:02}}{\textsf{\seka} {\tiny \RIGHTarrow} 12:17\hspace{2ex}}{\textsf{\svati} {\tiny \RIGHTarrow} 14:11\hspace{2ex}}{17:09-19:13}{08:52-10:56} 
&
\caldata{23}{\sunmonth{\mithuna}{9}{}}{\sundata{04:44}{21:18}{08:02}}{\textsf{\sdva} {\tiny \RIGHTarrow} 11:40\hspace{2ex}}{\textsf{\vishakha} {\tiny \RIGHTarrow} 14:20\hspace{2ex}}{13:01-15:05}{06:48-08:52} 
&
\caldata{24}{\sunmonth{\mithuna}{10}{}}{\sundata{04:44}{21:18}{08:02}}{\textsf{\stra} {\tiny \RIGHTarrow} 11:29\hspace{2ex}}{\textsf{\anuradha} {\tiny \RIGHTarrow} 14:56\hspace{2ex}}{15:05-17:09}{04:44-06:48} 
&
\caldata{25}{\sunmonth{\mithuna}{11}{}}{\sundata{04:45}{21:18}{08:03}}{\textsf{\schaturdashi} {\tiny \RIGHTarrow} 11:46\hspace{2ex}}{\textsf{\jyeshtha} {\tiny \RIGHTarrow} 15:59\hspace{2ex}}{10:57-13:01}{17:09-19:13} 
&
\caldata{26}{\sunmonth{\mithuna}{12}{}}{\sundata{04:45}{21:18}{08:03}}{\textsf{\purnima} {\tiny \RIGHTarrow} 12:33\hspace{2ex}}{\textsf{\mula} {\tiny \RIGHTarrow} 17:30\hspace{2ex}}{08:53-10:57}{15:05-17:09} 
\\ \hline
\caldata{27}{\sunmonth{\mithuna}{13}{}}{\sundata{04:46}{21:18}{08:04}}{\textsf{\kpra} {\tiny \RIGHTarrow} 13:47\hspace{2ex}}{\textsf{\purvashadha} {\tiny \RIGHTarrow} 19:28\hspace{2ex}}{19:14-21:18}{13:02-15:06} 
&
\caldata{28}{\sunmonth{\mithuna}{14}{}}{\sundata{04:46}{21:18}{08:04}}{\textsf{\kdvi} {\tiny \RIGHTarrow} 15:28\hspace{2ex}}{\textsf{\uttarashadha} {\tiny \RIGHTarrow} 21:51\hspace{2ex}}{06:50-08:54}{10:58-13:02} 
&
\caldata{29}{\sunmonth{\mithuna}{15}{}}{\sundata{04:47}{21:18}{08:05}}{\textsf{\ktri} {\tiny \RIGHTarrow} 17:31\hspace{2ex}}{\textsf{\shravana} {\tiny \RIGHTarrow} 00:34(+1)}{17:10-19:14}{08:54-10:58} 
&
\caldata{30}{\sunmonth{\mithuna}{16}{}}{\sundata{04:47}{21:18}{08:05}}{\textsf{\kcha} {\tiny \RIGHTarrow} 19:50\hspace{2ex}}{\textsf{\shravishtha} {\tiny \RIGHTarrow} 03:31(+1)}{13:02-15:06}{06:50-08:54} 
&
{}  &
{}  &
\\ \hline
\end{tabular}


%\clearpage
\begin{tabular}{|c|c|c|c|c|c|c|}
\multicolumn{7}{c}{\Large \bfseries JULY 2010}\\
\hline
\textbf{SUN} & \textbf{MON} & \textbf{TUE} & \textbf{WED} & \textbf{THU} & \textbf{FRI} & \textbf{SAT} \\ \hline
{}  &
{}  &
{}  &
{}  &
\caldata{1}{\sunmonth{\mithuna}{17}{}}{\sundata{04:48}{21:18}{08:06}}{\textsf{\kpanc} {\tiny \RIGHTarrow} 22:15\hspace{2ex}}{\textsf{\shatabhishak} {\tiny \RIGHTarrow} \ahoratram}{15:06-17:10}{04:48-06:51} 
&
\caldata{2}{\sunmonth{\mithuna}{18}{}}{\sundata{04:49}{21:17}{08:06}}{\textsf{\ksha} {\tiny \RIGHTarrow} 00:36(+1)}{\textsf{\shatabhishak} {\tiny \RIGHTarrow} 06:31\hspace{2ex}}{10:59-13:03}{17:10-19:13} 
&
\caldata{3}{\sunmonth{\mithuna}{19}{}}{\sundata{04:49}{21:17}{08:06}}{\textsf{\ksap} {\tiny \RIGHTarrow} 02:41(+1)}{\textsf{\proshthapada} {\tiny \RIGHTarrow} 09:22\hspace{2ex}}{08:56-10:59}{15:06-17:10} 
\\ \hline
\caldata{4}{\sunmonth{\mithuna}{20}{}}{\sundata{04:50}{21:16}{08:07}}{\textsf{\kasht} {\tiny \RIGHTarrow} 04:19(+1)}{\textsf{\uttaraproshthapada} {\tiny \RIGHTarrow} 11:53\hspace{2ex}}{19:12-21:16}{13:03-15:06} 
&
\caldata{5}{\sunmonth{\mithuna}{21}{}}{\sundata{04:51}{21:16}{08:08}}{\textsf{\knav} {\tiny \RIGHTarrow} \ahoratram}{\textsf{\revati} {\tiny \RIGHTarrow} 13:54\hspace{2ex}}{06:54-08:57}{11:00-13:03} 
&
\caldata{6}{\sunmonth{\mithuna}{22}{}}{\sundata{04:52}{21:15}{08:08}}{\textsf{\knav} {\tiny \RIGHTarrow} 05:18\hspace{2ex}}{\textsf{\ashwini} {\tiny \RIGHTarrow} 15:17\hspace{2ex}}{17:09-19:12}{08:57-11:00} 
&
\caldata{7}{\sunmonth{\mithuna}{23}{}}{\sundata{04:53}{21:15}{08:09}}{\textsf{\kdas} {\tiny \RIGHTarrow} 05:35\hspace{2ex}}{\textsf{\apabharani} {\tiny \RIGHTarrow} 15:56\hspace{2ex}}{13:04-15:06}{06:55-08:58} 
&
\caldata{8}{\sunmonth{\mithuna}{24}{}}{\sundata{04:54}{21:14}{08:10}}{\textsf{\keka} {\tiny \RIGHTarrow} 05:07\hspace{2ex}}{\textsf{\krittika} {\tiny \RIGHTarrow} 15:51\hspace{2ex}}{15:06-17:09}{04:54-06:56} 
&
\caldata{9}{\sunmonth{\mithuna}{25}{}}{\sundata{04:55}{21:13}{08:10}}{\textsf{\ktra} {\tiny \RIGHTarrow} 01:58(+1)}{\textsf{\rohini} {\tiny \RIGHTarrow} 15:04\hspace{2ex}}{11:01-13:04}{17:08-19:10} 
&
\caldata{10}{\sunmonth{\mithuna}{26}{}}{\sundata{04:56}{21:13}{08:11}}{\textsf{\kchaturdashi} {\tiny \RIGHTarrow} 23:30\hspace{2ex}}{\textsf{\mrigashirsha} {\tiny \RIGHTarrow} 13:41\hspace{2ex}}{09:00-11:02}{15:06-17:08} 
\\ \hline
\caldata{11}{\sunmonth{\mithuna}{27}{}}{\sundata{04:57}{21:12}{08:12}}{\textsf{\ama} {\tiny \RIGHTarrow} 20:37\hspace{2ex}}{\textsf{\ardra} {\tiny \RIGHTarrow} 11:48\hspace{2ex}}{19:10-21:12}{13:04-15:06} 
&
\caldata{12}{\sunmonth{\mithuna}{28}{}}{\sundata{04:58}{21:11}{08:12}}{\textsf{\spra} {\tiny \RIGHTarrow} 17:27\hspace{2ex}}{\textsf{\punarvasu} {\tiny \RIGHTarrow} 09:33\hspace{2ex}}{06:59-09:01}{11:02-13:04} 
&
\caldata{13}{\sunmonth{\mithuna}{29}{}}{\sundata{04:59}{21:10}{08:13}}{\textsf{\sdvi} {\tiny \RIGHTarrow} 14:08\hspace{2ex}}{\textsf{\pushya} {\tiny \RIGHTarrow} 07:06\hspace{2ex}}{17:07-19:08}{09:01-11:03} 
&
\caldata{14}{\sunmonth{\mithuna}{30}{}}{\sundata{05:00}{21:09}{08:13}}{\textsf{\stri} {\tiny \RIGHTarrow} 10:49\hspace{2ex}}{\textsf{\magha} {\tiny \RIGHTarrow} 02:08(+1)}{13:04-15:05}{07:01-09:02} 
&
\caldata{15}{\sunmonth{\mithuna}{31}{}}{\sundata{05:01}{21:08}{08:14}}{\textsf{\scha} {\tiny \RIGHTarrow} 07:36\hspace{2ex}}{\textsf{\purvaphalguni} {\tiny \RIGHTarrow} 23:58\hspace{2ex}}{15:05-17:06}{05:01-07:01} 
&
\caldata{16}{\sunmonth{\karkataka}{1}{{\textsf{\mithuna} {\tiny \RIGHTarrow} (16:33\hspace{2ex})}}}{\sundata{05:02}{21:07}{08:15}}{\textsf{\ssha} {\tiny \RIGHTarrow} 02:07(+1)}{\textsf{\uttaraphalguni} {\tiny \RIGHTarrow} 22:09\hspace{2ex}}{11:03-13:04}{17:05-19:06} 
&
\caldata{17}{\sunmonth{\karkataka}{2}{}}{\sundata{05:03}{21:06}{08:15}}{\textsf{\ssap} {\tiny \RIGHTarrow} 00:04(+1)}{\textsf{\hasta} {\tiny \RIGHTarrow} 20:47\hspace{2ex}}{09:03-11:04}{15:04-17:05} 
\\ \hline
\caldata{18}{\sunmonth{\karkataka}{3}{}}{\sundata{05:05}{21:05}{08:17}}{\textsf{\sasht} {\tiny \RIGHTarrow} 22:32\hspace{2ex}}{\textsf{\chitra} {\tiny \RIGHTarrow} 19:56\hspace{2ex}}{19:05-21:05}{13:05-15:05} 
&
\caldata{19}{\sunmonth{\karkataka}{4}{}}{\sundata{05:06}{21:04}{08:17}}{\textsf{\snav} {\tiny \RIGHTarrow} 21:34\hspace{2ex}}{\textsf{\svati} {\tiny \RIGHTarrow} 19:39\hspace{2ex}}{07:05-09:05}{11:05-13:05} 
&
\caldata{20}{\sunmonth{\karkataka}{5}{}}{\sundata{05:07}{21:03}{08:18}}{\textsf{\sdas} {\tiny \RIGHTarrow} 21:10\hspace{2ex}}{\textsf{\vishakha} {\tiny \RIGHTarrow} 19:57\hspace{2ex}}{17:04-19:03}{09:06-11:05} 
&
\caldata{21}{\sunmonth{\karkataka}{6}{}}{\sundata{05:09}{21:02}{08:19}}{\textsf{\seka} {\tiny \RIGHTarrow} 21:20\hspace{2ex}}{\textsf{\anuradha} {\tiny \RIGHTarrow} 20:47\hspace{2ex}}{13:05-15:04}{07:08-09:07} 
&
\caldata{22}{\sunmonth{\karkataka}{7}{}}{\sundata{05:10}{21:00}{08:20}}{\textsf{\sdva} {\tiny \RIGHTarrow} 22:01\hspace{2ex}}{\textsf{\jyeshtha} {\tiny \RIGHTarrow} 22:08\hspace{2ex}}{15:03-17:02}{05:10-07:08} 
&
\caldata{23}{\sunmonth{\karkataka}{8}{}}{\sundata{05:11}{20:59}{08:20}}{\textsf{\stra} {\tiny \RIGHTarrow} 23:10\hspace{2ex}}{\textsf{\mula} {\tiny \RIGHTarrow} 23:55\hspace{2ex}}{11:06-13:05}{17:02-19:00} 
&
\caldata{24}{\sunmonth{\karkataka}{9}{}}{\sundata{05:13}{20:58}{08:22}}{\textsf{\schaturdashi} {\tiny \RIGHTarrow} 00:43(+1)}{\textsf{\purvashadha} {\tiny \RIGHTarrow} 02:07(+1)}{09:09-11:07}{15:03-17:01} 
\\ \hline
\caldata{25}{\sunmonth{\karkataka}{10}{}}{\sundata{05:14}{20:56}{08:22}}{\textsf{\purnima} {\tiny \RIGHTarrow} 02:37(+1)}{\textsf{\uttarashadha} {\tiny \RIGHTarrow} 04:37(+1)}{18:58-20:56}{13:05-15:02} 
&
\caldata{26}{\sunmonth{\karkataka}{11}{}}{\sundata{05:15}{20:55}{08:23}}{\textsf{\kpra} {\tiny \RIGHTarrow} 04:47(+1)}{\textsf{\shravana} {\tiny \RIGHTarrow} \ahoratram}{07:12-09:10}{11:07-13:05} 
&
\caldata{27}{\sunmonth{\karkataka}{12}{}}{\sundata{05:17}{20:53}{08:24}}{\textsf{\kdvi} {\tiny \RIGHTarrow} \ahoratram}{\textsf{\shravana} {\tiny \RIGHTarrow} 07:24\hspace{2ex}}{16:59-18:56}{09:11-11:08} 
&
\caldata{28}{\sunmonth{\karkataka}{13}{}}{\sundata{05:18}{20:52}{08:24}}{\textsf{\kdvi} {\tiny \RIGHTarrow} 07:08\hspace{2ex}}{\textsf{\shravishtha} {\tiny \RIGHTarrow} 10:21\hspace{2ex}}{13:05-15:01}{07:14-09:11} 
&
\caldata{29}{\sunmonth{\karkataka}{14}{}}{\sundata{05:20}{20:50}{08:26}}{\textsf{\ktri} {\tiny \RIGHTarrow} 09:34\hspace{2ex}}{\textsf{\shatabhishak} {\tiny \RIGHTarrow} 13:21\hspace{2ex}}{15:01-16:57}{05:20-07:16} 
&
\caldata{30}{\sunmonth{\karkataka}{15}{}}{\sundata{05:21}{20:49}{08:26}}{\textsf{\kcha} {\tiny \RIGHTarrow} 11:58\hspace{2ex}}{\textsf{\proshthapada} {\tiny \RIGHTarrow} 16:18\hspace{2ex}}{11:09-13:05}{16:57-18:53} 
&
\caldata{31}{\sunmonth{\karkataka}{16}{}}{\sundata{05:23}{20:47}{08:27}}{\textsf{\kpanc} {\tiny \RIGHTarrow} 14:11\hspace{2ex}}{\textsf{\uttaraproshthapada} {\tiny \RIGHTarrow} 19:03\hspace{2ex}}{09:14-11:09}{15:00-16:56} 
\\ \hline
\end{tabular}


%\clearpage
\begin{tabular}{|c|c|c|c|c|c|c|}
\multicolumn{7}{c}{\Large \bfseries AUGUST 2010}\\
\hline
\textbf{SUN} & \textbf{MON} & \textbf{TUE} & \textbf{WED} & \textbf{THU} & \textbf{FRI} & \textbf{SAT} \\ \hline
\caldata{1}{\sunmonth{\karkataka}{17}{}}{\sundata{05:24}{20:46}{08:28}}{\textsf{\ksha} {\tiny \RIGHTarrow} 16:03\hspace{2ex}}{\textsf{\revati} {\tiny \RIGHTarrow} 21:27\hspace{2ex}}{18:50-20:46}{13:05-15:00} 
&
\caldata{2}{\sunmonth{\karkataka}{18}{}}{\sundata{05:26}{20:44}{08:29}}{\textsf{\ksap} {\tiny \RIGHTarrow} 17:26\hspace{2ex}}{\textsf{\ashwini} {\tiny \RIGHTarrow} 23:20\hspace{2ex}}{07:20-09:15}{11:10-13:05} 
&
\caldata{3}{\sunmonth{\karkataka}{19}{}}{\sundata{05:27}{20:42}{08:30}}{\textsf{\kasht} {\tiny \RIGHTarrow} 18:11\hspace{2ex}}{\textsf{\apabharani} {\tiny \RIGHTarrow} 00:36(+1)}{16:53-18:47}{09:15-11:10} 
&
\caldata{4}{\sunmonth{\karkataka}{20}{}}{\sundata{05:29}{20:41}{08:31}}{\textsf{\knav} {\tiny \RIGHTarrow} 18:12\hspace{2ex}}{\textsf{\krittika} {\tiny \RIGHTarrow} 01:08(+1)}{13:05-14:59}{07:23-09:17} 
&
\caldata{5}{\sunmonth{\karkataka}{21}{}}{\sundata{05:30}{20:39}{08:31}}{\textsf{\kdas} {\tiny \RIGHTarrow} 17:28\hspace{2ex}}{\textsf{\rohini} {\tiny \RIGHTarrow} 00:53(+1)}{14:58-16:51}{05:30-07:23} 
&
\caldata{6}{\sunmonth{\karkataka}{22}{}}{\sundata{05:32}{20:37}{08:33}}{\textsf{\keka} {\tiny \RIGHTarrow} 15:58\hspace{2ex}}{\textsf{\mrigashirsha} {\tiny \RIGHTarrow} 23:54\hspace{2ex}}{11:11-13:04}{16:50-18:43} 
&
\caldata{7}{\sunmonth{\karkataka}{23}{}}{\sundata{05:33}{20:35}{08:33}}{\textsf{\kdva} {\tiny \RIGHTarrow} 13:47\hspace{2ex}}{\textsf{\ardra} {\tiny \RIGHTarrow} 22:14\hspace{2ex}}{09:18-11:11}{14:56-16:49} 
\\ \hline
\caldata{8}{\sunmonth{\karkataka}{24}{}}{\sundata{05:35}{20:34}{08:34}}{\textsf{\ktra} {\tiny \RIGHTarrow} 11:00\hspace{2ex}}{\textsf{\punarvasu} {\tiny \RIGHTarrow} 20:01\hspace{2ex}}{18:41-20:34}{13:04-14:56} 
&
\caldata{9}{\sunmonth{\karkataka}{25}{}}{\sundata{05:36}{20:32}{08:35}}{\textsf{\kchaturdashi} {\tiny \RIGHTarrow} 07:44\hspace{2ex}}{\textsf{\pushya} {\tiny \RIGHTarrow} 17:24\hspace{2ex}}{07:28-09:20}{11:12-13:04} 
&
\caldata{10}{\sunmonth{\karkataka}{26}{}}{\sundata{05:38}{20:30}{08:36}}{\textsf{\spra} {\tiny \RIGHTarrow} 00:21(+1)}{\textsf{\ashresha} {\tiny \RIGHTarrow} 14:32\hspace{2ex}}{16:47-18:38}{09:21-11:12} 
&
\caldata{11}{\sunmonth{\karkataka}{27}{}}{\sundata{05:39}{20:28}{08:36}}{\textsf{\sdvi} {\tiny \RIGHTarrow} 20:37\hspace{2ex}}{\textsf{\magha} {\tiny \RIGHTarrow} 11:36\hspace{2ex}}{13:03-14:54}{07:30-09:21} 
&
\caldata{12}{\sunmonth{\karkataka}{28}{}}{\sundata{05:41}{20:26}{08:38}}{\textsf{\stri} {\tiny \RIGHTarrow} 17:05\hspace{2ex}}{\textsf{\purvaphalguni} {\tiny \RIGHTarrow} 08:46\hspace{2ex}}{14:54-16:44}{05:41-07:31} 
&
\caldata{13}{\sunmonth{\karkataka}{29}{}}{\sundata{05:43}{20:24}{08:39}}{\textsf{\scha} {\tiny \RIGHTarrow} 13:52\hspace{2ex}}{\textsf{\uttaraphalguni} {\tiny \RIGHTarrow} 06:11\hspace{2ex}}{11:13-13:03}{16:43-18:33} 
&
\caldata{14}{\sunmonth{\karkataka}{30}{}}{\sundata{05:44}{20:22}{08:39}}{\textsf{\spanc} {\tiny \RIGHTarrow} 11:09\hspace{2ex}}{\textsf{\chitra} {\tiny \RIGHTarrow} 02:37(+1)}{09:23-11:13}{14:52-16:42} 
\\ \hline
\caldata{15}{\sunmonth{\karkataka}{31}{}}{\sundata{05:46}{20:20}{08:40}}{\textsf{\ssha} {\tiny \RIGHTarrow} 09:03\hspace{2ex}}{\textsf{\svati} {\tiny \RIGHTarrow} 01:49(+1)}{18:30-20:20}{13:03-14:52} 
&
\caldata{16}{\sunmonth{\karkataka}{32}{{\textsf{\karkataka} {\tiny \RIGHTarrow} (00:56(+1))}}}{\sundata{05:47}{20:18}{08:41}}{\textsf{\ssap} {\tiny \RIGHTarrow} 07:40\hspace{2ex}}{\textsf{\vishakha} {\tiny \RIGHTarrow} 01:45(+1)}{07:35-09:24}{11:13-13:02} 
&
\caldata{17}{\sunmonth{\simha}{1}{}}{\sundata{05:49}{20:16}{08:42}}{\textsf{\sasht} {\tiny \RIGHTarrow} 07:02\hspace{2ex}}{\textsf{\anuradha} {\tiny \RIGHTarrow} 02:23(+1)}{16:39-18:27}{09:25-11:14} 
&
\caldata{18}{\sunmonth{\simha}{2}{}}{\sundata{05:51}{20:14}{08:43}}{\textsf{\snav} {\tiny \RIGHTarrow} 07:09\hspace{2ex}}{\textsf{\jyeshtha} {\tiny \RIGHTarrow} 03:43(+1)}{13:02-14:50}{07:38-09:26} 
&
\caldata{19}{\sunmonth{\simha}{3}{}}{\sundata{05:52}{20:12}{08:44}}{\textsf{\sdas} {\tiny \RIGHTarrow} 07:57\hspace{2ex}}{\textsf{\mula} {\tiny \RIGHTarrow} 05:36(+1)}{14:49-16:37}{05:52-07:39} 
&
\caldata{20}{\sunmonth{\simha}{4}{}}{\sundata{05:54}{20:10}{08:45}}{\textsf{\seka} {\tiny \RIGHTarrow} 09:20\hspace{2ex}}{\textsf{\purvashadha} {\tiny \RIGHTarrow} \ahoratram}{11:15-13:02}{16:36-18:23} 
&
\caldata{21}{\sunmonth{\simha}{5}{}}{\sundata{05:55}{20:08}{08:45}}{\textsf{\sdva} {\tiny \RIGHTarrow} 11:10\hspace{2ex}}{\textsf{\purvashadha} {\tiny \RIGHTarrow} 07:59\hspace{2ex}}{09:28-11:14}{14:48-16:34} 
\\ \hline
\caldata{22}{\sunmonth{\simha}{6}{}}{\sundata{05:57}{20:06}{08:46}}{\textsf{\stra} {\tiny \RIGHTarrow} 13:18\hspace{2ex}}{\textsf{\uttarashadha} {\tiny \RIGHTarrow} 10:40\hspace{2ex}}{18:19-20:06}{13:01-14:47} 
&
\caldata{23}{\sunmonth{\simha}{7}{}}{\sundata{05:58}{20:04}{08:47}}{\textsf{\schaturdashi} {\tiny \RIGHTarrow} 15:38\hspace{2ex}}{\textsf{\shravana} {\tiny \RIGHTarrow} 13:33\hspace{2ex}}{07:43-09:29}{11:15-13:01} 
&
\caldata{24}{\sunmonth{\simha}{8}{}}{\sundata{06:00}{20:02}{08:48}}{\textsf{\purnima} {\tiny \RIGHTarrow} 18:04\hspace{2ex}}{\textsf{\shravishtha} {\tiny \RIGHTarrow} 16:32\hspace{2ex}}{16:31-18:16}{09:30-11:15} 
&
\caldata{25}{\sunmonth{\simha}{9}{}}{\sundata{06:02}{20:00}{08:49}}{\textsf{\kpra} {\tiny \RIGHTarrow} 20:30\hspace{2ex}}{\textsf{\shatabhishak} {\tiny \RIGHTarrow} 19:31\hspace{2ex}}{13:01-14:45}{07:46-09:31} 
&
\caldata{26}{\sunmonth{\simha}{10}{}}{\sundata{06:03}{19:57}{08:49}}{\textsf{\kdvi} {\tiny \RIGHTarrow} 22:52\hspace{2ex}}{\textsf{\proshthapada} {\tiny \RIGHTarrow} 22:26\hspace{2ex}}{14:44-16:28}{06:03-07:47} 
&
\caldata{27}{\sunmonth{\simha}{11}{}}{\sundata{06:05}{19:55}{08:51}}{\textsf{\ktri} {\tiny \RIGHTarrow} 01:05(+1)}{\textsf{\uttaraproshthapada} {\tiny \RIGHTarrow} 01:13(+1)}{11:16-13:00}{16:27-18:11} 
&
\caldata{28}{\sunmonth{\simha}{12}{}}{\sundata{06:06}{19:53}{08:51}}{\textsf{\kcha} {\tiny \RIGHTarrow} 03:03(+1)}{\textsf{\revati} {\tiny \RIGHTarrow} 03:45(+1)}{09:32-11:16}{14:42-16:26} 
\\ \hline
\caldata{29}{\sunmonth{\simha}{13}{}}{\sundata{06:08}{19:51}{08:52}}{\textsf{\kpanc} {\tiny \RIGHTarrow} 04:39(+1)}{\textsf{\ashwini} {\tiny \RIGHTarrow} 05:56(+1)}{18:08-19:51}{12:59-14:42} 
&
\caldata{30}{\sunmonth{\simha}{14}{}}{\sundata{06:10}{19:49}{08:53}}{\textsf{\ksha} {\tiny \RIGHTarrow} 05:47(+1)}{\textsf{\apabharani} {\tiny \RIGHTarrow} \ahoratram}{07:52-09:34}{11:17-12:59} 
&
\caldata{31}{\sunmonth{\simha}{15}{}}{\sundata{06:11}{19:47}{08:54}}{\textsf{\ksap} {\tiny \RIGHTarrow} \ahoratram}{\textsf{\apabharani} {\tiny \RIGHTarrow} 07:39\hspace{2ex}}{16:23-18:05}{09:35-11:17} 
&
{}  &
{}  &
{}  &
\\ \hline
\end{tabular}


%\clearpage
\begin{tabular}{|c|c|c|c|c|c|c|}
\multicolumn{7}{c}{\Large \bfseries SEPTEMBER 2010}\\
\hline
\textbf{SUN} & \textbf{MON} & \textbf{TUE} & \textbf{WED} & \textbf{THU} & \textbf{FRI} & \textbf{SAT} \\ \hline
{}  &
{}  &
{}  &
\caldata{1}{\sunmonth{\simha}{16}{}}{\sundata{06:13}{19:44}{08:55}}{\textsf{\ksap} {\tiny \RIGHTarrow} 06:20\hspace{2ex}}{\textsf{\krittika} {\tiny \RIGHTarrow} 08:46\hspace{2ex}}{12:58-14:39}{07:54-09:35} 
&
\caldata{2}{\sunmonth{\simha}{17}{}}{\sundata{06:14}{19:42}{08:55}}{\textsf{\knav} {\tiny \RIGHTarrow} 05:19(+1)}{\textsf{\rohini} {\tiny \RIGHTarrow} 09:14\hspace{2ex}}{14:39-16:20}{06:14-07:55} 
&
\caldata{3}{\sunmonth{\simha}{18}{}}{\sundata{06:16}{19:40}{08:56}}{\textsf{\kdas} {\tiny \RIGHTarrow} 03:41(+1)}{\textsf{\mrigashirsha} {\tiny \RIGHTarrow} 08:58\hspace{2ex}}{11:17-12:58}{16:19-17:59} 
&
\caldata{4}{\sunmonth{\simha}{19}{}}{\sundata{06:18}{19:38}{08:58}}{\textsf{\keka} {\tiny \RIGHTarrow} 01:22(+1)}{\textsf{\ardra} {\tiny \RIGHTarrow} 08:00\hspace{2ex}}{09:38-11:18}{14:38-16:18} 
\\ \hline
\caldata{5}{\sunmonth{\simha}{20}{}}{\sundata{06:19}{19:35}{08:58}}{\textsf{\kdva} {\tiny \RIGHTarrow} 22:26\hspace{2ex}}{\textsf{\punarvasu} {\tiny \RIGHTarrow} 06:22\hspace{2ex}}{17:55-19:35}{12:57-14:36} 
&
\caldata{6}{\sunmonth{\simha}{21}{}}{\sundata{06:21}{19:33}{08:59}}{\textsf{\ktra} {\tiny \RIGHTarrow} 19:03\hspace{2ex}}{\textsf{\ashresha} {\tiny \RIGHTarrow} 01:22(+1)}{08:00-09:39}{11:18-12:57} 
&
\caldata{7}{\sunmonth{\simha}{22}{}}{\sundata{06:22}{19:31}{08:59}}{\textsf{\kchaturdashi} {\tiny \RIGHTarrow} 15:21\hspace{2ex}}{\textsf{\magha} {\tiny \RIGHTarrow} 22:24\hspace{2ex}}{16:13-17:52}{09:39-11:17} 
&
\caldata{8}{\sunmonth{\simha}{23}{}}{\sundata{06:24}{19:29}{09:01}}{\textsf{\ama} {\tiny \RIGHTarrow} 11:30\hspace{2ex}}{\textsf{\purvaphalguni} {\tiny \RIGHTarrow} 19:20\hspace{2ex}}{12:56-14:34}{08:02-09:40} 
&
\caldata{9}{\sunmonth{\simha}{24}{}}{\sundata{06:25}{19:26}{09:01}}{\textsf{\spra} {\tiny \RIGHTarrow} 07:38\hspace{2ex}}{\textsf{\uttaraphalguni} {\tiny \RIGHTarrow} 16:23\hspace{2ex}}{14:33-16:10}{06:25-08:02} 
&
\caldata{10}{\sunmonth{\simha}{25}{}}{\sundata{06:27}{19:24}{09:02}}{\textsf{\stri} {\tiny \RIGHTarrow} 00:48(+1)}{\textsf{\hasta} {\tiny \RIGHTarrow} 13:43\hspace{2ex}}{11:18-12:55}{16:09-17:46} 
&
\caldata{11}{\sunmonth{\simha}{26}{}}{\sundata{06:29}{19:22}{09:03}}{\textsf{\scha} {\tiny \RIGHTarrow} 22:09\hspace{2ex}}{\textsf{\chitra} {\tiny \RIGHTarrow} 11:31\hspace{2ex}}{09:42-11:18}{14:32-16:08} 
\\ \hline
\caldata{12}{\sunmonth{\simha}{27}{}}{\sundata{06:30}{19:19}{09:03}}{\textsf{\spanc} {\tiny \RIGHTarrow} 20:13\hspace{2ex}}{\textsf{\svati} {\tiny \RIGHTarrow} 09:57\hspace{2ex}}{17:42-19:19}{12:54-14:30} 
&
\caldata{13}{\sunmonth{\simha}{28}{}}{\sundata{06:32}{19:17}{09:05}}{\textsf{\ssha} {\tiny \RIGHTarrow} 19:04\hspace{2ex}}{\textsf{\vishakha} {\tiny \RIGHTarrow} 09:10\hspace{2ex}}{08:07-09:43}{11:18-12:54} 
&
\caldata{14}{\sunmonth{\simha}{29}{}}{\sundata{06:33}{19:15}{09:05}}{\textsf{\ssap} {\tiny \RIGHTarrow} 18:45\hspace{2ex}}{\textsf{\anuradha} {\tiny \RIGHTarrow} 09:13\hspace{2ex}}{16:04-17:39}{09:43-11:18} 
&
\caldata{15}{\sunmonth{\simha}{30}{}}{\sundata{06:35}{19:13}{09:06}}{\textsf{\sasht} {\tiny \RIGHTarrow} 19:16\hspace{2ex}}{\textsf{\jyeshtha} {\tiny \RIGHTarrow} 10:06\hspace{2ex}}{12:54-14:28}{08:09-09:44} 
&
\caldata{16}{\sunmonth{\simha}{31}{{\textsf{\simha} {\tiny \RIGHTarrow} (00:52(+1))}}}{\sundata{06:37}{19:10}{09:07}}{\textsf{\snav} {\tiny \RIGHTarrow} 20:31\hspace{2ex}}{\textsf{\mula} {\tiny \RIGHTarrow} 11:44\hspace{2ex}}{14:27-16:01}{06:37-08:11} 
&
\caldata{17}{\sunmonth{\kanya}{1}{}}{\sundata{06:38}{19:08}{09:08}}{\textsf{\sdas} {\tiny \RIGHTarrow} 22:21\hspace{2ex}}{\textsf{\purvashadha} {\tiny \RIGHTarrow} 13:58\hspace{2ex}}{11:19-12:53}{16:00-17:34} 
&
\caldata{18}{\sunmonth{\kanya}{2}{}}{\sundata{06:40}{19:06}{09:09}}{\textsf{\seka} {\tiny \RIGHTarrow} 00:35(+1)}{\textsf{\uttarashadha} {\tiny \RIGHTarrow} 16:39\hspace{2ex}}{09:46-11:19}{14:26-15:59} 
\\ \hline
\caldata{19}{\sunmonth{\kanya}{3}{}}{\sundata{06:41}{19:03}{09:09}}{\textsf{\sdva} {\tiny \RIGHTarrow} 03:02(+1)}{\textsf{\shravana} {\tiny \RIGHTarrow} 19:35\hspace{2ex}}{17:30-19:03}{12:52-14:24} 
&
\caldata{20}{\sunmonth{\kanya}{4}{}}{\sundata{06:43}{19:01}{09:10}}{\textsf{\stra} {\tiny \RIGHTarrow} 05:33(+1)}{\textsf{\shravishtha} {\tiny \RIGHTarrow} 22:36\hspace{2ex}}{08:15-09:47}{11:19-12:52} 
&
\caldata{21}{\sunmonth{\kanya}{5}{}}{\sundata{06:45}{18:59}{09:11}}{\textsf{\schaturdashi} {\tiny \RIGHTarrow} \ahoratram}{\textsf{\shatabhishak} {\tiny \RIGHTarrow} 01:35(+1)}{15:55-17:27}{09:48-11:20} 
&
\caldata{22}{\sunmonth{\kanya}{6}{}}{\sundata{06:46}{18:56}{09:12}}{\textsf{\schaturdashi} {\tiny \RIGHTarrow} 08:00\hspace{2ex}}{\textsf{\proshthapada} {\tiny \RIGHTarrow} 04:26(+1)}{12:51-14:22}{08:17-09:48} 
&
\caldata{23}{\sunmonth{\kanya}{7}{}}{\sundata{06:48}{18:54}{09:13}}{\textsf{\purnima} {\tiny \RIGHTarrow} 10:16\hspace{2ex}}{\textsf{\uttaraproshthapada} {\tiny \RIGHTarrow} \ahoratram}{14:21-15:52}{06:48-08:18} 
&
\caldata{24}{\sunmonth{\kanya}{8}{}}{\sundata{06:49}{18:52}{09:13}}{\textsf{\kpra} {\tiny \RIGHTarrow} 12:19\hspace{2ex}}{\textsf{\uttaraproshthapada} {\tiny \RIGHTarrow} 07:05\hspace{2ex}}{11:20-12:50}{15:51-17:21} 
&
\caldata{25}{\sunmonth{\kanya}{9}{}}{\sundata{06:51}{18:49}{09:14}}{\textsf{\kdvi} {\tiny \RIGHTarrow} 14:06\hspace{2ex}}{\textsf{\revati} {\tiny \RIGHTarrow} 09:29\hspace{2ex}}{09:50-11:20}{14:19-15:49} 
\\ \hline
\caldata{26}{\sunmonth{\kanya}{10}{}}{\sundata{06:53}{18:47}{09:15}}{\textsf{\ktri} {\tiny \RIGHTarrow} 15:35\hspace{2ex}}{\textsf{\ashwini} {\tiny \RIGHTarrow} 11:35\hspace{2ex}}{17:17-18:47}{12:50-14:19} 
&
\caldata{27}{\sunmonth{\kanya}{11}{}}{\sundata{06:54}{18:45}{09:16}}{\textsf{\kcha} {\tiny \RIGHTarrow} 16:42\hspace{2ex}}{\textsf{\apabharani} {\tiny \RIGHTarrow} 13:20\hspace{2ex}}{08:22-09:51}{11:20-12:49} 
&
\caldata{28}{\sunmonth{\kanya}{12}{}}{\sundata{06:56}{18:43}{09:17}}{\textsf{\kpanc} {\tiny \RIGHTarrow} 17:23\hspace{2ex}}{\textsf{\krittika} {\tiny \RIGHTarrow} 14:42\hspace{2ex}}{15:46-17:14}{09:52-11:21} 
&
\caldata{29}{\sunmonth{\kanya}{13}{}}{\sundata{06:58}{18:40}{09:18}}{\textsf{\ksha} {\tiny \RIGHTarrow} 17:34\hspace{2ex}}{\textsf{\rohini} {\tiny \RIGHTarrow} 15:34\hspace{2ex}}{12:49-14:16}{08:25-09:53} 
&
\caldata{30}{\sunmonth{\kanya}{14}{}}{\sundata{06:59}{18:38}{09:18}}{\textsf{\ksap} {\tiny \RIGHTarrow} 17:12\hspace{2ex}}{\textsf{\mrigashirsha} {\tiny \RIGHTarrow} 15:54\hspace{2ex}}{14:15-15:43}{06:59-08:26} 
&
{}  &
\\ \hline
\end{tabular}


%\clearpage
\begin{tabular}{|c|c|c|c|c|c|c|}
\multicolumn{7}{c}{\Large \bfseries OCTOBER 2010}\\
\hline
\textbf{SUN} & \textbf{MON} & \textbf{TUE} & \textbf{WED} & \textbf{THU} & \textbf{FRI} & \textbf{SAT} \\ \hline
{}  &
{}  &
{}  &
{}  &
{}  &
\caldata{1}{\sunmonth{\kanya}{15}{}}{\sundata{07:01}{18:36}{09:20}}{\textsf{\kasht} {\tiny \RIGHTarrow} 16:12\hspace{2ex}}{\textsf{\ardra} {\tiny \RIGHTarrow} 15:37\hspace{2ex}}{11:21-12:48}{15:42-17:09} 
&
\caldata{2}{\sunmonth{\kanya}{16}{}}{\sundata{07:02}{18:33}{09:20}}{\textsf{\knav} {\tiny \RIGHTarrow} 14:35\hspace{2ex}}{\textsf{\punarvasu} {\tiny \RIGHTarrow} 14:44\hspace{2ex}}{09:54-11:21}{14:13-15:40} 
\\ \hline
\caldata{3}{\sunmonth{\kanya}{17}{}}{\sundata{07:04}{18:31}{09:21}}{\textsf{\kdas} {\tiny \RIGHTarrow} 12:23\hspace{2ex}}{\textsf{\pushya} {\tiny \RIGHTarrow} 13:15\hspace{2ex}}{17:05-18:31}{12:47-14:13} 
&
\caldata{4}{\sunmonth{\kanya}{18}{}}{\sundata{07:06}{18:29}{09:22}}{\textsf{\keka} {\tiny \RIGHTarrow} 09:38\hspace{2ex}}{\textsf{\ashresha} {\tiny \RIGHTarrow} 11:14\hspace{2ex}}{08:31-09:56}{11:22-12:47} 
&
\caldata{5}{\sunmonth{\kanya}{19}{}}{\sundata{07:07}{18:27}{09:23}}{\textsf{\ktra} {\tiny \RIGHTarrow} 02:56(+1)}{\textsf{\magha} {\tiny \RIGHTarrow} 08:47\hspace{2ex}}{15:37-17:02}{09:57-11:22} 
&
\caldata{6}{\sunmonth{\kanya}{20}{}}{\sundata{07:09}{18:24}{09:24}}{\textsf{\kchaturdashi} {\tiny \RIGHTarrow} 23:19\hspace{2ex}}{\textsf{\uttaraphalguni} {\tiny \RIGHTarrow} 03:13(+1)}{12:46-14:10}{08:33-09:57} 
&
\caldata{7}{\sunmonth{\kanya}{21}{}}{\sundata{07:11}{18:22}{09:25}}{\textsf{\ama} {\tiny \RIGHTarrow} 19:46\hspace{2ex}}{\textsf{\hasta} {\tiny \RIGHTarrow} 00:28(+1)}{14:10-15:34}{07:11-08:34} 
&
\caldata{8}{\sunmonth{\kanya}{22}{}}{\sundata{07:12}{18:20}{09:25}}{\textsf{\spra} {\tiny \RIGHTarrow} 16:27\hspace{2ex}}{\textsf{\chitra} {\tiny \RIGHTarrow} 22:01\hspace{2ex}}{11:22-12:46}{15:33-16:56} 
&
\caldata{9}{\sunmonth{\kanya}{23}{}}{\sundata{07:14}{18:18}{09:26}}{\textsf{\sdvi} {\tiny \RIGHTarrow} 13:31\hspace{2ex}}{\textsf{\svati} {\tiny \RIGHTarrow} 20:01\hspace{2ex}}{10:00-11:23}{14:09-15:32} 
\\ \hline
\caldata{10}{\sunmonth{\kanya}{24}{}}{\sundata{07:16}{18:16}{09:28}}{\textsf{\stri} {\tiny \RIGHTarrow} 11:10\hspace{2ex}}{\textsf{\vishakha} {\tiny \RIGHTarrow} 18:39\hspace{2ex}}{16:53-18:16}{12:46-14:08} 
&
\caldata{11}{\sunmonth{\kanya}{25}{}}{\sundata{07:17}{18:13}{09:28}}{\textsf{\scha} {\tiny \RIGHTarrow} 09:32\hspace{2ex}}{\textsf{\anuradha} {\tiny \RIGHTarrow} 18:01\hspace{2ex}}{08:39-10:01}{11:23-12:45} 
&
\caldata{12}{\sunmonth{\kanya}{26}{}}{\sundata{07:19}{18:11}{09:29}}{\textsf{\spanc} {\tiny \RIGHTarrow} 08:44\hspace{2ex}}{\textsf{\jyeshtha} {\tiny \RIGHTarrow} 18:13\hspace{2ex}}{15:28-16:49}{10:02-11:23} 
&
\caldata{13}{\sunmonth{\kanya}{27}{}}{\sundata{07:21}{18:09}{09:30}}{\textsf{\ssha} {\tiny \RIGHTarrow} 08:49\hspace{2ex}}{\textsf{\mula} {\tiny \RIGHTarrow} 19:15\hspace{2ex}}{12:45-14:06}{08:42-10:03} 
&
\caldata{14}{\sunmonth{\kanya}{28}{}}{\sundata{07:22}{18:07}{09:31}}{\textsf{\ssap} {\tiny \RIGHTarrow} 09:44\hspace{2ex}}{\textsf{\purvashadha} {\tiny \RIGHTarrow} 21:02\hspace{2ex}}{14:05-15:25}{07:22-08:42} 
&
\caldata{15}{\sunmonth{\kanya}{29}{}}{\sundata{07:24}{18:05}{09:32}}{\textsf{\sasht} {\tiny \RIGHTarrow} 11:22\hspace{2ex}}{\textsf{\uttarashadha} {\tiny \RIGHTarrow} 23:25\hspace{2ex}}{11:24-12:44}{15:24-16:44} 
&
\caldata{16}{\sunmonth{\kanya}{30}{}}{\sundata{07:26}{18:03}{09:33}}{\textsf{\snav} {\tiny \RIGHTarrow} 13:31\hspace{2ex}}{\textsf{\shravana} {\tiny \RIGHTarrow} 02:13(+1)}{10:05-11:24}{14:04-15:23} 
\\ \hline
\caldata{17}{\sunmonth{\tula}{1}{{\textsf{\kanya} {\tiny \RIGHTarrow} (12:50\hspace{2ex})}}}{\sundata{07:28}{18:00}{09:34}}{\textsf{\sdas} {\tiny \RIGHTarrow} 15:58\hspace{2ex}}{\textsf{\shravishtha} {\tiny \RIGHTarrow} 05:13(+1)}{16:41-18:00}{12:44-14:03} 
&
\caldata{18}{\sunmonth{\tula}{2}{}}{\sundata{07:29}{17:58}{09:34}}{\textsf{\seka} {\tiny \RIGHTarrow} 18:31\hspace{2ex}}{\textsf{\shatabhishak} {\tiny \RIGHTarrow} \ahoratram}{08:47-10:06}{11:24-12:43} 
&
\caldata{19}{\sunmonth{\tula}{3}{}}{\sundata{07:31}{17:56}{09:36}}{\textsf{\sdva} {\tiny \RIGHTarrow} 20:57\hspace{2ex}}{\textsf{\shatabhishak} {\tiny \RIGHTarrow} 08:13\hspace{2ex}}{15:19-16:37}{10:07-11:25} 
&
\caldata{20}{\sunmonth{\tula}{4}{}}{\sundata{07:33}{17:54}{09:37}}{\textsf{\stra} {\tiny \RIGHTarrow} 23:09\hspace{2ex}}{\textsf{\proshthapada} {\tiny \RIGHTarrow} 11:02\hspace{2ex}}{12:43-14:01}{08:50-10:08} 
&
\caldata{21}{\sunmonth{\tula}{5}{}}{\sundata{07:34}{17:52}{09:37}}{\textsf{\schaturdashi} {\tiny \RIGHTarrow} 01:03(+1)}{\textsf{\uttaraproshthapada} {\tiny \RIGHTarrow} 13:35\hspace{2ex}}{14:00-15:17}{07:34-08:51} 
&
\caldata{22}{\sunmonth{\tula}{6}{}}{\sundata{07:36}{17:50}{09:38}}{\textsf{\purnima} {\tiny \RIGHTarrow} 02:34(+1)}{\textsf{\revati} {\tiny \RIGHTarrow} 15:47\hspace{2ex}}{11:26-12:43}{15:16-16:33} 
&
\caldata{23}{\sunmonth{\tula}{7}{}}{\sundata{07:38}{17:48}{09:40}}{\textsf{\kpra} {\tiny \RIGHTarrow} 03:43(+1)}{\textsf{\ashwini} {\tiny \RIGHTarrow} 17:38\hspace{2ex}}{10:10-11:26}{13:59-15:15} 
\\ \hline
\caldata{24}{\sunmonth{\tula}{8}{}}{\sundata{07:40}{17:46}{09:41}}{\textsf{\kdvi} {\tiny \RIGHTarrow} 04:30(+1)}{\textsf{\apabharani} {\tiny \RIGHTarrow} 19:08\hspace{2ex}}{16:30-17:46}{12:43-13:58} 
&
\caldata{25}{\sunmonth{\tula}{9}{}}{\sundata{07:41}{17:44}{09:41}}{\textsf{\ktri} {\tiny \RIGHTarrow} 04:54(+1)}{\textsf{\krittika} {\tiny \RIGHTarrow} 20:15\hspace{2ex}}{08:56-10:11}{11:27-12:42} 
&
\caldata{26}{\sunmonth{\tula}{10}{}}{\sundata{07:43}{17:42}{09:42}}{\textsf{\kcha} {\tiny \RIGHTarrow} 04:56(+1)}{\textsf{\rohini} {\tiny \RIGHTarrow} 21:01\hspace{2ex}}{15:12-16:27}{10:12-11:27} 
&
\caldata{27}{\sunmonth{\tula}{11}{}}{\sundata{07:45}{17:40}{09:44}}{\textsf{\kpanc} {\tiny \RIGHTarrow} 04:33(+1)}{\textsf{\mrigashirsha} {\tiny \RIGHTarrow} 21:24\hspace{2ex}}{12:42-13:56}{08:59-10:13} 
&
\caldata{28}{\sunmonth{\tula}{12}{}}{\sundata{07:47}{17:38}{09:45}}{\textsf{\ksha} {\tiny \RIGHTarrow} 03:45(+1)}{\textsf{\ardra} {\tiny \RIGHTarrow} 21:23\hspace{2ex}}{13:56-15:10}{07:47-09:00} 
&
\caldata{29}{\sunmonth{\tula}{13}{}}{\sundata{07:48}{17:36}{09:45}}{\textsf{\ksap} {\tiny \RIGHTarrow} 02:30(+1)}{\textsf{\punarvasu} {\tiny \RIGHTarrow} 20:56\hspace{2ex}}{11:28-12:42}{15:09-16:22} 
&
\caldata{30}{\sunmonth{\tula}{14}{}}{\sundata{07:50}{17:34}{09:46}}{\textsf{\kasht} {\tiny \RIGHTarrow} 00:48(+1)}{\textsf{\pushya} {\tiny \RIGHTarrow} 20:03\hspace{2ex}}{10:16-11:29}{13:55-15:08} 
\\ \hline
\caldata{31}{\sunmonth{\tula}{15}{}}{\sundata{06:52}{16:33}{08:48}}{\textsf{\knav} {\tiny \RIGHTarrow} 21:41\hspace{2ex}}{\textsf{\ashresha} {\tiny \RIGHTarrow} 17:44\hspace{2ex}}{15:20-16:33}{11:42-12:55} 
&
{}  &
{}  &
{}  &
{}  &
{}  &
\\ \hline
\end{tabular}


%\clearpage
\begin{tabular}{|c|c|c|c|c|c|c|}
\multicolumn{7}{c}{\Large \bfseries NOVEMBER 2010}\\
\hline
\textbf{SUN} & \textbf{MON} & \textbf{TUE} & \textbf{WED} & \textbf{THU} & \textbf{FRI} & \textbf{SAT} \\ \hline
{}  &
\caldata{1}{\sunmonth{\tula}{16}{}}{\sundata{06:54}{16:31}{08:49}}{\textsf{\kdas} {\tiny \RIGHTarrow} 19:12\hspace{2ex}}{\textsf{\magha} {\tiny \RIGHTarrow} 16:02\hspace{2ex}}{08:06-09:18}{10:30-11:42} 
&
\caldata{2}{\sunmonth{\tula}{17}{}}{\sundata{06:56}{16:29}{08:50}}{\textsf{\keka} {\tiny \RIGHTarrow} 16:26\hspace{2ex}}{\textsf{\purvaphalguni} {\tiny \RIGHTarrow} 14:02\hspace{2ex}}{14:05-15:17}{09:19-10:30} 
&
\caldata{3}{\sunmonth{\tula}{18}{}}{\sundata{06:57}{16:27}{08:51}}{\textsf{\kdva} {\tiny \RIGHTarrow} 13:29\hspace{2ex}}{\textsf{\uttaraphalguni} {\tiny \RIGHTarrow} 11:50\hspace{2ex}}{11:42-12:53}{08:08-09:19} 
&
\caldata{4}{\sunmonth{\tula}{19}{}}{\sundata{06:59}{16:25}{08:52}}{\textsf{\ktra} {\tiny \RIGHTarrow} 10:28\hspace{2ex}}{\textsf{\hasta} {\tiny \RIGHTarrow} 09:34\hspace{2ex}}{12:52-14:03}{06:59-08:09} 
&
\caldata{5}{\sunmonth{\tula}{20}{}}{\sundata{07:01}{16:24}{08:53}}{\textsf{\kchaturdashi} {\tiny \RIGHTarrow} 07:32\hspace{2ex}}{\textsf{\chitra} {\tiny \RIGHTarrow} 07:22\hspace{2ex}}{10:32-11:42}{14:03-15:13} 
&
\caldata{6}{\sunmonth{\tula}{21}{}}{\sundata{07:03}{16:22}{08:54}}{\textsf{\spra} {\tiny \RIGHTarrow} 02:39(+1)}{\textsf{\vishakha} {\tiny \RIGHTarrow} 03:56(+1)}{09:22-10:32}{12:52-14:02} 
\\ \hline
\caldata{7}{\sunmonth{\tula}{22}{}}{\sundata{07:04}{16:20}{08:55}}{\textsf{\sdvi} {\tiny \RIGHTarrow} 00:59(+1)}{\textsf{\anuradha} {\tiny \RIGHTarrow} 03:00(+1)}{15:10-16:20}{11:42-12:51} 
&
\caldata{8}{\sunmonth{\tula}{23}{}}{\sundata{07:06}{16:19}{08:56}}{\textsf{\stri} {\tiny \RIGHTarrow} 00:00(+1)}{\textsf{\jyeshtha} {\tiny \RIGHTarrow} 02:45(+1)}{08:15-09:24}{10:33-11:42} 
&
\caldata{9}{\sunmonth{\tula}{24}{}}{\sundata{07:08}{16:17}{08:57}}{\textsf{\scha} {\tiny \RIGHTarrow} 23:47\hspace{2ex}}{\textsf{\mula} {\tiny \RIGHTarrow} 03:14(+1)}{13:59-15:08}{09:25-10:33} 
&
\caldata{10}{\sunmonth{\tula}{25}{}}{\sundata{07:10}{16:16}{08:59}}{\textsf{\spanc} {\tiny \RIGHTarrow} 00:20(+1)}{\textsf{\purvashadha} {\tiny \RIGHTarrow} 04:28(+1)}{11:43-12:51}{08:18-09:26} 
&
\caldata{11}{\sunmonth{\tula}{26}{}}{\sundata{07:11}{16:14}{08:59}}{\textsf{\ssha} {\tiny \RIGHTarrow} 01:37(+1)}{\textsf{\uttarashadha} {\tiny \RIGHTarrow} 06:24(+1)}{12:50-13:58}{07:11-08:18} 
&
\caldata{12}{\sunmonth{\tula}{27}{}}{\sundata{07:13}{16:13}{09:01}}{\textsf{\ssap} {\tiny \RIGHTarrow} 03:31(+1)}{\textsf{\shravana} {\tiny \RIGHTarrow} \ahoratram}{10:35-11:43}{13:58-15:05} 
&
\caldata{13}{\sunmonth{\tula}{28}{}}{\sundata{07:15}{16:11}{09:02}}{\textsf{\sasht} {\tiny \RIGHTarrow} 05:51(+1)}{\textsf{\shravana} {\tiny \RIGHTarrow} 08:54\hspace{2ex}}{09:29-10:36}{12:50-13:57} 
\\ \hline
\caldata{14}{\sunmonth{\tula}{29}{}}{\sundata{07:17}{16:10}{09:03}}{\textsf{\snav} {\tiny \RIGHTarrow} \ahoratram}{\textsf{\shravishtha} {\tiny \RIGHTarrow} 11:46\hspace{2ex}}{15:03-16:10}{11:43-12:50} 
&
\caldata{15}{\sunmonth{\tula}{30}{}}{\sundata{07:18}{16:08}{09:04}}{\textsf{\snav} {\tiny \RIGHTarrow} 08:23\hspace{2ex}}{\textsf{\shatabhishak} {\tiny \RIGHTarrow} 14:44\hspace{2ex}}{08:24-09:30}{10:36-11:43} 
&
\caldata{16}{\sunmonth{\vrishchika}{1}{{\textsf{\tula} {\tiny \RIGHTarrow} (11:41\hspace{2ex})}}}{\sundata{07:20}{16:07}{09:05}}{\textsf{\sdas} {\tiny \RIGHTarrow} 10:51\hspace{2ex}}{\textsf{\proshthapada} {\tiny \RIGHTarrow} 17:36\hspace{2ex}}{13:55-15:01}{09:31-10:37} 
&
\caldata{17}{\sunmonth{\vrishchika}{2}{}}{\sundata{07:22}{16:06}{09:06}}{\textsf{\seka} {\tiny \RIGHTarrow} 13:04\hspace{2ex}}{\textsf{\uttaraproshthapada} {\tiny \RIGHTarrow} 20:12\hspace{2ex}}{11:44-12:49}{08:27-09:33} 
&
\caldata{18}{\sunmonth{\vrishchika}{3}{}}{\sundata{07:24}{16:04}{09:08}}{\textsf{\sdva} {\tiny \RIGHTarrow} 14:53\hspace{2ex}}{\textsf{\revati} {\tiny \RIGHTarrow} 22:24\hspace{2ex}}{12:49-13:54}{07:24-08:29} 
&
\caldata{19}{\sunmonth{\vrishchika}{4}{}}{\sundata{07:25}{16:03}{09:08}}{\textsf{\stra} {\tiny \RIGHTarrow} 16:13\hspace{2ex}}{\textsf{\ashwini} {\tiny \RIGHTarrow} 00:07(+1)}{10:39-11:44}{13:53-14:58} 
&
\caldata{20}{\sunmonth{\vrishchika}{5}{}}{\sundata{07:27}{16:02}{09:10}}{\textsf{\schaturdashi} {\tiny \RIGHTarrow} 17:03\hspace{2ex}}{\textsf{\apabharani} {\tiny \RIGHTarrow} 01:21(+1)}{09:35-10:40}{12:48-13:53} 
\\ \hline
\caldata{21}{\sunmonth{\vrishchika}{6}{}}{\sundata{07:29}{16:01}{09:11}}{\textsf{\purnima} {\tiny \RIGHTarrow} 17:23\hspace{2ex}}{\textsf{\krittika} {\tiny \RIGHTarrow} 02:07(+1)}{14:57-16:01}{11:45-12:49} 
&
\caldata{22}{\sunmonth{\vrishchika}{7}{}}{\sundata{07:30}{16:00}{09:12}}{\textsf{\kpra} {\tiny \RIGHTarrow} 17:16\hspace{2ex}}{\textsf{\rohini} {\tiny \RIGHTarrow} 02:27(+1)}{08:33-09:37}{10:41-11:45} 
&
\caldata{23}{\sunmonth{\vrishchika}{8}{}}{\sundata{07:32}{15:59}{09:13}}{\textsf{\kdvi} {\tiny \RIGHTarrow} 16:45\hspace{2ex}}{\textsf{\mrigashirsha} {\tiny \RIGHTarrow} 02:24(+1)}{13:52-14:55}{09:38-10:42} 
&
\caldata{24}{\sunmonth{\vrishchika}{9}{}}{\sundata{07:33}{15:58}{09:14}}{\textsf{\ktri} {\tiny \RIGHTarrow} 15:52\hspace{2ex}}{\textsf{\ardra} {\tiny \RIGHTarrow} 02:02(+1)}{11:45-12:48}{08:36-09:39} 
&
\caldata{25}{\sunmonth{\vrishchika}{10}{}}{\sundata{07:35}{15:57}{09:15}}{\textsf{\kcha} {\tiny \RIGHTarrow} 14:41\hspace{2ex}}{\textsf{\punarvasu} {\tiny \RIGHTarrow} 01:22(+1)}{12:48-13:51}{07:35-08:37} 
&
\caldata{26}{\sunmonth{\vrishchika}{11}{}}{\sundata{07:36}{15:56}{09:16}}{\textsf{\kpanc} {\tiny \RIGHTarrow} 13:13\hspace{2ex}}{\textsf{\pushya} {\tiny \RIGHTarrow} 00:27(+1)}{10:43-11:46}{13:51-14:53} 
&
\caldata{27}{\sunmonth{\vrishchika}{12}{}}{\sundata{07:38}{15:55}{09:17}}{\textsf{\ksha} {\tiny \RIGHTarrow} 11:31\hspace{2ex}}{\textsf{\ashresha} {\tiny \RIGHTarrow} 23:19\hspace{2ex}}{09:42-10:44}{12:48-13:50} 
\\ \hline
\caldata{28}{\sunmonth{\vrishchika}{13}{}}{\sundata{07:39}{15:54}{09:18}}{\textsf{\ksap} {\tiny \RIGHTarrow} 09:37\hspace{2ex}}{\textsf{\magha} {\tiny \RIGHTarrow} 22:00\hspace{2ex}}{14:52-15:54}{11:46-12:48} 
&
\caldata{29}{\sunmonth{\vrishchika}{14}{}}{\sundata{07:41}{15:53}{09:19}}{\textsf{\knav} {\tiny \RIGHTarrow} 05:19(+1)}{\textsf{\purvaphalguni} {\tiny \RIGHTarrow} 20:32\hspace{2ex}}{08:42-09:44}{10:45-11:47} 
&
\caldata{30}{\sunmonth{\vrishchika}{15}{}}{\sundata{07:42}{15:53}{09:20}}{\textsf{\kdas} {\tiny \RIGHTarrow} 03:02(+1)}{\textsf{\uttaraphalguni} {\tiny \RIGHTarrow} 18:59\hspace{2ex}}{13:50-14:51}{09:44-10:46} 
&
{}  &
{}  &
{}  &
\\ \hline
\end{tabular}


%\clearpage
\begin{tabular}{|c|c|c|c|c|c|c|}
\multicolumn{7}{c}{\Large \bfseries DECEMBER 2010}\\
\hline
\textbf{SUN} & \textbf{MON} & \textbf{TUE} & \textbf{WED} & \textbf{THU} & \textbf{FRI} & \textbf{SAT} \\ \hline
{}  &
{}  &
{}  &
\caldata{1}{\sunmonth{\vrishchika}{16}{}}{\sundata{07:44}{15:52}{09:21}}{\textsf{\keka} {\tiny \RIGHTarrow} 00:46(+1)}{\textsf{\hasta} {\tiny \RIGHTarrow} 17:24\hspace{2ex}}{11:48-12:49}{08:45-09:46} 
&
\caldata{2}{\sunmonth{\vrishchika}{17}{}}{\sundata{07:45}{15:51}{09:22}}{\textsf{\kdva} {\tiny \RIGHTarrow} 22:35\hspace{2ex}}{\textsf{\chitra} {\tiny \RIGHTarrow} 15:53\hspace{2ex}}{12:48-13:49}{07:45-08:45} 
&
\caldata{3}{\sunmonth{\vrishchika}{18}{}}{\sundata{07:47}{15:51}{09:23}}{\textsf{\ktra} {\tiny \RIGHTarrow} 20:37\hspace{2ex}}{\textsf{\svati} {\tiny \RIGHTarrow} 14:31\hspace{2ex}}{10:48-11:49}{13:50-14:50} 
&
\caldata{4}{\sunmonth{\vrishchika}{19}{}}{\sundata{07:48}{15:50}{09:24}}{\textsf{\kchaturdashi} {\tiny \RIGHTarrow} 18:56\hspace{2ex}}{\textsf{\vishakha} {\tiny \RIGHTarrow} 13:25\hspace{2ex}}{09:48-10:48}{12:49-13:49} 
\\ \hline
\caldata{5}{\sunmonth{\vrishchika}{20}{}}{\sundata{07:49}{15:50}{09:25}}{\textsf{\ama} {\tiny \RIGHTarrow} 17:39\hspace{2ex}}{\textsf{\anuradha} {\tiny \RIGHTarrow} 12:41\hspace{2ex}}{14:49-15:50}{11:49-12:49} 
&
\caldata{6}{\sunmonth{\vrishchika}{21}{}}{\sundata{07:50}{15:49}{09:25}}{\textsf{\spra} {\tiny \RIGHTarrow} 16:54\hspace{2ex}}{\textsf{\jyeshtha} {\tiny \RIGHTarrow} 12:27\hspace{2ex}}{08:49-09:49}{10:49-11:49} 
&
\caldata{7}{\sunmonth{\vrishchika}{22}{}}{\sundata{07:52}{15:49}{09:27}}{\textsf{\sdvi} {\tiny \RIGHTarrow} 16:44\hspace{2ex}}{\textsf{\mula} {\tiny \RIGHTarrow} 12:46\hspace{2ex}}{13:49-14:49}{09:51-10:50} 
&
\caldata{8}{\sunmonth{\vrishchika}{23}{}}{\sundata{07:53}{15:49}{09:28}}{\textsf{\stri} {\tiny \RIGHTarrow} 17:13\hspace{2ex}}{\textsf{\purvashadha} {\tiny \RIGHTarrow} 13:44\hspace{2ex}}{11:51-12:50}{08:52-09:52} 
&
\caldata{9}{\sunmonth{\vrishchika}{24}{}}{\sundata{07:54}{15:48}{09:28}}{\textsf{\scha} {\tiny \RIGHTarrow} 18:21\hspace{2ex}}{\textsf{\uttarashadha} {\tiny \RIGHTarrow} 15:18\hspace{2ex}}{12:50-13:49}{07:54-08:53} 
&
\caldata{10}{\sunmonth{\vrishchika}{25}{}}{\sundata{07:55}{15:48}{09:29}}{\textsf{\spanc} {\tiny \RIGHTarrow} 20:04\hspace{2ex}}{\textsf{\shravana} {\tiny \RIGHTarrow} 17:27\hspace{2ex}}{10:52-11:51}{13:49-14:48} 
&
\caldata{11}{\sunmonth{\vrishchika}{26}{}}{\sundata{07:56}{15:48}{09:30}}{\textsf{\ssha} {\tiny \RIGHTarrow} 22:15\hspace{2ex}}{\textsf{\shravishtha} {\tiny \RIGHTarrow} 20:03\hspace{2ex}}{09:54-10:53}{12:51-13:50} 
\\ \hline
\caldata{12}{\sunmonth{\vrishchika}{27}{}}{\sundata{07:57}{15:48}{09:31}}{\textsf{\ssap} {\tiny \RIGHTarrow} 00:42(+1)}{\textsf{\shatabhishak} {\tiny \RIGHTarrow} 22:56\hspace{2ex}}{14:49-15:48}{11:52-12:51} 
&
\caldata{13}{\sunmonth{\vrishchika}{28}{}}{\sundata{07:58}{15:48}{09:32}}{\textsf{\sasht} {\tiny \RIGHTarrow} 03:13(+1)}{\textsf{\proshthapada} {\tiny \RIGHTarrow} 01:52(+1)}{08:56-09:55}{10:54-11:53} 
&
\caldata{14}{\sunmonth{\vrishchika}{29}{}}{\sundata{07:59}{15:48}{09:32}}{\textsf{\snav} {\tiny \RIGHTarrow} 05:33(+1)}{\textsf{\uttaraproshthapada} {\tiny \RIGHTarrow} 04:39(+1)}{13:50-14:49}{09:56-10:54} 
&
\caldata{15}{\sunmonth{\vrishchika}{30}{{\textsf{\vrishchika} {\tiny \RIGHTarrow} (02:22(+1))}}}{\sundata{08:00}{15:48}{09:33}}{\textsf{\sdas} {\tiny \RIGHTarrow} 07:31(+1)}{\textsf{\revati} {\tiny \RIGHTarrow} 07:05(+1)}{11:54-12:52}{08:58-09:57} 
&
\caldata{16}{\sunmonth{\dhanur}{1}{}}{\sundata{08:01}{15:48}{09:34}}{\textsf{\seka} {\tiny \RIGHTarrow} \ahoratram}{\textsf{\ashwini} {\tiny \RIGHTarrow} \ahoratram}{12:52-13:51}{08:01-08:59} 
&
\caldata{17}{\sunmonth{\dhanur}{2}{}}{\sundata{08:01}{15:49}{09:34}}{\textsf{\seka} {\tiny \RIGHTarrow} 08:55\hspace{2ex}}{\textsf{\ashwini} {\tiny \RIGHTarrow} 08:59\hspace{2ex}}{10:56-11:55}{13:52-14:50} 
&
\caldata{18}{\sunmonth{\dhanur}{3}{}}{\sundata{08:02}{15:49}{09:35}}{\textsf{\sdva} {\tiny \RIGHTarrow} 09:41\hspace{2ex}}{\textsf{\apabharani} {\tiny \RIGHTarrow} 10:17\hspace{2ex}}{09:58-10:57}{12:53-13:52} 
\\ \hline
\caldata{19}{\sunmonth{\dhanur}{4}{}}{\sundata{08:03}{15:49}{09:36}}{\textsf{\stra} {\tiny \RIGHTarrow} 09:47\hspace{2ex}}{\textsf{\krittika} {\tiny \RIGHTarrow} 10:56\hspace{2ex}}{14:50-15:49}{11:56-12:54} 
&
\caldata{20}{\sunmonth{\dhanur}{5}{}}{\sundata{08:03}{15:50}{09:36}}{\textsf{\schaturdashi} {\tiny \RIGHTarrow} 09:17\hspace{2ex}}{\textsf{\rohini} {\tiny \RIGHTarrow} 11:00\hspace{2ex}}{09:01-09:59}{10:58-11:56} 
&
\caldata{21}{\sunmonth{\dhanur}{6}{}}{\sundata{08:04}{15:50}{09:37}}{\textsf{\purnima} {\tiny \RIGHTarrow} 08:13\hspace{2ex}}{\textsf{\mrigashirsha} {\tiny \RIGHTarrow} 10:31\hspace{2ex}}{13:53-14:51}{10:00-10:58} 
&
\caldata{22}{\sunmonth{\dhanur}{7}{}}{\sundata{08:05}{15:51}{09:38}}{\textsf{\kdvi} {\tiny \RIGHTarrow} 04:45(+1)}{\textsf{\ardra} {\tiny \RIGHTarrow} 09:36\hspace{2ex}}{11:58-12:56}{09:03-10:01} 
&
\caldata{23}{\sunmonth{\dhanur}{8}{}}{\sundata{08:05}{15:51}{09:38}}{\textsf{\ktri} {\tiny \RIGHTarrow} 02:35(+1)}{\textsf{\punarvasu} {\tiny \RIGHTarrow} 08:20\hspace{2ex}}{12:56-13:54}{08:05-09:03} 
&
\caldata{24}{\sunmonth{\dhanur}{9}{}}{\sundata{08:05}{15:52}{09:38}}{\textsf{\kcha} {\tiny \RIGHTarrow} 00:17(+1)}{\textsf{\ashresha} {\tiny \RIGHTarrow} 05:11(+1)}{11:00-11:58}{13:55-14:53} 
&
\caldata{25}{\sunmonth{\dhanur}{10}{}}{\sundata{08:06}{15:52}{09:39}}{\textsf{\kpanc} {\tiny \RIGHTarrow} 21:56\hspace{2ex}}{\textsf{\magha} {\tiny \RIGHTarrow} 03:32(+1)}{10:02-11:00}{12:57-13:55} 
\\ \hline
\caldata{26}{\sunmonth{\dhanur}{11}{}}{\sundata{08:06}{15:53}{09:39}}{\textsf{\ksha} {\tiny \RIGHTarrow} 19:36\hspace{2ex}}{\textsf{\purvaphalguni} {\tiny \RIGHTarrow} 01:55(+1)}{14:54-15:53}{11:59-12:57} 
&
\caldata{27}{\sunmonth{\dhanur}{12}{}}{\sundata{08:06}{15:54}{09:39}}{\textsf{\ksap} {\tiny \RIGHTarrow} 17:23\hspace{2ex}}{\textsf{\uttaraphalguni} {\tiny \RIGHTarrow} 00:25(+1)}{09:04-10:03}{11:01-12:00} 
&
\caldata{28}{\sunmonth{\dhanur}{13}{}}{\sundata{08:06}{15:55}{09:39}}{\textsf{\kasht} {\tiny \RIGHTarrow} 15:19\hspace{2ex}}{\textsf{\hasta} {\tiny \RIGHTarrow} 23:07\hspace{2ex}}{13:57-14:56}{10:03-11:01} 
&
\caldata{29}{\sunmonth{\dhanur}{14}{}}{\sundata{08:07}{15:56}{09:40}}{\textsf{\knav} {\tiny \RIGHTarrow} 13:28\hspace{2ex}}{\textsf{\chitra} {\tiny \RIGHTarrow} 22:01\hspace{2ex}}{12:01-13:00}{09:05-10:04} 
&
\caldata{30}{\sunmonth{\dhanur}{15}{}}{\sundata{08:07}{15:57}{09:41}}{\textsf{\kdas} {\tiny \RIGHTarrow} 11:51\hspace{2ex}}{\textsf{\svati} {\tiny \RIGHTarrow} 21:12\hspace{2ex}}{13:00-13:59}{08:07-09:05} 
&
\caldata{31}{\sunmonth{\dhanur}{16}{}}{\sundata{08:07}{15:58}{09:41}}{\textsf{\keka} {\tiny \RIGHTarrow} 10:33\hspace{2ex}}{\textsf{\vishakha} {\tiny \RIGHTarrow} 20:40\hspace{2ex}}{11:03-12:02}{14:00-14:59} 
&
\\ \hline
\end{tabular}


%\clearpage

\end{center}
\end{document}
