\documentclass[a4paper,11pt,landscape]{article}
\usepackage[sort&compress,square,numbers]{natbib}

\usepackage[xetex]{graphicx}
%\usepackage{fullpage}
\usepackage{multirow}
\usepackage[normalsections]{savetrees}
\usepackage{euler}
\usepackage{fontspec}
\usepackage{xltxtra}
\usepackage{url}
\usepackage{multicol}
\usepackage{bbding}
% PDF SETUP
% ---- FILL IN HERE THE DOC TITLE AND AUTHOR
%\usepackage[bookmarks, colorlinks, breaklinks, pdftitle={Karthik Raman - vita},pdfauthor={Karthik Raman}]{hyperref} 
\usepackage[dvipsnames]{xcolor} 
\usepackage{wasysym} 
%\hypersetup{linkcolor=Sepia,citecolor=blue,filecolor=black,urlcolor=Blue} 


\defaultfontfeatures{Scale=MatchLowercase,Mapping=tex-text}
\setmainfont{Scala Sans LF}
\setsansfont{Sanskrit 2003:script=deva}

\newcommand{\caldata}[8]{%
\begin{minipage}{2cm}
\begin{minipage}[t]{1.2cm}
\vspace{.2ex}
%\mbox{}\\
%\SunshineOpenCircled
\mbox{{\sun\tiny\UParrow} \small #2}\\
\mbox{{\sun\tiny\DOWNarrow} \small  #6}\\
\scriptsize
\mbox{(\textsf{स} #3)}\\
%\textsf{राहुकालम्:} #4\\
\mbox{\textsf{#7}}\\
\mbox{\textsf{#8}}\\
\mbox{\textsf{राहु~} #4}\\
%{\textsf{यमकण्डम्:}} #5\\
\mbox{\textsf{यम~} #5}\\
\end{minipage}\begin{minipage}[c]{1.cm}
\vspace{.4ex}
\begin{flushright} \textcolor{blue}{\font\x="Plantin Std" at 24 pt\x #1}
\end{flushright}
\end{minipage}
\end{minipage}
}

\addtolength{\headsep}{-3ex}
\pagestyle{empty}
\newcommand{\ashwini}{अश्विनी}
\newcommand{\apabharani}{अपभरणी}
\newcommand{\krittika}{कृत्तिका}
\newcommand{\rohini}{रोहिणी}
\newcommand{\mrigashirsha}{मृगशीर्ष}
\newcommand{\ardra}{आर्द्रा}
\newcommand{\punarvasu}{पुनर्वसू}
\newcommand{\pushya}{पुष्य}
\newcommand{\ashresha}{आश्रेषा}
\newcommand{\magha}{मघा}
\newcommand{\purvaphalguni}{पूर्वफल्गुनी}
\newcommand{\uttaraphalguni}{उत्तरफल्गुनी}
\newcommand{\hasta}{हस्त}
\newcommand{\chitra}{चित्रा}
\newcommand{\svati}{स्वाति}
\newcommand{\vishakha}{विशाखा}
\newcommand{\anuradha}{अनूराधा}
\newcommand{\jyeshtha}{ज्येष्ठा}
\newcommand{\mula}{मूला}
\newcommand{\purvashadha}{पूर्वाषाढा}
\newcommand{\uttarashadha}{उत्तराषाढा}
\newcommand{\shravana}{श्रवण}
\newcommand{\shravishtha}{श्रविष्ठा}
\newcommand{\shatabhishak}{शतभिषक्}
\newcommand{\proshthapada}{प्रोष्ठपदा}
\newcommand{\uttaraproshthapada}{उत्तरप्रोष्ठपदा}
\newcommand{\revati}{रेवती}

\newcommand{\spra}{शुक्ल प्रथमा}
\newcommand{\sdvi}{शुक्ल द्वितीया}
\newcommand{\stri}{शुक्ल तृतीया}
\newcommand{\scha}{शुक्ल चतुर्थी}
\newcommand{\spanc}{शुक्ल पञ्चमी}
\newcommand{\ssha}{शुक्ल षष्ठी}
\newcommand{\ssap}{शुक्ल सप्तमी}
\newcommand{\sasht}{शुक्ल अष्टमी}
\newcommand{\snav}{शुक्ल नवमी}
\newcommand{\sdas}{शुक्ल दशमी}
\newcommand{\seka}{शुक्ल एकादशी}
\newcommand{\sdva}{शुक्ल द्वादशी}
\newcommand{\stra}{शुक्ल त्रयोदशी}
\newcommand{\schaturdashi}{शुक्ल चतुर्दशी}
\newcommand{\purnima}{पूर्णिमा}
\newcommand{\kpra}{कृष्ण प्रथमा}
\newcommand{\kdvi}{कृष्ण द्वितीया}
\newcommand{\ktri}{कृष्ण तृतीया}
\newcommand{\kcha}{कृष्ण चतुर्थी}
\newcommand{\kpanc}{कृष्ण पञ्चमी}
\newcommand{\ksha}{कृष्ण षष्ठी}
\newcommand{\ksap}{कृष्ण सप्तमी}
\newcommand{\kasht}{कृष्ण अष्टमी}
\newcommand{\knav}{कृष्ण नवमी}
\newcommand{\kdas}{कृष्ण दशमी}
\newcommand{\keka}{कृष्ण एकादशी}
\newcommand{\kdva}{कृष्ण द्वादशी}
\newcommand{\ktra}{कृष्ण त्रयोदशी}
\newcommand{\kchaturdashi}{कृष्ण चतुर्दशी}
\newcommand{\ama}{अमावस्या}
\begin{document}
\pagestyle{empty}
\begin{center}
\mbox{}\\[2.5in]
\hrule\mbox{}
\mbox{}\\[1ex]
\mbox{}
{\font\x="Warnock Pro" at 60 pt\x 2010\\[0.5cm]}
\mbox{}
{\font\x="Warnock Pro" at 48 pt\x \uppercase{Chennai}\\[0.3cm]}
\hrule
\begin{tabular}{|c|c|c|c|c|c|c|}
\multicolumn{7}{c}{\Large \bfseries JANUARY 2010}\\
\hline
\textbf{SUN} & \textbf{MON} & \textbf{TUE} & \textbf{WED} & \textbf{THU} & \textbf{FRI} & \textbf{SAT} \\ \hline
{}  &
{}  &
{}  &
{}  &
{}  &
\caldata{1}{06:32}{08:48}{10:47-12:12}{15:02-16:27}{17:52}{\kpra -- 21:13}{\ardra -- 08:06} 
&
\caldata{2}{06:32}{08:48}{09:22-10:47}{13:37-15:02}{17:53}{\kdvi -- 17:44}{\pushya -- 26:36} 
\\ \hline
\caldata{3}{06:32}{08:48}{16:27-17:53}{12:12-13:37}{17:53}{\ktri -- 14:23}{\ashresha -- 24:05} 
&
\caldata{4}{06:33}{08:49}{07:58-09:23}{10:48-12:13}{17:54}{\kcha -- 11:16}{\magha -- 21:54} 
&
\caldata{5}{06:33}{08:49}{15:03-16:28}{09:23-10:48}{17:54}{\kpanc -- 08:32}{\purvaphalguni -- 20:09} 
&
\caldata{6}{06:33}{08:49}{12:14-13:39}{07:58-09:23}{17:55}{\ksap -- 28:44}{\uttaraphalguni -- 18:57} 
&
\caldata{7}{06:34}{08:50}{13:39-15:04}{06:34-07:59}{17:55}{\kasht -- 27:47}{\hasta -- 18:21} 
&
\caldata{8}{06:34}{08:50}{10:49-12:15}{15:05-16:30}{17:56}{\knav -- 27:28}{\chitra -- 18:24} 
&
\caldata{9}{06:34}{08:50}{09:24-10:50}{13:40-15:06}{17:57}{\kdas -- 27:47}{\svati -- 19:03} 
\\ \hline
\caldata{10}{06:34}{08:50}{16:31-17:57}{12:15-13:40}{17:57}{\keka -- 28:41}{\vishakha -- 20:17} 
&
\caldata{11}{06:35}{08:51}{08:00-09:25}{10:51-12:16}{17:58}{\kdva -- 30:06}{\anuradha -- 22:03} 
&
\caldata{12}{06:35}{08:51}{15:07-16:32}{09:25-10:51}{17:58}{\ktra -- 31:58}{\jyeshtha -- 24:15} 
&
\caldata{13}{06:35}{08:51}{12:17-13:42}{08:00-09:26}{17:59}{\ktra -- 07:59}{\mula -- 26:49} 
&
\caldata{14}{06:35}{08:51}{13:42-15:08}{06:35-08:00}{17:59}{\kchaturdashi -- 10:12}{\purvashadha -- 29:39} 
&
\caldata{15}{06:36}{08:52}{10:52-12:18}{15:09-16:34}{18:00}{\ama -- 12:42}{\uttarashadha -- 32:41} 
&
\caldata{16}{06:36}{08:52}{09:27-10:52}{13:43-15:09}{18:00}{\spra -- 15:21}{\uttarashadha -- 08:41} 
\\ \hline
\caldata{17}{06:36}{08:53}{16:35-18:01}{12:18-13:44}{18:01}{\sdvi -- 18:04}{\shravana -- 11:48} 
&
\caldata{18}{06:36}{08:53}{08:01-09:27}{10:53-12:19}{18:02}{\stri -- 20:44}{\shravishtha -- 14:54} 
&
\caldata{19}{06:36}{08:53}{15:10-16:36}{09:27-10:53}{18:02}{\scha -- 23:13}{\shatabhishak -- 17:51} 
&
\caldata{20}{06:36}{08:53}{12:19-13:45}{08:01-09:27}{18:03}{\spanc -- 06:00}{\proshthapada -- 06:07} 
&
\caldata{21}{06:36}{08:53}{13:45-15:11}{06:36-08:01}{18:03}{\ssha -- 27:02}{\uttaraproshthapada -- 22:45} 
&
\caldata{22}{06:36}{08:53}{10:54-12:20}{15:12-16:38}{18:04}{\ssap -- 28:05}{\revati -- 24:25} 
&
\caldata{23}{06:36}{08:53}{09:28-10:54}{13:46-15:12}{18:04}{\sasht -- 28:25}{\ashwini -- 25:24} 
\\ \hline
\caldata{24}{06:36}{08:53}{16:38-18:05}{12:20-13:46}{18:05}{\snav -- 27:58}{\apabharani -- 25:39} 
&
\caldata{25}{06:36}{08:53}{08:02-09:28}{10:54-12:20}{18:05}{\sdas -- 26:43}{\krittika -- 25:07} 
&
\caldata{26}{06:36}{08:54}{15:13-16:39}{09:28-10:54}{18:06}{\seka -- 24:44}{\rohini -- 23:52} 
&
\caldata{27}{06:36}{08:54}{12:21-13:47}{08:02-09:28}{18:06}{\sdva -- 22:07}{\mrigashirsha -- 21:59} 
&
\caldata{28}{06:36}{08:54}{13:47-15:13}{06:36-08:02}{18:06}{\stra -- 18:59}{\ardra -- 19:34} 
&
\caldata{29}{06:36}{08:54}{10:55-12:21}{15:14-16:40}{18:07}{\schaturdashi -- 15:30}{\punarvasu -- 16:48} 
&
\caldata{30}{06:36}{08:54}{09:28-10:55}{13:47-15:14}{18:07}{\purnima -- 11:47}{\pushya -- 13:50} 
\\ \hline
\caldata{31}{06:36}{08:54}{16:41-18:08}{12:22-13:48}{18:08}{\kpra -- 08:00}{\ashresha -- 10:49} 
&
{}  &
{}  &
{}  &
{}  &
{}  &
\\ \hline
\end{tabular}


%\clearpage
\begin{tabular}{|c|c|c|c|c|c|c|}
\multicolumn{7}{c}{\Large \bfseries FEBRUARY 2010}\\
\hline
\textbf{SUN} & \textbf{MON} & \textbf{TUE} & \textbf{WED} & \textbf{THU} & \textbf{FRI} & \textbf{SAT} \\ \hline
{}  &
\caldata{1}{06:36}{08:54}{08:02-09:29}{10:55-12:22}{18:08}{\ktri -- 25:00}{\magha -- 07:55} 
&
\caldata{2}{06:36}{08:54}{15:15-16:42}{09:29-10:55}{18:09}{\kcha -- 22:07}{\uttaraphalguni -- 27:20} 
&
\caldata{3}{06:35}{08:53}{12:22-13:48}{08:01-09:28}{18:09}{\kpanc -- 19:50}{\hasta -- 25:55} 
&
\caldata{4}{06:35}{08:53}{13:48-15:15}{06:35-08:01}{18:09}{\ksha -- 18:16}{\chitra -- 25:14} 
&
\caldata{5}{06:35}{08:54}{10:55-12:22}{15:16-16:43}{18:10}{\ksap -- 17:29}{\svati -- 25:20} 
&
\caldata{6}{06:35}{08:54}{09:28-10:55}{13:49-15:16}{18:10}{\kasht -- 17:31}{\vishakha -- 26:13} 
\\ \hline
\caldata{7}{06:35}{08:54}{16:44-18:11}{12:23-13:50}{18:11}{\knav -- 18:20}{\anuradha -- 27:49} 
&
\caldata{8}{06:34}{08:53}{08:01-09:28}{10:55-12:22}{18:11}{\kdas -- 19:50}{\jyeshtha -- 30:01} 
&
\caldata{9}{06:34}{08:53}{15:16-16:43}{09:28-10:55}{18:11}{\keka -- 21:53}{\mula -- 32:42} 
&
\caldata{10}{06:34}{08:53}{12:23-13:50}{08:01-09:28}{18:12}{\kdva -- 24:18}{\mula -- 08:43} 
&
\caldata{11}{06:33}{08:52}{13:49-15:17}{06:33-08:00}{18:12}{\ktra -- 26:57}{\purvashadha -- 11:42} 
&
\caldata{12}{06:33}{08:52}{10:55-12:22}{15:17-16:44}{18:12}{\kchaturdashi -- 29:41}{\uttarashadha -- 14:50} 
&
\caldata{13}{06:33}{08:53}{09:28-10:55}{13:50-15:18}{18:13}{\ama -- 32:21}{\shravana -- 17:57} 
\\ \hline
\caldata{14}{06:32}{08:52}{16:45-18:13}{12:22-13:50}{18:13}{\ama -- 08:20}{\shravishtha -- 20:59} 
&
\caldata{15}{06:32}{08:52}{07:59-09:27}{10:54-12:22}{18:13}{\spra -- 10:51}{\shatabhishak -- 23:50} 
&
\caldata{16}{06:32}{08:52}{15:17-16:45}{09:27-10:54}{18:13}{\sdvi -- 06:19}{\proshthapada -- 05:50} 
&
\caldata{17}{06:31}{08:51}{12:22-13:50}{07:58-09:26}{18:14}{\stri -- 15:04}{\uttaraproshthapada -- 28:43} 
&
\caldata{18}{06:31}{08:51}{13:50-15:18}{06:31-07:58}{18:14}{\scha -- 16:39}{\revati -- 30:36} 
&
\caldata{19}{06:30}{08:50}{10:54-12:22}{15:18-16:46}{18:14}{\spanc -- 17:46}{\revati -- 06:36} 
&
\caldata{20}{06:30}{08:50}{09:26-10:54}{13:50-15:18}{18:14}{\ssha -- 18:22}{\ashwini -- 07:59} 
\\ \hline
\caldata{21}{06:29}{08:50}{16:46-18:15}{12:22-13:50}{18:15}{\ssap -- 18:21}{\apabharani -- 08:49} 
&
\caldata{22}{06:29}{08:50}{07:57-09:25}{10:53-12:22}{18:15}{\sasht -- 17:42}{\krittika -- 09:02} 
&
\caldata{23}{06:29}{08:50}{15:18-16:46}{09:25-10:53}{18:15}{\snav -- 16:23}{\rohini -- 08:38} 
&
\caldata{24}{06:28}{08:49}{12:21-13:49}{07:56-09:24}{18:15}{\sdas -- 14:27}{\mrigashirsha -- 07:35} 
&
\caldata{25}{06:28}{08:49}{13:50-15:19}{06:28-07:56}{18:16}{\seka -- 11:55}{\punarvasu -- 27:43} 
&
\caldata{26}{06:27}{08:48}{10:52-12:21}{15:18-16:47}{18:16}{\sdva -- 08:54}{\pushya -- 25:07} 
&
\caldata{27}{06:27}{08:48}{09:24-10:52}{13:50-15:18}{18:16}{\schaturdashi -- 25:50}{\ashresha -- 22:16} 
\\ \hline
\caldata{28}{06:26}{08:48}{16:47-18:16}{12:21-13:49}{18:16}{\purnima -- 22:08}{\magha -- 19:21} 
&
{}  &
{}  &
{}  &
{}  &
{}  &
\\ \hline
\end{tabular}


%\clearpage
\begin{tabular}{|c|c|c|c|c|c|c|}
\multicolumn{7}{c}{\Large \bfseries MARCH 2010}\\
\hline
\textbf{SUN} & \textbf{MON} & \textbf{TUE} & \textbf{WED} & \textbf{THU} & \textbf{FRI} & \textbf{SAT} \\ \hline
{}  &
\caldata{1}{06:25}{08:47}{07:53-09:22}{10:51-12:20}{18:16}{\kpra -- 18:35}{\purvaphalguni -- 16:31} 
&
\caldata{2}{06:25}{08:47}{15:18-16:47}{09:22-10:51}{18:16}{\kdvi -- 15:18}{\uttaraphalguni -- 13:58} 
&
\caldata{3}{06:24}{08:46}{12:20-13:49}{07:53-09:22}{18:17}{\ktri -- 12:29}{\hasta -- 11:52} 
&
\caldata{4}{06:24}{08:46}{13:49-15:18}{06:24-07:53}{18:17}{\kcha -- 10:18}{\chitra -- 10:23} 
&
\caldata{5}{06:23}{08:45}{10:50-12:20}{15:18-16:47}{18:17}{\kpanc -- 08:54}{\svati -- 09:41} 
&
\caldata{6}{06:23}{08:45}{09:21-10:50}{13:49-15:18}{18:17}{\ksha -- 08:21}{\vishakha -- 09:50} 
\\ \hline
\caldata{7}{06:22}{08:45}{16:47-18:17}{12:19-13:48}{18:17}{\ksap -- 08:43}{\anuradha -- 10:52} 
&
\caldata{8}{06:21}{08:44}{07:50-09:20}{10:49-12:19}{18:17}{\kasht -- 09:55}{\jyeshtha -- 12:40} 
&
\caldata{9}{06:21}{08:44}{15:18-16:47}{09:20-10:49}{18:17}{\knav -- 11:49}{\mula -- 15:08} 
&
\caldata{10}{06:20}{08:43}{12:18-13:48}{07:49-09:19}{18:17}{\kdas -- 14:12}{\purvashadha -- 18:01} 
&
\caldata{11}{06:20}{08:43}{13:48-15:18}{06:20-07:49}{18:18}{\keka -- 16:52}{\uttarashadha -- 21:09} 
&
\caldata{12}{06:19}{08:42}{10:48-12:18}{15:18-16:48}{18:18}{\kdva -- 19:34}{\shravana -- 24:19} 
&
\caldata{13}{06:18}{08:42}{09:18-10:48}{13:48-15:18}{18:18}{\ktra -- 22:10}{\shravishtha -- 27:20} 
\\ \hline
\caldata{14}{06:18}{08:42}{16:48-18:18}{12:18-13:48}{18:18}{\kchaturdashi -- 24:30}{\shatabhishak -- 30:06} 
&
\caldata{15}{06:17}{08:41}{07:47-09:17}{10:47-12:17}{18:18}{\ama -- 26:29}{\proshthapada -- 32:32} 
&
\caldata{16}{06:16}{08:40}{15:17-16:47}{09:16-10:46}{18:18}{\spra -- 05:34}{\proshthapada -- 06:12} 
&
\caldata{17}{06:16}{08:40}{12:17-13:47}{07:46-09:16}{18:18}{\sdvi -- 29:19}{\uttaraproshthapada -- 10:32} 
&
\caldata{18}{06:15}{08:39}{13:46-15:17}{06:15-07:45}{18:18}{\stri -- 30:08}{\revati -- 12:12} 
&
\caldata{19}{06:14}{08:38}{10:45-12:16}{15:17-16:47}{18:18}{\scha -- 30:33}{\ashwini -- 13:29} 
&
\caldata{20}{06:14}{08:38}{09:15-10:45}{13:46-15:17}{18:18}{\scha -- 06:14}{\apabharani -- 14:21} 
\\ \hline
\caldata{21}{06:13}{08:38}{16:47-18:18}{12:15-13:46}{18:18}{\spanc -- 06:32}{\krittika -- 14:50} 
&
\caldata{22}{06:12}{08:37}{07:42-09:13}{10:44-12:15}{18:18}{\ssap -- 29:08}{\rohini -- 14:51} 
&
\caldata{23}{06:12}{08:37}{15:16-16:47}{09:13-10:44}{18:18}{\sasht -- 27:42}{\mrigashirsha -- 14:25} 
&
\caldata{24}{06:11}{08:36}{12:15-13:46}{07:42-09:13}{18:19}{\snav -- 25:46}{\ardra -- 13:29} 
&
\caldata{25}{06:10}{08:35}{13:45-15:16}{06:10-07:41}{18:19}{\sdas -- 23:23}{\punarvasu -- 12:06} 
&
\caldata{26}{06:10}{08:35}{10:43-12:14}{15:16-16:47}{18:19}{\seka -- 20:38}{\pushya -- 10:18} 
&
\caldata{27}{06:09}{08:35}{09:11-10:42}{13:45-15:16}{18:19}{\sdva -- 17:34}{\ashresha -- 08:08} 
\\ \hline
\caldata{28}{06:08}{08:34}{16:47-18:19}{12:13-13:44}{18:19}{\stra -- 14:21}{\purvaphalguni -- 27:12} 
&
\caldata{29}{06:08}{08:34}{07:39-09:10}{10:42-12:13}{18:19}{\schaturdashi -- 11:05}{\uttaraphalguni -- 24:47} 
&
\caldata{30}{06:07}{08:33}{15:16-16:47}{09:10-10:41}{18:19}{\purnima -- 07:56}{\hasta -- 22:37} 
&
\caldata{31}{06:06}{08:32}{12:12-13:44}{07:37-09:09}{18:19}{\kdvi -- 26:45}{\chitra -- 20:52} 
&
{}  &
{}  &
\\ \hline
\end{tabular}


%\clearpage
\begin{tabular}{|c|c|c|c|c|c|c|}
\multicolumn{7}{c}{\Large \bfseries APRIL 2010}\\
\hline
\textbf{SUN} & \textbf{MON} & \textbf{TUE} & \textbf{WED} & \textbf{THU} & \textbf{FRI} & \textbf{SAT} \\ \hline
{}  &
{}  &
{}  &
{}  &
\caldata{1}{06:06}{08:32}{13:44-15:15}{06:06-07:37}{18:19}{\ktri -- 25:02}{\svati -- 19:42} 
&
\caldata{2}{06:05}{08:31}{10:40-12:12}{15:15-16:47}{18:19}{\kcha -- 24:05}{\vishakha -- 19:15} 
&
\caldata{3}{06:04}{08:31}{09:07-10:39}{13:43-15:15}{18:19}{\kpanc -- 23:58}{\anuradha -- 19:37} 
\\ \hline
\caldata{4}{06:04}{08:31}{16:47-18:19}{12:11-13:43}{18:19}{\ksha -- 24:40}{\jyeshtha -- 20:48} 
&
\caldata{5}{06:03}{08:30}{07:35-09:07}{10:39-12:11}{18:19}{\ksap -- 26:09}{\mula -- 22:45} 
&
\caldata{6}{06:03}{08:30}{15:15-16:47}{09:07-10:39}{18:19}{\kasht -- 28:14}{\purvashadha -- 25:18} 
&
\caldata{7}{06:02}{08:29}{12:10-13:42}{07:34-09:06}{18:19}{\knav -- 30:44}{\uttarashadha -- 28:15} 
&
\caldata{8}{06:01}{08:28}{13:42-15:15}{06:01-07:33}{18:20}{\knav -- 06:44}{\shravana -- 31:23} 
&
\caldata{9}{06:01}{08:28}{10:38-12:10}{15:15-16:47}{18:20}{\kdas -- 09:22}{\shravana -- 07:23} 
&
\caldata{10}{06:00}{08:28}{09:05-10:37}{13:42-15:15}{18:20}{\keka -- 11:52}{\shravishtha -- 10:24} 
\\ \hline
\caldata{11}{05:59}{08:27}{16:47-18:20}{12:09-13:42}{18:20}{\kdva -- 14:04}{\shatabhishak -- 13:08} 
&
\caldata{12}{05:59}{08:27}{07:31-09:04}{10:36-12:09}{18:20}{\ktra -- 05:40}{\proshthapada -- 05:39} 
&
\caldata{13}{05:58}{08:26}{15:14-16:47}{09:03-10:36}{18:20}{\kchaturdashi -- 17:08}{\uttaraproshthapada -- 17:20} 
&
\caldata{14}{05:58}{08:26}{12:09-13:41}{07:30-09:03}{18:20}{\ama -- 17:55}{\revati -- 18:44} 
&
\caldata{15}{05:57}{08:25}{13:41-15:14}{05:57-07:29}{18:20}{\spra -- 18:15}{\ashwini -- 19:41} 
&
\caldata{16}{05:56}{08:24}{10:35-12:08}{15:14-16:47}{18:20}{\sdvi -- 18:09}{\apabharani -- 20:13} 
&
\caldata{17}{05:56}{08:24}{09:02-10:35}{13:41-15:14}{18:20}{\stri -- 17:41}{\krittika -- 20:24} 
\\ \hline
\caldata{18}{05:55}{08:24}{16:47-18:21}{12:08-13:41}{18:21}{\scha -- 16:54}{\rohini -- 20:16} 
&
\caldata{19}{05:55}{08:24}{07:28-09:01}{10:34-12:08}{18:21}{\spanc -- 15:48}{\mrigashirsha -- 19:50} 
&
\caldata{20}{05:54}{08:23}{15:14-16:47}{09:00-10:34}{18:21}{\ssha -- 14:25}{\ardra -- 19:06} 
&
\caldata{21}{05:54}{08:23}{12:07-13:40}{07:27-09:00}{18:21}{\ssap -- 12:44}{\punarvasu -- 18:06} 
&
\caldata{22}{05:53}{08:22}{13:40-15:14}{05:53-07:26}{18:21}{\sasht -- 10:47}{\pushya -- 16:49} 
&
\caldata{23}{05:52}{08:21}{10:32-12:06}{15:13-16:47}{18:21}{\snav -- 08:34}{\ashresha -- 15:18} 
&
\caldata{24}{05:52}{08:21}{08:59-10:32}{13:40-15:13}{18:21}{\sdas -- 06:09}{\magha -- 13:35} 
\\ \hline
\caldata{25}{05:51}{08:21}{16:47-18:21}{12:06-13:39}{18:21}{\sdva -- 24:54}{\purvaphalguni -- 11:45} 
&
\caldata{26}{05:51}{08:21}{07:24-08:58}{10:32-12:06}{18:22}{\stra -- 22:19}{\uttaraphalguni -- 09:53} 
&
\caldata{27}{05:50}{08:20}{15:14-16:48}{08:58-10:32}{18:22}{\schaturdashi -- 19:56}{\hasta -- 08:08} 
&
\caldata{28}{05:50}{08:20}{12:06-13:40}{07:24-08:58}{18:22}{\purnima -- 17:52}{\chitra -- 06:36} 
&
\caldata{29}{05:50}{08:20}{13:40-15:14}{05:50-07:24}{18:22}{\kpra -- 16:16}{\vishakha -- 28:55} 
&
\caldata{30}{05:49}{08:19}{10:31-12:05}{15:13-16:47}{18:22}{\kdvi -- 15:15}{\anuradha -- 29:00} 
&
\\ \hline
\end{tabular}


%\clearpage
\begin{tabular}{|c|c|c|c|c|c|c|}
\multicolumn{7}{c}{\Large \bfseries MAY 2010}\\
\hline
\textbf{SUN} & \textbf{MON} & \textbf{TUE} & \textbf{WED} & \textbf{THU} & \textbf{FRI} & \textbf{SAT} \\ \hline
{}  &
{}  &
{}  &
{}  &
{}  &
{}  &
\caldata{1}{05:49}{08:19}{08:57-10:31}{13:39-15:13}{18:22}{\ktri -- 14:56}{\jyeshtha -- 29:47} 
\\ \hline
\caldata{2}{05:48}{08:19}{16:48-18:23}{12:05-13:39}{18:23}{\kcha -- 15:22}{\mula -- 31:17} 
&
\caldata{3}{05:48}{08:19}{07:22-08:56}{10:31-12:05}{18:23}{\kpanc -- 16:31}{\mula -- 07:19} 
&
\caldata{4}{05:47}{08:18}{15:14-16:48}{08:56-10:30}{18:23}{\ksha -- 18:18}{\purvashadha -- 09:31} 
&
\caldata{5}{05:47}{08:18}{12:05-13:39}{07:21-08:56}{18:23}{\ksap -- 20:32}{\uttarashadha -- 12:12} 
&
\caldata{6}{05:47}{08:18}{13:39-15:14}{05:47-07:21}{18:23}{\kasht -- 23:00}{\shravana -- 15:10} 
&
\caldata{7}{05:46}{08:17}{10:30-12:05}{15:14-16:49}{18:24}{\knav -- 25:27}{\shravishtha -- 18:11} 
&
\caldata{8}{05:46}{08:17}{08:55-10:30}{13:39-15:14}{18:24}{\kdas -- 27:39}{\shatabhishak -- 21:00} 
\\ \hline
\caldata{9}{05:46}{08:17}{16:49-18:24}{12:05-13:39}{18:24}{\keka -- 05:00}{\proshthapada -- 05:09} 
&
\caldata{10}{05:45}{08:16}{07:19-08:54}{10:29-12:04}{18:24}{\kdva -- 30:36}{\uttaraproshthapada -- 25:22} 
&
\caldata{11}{05:45}{08:17}{15:15-16:50}{08:55-10:30}{18:25}{\kdva -- 06:35}{\revati -- 26:42} 
&
\caldata{12}{05:45}{08:17}{12:05-13:40}{07:20-08:55}{18:25}{\ktra -- 07:08}{\ashwini -- 27:27} 
&
\caldata{13}{05:44}{08:16}{13:39-15:14}{05:44-07:19}{18:25}{\kchaturdashi -- 07:07}{\apabharani -- 27:40} 
&
\caldata{14}{05:44}{08:16}{10:29-12:04}{15:14-16:49}{18:25}{\ama -- 06:33}{\krittika -- 27:26} 
&
\caldata{15}{05:44}{08:16}{08:54-10:29}{13:40-15:15}{18:26}{\sdvi -- 28:09}{\rohini -- 26:49} 
\\ \hline
\caldata{16}{05:44}{08:16}{16:50-18:26}{12:05-13:40}{18:26}{\stri -- 26:29}{\mrigashirsha -- 25:55} 
&
\caldata{17}{05:43}{08:15}{07:18-08:53}{10:29-12:04}{18:26}{\scha -- 24:37}{\ardra -- 24:48} 
&
\caldata{18}{05:43}{08:15}{15:15-16:50}{08:53-10:29}{18:26}{\spanc -- 22:36}{\punarvasu -- 23:33} 
&
\caldata{19}{05:43}{08:15}{12:05-13:40}{07:18-08:54}{18:27}{\ssha -- 20:29}{\pushya -- 22:12} 
&
\caldata{20}{05:43}{08:15}{13:40-15:16}{05:43-07:18}{18:27}{\ssap -- 18:18}{\ashresha -- 20:48} 
&
\caldata{21}{05:43}{08:15}{10:29-12:05}{15:16-16:51}{18:27}{\sasht -- 16:06}{\magha -- 19:23} 
&
\caldata{22}{05:43}{08:16}{08:54-10:29}{13:41-15:16}{18:28}{\snav -- 13:55}{\purvaphalguni -- 17:59} 
\\ \hline
\caldata{23}{05:42}{08:15}{16:52-18:28}{12:05-13:40}{18:28}{\sdas -- 11:48}{\uttaraphalguni -- 16:39} 
&
\caldata{24}{05:42}{08:15}{07:17-08:53}{10:29-12:05}{18:28}{\seka -- 09:47}{\hasta -- 15:28} 
&
\caldata{25}{05:42}{08:15}{15:16-16:52}{08:53-10:29}{18:28}{\sdva -- 07:59}{\chitra -- 14:30} 
&
\caldata{26}{05:42}{08:15}{12:05-13:41}{07:17-08:53}{18:29}{\stra -- 06:27}{\svati -- 13:51} 
&
\caldata{27}{05:42}{08:15}{13:41-15:17}{05:42-07:17}{18:29}{\purnima -- 28:38}{\vishakha -- 13:36} 
&
\caldata{28}{05:42}{08:15}{10:29-12:05}{15:17-16:53}{18:29}{\kpra -- 28:30}{\anuradha -- 13:50} 
&
\caldata{29}{05:42}{08:15}{08:54-10:30}{13:42-15:18}{18:30}{\kdvi -- 28:57}{\jyeshtha -- 14:37} 
\\ \hline
\caldata{30}{05:42}{08:15}{16:54-18:30}{12:06-13:42}{18:30}{\ktri -- 30:01}{\mula -- 16:00} 
&
\caldata{31}{05:42}{08:15}{07:18-08:54}{10:30-12:06}{18:30}{\ktri -- 06:01}{\purvashadha -- 17:57} 
&
{}  &
{}  &
{}  &
{}  &
\\ \hline
\end{tabular}


%\clearpage
\begin{tabular}{|c|c|c|c|c|c|c|}
\multicolumn{7}{c}{\Large \bfseries JUNE 2010}\\
\hline
\textbf{SUN} & \textbf{MON} & \textbf{TUE} & \textbf{WED} & \textbf{THU} & \textbf{FRI} & \textbf{SAT} \\ \hline
{}  &
{}  &
\caldata{1}{05:42}{08:15}{15:18-16:54}{08:54-10:30}{18:31}{\kcha -- 07:40}{\uttarashadha -- 20:24} 
&
\caldata{2}{05:42}{08:15}{12:06-13:42}{07:18-08:54}{18:31}{\kpanc -- 09:45}{\shravana -- 23:13} 
&
\caldata{3}{05:42}{08:15}{13:42-15:18}{05:42-07:18}{18:31}{\ksha -- 12:07}{\shravishtha -- 26:12} 
&
\caldata{4}{05:42}{08:15}{10:30-12:06}{15:18-16:54}{18:31}{\ksap -- 14:31}{\shatabhishak -- 29:08} 
&
\caldata{5}{05:42}{08:16}{08:54-10:30}{13:43-15:19}{18:32}{\kasht -- 16:46}{\proshthapada -- 31:48} 
\\ \hline
\caldata{6}{05:42}{08:16}{16:55-18:32}{12:07-13:43}{18:32}{\knav -- 05:17}{\proshthapada -- 05:38} 
&
\caldata{7}{05:42}{08:16}{07:18-08:54}{10:30-12:07}{18:32}{\kdas -- 19:56}{\uttaraproshthapada -- 09:56} 
&
\caldata{8}{05:42}{08:16}{15:20-16:56}{08:54-10:31}{18:33}{\keka -- 20:35}{\revati -- 11:29} 
&
\caldata{9}{05:42}{08:16}{12:07-13:43}{07:18-08:54}{18:33}{\kdva -- 20:32}{\ashwini -- 12:22} 
&
\caldata{10}{05:42}{08:16}{13:43-15:20}{05:42-07:18}{18:33}{\ktra -- 19:49}{\apabharani -- 12:35} 
&
\caldata{11}{05:42}{08:16}{10:31-12:07}{15:20-16:56}{18:33}{\kchaturdashi -- 18:30}{\krittika -- 12:12} 
&
\caldata{12}{05:43}{08:17}{08:55-10:32}{13:44-15:21}{18:34}{\ama -- 16:41}{\rohini -- 11:16} 
\\ \hline
\caldata{13}{05:43}{08:17}{16:57-18:34}{12:08-13:44}{18:34}{\spra -- 14:28}{\mrigashirsha -- 09:56} 
&
\caldata{14}{05:43}{08:17}{07:19-08:55}{10:32-12:08}{18:34}{\sdvi -- 11:59}{\ardra -- 08:17} 
&
\caldata{15}{05:43}{08:17}{15:22-16:58}{08:56-10:32}{18:35}{\stri -- 09:20}{\punarvasu -- 06:26} 
&
\caldata{16}{05:43}{08:17}{12:09-13:45}{07:19-08:56}{18:35}{\scha -- 06:37}{\ashresha -- 26:40} 
&
\caldata{17}{05:43}{08:17}{13:45-15:22}{05:43-07:19}{18:35}{\ssha -- 25:25}{\magha -- 24:55} 
&
\caldata{18}{05:44}{08:18}{10:33-12:09}{15:22-16:58}{18:35}{\ssap -- 23:05}{\purvaphalguni -- 23:23} 
&
\caldata{19}{05:44}{08:18}{08:57-10:33}{13:46-15:23}{18:36}{\sasht -- 21:01}{\uttaraphalguni -- 22:06} 
\\ \hline
\caldata{20}{05:44}{08:18}{16:59-18:36}{12:10-13:46}{18:36}{\snav -- 19:15}{\hasta -- 21:08} 
&
\caldata{21}{05:44}{08:18}{07:20-08:57}{10:33-12:10}{18:36}{\sdas -- 17:50}{\chitra -- 20:30} 
&
\caldata{22}{05:44}{08:18}{15:23-16:59}{08:57-10:33}{18:36}{\seka -- 16:47}{\svati -- 20:16} 
&
\caldata{23}{05:45}{08:19}{12:10-13:46}{07:21-08:57}{18:36}{\sdva -- 16:10}{\vishakha -- 20:27} 
&
\caldata{24}{05:45}{08:19}{13:47-15:24}{05:45-07:21}{18:37}{\stra -- 15:59}{\anuradha -- 21:04} 
&
\caldata{25}{05:45}{08:19}{10:34-12:11}{15:24-17:00}{18:37}{\schaturdashi -- 16:17}{\jyeshtha -- 22:09} 
&
\caldata{26}{05:45}{08:19}{08:58-10:34}{13:47-15:24}{18:37}{\purnima -- 17:03}{\mula -- 23:42} 
\\ \hline
\caldata{27}{05:46}{08:20}{17:00-18:37}{12:11-13:47}{18:37}{\kpra -- 18:18}{\purvashadha -- 25:41} 
&
\caldata{28}{05:46}{08:20}{07:22-08:58}{10:35-12:11}{18:37}{\kdvi -- 19:58}{\uttarashadha -- 28:06} 
&
\caldata{29}{05:46}{08:20}{15:24-17:00}{08:58-10:35}{18:37}{\ktri -- 22:01}{\shravana -- 30:50} 
&
\caldata{30}{05:46}{08:20}{12:12-13:48}{07:22-08:59}{18:38}{\kcha -- 24:20}{\shravana -- 06:50} 
&
{}  &
{}  &
\\ \hline
\end{tabular}


%\clearpage
\begin{tabular}{|c|c|c|c|c|c|c|}
\multicolumn{7}{c}{\Large \bfseries JULY 2010}\\
\hline
\textbf{SUN} & \textbf{MON} & \textbf{TUE} & \textbf{WED} & \textbf{THU} & \textbf{FRI} & \textbf{SAT} \\ \hline
{}  &
{}  &
{}  &
{}  &
\caldata{1}{05:47}{08:21}{13:48-15:25}{05:47-07:23}{18:38}{\kpanc -- 26:45}{\shravishtha -- 09:48} 
&
\caldata{2}{05:47}{08:21}{10:36-12:12}{15:25-17:01}{18:38}{\ksha -- 29:07}{\shatabhishak -- 12:47} 
&
\caldata{3}{05:47}{08:21}{08:59-10:36}{13:48-15:25}{18:38}{\ksap -- 04:59}{\proshthapada -- 05:27} 
\\ \hline
\caldata{4}{05:47}{08:21}{17:01-18:38}{12:12-13:48}{18:38}{\ksap -- 07:11}{\uttaraproshthapada -- 18:07} 
&
\caldata{5}{05:48}{08:22}{07:24-09:00}{10:36-12:13}{18:38}{\kasht -- 08:47}{\revati -- 20:06} 
&
\caldata{6}{05:48}{08:22}{15:25-17:01}{09:00-10:36}{18:38}{\knav -- 09:46}{\ashwini -- 21:26} 
&
\caldata{7}{05:48}{08:22}{12:13-13:49}{07:24-09:00}{18:38}{\kdas -- 10:02}{\apabharani -- 22:03} 
&
\caldata{8}{05:49}{08:22}{13:49-15:25}{05:49-07:25}{18:38}{\keka -- 09:34}{\krittika -- 21:55} 
&
\caldata{9}{05:49}{08:22}{10:37-12:13}{15:25-17:01}{18:38}{\kdva -- 08:22}{\rohini -- 21:05} 
&
\caldata{10}{05:49}{08:22}{09:01-10:37}{13:49-15:25}{18:38}{\ktra -- 06:30}{\mrigashirsha -- 19:39} 
\\ \hline
\caldata{11}{05:49}{08:22}{17:01-18:38}{12:13-13:49}{18:38}{\ama -- 25:08}{\ardra -- 17:44} 
&
\caldata{12}{05:50}{08:23}{07:26-09:02}{10:38-12:14}{18:38}{\spra -- 21:57}{\punarvasu -- 15:29} 
&
\caldata{13}{05:50}{08:23}{15:26-17:02}{09:02-10:38}{18:38}{\sdvi -- 18:38}{\pushya -- 13:01} 
&
\caldata{14}{05:50}{08:23}{12:14-13:50}{07:26-09:02}{18:38}{\stri -- 15:19}{\ashresha -- 10:30} 
&
\caldata{15}{05:51}{08:24}{13:50-15:26}{05:51-07:26}{18:38}{\scha -- 12:08}{\magha -- 08:05} 
&
\caldata{16}{05:51}{08:24}{10:38-12:14}{15:26-17:02}{18:38}{\spanc -- 09:11}{\purvaphalguni -- 05:53} 
&
\caldata{17}{05:51}{08:24}{09:02-10:38}{13:50-15:26}{18:38}{\ssha -- 06:36}{\hasta -- 26:46} 
\\ \hline
\caldata{18}{05:51}{08:24}{17:02-18:38}{12:14-13:50}{18:38}{\sasht -- 27:00}{\chitra -- 25:58} 
&
\caldata{19}{05:52}{08:25}{07:27-09:03}{10:39-12:15}{18:38}{\snav -- 26:02}{\svati -- 25:43} 
&
\caldata{20}{05:52}{08:25}{15:25-17:01}{09:03-10:38}{18:37}{\sdas -- 25:39}{\vishakha -- 26:03} 
&
\caldata{21}{05:52}{08:25}{12:14-13:50}{07:27-09:03}{18:37}{\seka -- 25:49}{\anuradha -- 26:55} 
&
\caldata{22}{05:52}{08:25}{13:50-15:25}{05:52-07:27}{18:37}{\sdva -- 26:30}{\jyeshtha -- 28:18} 
&
\caldata{23}{05:53}{08:25}{10:39-12:15}{15:26-17:01}{18:37}{\stra -- 27:39}{\mula -- 30:07} 
&
\caldata{24}{05:53}{08:25}{09:04-10:39}{13:50-15:26}{18:37}{\schaturdashi -- 29:12}{\mula -- 06:07} 
\\ \hline
\caldata{25}{05:53}{08:25}{17:01-18:37}{12:15-13:50}{18:37}{\purnima -- 31:06}{\purvashadha -- 08:21} 
&
\caldata{26}{05:53}{08:25}{07:28-09:03}{10:39-12:14}{18:36}{\purnima -- 07:06}{\uttarashadha -- 10:54} 
&
\caldata{27}{05:54}{08:26}{15:25-17:00}{09:04-10:39}{18:36}{\kpra -- 09:17}{\shravana -- 13:42} 
&
\caldata{28}{05:54}{08:26}{12:15-13:50}{07:29-09:04}{18:36}{\kdvi -- 11:38}{\shravishtha -- 16:38} 
&
\caldata{29}{05:54}{08:26}{13:50-15:25}{05:54-07:29}{18:36}{\ktri -- 14:04}{\shatabhishak -- 19:39} 
&
\caldata{30}{05:54}{08:26}{10:39-12:14}{15:24-16:59}{18:35}{\kcha -- 05:34}{\proshthapada -- 05:20} 
&
\caldata{31}{05:54}{08:26}{09:04-10:39}{13:49-15:24}{18:35}{\kpanc -- 18:41}{\uttaraproshthapada -- 25:19} 
\\ \hline
\end{tabular}


%\clearpage
\begin{tabular}{|c|c|c|c|c|c|c|}
\multicolumn{7}{c}{\Large \bfseries AUGUST 2010}\\
\hline
\textbf{SUN} & \textbf{MON} & \textbf{TUE} & \textbf{WED} & \textbf{THU} & \textbf{FRI} & \textbf{SAT} \\ \hline
\caldata{1}{05:55}{08:27}{17:00-18:35}{12:15-13:50}{18:35}{\ksha -- 20:33}{\revati -- 27:42} 
&
\caldata{2}{05:55}{08:26}{07:29-09:04}{10:39-12:14}{18:34}{\ksap -- 21:56}{\ashwini -- 29:35} 
&
\caldata{3}{05:55}{08:26}{15:24-16:59}{09:04-10:39}{18:34}{\kasht -- 22:42}{\apabharani -- 30:49} 
&
\caldata{4}{05:55}{08:26}{12:14-13:49}{07:29-09:04}{18:34}{\knav -- 22:43}{\apabharani -- 06:47} 
&
\caldata{5}{05:55}{08:26}{13:48-15:23}{05:55-07:29}{18:33}{\kdas -- 21:59}{\krittika -- 07:16} 
&
\caldata{6}{05:56}{08:27}{10:39-12:14}{15:23-16:58}{18:33}{\keka -- 20:28}{\rohini -- 06:59} 
&
\caldata{7}{05:56}{08:27}{09:05-10:39}{13:48-15:23}{18:32}{\kdva -- 18:16}{\mrigashirsha -- 05:58} 
\\ \hline
\caldata{8}{05:56}{08:27}{16:57-18:32}{12:14-13:48}{18:32}{\ktra -- 15:28}{\punarvasu -- 25:58} 
&
\caldata{9}{05:56}{08:27}{07:30-09:05}{10:39-12:14}{18:32}{\kchaturdashi -- 12:13}{\pushya -- 23:19} 
&
\caldata{10}{05:56}{08:27}{15:22-16:56}{09:04-10:39}{18:31}{\ama -- 08:37}{\ashresha -- 20:26} 
&
\caldata{11}{05:56}{08:27}{12:13-13:47}{07:30-09:04}{18:31}{\sdvi -- 25:06}{\magha -- 17:30} 
&
\caldata{12}{05:57}{08:27}{13:47-15:21}{05:57-07:31}{18:30}{\stri -- 21:34}{\purvaphalguni -- 14:42} 
&
\caldata{13}{05:57}{08:27}{10:39-12:13}{15:21-16:55}{18:30}{\scha -- 18:23}{\uttaraphalguni -- 12:11} 
&
\caldata{14}{05:57}{08:27}{09:05-10:39}{13:47-15:21}{18:29}{\spanc -- 15:41}{\hasta -- 10:06} 
\\ \hline
\caldata{15}{05:57}{08:27}{16:55-18:29}{12:13-13:47}{18:29}{\ssha -- 13:35}{\chitra -- 08:38} 
&
\caldata{16}{05:57}{08:27}{07:30-09:04}{10:38-12:12}{18:28}{\ssap -- 12:12}{\svati -- 07:51} 
&
\caldata{17}{05:57}{08:27}{15:20-16:54}{09:04-10:38}{18:28}{\sasht -- 11:35}{\vishakha -- 07:50} 
&
\caldata{18}{05:57}{08:27}{12:12-13:45}{07:30-09:04}{18:27}{\snav -- 11:42}{\anuradha -- 08:32} 
&
\caldata{19}{05:57}{08:27}{13:45-15:19}{05:57-07:30}{18:27}{\sdas -- 12:29}{\jyeshtha -- 09:55} 
&
\caldata{20}{05:57}{08:26}{10:37-12:11}{15:18-16:52}{18:26}{\seka -- 13:51}{\mula -- 11:52} 
&
\caldata{21}{05:58}{08:27}{09:04-10:38}{13:44-15:18}{18:25}{\sdva -- 15:40}{\purvashadha -- 14:15} 
\\ \hline
\caldata{22}{05:58}{08:27}{16:51-18:25}{12:11-13:44}{18:25}{\stra -- 17:48}{\uttarashadha -- 16:57} 
&
\caldata{23}{05:58}{08:27}{07:31-09:04}{10:37-12:11}{18:24}{\schaturdashi -- 20:08}{\shravana -- 19:51} 
&
\caldata{24}{05:58}{08:27}{15:17-16:50}{09:04-10:37}{18:24}{\purnima -- 22:34}{\shravishtha -- 22:49} 
&
\caldata{25}{05:58}{08:27}{12:10-13:43}{07:31-09:04}{18:23}{\kpra -- 25:01}{\shatabhishak -- 25:49} 
&
\caldata{26}{05:58}{08:26}{13:43-15:16}{05:58-07:31}{18:22}{\kdvi -- 27:23}{\proshthapada -- 28:44} 
&
\caldata{27}{05:58}{08:26}{10:37-12:10}{15:16-16:49}{18:22}{\ktri -- 05:13}{\uttaraproshthapada -- 05:05} 
&
\caldata{28}{05:58}{08:26}{09:03-10:36}{13:42-15:15}{18:21}{\kcha -- 31:34}{\uttaraproshthapada -- 07:29} 
\\ \hline
\caldata{29}{05:58}{08:26}{16:48-18:21}{12:09-13:42}{18:21}{\kcha -- 07:33}{\revati -- 09:59} 
&
\caldata{30}{05:58}{08:26}{07:30-09:03}{10:36-12:09}{18:20}{\kpanc -- 09:08}{\ashwini -- 12:07} 
&
\caldata{31}{05:58}{08:26}{15:13-16:46}{09:03-10:35}{18:19}{\ksha -- 10:15}{\apabharani -- 13:47} 
&
{}  &
{}  &
{}  &
\\ \hline
\end{tabular}


%\clearpage
\begin{tabular}{|c|c|c|c|c|c|c|}
\multicolumn{7}{c}{\Large \bfseries SEPTEMBER 2010}\\
\hline
\textbf{SUN} & \textbf{MON} & \textbf{TUE} & \textbf{WED} & \textbf{THU} & \textbf{FRI} & \textbf{SAT} \\ \hline
{}  &
{}  &
{}  &
\caldata{1}{05:58}{08:26}{12:08-13:41}{07:30-09:03}{18:19}{\ksap -- 10:47}{\krittika -- 14:52} 
&
\caldata{2}{05:58}{08:26}{13:40-15:13}{05:58-07:30}{18:18}{\kasht -- 10:38}{\rohini -- 15:17} 
&
\caldata{3}{05:58}{08:25}{10:35-12:07}{15:12-16:44}{18:17}{\knav -- 09:47}{\mrigashirsha -- 14:58} 
&
\caldata{4}{05:58}{08:25}{09:02-10:35}{13:39-15:12}{18:17}{\kdas -- 08:12}{\ardra -- 13:57} 
\\ \hline
\caldata{5}{05:58}{08:25}{16:43-18:16}{12:07-13:39}{18:16}{\kdva -- 26:58}{\punarvasu -- 12:15} 
&
\caldata{6}{05:58}{08:25}{07:30-09:02}{10:34-12:06}{18:15}{\ktra -- 23:34}{\pushya -- 10:00} 
&
\caldata{7}{05:58}{08:25}{15:10-16:42}{09:02-10:34}{18:15}{\kchaturdashi -- 19:51}{\ashresha -- 07:18} 
&
\caldata{8}{05:58}{08:25}{12:06-13:38}{07:30-09:02}{18:14}{\ama -- 16:00}{\purvaphalguni -- 25:13} 
&
\caldata{9}{05:58}{08:25}{13:37-15:09}{05:58-07:29}{18:13}{\spra -- 12:10}{\uttaraphalguni -- 22:17} 
&
\caldata{10}{05:58}{08:24}{10:33-12:05}{15:08-16:40}{18:12}{\sdvi -- 08:30}{\hasta -- 19:39} 
&
\caldata{11}{05:58}{08:24}{09:01-10:33}{13:36-15:08}{18:12}{\scha -- 26:37}{\chitra -- 17:31} 
\\ \hline
\caldata{12}{05:58}{08:24}{16:39-18:11}{12:04-13:36}{18:11}{\spanc -- 24:41}{\svati -- 16:01} 
&
\caldata{13}{05:58}{08:24}{07:29-09:01}{10:32-12:04}{18:10}{\ssha -- 23:32}{\vishakha -- 15:17} 
&
\caldata{14}{05:58}{08:24}{15:07-16:38}{09:01-10:32}{18:10}{\ssap -- 23:14}{\anuradha -- 15:23} 
&
\caldata{15}{05:58}{08:24}{12:03-13:34}{07:29-09:00}{18:09}{\sasht -- 23:45}{\jyeshtha -- 16:19} 
&
\caldata{16}{05:58}{08:24}{13:34-15:05}{05:58-07:29}{18:08}{\snav -- 25:00}{\mula -- 17:58} 
&
\caldata{17}{05:58}{08:23}{10:31-12:02}{15:04-16:35}{18:07}{\sdas -- 26:50}{\purvashadha -- 20:14} 
&
\caldata{18}{05:58}{08:23}{09:00-10:31}{13:33-15:04}{18:07}{\seka -- 29:04}{\uttarashadha -- 22:56} 
\\ \hline
\caldata{19}{05:58}{08:23}{16:35-18:06}{12:02-13:33}{18:06}{\sdva -- 31:32}{\shravana -- 25:52} 
&
\caldata{20}{05:58}{08:23}{07:28-08:59}{10:30-12:01}{18:05}{\sdva -- 07:32}{\shravishtha -- 28:54} 
&
\caldata{21}{05:58}{08:23}{15:03-16:34}{08:59-10:30}{18:05}{\stra -- 10:03}{\shatabhishak -- 31:53} 
&
\caldata{22}{05:58}{08:23}{12:01-13:31}{07:28-08:59}{18:04}{\schaturdashi -- 12:29}{\shatabhishak -- 07:52} 
&
\caldata{23}{05:58}{08:23}{13:31-15:01}{05:58-07:28}{18:03}{\purnima -- 05:42}{\proshthapada -- 05:49} 
&
\caldata{24}{05:58}{08:22}{10:29-12:00}{15:01-16:31}{18:02}{\kpra -- 16:49}{\uttaraproshthapada -- 13:20} 
&
\caldata{25}{05:58}{08:22}{08:59-10:29}{13:30-15:01}{18:02}{\kdvi -- 18:36}{\revati -- 15:42} 
\\ \hline
\caldata{26}{05:58}{08:22}{16:30-18:01}{11:59-13:29}{18:01}{\ktri -- 20:05}{\ashwini -- 17:47} 
&
\caldata{27}{05:58}{08:22}{07:28-08:58}{10:28-11:59}{18:00}{\kcha -- 21:12}{\apabharani -- 19:32} 
&
\caldata{28}{05:58}{08:22}{14:59-16:29}{08:58-10:28}{18:00}{\kpanc -- 21:54}{\krittika -- 20:51} 
&
\caldata{29}{05:58}{08:22}{11:58-13:28}{07:28-08:58}{17:59}{\ksha -- 22:05}{\rohini -- 21:42} 
&
\caldata{30}{05:59}{08:22}{13:28-14:58}{05:59-07:28}{17:58}{\ksap -- 21:42}{\mrigashirsha -- 22:00} 
&
{}  &
\\ \hline
\end{tabular}


%\clearpage
\begin{tabular}{|c|c|c|c|c|c|c|}
\multicolumn{7}{c}{\Large \bfseries OCTOBER 2010}\\
\hline
\textbf{SUN} & \textbf{MON} & \textbf{TUE} & \textbf{WED} & \textbf{THU} & \textbf{FRI} & \textbf{SAT} \\ \hline
{}  &
{}  &
{}  &
{}  &
{}  &
\caldata{1}{05:59}{08:22}{10:28-11:58}{14:58-16:28}{17:58}{\kasht -- 20:42}{\ardra -- 21:40} 
&
\caldata{2}{05:59}{08:22}{08:58-10:28}{13:27-14:57}{17:57}{\knav -- 19:05}{\punarvasu -- 20:44} 
\\ \hline
\caldata{3}{05:59}{08:22}{16:26-17:56}{11:57-13:27}{17:56}{\kdas -- 16:51}{\pushya -- 19:12} 
&
\caldata{4}{05:59}{08:22}{07:28-08:58}{10:27-11:57}{17:56}{\keka -- 14:06}{\ashresha -- 17:08} 
&
\caldata{5}{05:59}{08:22}{14:56-16:25}{08:58-10:27}{17:55}{\kdva -- 10:55}{\magha -- 14:41} 
&
\caldata{6}{05:59}{08:22}{11:56-13:25}{07:28-08:57}{17:54}{\ktra -- 07:27}{\purvaphalguni -- 11:57} 
&
\caldata{7}{05:59}{08:22}{13:25-14:55}{05:59-07:28}{17:54}{\ama -- 24:15}{\uttaraphalguni -- 09:07} 
&
\caldata{8}{05:59}{08:21}{10:26-11:56}{14:54-16:23}{17:53}{\spra -- 20:57}{\hasta -- 06:22} 
&
\caldata{9}{05:59}{08:21}{08:57-10:26}{13:24-14:53}{17:52}{\sdvi -- 18:03}{\svati -- 25:57} 
\\ \hline
\caldata{10}{05:59}{08:21}{16:22-17:52}{11:55-13:24}{17:52}{\stri -- 15:43}{\vishakha -- 24:38} 
&
\caldata{11}{05:59}{08:21}{07:28-08:57}{10:26-11:55}{17:51}{\scha -- 14:06}{\anuradha -- 24:04} 
&
\caldata{12}{05:59}{08:21}{14:52-16:21}{08:56-10:25}{17:50}{\spanc -- 13:18}{\jyeshtha -- 24:19} 
&
\caldata{13}{05:59}{08:21}{11:54-13:23}{07:27-08:56}{17:50}{\ssha -- 13:23}{\mula -- 25:24} 
&
\caldata{14}{05:59}{08:21}{13:22-14:51}{05:59-07:27}{17:49}{\ssap -- 14:17}{\purvashadha -- 27:14} 
&
\caldata{15}{06:00}{08:21}{10:25-11:54}{14:51-16:20}{17:49}{\sasht -- 15:54}{\uttarashadha -- 29:39} 
&
\caldata{16}{06:00}{08:21}{08:57-10:25}{13:22-14:51}{17:48}{\snav -- 18:02}{\shravana -- 32:29} 
\\ \hline
\caldata{17}{06:00}{08:21}{16:19-17:48}{11:54-13:22}{17:48}{\sdas -- 20:28}{\shravana -- 08:30} 
&
\caldata{18}{06:00}{08:21}{07:28-08:56}{10:25-11:53}{17:47}{\seka -- 23:01}{\shravishtha -- 11:30} 
&
\caldata{19}{06:00}{08:21}{14:49-16:17}{08:56-10:24}{17:46}{\sdva -- 25:28}{\shatabhishak -- 14:29} 
&
\caldata{20}{06:00}{08:21}{11:53-13:21}{07:28-08:56}{17:46}{\stra -- 05:19}{\proshthapada -- 05:37} 
&
\caldata{21}{06:00}{08:21}{13:20-14:48}{06:00-07:28}{17:45}{\schaturdashi -- 29:34}{\uttaraproshthapada -- 19:49} 
&
\caldata{22}{06:01}{08:21}{10:25-11:53}{14:49-16:17}{17:45}{\purnima -- 31:06}{\revati -- 22:01} 
&
\caldata{23}{06:01}{08:21}{08:56-10:24}{13:20-14:48}{17:44}{\purnima -- 07:06}{\ashwini -- 23:51} 
\\ \hline
\caldata{24}{06:01}{08:21}{16:16-17:44}{11:52-13:20}{17:44}{\kpra -- 08:14}{\apabharani -- 25:19} 
&
\caldata{25}{06:01}{08:21}{07:28-08:56}{10:24-11:52}{17:44}{\kdvi -- 09:00}{\krittika -- 26:26} 
&
\caldata{26}{06:01}{08:21}{14:47-16:15}{08:56-10:24}{17:43}{\ktri -- 09:24}{\rohini -- 27:11} 
&
\caldata{27}{06:02}{08:22}{11:52-13:20}{07:29-08:57}{17:43}{\kcha -- 09:25}{\mrigashirsha -- 27:32} 
&
\caldata{28}{06:02}{08:22}{13:19-14:47}{06:02-07:29}{17:42}{\kpanc -- 09:03}{\ardra -- 27:29} 
&
\caldata{29}{06:02}{08:22}{10:24-11:52}{14:47-16:14}{17:42}{\ksha -- 08:16}{\punarvasu -- 27:01} 
&
\caldata{30}{06:02}{08:22}{08:57-10:24}{13:19-14:47}{17:42}{\ksap -- 07:02}{\pushya -- 26:05} 
\\ \hline
\caldata{31}{06:03}{08:22}{16:13-17:41}{11:52-13:19}{17:41}{\knav -- 27:13}{\ashresha -- 24:44} 
&
{}  &
{}  &
{}  &
{}  &
{}  &
\\ \hline
\end{tabular}


%\clearpage
\begin{tabular}{|c|c|c|c|c|c|c|}
\multicolumn{7}{c}{\Large \bfseries NOVEMBER 2010}\\
\hline
\textbf{SUN} & \textbf{MON} & \textbf{TUE} & \textbf{WED} & \textbf{THU} & \textbf{FRI} & \textbf{SAT} \\ \hline
{}  &
\caldata{1}{06:03}{08:22}{07:30-08:57}{10:24-11:52}{17:41}{\kdas -- 24:43}{\magha -- 23:00} 
&
\caldata{2}{06:03}{08:22}{14:45-16:12}{08:57-10:24}{17:40}{\keka -- 21:56}{\purvaphalguni -- 20:59} 
&
\caldata{3}{06:04}{08:23}{11:52-13:19}{07:31-08:58}{17:40}{\kdva -- 18:59}{\uttaraphalguni -- 18:46} 
&
\caldata{4}{06:04}{08:23}{13:19-14:46}{06:04-07:31}{17:40}{\ktra -- 15:58}{\hasta -- 16:31} 
&
\caldata{5}{06:04}{08:23}{10:25-11:52}{14:46-16:13}{17:40}{\kchaturdashi -- 13:04}{\chitra -- 14:21} 
&
\caldata{6}{06:05}{08:23}{08:58-10:25}{13:18-14:45}{17:39}{\ama -- 10:24}{\svati -- 12:27} 
\\ \hline
\caldata{7}{06:05}{08:23}{16:12-17:39}{11:52-13:18}{17:39}{\spra -- 08:08}{\vishakha -- 10:58} 
&
\caldata{8}{06:05}{08:23}{07:31-08:58}{10:25-11:52}{17:39}{\sdvi -- 06:25}{\anuradha -- 10:03} 
&
\caldata{9}{06:06}{08:24}{14:45-16:12}{08:59-10:25}{17:39}{\scha -- 29:13}{\jyeshtha -- 09:50} 
&
\caldata{10}{06:06}{08:24}{11:52-13:18}{07:32-08:59}{17:38}{\spanc -- 29:46}{\mula -- 10:23} 
&
\caldata{11}{06:06}{08:24}{13:18-14:45}{06:06-07:32}{17:38}{\ssha -- 31:04}{\purvashadha -- 11:41} 
&
\caldata{12}{06:07}{08:25}{10:26-11:52}{14:45-16:11}{17:38}{\ssha -- 07:05}{\uttarashadha -- 13:41} 
&
\caldata{13}{06:07}{08:25}{08:59-10:26}{13:18-14:45}{17:38}{\ssap -- 09:01}{\shravana -- 16:12} 
\\ \hline
\caldata{14}{06:08}{08:26}{16:11-17:38}{11:53-13:19}{17:38}{\sasht -- 11:21}{\shravishtha -- 19:03} 
&
\caldata{15}{06:08}{08:26}{07:34-09:00}{10:26-11:53}{17:38}{\snav -- 13:52}{\shatabhishak -- 22:01} 
&
\caldata{16}{06:08}{08:26}{14:45-16:11}{09:00-10:26}{17:38}{\sdas -- 05:49}{\proshthapada -- 05:30} 
&
\caldata{17}{06:09}{08:26}{11:53-13:19}{07:35-09:01}{17:38}{\seka -- 18:33}{\uttaraproshthapada -- 27:29} 
&
\caldata{18}{06:09}{08:26}{13:19-14:45}{06:09-07:35}{17:38}{\sdva -- 20:23}{\revati -- 29:40} 
&
\caldata{19}{06:10}{08:27}{10:28-11:54}{14:46-16:12}{17:38}{\stra -- 21:43}{\ashwini -- 31:22} 
&
\caldata{20}{06:10}{08:27}{09:02-10:28}{13:20-14:46}{17:38}{\schaturdashi -- 22:34}{\ashwini -- 07:21} 
\\ \hline
\caldata{21}{06:11}{08:28}{16:12-17:38}{11:54-13:20}{17:38}{\purnima -- 22:54}{\apabharani -- 08:32} 
&
\caldata{22}{06:11}{08:28}{07:36-09:02}{10:28-11:54}{17:38}{\kpra -- 22:47}{\krittika -- 09:15} 
&
\caldata{23}{06:12}{08:29}{14:46-16:12}{09:03-10:29}{17:38}{\kdvi -- 22:15}{\rohini -- 09:33} 
&
\caldata{24}{06:12}{08:29}{11:55-13:20}{07:37-09:03}{17:38}{\ktri -- 21:22}{\mrigashirsha -- 09:29} 
&
\caldata{25}{06:13}{08:30}{13:21-14:46}{06:13-07:38}{17:38}{\kcha -- 20:10}{\ardra -- 09:06} 
&
\caldata{26}{06:13}{08:30}{10:29-11:55}{14:46-16:12}{17:38}{\kpanc -- 18:42}{\punarvasu -- 08:25} 
&
\caldata{27}{06:14}{08:30}{09:05-10:30}{13:21-14:47}{17:38}{\ksha -- 17:00}{\pushya -- 07:29} 
\\ \hline
\caldata{28}{06:14}{08:30}{16:12-17:38}{11:56-13:21}{17:38}{\ksap -- 15:06}{\ashresha -- 06:21} 
&
\caldata{29}{06:15}{08:31}{07:40-09:05}{10:31-11:56}{17:38}{\kasht -- 13:01}{\purvaphalguni -- 27:32} 
&
\caldata{30}{06:15}{08:31}{14:48-16:13}{09:06-10:31}{17:39}{\knav -- 10:49}{\uttaraphalguni -- 25:58} 
&
{}  &
{}  &
{}  &
\\ \hline
\end{tabular}


%\clearpage
\begin{tabular}{|c|c|c|c|c|c|c|}
\multicolumn{7}{c}{\Large \bfseries DECEMBER 2010}\\
\hline
\textbf{SUN} & \textbf{MON} & \textbf{TUE} & \textbf{WED} & \textbf{THU} & \textbf{FRI} & \textbf{SAT} \\ \hline
{}  &
{}  &
{}  &
\caldata{1}{06:16}{08:32}{11:57-13:22}{07:41-09:06}{17:39}{\kdas -- 08:32}{\hasta -- 24:23} 
&
\caldata{2}{06:16}{08:32}{13:22-14:48}{06:16-07:41}{17:39}{\kdva -- 28:04}{\chitra -- 22:52} 
&
\caldata{3}{06:17}{08:33}{10:32-11:58}{14:48-16:13}{17:39}{\ktra -- 26:05}{\svati -- 21:31} 
&
\caldata{4}{06:17}{08:33}{09:07-10:32}{13:23-14:48}{17:39}{\kchaturdashi -- 24:25}{\vishakha -- 20:27} 
\\ \hline
\caldata{5}{06:18}{08:34}{16:14-17:40}{11:59-13:24}{17:40}{\ama -- 23:09}{\anuradha -- 19:46} 
&
\caldata{6}{06:18}{08:34}{07:43-09:08}{10:33-11:59}{17:40}{\spra -- 22:23}{\jyeshtha -- 19:34} 
&
\caldata{7}{06:19}{08:35}{14:49-16:14}{09:09-10:34}{17:40}{\sdvi -- 22:14}{\mula -- 19:56} 
&
\caldata{8}{06:20}{08:36}{12:00-13:25}{07:45-09:10}{17:41}{\stri -- 22:43}{\purvashadha -- 20:55} 
&
\caldata{9}{06:20}{08:36}{13:25-14:50}{06:20-07:45}{17:41}{\scha -- 23:50}{\uttarashadha -- 22:32} 
&
\caldata{10}{06:21}{08:37}{10:36-12:01}{14:51-16:16}{17:41}{\spanc -- 25:33}{\shravana -- 24:42} 
&
\caldata{11}{06:21}{08:37}{09:11-10:36}{13:26-14:51}{17:42}{\ssha -- 27:44}{\shravishtha -- 27:19} 
\\ \hline
\caldata{12}{06:22}{08:38}{16:17-17:42}{12:02-13:27}{17:42}{\ssap -- 30:12}{\shatabhishak -- 30:12} 
&
\caldata{13}{06:22}{08:38}{07:47-09:12}{10:37-12:02}{17:42}{\sasht -- 32:44}{\proshthapada -- 33:10} 
&
\caldata{14}{06:23}{08:39}{14:53-16:18}{09:13-10:38}{17:43}{\sasht -- 06:18}{\proshthapada -- 06:17} 
&
\caldata{15}{06:23}{08:39}{12:03-13:28}{07:48-09:13}{17:43}{\snav -- 11:03}{\uttaraproshthapada -- 11:54} 
&
\caldata{16}{06:24}{08:40}{13:29-14:54}{06:24-07:49}{17:44}{\sdas -- 12:58}{\revati -- 14:17} 
&
\caldata{17}{06:24}{08:40}{10:39-12:04}{14:54-16:19}{17:44}{\seka -- 14:22}{\ashwini -- 16:08} 
&
\caldata{18}{06:25}{08:41}{09:15-10:40}{13:30-14:55}{17:45}{\sdva -- 15:08}{\apabharani -- 17:23} 
\\ \hline
\caldata{19}{06:26}{08:41}{16:20-17:45}{12:05-13:30}{17:45}{\stra -- 15:14}{\krittika -- 18:00} 
&
\caldata{20}{06:26}{08:42}{07:51-09:16}{10:41-12:06}{17:46}{\schaturdashi -- 14:44}{\rohini -- 18:02} 
&
\caldata{21}{06:27}{08:42}{14:56-16:21}{09:16-10:41}{17:46}{\purnima -- 13:40}{\mrigashirsha -- 17:31} 
&
\caldata{22}{06:27}{08:43}{12:07-13:32}{07:52-09:17}{17:47}{\kpra -- 12:08}{\ardra -- 16:35} 
&
\caldata{23}{06:28}{08:43}{13:32-14:57}{06:28-07:52}{17:47}{\kdvi -- 10:15}{\punarvasu -- 15:18} 
&
\caldata{24}{06:28}{08:44}{10:43-12:08}{14:58-16:23}{17:48}{\ktri -- 08:06}{\pushya -- 13:48} 
&
\caldata{25}{06:28}{08:44}{09:18-10:43}{13:33-14:58}{17:48}{\kpanc -- 27:25}{\ashresha -- 12:10} 
\\ \hline
\caldata{26}{06:29}{08:45}{16:24-17:49}{12:09-13:34}{17:49}{\ksha -- 25:06}{\magha -- 10:31} 
&
\caldata{27}{06:29}{08:45}{07:54-09:19}{10:44-12:09}{17:49}{\ksap -- 22:53}{\purvaphalguni -- 08:54} 
&
\caldata{28}{06:30}{08:46}{15:00-16:25}{09:20-10:45}{17:50}{\kasht -- 20:49}{\uttaraphalguni -- 07:24} 
&
\caldata{29}{06:30}{08:46}{12:10-13:35}{07:55-09:20}{17:50}{\knav -- 18:58}{\chitra -- 29:01} 
&
\caldata{30}{06:31}{08:47}{13:36-15:01}{06:31-07:56}{17:51}{\kdas -- 17:23}{\svati -- 28:13} 
&
\caldata{31}{06:31}{08:47}{10:46-12:11}{15:01-16:26}{17:51}{\keka -- 16:04}{\vishakha -- 27:43} 
&
\\ \hline
\end{tabular}


%\clearpage

\end{center}
\end{document}
